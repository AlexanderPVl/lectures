\documentclass[a4paper, 12pt]{article}
\usepackage{header}


\begin{document}
\pagestyle{fancy}

\section{Лекция 2 от 06.09.2016. Быстрое преобразование Фурье}
\epigraph{Чтобы быть успешным программистом, надо знать 3 вещи: 
\begin{itemize}
\item Cортировки;
\item Хэширование;
\item Преобразование Фурье.
\end{itemize}
\leavevmode}{Глеб}

В этой лекции будет разобрано дискретное преобразование Фурье (Discrete Fourier
Transform).

\subsection{Применение преобразования Фурье}

Допустим, что мы хотим решить такую задачу:

\begin{Examples}
  Даны 2 бинарные строки $A$ и $B$ длины $n$ и $m$ соответственно. Мы хотим найти, какая подстрока в $A$ наиболее
  похожа на $B$. Наивная реализация решает эту задачу в худшем случае за
  $\O(n^2)$. Преобразование
  Фурье поможет решить эту задачу за $\O(n\log n)$, а именно научимся решать другую
  задачу:

{\bf Цель.}
  Хотим научиться перемножать многочлены одной степени \newline $A(x) = a_0 + a_1x + \ldots + a_{n - 1}x^{n - 1}$
  и $B(x) = b_0 + b_1x + \ldots + b_{n - 1}x^{n - 1}$ так, что 
  $C(x) = A(x)B(x)$, то есть считать \textit{свёртку} (найти все коэффициенты, если по-другому) 
  $\sum\limits_{i = 0}^{n - 1} \sum\limits_{j = 0}^i a_jb_{i - j}x^i$ за $\O(n\log n)$.

Вернёмся к нашему примеру. Поймём как с помощью нашей {\bf цели} решать задачу
про бинарные строки.

Пусть $A = a_0 \ldots a_{n - 1}, B = b_0\ldots b_{n - 1}$. Их можно считать одной длины
(просто добавим нулей в конец $b$ при надобности). Теперь задача 
переформулировывается как нахождение максимального скалярного произведение
$B$ и некоторых циклических сдвигов $A$ (до $n - m + 1$).

Инвертируем массив $B$ и припишем в конец $n$ нулей, а к массиву
$A$ припишем самого себя. Посмотрим на все коэффициенты перемножения:

\[
  c_k = \sum\limits_{i + j = k} a_ib_j
\]

Но $b_i = 0$ при $i \geqslant n$, поэтому при $k \geqslant n$:

\[
  c_k = \sum\limits_{i = 0}^{n - 1} b_ia_{k - i}
\]

Выбрав нужные коэффициенты, мы решили эту задачу.

\end{Examples}

\subsection{Алгоритм быстрого преобразования Фурье}

Основная идея алгоритма заключается в том, чтобы представить каждый многочлен
через набор $n$ точек и значений многочлена в этих точках, быстро (за $\O(n\log n)$) вычислить значения в каких-то $n$ точках для обоих многочленов, потом
перемножить за $\O(n)$ сами значения. Потом применить обратное преобразование
Фурье и получить коэффициенты $C(x) = A(x)B(x)$.

Итак, для начала будем считать, что $n = 2^k$ (просто добавим нулей до степени двойки).

Рассмотрим циклическую группу корней из 1 --- $W_n = \{\e^{i\frac{2\pi k}{n}} \ 
\forall \ k = 0, \ldots, n - 1\}$. Обозначим за $w_n = \e^{i\frac{2\pi}{n}}$.
Одно из самых главных свойств, что $w_n^{p} \cdot w_n^{q} = w_n^{p + q}$,
которым мы будем пользоваться в дальнейшем.

Воспользуемся идеей метода <<разделяй и властвуй>>.

Пусть $A(x) = a_0 + \ldots a_{n - 1}x^{n - 1}$.

Представим $A(x) = A_l(x^2) + xA_r(x^2)$ так, что

\begin{center}
  $A_l(x^2) = a_0 + a_2x^2 + \ldots + a_{n - 2}x^{n - 2}$

  $A_r(x^2) = a_1 + a_3x^2 + \ldots + a_{n - 1}x^{n - 2}$
\end{center}

\begin{Def}
  \rm{Назовём \textit{Фурье-образом} многочлена $P(x) = p_0 + \ldots + 
  p_{m - 1}x^{m - 1}$ вектор из $m$ элементов ---  $\langle P(1), P(w_m), P(w_m^2), \ldots, P(w_m^{m - 1}) \rangle$.}
\end{Def}

Теперь рекурсивно запускаемся от многочленов меньшей степени. Так как для 
любого целого неотрицательного $k$ следует, что $2k$ четное число, то $w_n^{2k} = w_{n/2}^k \in W_{n/2}$,
то есть мы можем уже использовать значения Фурье-образа для вычисления $A(x)$.

Если мы сможем за линейное время вычислить сумму $A_l(x^2) + xA_r(x^2)$, то
суммарное время работы будет $\O(n\log n)$, так как $A_l(x), A_r(x)$ имеют степень
в 2 раза меньше, чем $A(x)$.

Действительно это легко сделать из псевдокода, который приведен ниже:

\begin{algorithm}
  \caption{FFT}
  \begin{algorithmic}[1]
    \Function{FFT}{$A$} \Comment{A --- массив из комплексных чисел, функция 
    возвращает Фурье-образ}
    \Let{$n$}{length($A$)}
    \If{$n == 1$}
      \State{return A}
    \EndIf
    \Let{$A_l$}{$\langle a_0, a_2, \ldots, a_{n - 2}\rangle$}
    \Let{$A_r$}{$\langle a_1, a_3, \ldots, a_{n - 1}\rangle$}
    \Let{$\hat A_l$}{FFT($A_l$)}
    \Let{$\hat A_r$}{FFT($A_r$)}
    \For{$k \gets 0 \textrm{ to } \frac{n}{2} - 1$}
      \Let{$A[k]$}{$\hat A_l[k] + \e^{i\frac{2\pi k}{n}}\hat A_r[k]$}
      \Let{$A[k + \frac{n}{2}]$}{$\hat A_l[k] - \e^{i\frac{2\pi k}{n}}\hat 
      A_r[k]$} \Comment{Здесь минус перед комплексным числом из-за того, что мы
      должны найти другой угол,
      удвоенный которого на окружности будет $\frac{2\pi (k + n/2)}{n}$}
    \EndFor
    \State return A
    \EndFunction
  \end{algorithmic}
\end{algorithm}

Теперь поговорим про обратное FFT. Этого материала не было на лекции на момент
написания:

Фактически, мы вычислили такую вещь за $\O(n\log n)$:


\[
  \begin{pmatrix}
    w_n^0 & w_n^0 & w_n^0 & w_n^0 & \cdots & w_n^0 \cr 
    w_n^0 & w_n^1 & w_n^2 & w_n^3 & \cdots & w_n^{n-1} \cr 
    w_n^0 & w_n^2 & w_n^4 & w_n^6 & \cdots & w_n^{2(n-1)} \cr 
    w_n^0 & w_n^3 & w_n^6 & w_n^9 & \cdots & w_n^{3(n-1)} \cr 
    \vdots & \vdots & \vdots & \vdots & \ddots & \vdots \cr 
    w_n^0 & w_n^{n-1} & w_n^{2(n-1)} & w_n^{3(n-1)} & \cdots & w_n^{(n-1)(n-1)}
  \end{pmatrix}
  \begin{pmatrix}
     a_0 \cr 
     a_1 \cr 
     a_2 \cr 
     a_3 \cr 
     \vdots \cr 
     a_{n-1}
  \end{pmatrix}
   =
  \begin{pmatrix}
     y_0 \cr 
     y_1 \cr 
     y_2 \cr 
     y_3 \cr 
     \vdots \cr 
     y_{n-1}
  \end{pmatrix}
\]

где $y_i$ --- Фурье-образ многочлена $A(x)$.

Фактически нам надо найти обратное преобразование. Магическим образом обратная
матрица к квадратной матрице выглядит почти также:

\[
  \frac{1}{n}
  \begin{pmatrix}
    w_n^0 & w_n^0 & w_n^0 & w_n^0 & \cdots & w_n^0 \cr 
    w_n^0 & w_n^{-1} & w_n^{-2} & w_n^{-3} & \cdots & w_n^{-(n-1)} \cr 
    w_n^0 & w_n^{-2} & w_n^{-4} & w_n^{-6} & \cdots & w_n^{-2(n-1)} \cr 
    w_n^0 & w_n^{-3} & w_n^{-6} & w_n^{-9} & \cdots & w_n^{-3(n-1)} \cr 
    \vdots & \vdots & \vdots & \vdots & \ddots & \vdots \cr 
    w_n^0 & w_n^{-(n-1)} & w_n^{-2(n-1)} & w_n^{-3(n-1)} & \cdots & w_n^{-(n-1)(n-1)}
  \end{pmatrix}
\]


\[
  \frac{1}{n}
  \begin{pmatrix}
    w_n^0 & w_n^0 & w_n^0 & w_n^0 & \cdots & w_n^0 \cr 
    w_n^0 & w_n^{-1} & w_n^{-2} & w_n^{-3} & \cdots & w_n^{-(n-1)} \cr 
    w_n^0 & w_n^{-2} & w_n^{-4} & w_n^{-6} & \cdots & w_n^{-2(n-1)} \cr 
    w_n^0 & w_n^{-3} & w_n^{-6} & w_n^{-9} & \cdots & w_n^{-3(n-1)} \cr 
    \vdots & \vdots & \vdots & \vdots & \ddots & \vdots \cr 
    w_n^0 & w_n^{-(n-1)} & w_n^{-2(n-1)} & w_n^{-3(n-1)} & \cdots & w_n^{-(n-1)(n-1)}
  \end{pmatrix}
  \begin{pmatrix}
    y_0 \cr 
    y_1 \cr 
    y_2 \cr 
    y_3 \cr 
    \vdots \cr 
    y_{n-1}
  \end{pmatrix}
  =
  \begin{pmatrix}
    a_0 \cr 
    a_1 \cr 
    a_2 \cr 
    a_3 \cr 
    \vdots \cr 
    a_{n-1}
  \end{pmatrix}
\]

Откуда получаем: $a_k = \frac{1}{n}\sum_{j = 0}^{n - 1} y_jw_n^{-kj}$.

Теперь напишем псевдокод обратного алгоритма:

\begin{algorithm}
  \caption{FFT\_inverted}
  \begin{algorithmic}[1]
    \Function{FFT\_inverted}{$A$} \Comment{A --- Фурье-образ, возвращает коэффициенты многочлена}
    \Let{$n$}{length($A$)}
    \If{$n == 1$}
      \State{return A}
    \EndIf
    \Let{$A_l$}{$\langle a_0, a_2, \ldots, a_{n - 2}\rangle$}
    \Let{$A_r$}{$\langle a_1, a_3, \ldots, a_{n - 1}\rangle$}
    \Let{$\hat A_l$}{FFT\_inverted($A_l$)}
    \Let{$\hat A_r$}{FFT\_inverted($A_r$)}
    \For{$k \gets 0 \textrm{ to } \frac{n}{2} - 1$}
      \Let{$A[k]$}{$\hat A_l[k] + \e^{i\frac{-2\pi k}{n}}\hat A_r[k]$} 
      \Comment{Здесь угол идёт с минусом}
      \Let{$A[k + \frac{n}{2}]$}{$\hat A_l[k] - \e^{i\frac{-2\pi k}{n}}\hat 
      A_r[k]$}
      \Let{$A[k]$}{$A[k]/2$} \Comment{Поделим на 2 $\log n$ раз, а значит 
      поделим на $n$ в итоге}
      \Let{$A[k + \frac{n}{2}]$}{$A[k + \frac{n}{2}]/2$} \Comment{Аналогично 
      строчке выше}
    \EndFor
    \State return A
    \EndFunction
  \end{algorithmic}
\end{algorithm}

\end{document}
