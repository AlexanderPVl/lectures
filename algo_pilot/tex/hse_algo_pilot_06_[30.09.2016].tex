\documentclass[a4paper, 12pt]{article}
\usepackage{header}

\begin{document}
\pagestyle{fancy}

\section{Лекция 6 от 30.09.2016. Разбор некоторых интересных
задач по матроидам.}

\textit{Здесь мы разберем 3 важных задачи, 2 из которых, скорее всего Глеб включит
куда-нибудь (экзамен или что-то такое)}.

Я везде отождествляю элементы как одноэлементные множества.

\subsection{Матроид паросочетаний}

\begin{Lemma}
  Пара $\langle V, I\rangle$, где $V$ --- множество вершин
  некоторого графа $G = (V, E)$, и $B \in I$, если существует
  паросочетание, покрывающее множество $B$, является матроидом. Также его называют
  матроидом паросочетаний.
\end{Lemma}

\begin{proof}
  Первые 2 свойства матроида и правда ясны без объяснения (проделайте сами).

  Будем доказывать 3-е свойство.

  Давайте возьмём 2 множества вершин $B_1, B_2 \in I$ такие, что $|B_1| < |B_2|$.

  Пусть $X_1, X_2$ --- паросочетания, насыщающие $B_1$ и $B_2$ 
  соответственно.

  Есть 2 случая:

  1. Если существует элемент $x \in B_2 \setminus B_1$, насыщенный в $X_1$, то всё
  отлично, так как $X_1$ насыщает $B_1 \cup x$ 

  2. Теперь мы предполагаем, что все $x \in B_2 \setminus B_1$ не задействованы
  в вершинах в $X_1$. Рассмотрим подграф $G' = X_1 \triangle X_2$ --- 
  симметрическая разность ребер. Теперь степень каждой вершины не более 2, поэтому
  наш граф разбивается на циклы и цепочки, притом в циклах идёт чередование ребер,
  в цепочках тоже.

  Ясно, что в каждой цепи концевые вершины будут в одном паросочении не насыщены, так
  как иначе будет 
  четное число ребер, содержащую данную вершину в $G'$, значит это не конец пути.

  Так как ни одна вершина из $B_2 \setminus B_1$ (а их там хотя бы 1) не насыщена
  первым паросочетанием, то эти вершины могут быть только концами путей.

  Докажем, что $|B_2 \setminus B_1| > |B_1 \setminus B_2|$. Пусть $|B_1 \cap B_2| 
  = \alpha$, тогда $|B_2 \setminus B_1| = |B_2| - \alpha, |B_1 \setminus B_2| = 
  |B_1| - \alpha$, так как из каждого множества мы убираем только элементы из 
  пересечения, откуда $|B_2 \setminus B_1| > |B_1 \setminus B_2|$.

  Так как $|B_2 \setminus B_1| > |B_1 \setminus B_2|$, то существует путь $P$ (не
  цикл),
  начинающаяся в $B_2 \setminus B_1$ и заканчивающаяся {\bf не} в $B_1 \setminus
  B_2$. Заканчиваться путь в $B_1 \cap B_2$ не может, так как в этом множестве все
  вершины имеют четную степень в $G'$, значит этот путь заканчивается вне $B_1$.

  Докажем, что $X_1 \triangle P$ будет паросочетанием, насыщающим 
  $B_1 \cup v_{start}$, где $v_{start} \in B_2 \setminus B_1$ и начинает путь 
  $P$ с одного из концов. Фактически мы написали, что мы поменяем цвета этих
  рёбер, то есть те рёбра, которые были в $X_1 \cap P$, больше не ребра паросочетания,
  а ребра из $X_2$ на этом пути будут теперь рёбрами $X_1 \triangle P$. Все вершины из
  $B_1$,
  лежащие внутри пути (не на концах) останутся быть вершинами паросочетания 
  $X_1 \triangle P$. Единственная проблема может возникнуть с концами --- $v_{start}$
  теперь вершина паросочетания $X_1 \triangle P$, а другой конец не входил в $B_1$,
  значит даже если там не берется ребро, то ничего страшного, эта вершина нам и не
  нужна была.
\end{proof}

\subsection{Worst-out Greedy}

\begin{Lemma}[Worst-out Greedy]
  Пусть дан некоторый матроид $M = \langle M \rangle$, элементам присвоены
  некоторые стоимости $c(w)$, причём элементы $w_1, \ldots, w_n$ упорядочены
  так, что $0 \leqslant c(w_1) \leqslant \ldots \leqslant c(w_n)$. Рассмотрим
  следующий жадный алгоритм:

  \begin{algorithm}
  \caption{Worst-out greedy}
    \begin{algorithmic}[1]
      \Let{$F$}{$X$}
      \For {$i \gets 1 \textbf{ to } n$}
        \If{$F \setminus w_i$ содержит базу} 
          \Let{$F$}{$F \setminus w_i$}
        \EndIf
      \EndFor
    \end{algorithmic}
\end{algorithm}
\end{Lemma}

  Введем понятие {\it двойственного матроида}.

{\it Двойственный матроид} к $M = \langle X, I \rangle$ это матроид $M^* = 
\langle X, I^*\rangle$, где $I^* = \{A \ | \ \exists \ B \in \mathfrak{B} : A
\cap B = \varnothing\}$.

Докажем, что это матроид:

\begin{proof}

Проверим все 3 свойства.

1. Пустое множество лежит в этом множестве.

2. Пусть $A_1 \subseteq A_2$ и $A_2 \in I^*$, тогда $A_2 \cap B = \varnothing$,
тогда $A_1 \cap B = \varnothing$, так как подмножество пустого множества будет
пустым.

3. Возьмём произвольные $A_1, A_2$ такие, что $|A_1| < |A_2|$. Из определения
следует, что $X \setminus A_2$ сожержит какую-то базу --- пусть это будет база 
$B$. Мы знаем, что $B \setminus A_1 \subseteq X \setminus A_1$, так как $B \subseteq X$.

Также пусть $B_1' \subseteq X \setminus A_1$ --- та база, которая содержится в
дополнении $A_1$. 

Рассмотрим множества $B \setminus A_1$ и $B_1'$. Будем дополнять по 3 аксиоме
матроида $M$ первое множество, пока оно не станет базой. Пусть мы в итоге
получили $B' = B \setminus A_1 \cup \{x_1, \ldots, x_{|A_1|}\}$. Мы получили
множество $B'$, которое является базой, содержит $B \setminus A_1$ и содержится
в $X \setminus A_1$, так как мы добавляли элементы только из $B_1'$, а $B_1' \subseteq X \setminus A_1$.

Отлично, мы нашли базу $B'$, что $B \setminus A_1\subseteq B' \subseteq X \setminus A_1$.

Предположим, что $A_2 \setminus A_1 \subseteq B'$.

Также нам понадобится факт $B \cap A_1 \subseteq A_1 \setminus A_2$, который 
объясняется тем, что $B$ не имеет общих элементов с $A_2$ по определению.

Теперь выпишем цепочку неравенств и равенств и докажем каждое из них поочереди:

$|B| = |B \cap A_1| + |B \setminus A_1|$ --- простое свойство из теории множеств
(это просто все элементы лежащие в $B$ и в 1-ом случае и в $A_1$, а во 2-ом не в $A_1$).

$|B \cap A_1| + |B \setminus A_1| \leqslant |A_1 \setminus A_2| + |B \setminus A_1|$ ---
см. свойство выше.

$|A_1 \setminus A_2| + |B \setminus A_1| < |A_2 \setminus A_1| + |B \setminus A_1|$ ---
так как $|A_1 \setminus A_2|<|A_2 \setminus A_1|$, так как $|A_2| > |A_1|$ (см. факт
из 1-ой леммы).

$|A_2 \setminus A_1| + |B \setminus A_1| \leqslant |B'|$ --- это так, так как 2
множества слева под модулями не пересекаются (так как $B$ и $A_2$ не пересекаются).
И каждое из этих множеств является подмножеством $B'$ (1-ое по предположению, 2-ое
доказано выше).

Откуда $|B| < |B'|$, что неверно, так как все базы равномощны между собой.

Значит $A_2 \setminus A_1 \not\subseteq B'$, откуда существует элемент $z \in
A_2 \setminus A_1$ такой, что $z \not\in B'$, откуда 

\noindent$(A_1 \cup z) \cap B' = \varnothing$,
что нам и требуется.

\end{proof}

Теперь перейдём к доказательству леммы:

\begin{proof}

{\bf Свойство двойственности баз.}

Понятно, что база матроида $M^*$ будет дополнением всех элементов из базы $\mathfrak{B}$,
потому что для каждого дополнения существует база, с которой он пересекается по
пустому множеству. И если существует множество из $I^*$, которое по мощности
больше, чем мощность дополнения любого элемента из $\mathfrak{B}$, то такое
множество по принципу Дирихле пересекается со всеми элементами $\mathfrak{B}$,
а значит это множество не лежит в $I^*$. И если есть элемент базы $M^*$, который является недополнением
какого-то
элемента из базы $\mathfrak{B}$, то такое множество тоже пересекается со всеми
базами, так как базы имеют одинаковую мощность в обоих случаях.

Теперь чуточку перепишем алгоритм, данный в лемме.

\begin{algorithm}
  \caption{Модификация алгоритма на матроиде $M^*$(именно на двойственном)}
    \begin{algorithmic}[1]
      \Let{$F^*$}{$\varnothing$}
      \For {$i \gets 1 \textbf{ to } n$}
        \If{$F^* \cup w_i \in I^*$} 
          \Let{$F^*$}{$F^* \cup w_i$}
        \EndIf
      \EndFor
    \end{algorithmic}
\end{algorithm}

Заметим, что этот алгоритм возьмёт все те и только те элементы, которые алгоритм
из леммы выкинет. Докажем это по индукции:

Утверждение. На каждом шаге $i$ от 0 до $n$ выполняется $F^* \cup F = X, F \cap
F^* = \varnothing$.

База $i = 0$. На нулевом шаге у нас $F^* \cup F = X, F^* \cap F = \varnothing$.

Переход $i - 1 \to i$:

Если $F$ выкидывает $w_i$, то значит существует база $B_i$ такая, что $B_i
\subseteq F \setminus w_i$. А значит в двойственном матроиде $F^* \cup w_i \subseteq
X \setminus B_i$ (здесь мы пользуемся предположением индукции) , то есть $F^* 
\cup w_i$ является подмножеством какой-то базы 
матроида $M^*$, а значит $F^* \cup w_i \in I^*$. Это всё следует из свойств,
которые мы доказывали выше. В обратную сторону аналогично --- если $w_i$ добавляется
к $F^*$, то и $F$ выкинет $w_i$. Переход доказан.

В 1-ой лекции было, что алгоритм из решения приносит максимальную сумму, если веса
расположены в невозрастающем порядке. {\bf Дословно} переносятся все утверждения,
когда порядок неубывающий, только в итоге мы получим минимальную сумму множества.

Поэтому мы получили множество $F^*$, которое набрало минимальную сумму. Поэтому
$F$ наберет максимальную, так как $F \cup F^* = X, F \cap F^* = \varnothing$, откуда
$c(F) + c(F^*) = c(X)$, а значит $c(F)$ набирает максимум. Так как $F^*$
будет элементом базы (из следствия на 1-ой лекции) в матроиде $M^*$, то и $F$ будет
элементом базы в $M$ из-за
свойства двойственности баз (см. выше).
\end{proof}


\subsection{Лемма об обмене}

\begin{Lemma}[Лемма об обмене]
  Пусть имеются 2 базы $B_1, B_2$, тогда $\forall \ x \in B_1 \setminus B_2
  \ \exists \ y \in B_2 \setminus B_1$, такие, что $(B_1 \setminus x) \cup y \in I$ и
  $(B_2 \setminus y) \cup x \in I$.
\end{Lemma}

Здесь у нас будет матроид $M = \langle S, I \rangle$

Введем понятие {\it цикла}. {\it Цикл} это наименьшее по включению зависимое
множество, то есть все собственные подмножества цикла принадлежат $I$.

Все ссылки на леммы, которые я делаю в этой теореме, это ссылки
на леммы <<задачи>>.

Докажем такие леммы:

{\bf Лемма задачи 1.} {\it $r(A) + r(B) \geqslant r(A \cup B) + r(A \cap B)$.}

\begin{proof}
  Пусть $X$ максимальное независимое подмножество $A \cap B$. Возьмём $Y'$ ---
  максимальное независимое множество $A \cup B$. Заметим, что $|X| \leqslant 
  |Y'|$, поэтому будем по 3 аксиоме матроидов добавлять к $X$ элементы из $Y' 
  \setminus X$. Получим максимальное независимое подмножество $A \cup B$, которое
  содержит $X$. Пусть это подмножество будет $Y \Rightarrow X \subseteq Y$.

  Разделим $Y$ на 3 категории множеств --- $Y = X \cup V \cup W$ так, что
  $V \subseteq A \setminus B, W \subseteq B \setminus A$. Так и будет, потому
  что из пересечения мы не могли добавить к $X$ больше элементов, иначе $X$ был
  бы не максимальным по включению.

  Получаем, что $X \cup V$ --- независимо в $A$ (аксиома 2), аналогично $X \cup W$
  независимо в $B$, откуда $r(A) \geqslant |X \cup V|, r(B) \geqslant |X \cup W|$.

  Откуда $r(A) + r(B) \geqslant |X \cup V| + |X \cup W| = |X| + |X| + |W| + |V| =
  r(A \cup B) + r(A \cap B)$.
\end{proof}

{\bf Лемма задачи 2.} {\it Пусть $C_1, C_2$ --- различные циклы одного матроида и $x \in
C_1 \cap C_2$. Тогда существует цикл $C_3 \subseteq (C_1 \cup C_2) \setminus x$.}

\begin{proof}
  Покажем, что $r(C') < |C'|$, где $C' = (C_1 \cup C_2) \setminus x$.

  Так как $C_1, C_2$ --- различные циклы, то $C_1 \cap C_2$ является собственным
  подмножеством $C_1$, то есть $r(C_1 \cap C_2) = |C_1 \cap C_2|$.

  Также мы знаем, что $r(C_1) = |C_1| - 1, r(C_2) = |C_2| - 1$, так как это циклы.

  По лемме 1 получаем

  \[
    r(C_1 \cup C_2) \leqslant r(C_1) + r(C_2) - r(C_1 \cap C_2) = |C_1| + |C_2|-
    |C_1 \cap C_2| - 2 = |C_1 \cup C_2| - 2 < |C'|
  \]

  Также мы знаем, что $r(C') \leqslant r(C_1 \cup C_2)$, так как $C' \subseteq
  C_1 \cup C_2$.

  Откуда $r(C') < |C'|$. Значит существует цикл в таком множестве.
\end{proof}

{\bf Лемма задачи 3.} {\it Если $A$ независимое множество, а $x \in S$, тогда $A \cup x$
содержит не более 1 цикла.}

\begin{proof}
  Пусть есть 2 различных цикла $C_1, C_2 \subseteq (A \cup x)$. Они оба содержат
  $x$, иначе они независимы. 

  Рассмотрим множество $(C_1 \cup C_2) \setminus x$. По лемме 2 в этом множестве
  есть цикл, а значит $(C_1 \cup C_2) \setminus x$ зависимо. Но $(C_1 \cup C_2)
  \setminus x \subseteq A$, что противоречит независимости $A$.
\end{proof}

Также введём ещё понятие для любого подмножества $A \subseteq S$~---
$span(A) = \{s \in S : r(A \cup s) = r(S)\}$. Тривиально, что $A \subseteq span(A)$.

{\bf Лемма задачи 4.} a) Если $A \subseteq B$, то $span(A) \subseteq span(B)$

b) Если $e \in span(A)$, то $span(A \cup e) = span(A)$.

\begin{proof}
  a) Пусть $e \in span(A)$. Если $e \in B$, то отсюда сразу следует, что $e \in
  span(B)$ (см. тривиальное свойство). По лемме 1 следует, что
  $r(A \cup e) + r(B) \geqslant r((A \cup e) \cap B) + r((A \cup e) \cup B)$

  Откуда сразу следует, что
  $r(A \cup e) + r(B) \geqslant r(A) + r(B \cup e)$, так как $e \not\in B$, поэтому
  $(A \cup e) \cap B = A, (A \cup e) \cup B = B \cup e$. Из определения следует,
  что $r(A \cup e) = r(A)$, значит

  $r(B) \geqslant r(B \cup e)$, но мы знаем, что все независимые подмножества $B$
  являются независимыми множествами $B \cup e$, значит в другую сторону
  неравенство выполняется очевидно.

  Поэтому $r(B) = r(B \cup e)$, откуда $e \in span(B)$.

  b) Из пункта a) следует, что $span(A) \subseteq span(A \cup e)$. Поэтому нам
  надо доказать для каждого $f \in span(A \cup e)$, что оно лежит в $span(A)$.
  Опять воспользуемся леммой 1:

  $r(A \cup e) + r(A \cup f) \geqslant r((A \cup e) \cap (A \cup f)) + r(A \cup e \cup A \cup f)$

  Случай $e = f$ очевиден, пусть $e \neq f$.

  Тогда

  $r(A \cup e) + r(A \cup f) \geqslant r(A) + r(A \cup e \cup f)$

  $r(A \cup e) = r(A)$ по определению. Значит

  $r(A \cup f) \geqslant r(A \cup e \cup f)$ откуда аналогично следует, что
  $r(A \cup f) = r(A \cup e \cup f)$.

  Откуда $e \in span(A \cup f)$, но $e \in span(A)$ и $f \in span(A \cup e)$, значит
  $r(A) = r(A \cup e) = r(A \cup e \cup f) = r(A \cup f)$ --- последнее
  равенство следует из выше доказанного. Поэтому $f \in span(A)$, что и требовалось.
\end{proof}

{\bf Доказательство леммы.}
\begin{proof}

Пусть $x \in B_1 \setminus B_2$, тогда $B_2 \cup x$ содержит ровно 1 цикл по
лемме 3 (так как $B_2$ независимо, а $B_2 \cup x$ зависимо). Пусть этот цикл
будет $C$. Мы знаем, что $x \in span(C \setminus x)$, так как добавление не 
меняет ранг. Поэтому $x \in span((B_1 \cup C) \setminus x)$ (см. лемма 4а). По 
лемме 4b следует, что
$span((B_1 \cup C) \setminus x) = span(B_1 \cup C) = S$, так как $B_1$ является 
базой. Получается, что какой-бы элемент к максимально независимому множеству 
в $(B_1 \cup C) \setminus x$ ни добавляй, получим, что ранг меняться не будет. 
Это возможно только если $r((B_1 \cup C) \setminus x) = |B_1|$, иначе мы могли
бы получить противоречие с аксиомой 3 матроидов.

Пусть $B'$ --- база, содержащаяся в $(B_1 \cup C) \setminus x$.

По аксиоме 3 (для $B_1 \setminus x$ и $B'$) в $B' \setminus (B_1 \setminus x)$ существует $y$, что 
$(B_1 \setminus x) \cup y$ --- база. 

$B' \setminus (B_1 \setminus x) \subseteq ((B_1 \cup C) \setminus x) \setminus
(B_1 \setminus x)$ (см. 2 абзаца выше).

$((B_1 \cup C) \setminus x) \setminus (B_1 \setminus x) \subseteq C \setminus x$
--- легко проверяется кругами Эйлера. То есть $y \in C \setminus x$. Но $C 
\subseteq (B_2 \cup x)$, поэтому $C \setminus x \subseteq B_2$. Значит $y \in B_2$,
значит $y \in B_2 \setminus B_1$ (см. выше, почему $y \not\in B_1$). Также важно отметить, что $x \neq y$, так как $B' \subseteq (B_1 \cup C)\setminus x$.

Докажем, что $(B_2 \setminus y) \cup x$ --- тоже база. Допустим, что это не так.
Тогда существует цикл
$C' \subseteq (B_2 \setminus y) \cup x \subseteq B_2 \cup x$. Причем $C' \neq C$,
так как $y \in C$ (см. выше), но $y \not\in C'$. Значит у $B_2 \cup x$ существовало 2 различных цикла,
что невозможно по лемме 3. Что и завершает наше доказательство.
\end{proof}

\end{document}
