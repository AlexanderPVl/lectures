\documentclass[a4paper, 12pt]{article}
\usepackage{header}

\begin{document}
\pagestyle{fancy}

\section{Лекция 7 от 04.10.2016. Венгерский алгоритм решение задачи о назначениях}

\subsection{Формальная постановка задачи}
Оригинальная постановка задачи о назначениях
выгдядит так: нам дана матрица $n \times n$

\[
  \begin{pmatrix}
    a_{11} & \cdots & a_{1n} \cr 
    \vdots & \ddots & \vdots \cr 
    a_{n1} & \cdots & a_{nn}
  \end{pmatrix},
\]

где строки отвечают за работников, которые требуют определенную сумму за то
или иное задание (столбцы).

Требуется найти такую перестановку $p(i)$, что $\sum\limits_{i = 1}^{n} 
a_{ip(i)} \to \min.$  Обозначим эту задачу за $A'$.

\subsection{Немного линейного программирования}

Выпишем задачу линейного программирования (назовём её $B'$).

\begin{align*}
  \sum\limits_{i = 1}^n \sum\limits_{j = 1}^n a_{ij}x_{ij} \to & \min\\
  \forall i \in \overline{1\ldots n} \implies \sum\limits_{j = 1}^n x_{ij} &= 1,\\
  \forall j \in \overline{1\ldots n} \implies \sum\limits_{i = 1}^n x_{ij} &= 1,\\
  x_{ij} &\geqslant 0.
\end{align*}

Ясно, что если есть решение задачи $A'$, то оно является каким-то решением задачи $B'$,
так как мы в каждой строке и столбце у нас ровно одно $x_{ij} = 1$, поэтому
все такие решения подойдут под $B'$. Значит решение задачи $B'$ не хуже, чем у
$A'$. Выпишем этой задаче двойственную (что, конечно же, было проделано
много раз на курсе дискретной математики). Двойственные переменные $u_i$ отвечают
за строки, $v_i$ за столбцы. Напомним пару двойственных задач:

\[
  \begin{cases}
    cx \to \max\\
    Ax \leqslant b.
  \end{cases}
  \text{ двойственна }
  \begin{cases}
    yb \to \min\\
    yA = c,\\
    y \geqslant 0.
  \end{cases}
\]

Поэтому двойственная к $B'$ будет

\begin{align*}
  \sum\limits_{i = 1}^n u_i + \sum\limits_{j = 1}^n v_j \to &\max\\
  \forall i, j \in \overline{1\ldots n} \implies u_i + v_j \leqslant a_{ij}.
\end{align*}

Это действительно так, потому что в строках матрицы $A$ задачи $B'$ будет лишь
2 элемента --- отвечающее за строку и столбец, в котором переменная $x_{ij}$
находится.

Обозначим эту задачу за $C'$.

Как мы знаем из курса дискретной математики, двойственная и прямая задачи
имеют одно и то же оптимальное значение, поэтому оптимальные значения $B'$ и $C'$
совпадают. Также легко убедиться, что каждая задача имеет хоть какое-то
решение.

Заметим, что любое решение задачи $A'$ является решением задачи $C'$. Действительно,

\begin{equation}\label{main}
  \sum\limits_{i = 1}^n a_{ip(i)} \geqslant \sum\limits_{i = 1}^n u_i + v_{p(i)} =
  \sum\limits_{i = 1}^n u_i + \sum\limits_{j = 1}^n v_j.
\end{equation}

То есть любое допустимая перестановка не меньше, чем любое значение целевой
функции задачи $C'$, поэтому если мы предъявим хоть какое-то решение, что 
значение целевой функции задачи $C'$ совпадает с каким-то из задачи $A'$, то мы найдём
оптимальное решение (достигается равенство выше). И тогда все оптимальные решения
задач $A', B', C'$ совпадают. Предъявим конструктивно алгоритм построения решения
$C'$.

\subsection{Венгерский алгоритм}

Нарисуем двудольный граф, где левая доля отвечает переменным $u_i$, правая
за $v_j$. 

\begin{Def}
  Назовём ребро $(i, j)$ \textbf{жёстким}, если $u_i + v_j = a_{ij}$.
\end{Def}

Заметим, что если мы найдём совершенное паросочетание на жёстких ребрах, то
мы найдём какую-то перестановку $p(i)$, тогда мы найдём какое-то решение, где
$u_i + v_{p(i)}$ будет входить в сумму целевой функции, как $a_{ip(i)}$, а значит
мы найдём какое-то решение задачи $A'$. Поэтому надо всего лишь <<подвигать
значения переменных>>, чтобы получить совершенное паросочетание на жёстких ребрах.

Мы будем пытаться добавлять вершины левой доли по одной на каждом шаге, в предположении,
что на предыдущих шагах мы смогли всё добавить. Пусть мы добавляем вершину $i$,
а на всех $i-1$ предыдущих максимальное паросочетание можно построить. Запустим
алгоритм Куна из $i$-ой вершины. Если мы нашли паросочетание и на $i$-ой вершине,
то отлично, мы увеличили паросочетание. Теперь допустим, что не смогли увеличить.

Обозначим за $R_{+}$ --- все вершины, которые мы достигли алгоритмом
Куна в правой доли из вершины $i$,
$L_{+}$ в левой. Аналогично определяются множества вершин
$R_{-}$ и $L_{-}$. В $R_{+}$ вершин меньше, так как для любой вершины из
$R_+$ алгоритм Куна посетит вместе с ней какую-то вершину правой доли из
паросочетания. 

Заметим, что нет жёстких ребёр из $L_+$ в $R_-$, так как иначе мы могли бы
найти удлиняющий путь (просто пройдя как-то по этой вершине).

Теперь введём обозначение $d = \min(a_{ij} - u_i - v_j), i \in L_+, j \in R_-$.
Как мы только что показали, $d > 0$, так как ни одно ребро не жёсткое.

Теперь применим данные действия к алгоритму:

\begin{itemize}
  \item Увеличим все $u_i$ на $d$ для $i \in L_+$,
  \item Уменьшим все $v_j$ на $d$ для $j \in R_+$. 
\end{itemize}

Заметим, что целевая функция увеличилась хотя бы на $d$, так как в $L_+$ больше
вершин, чем в $R_+$. Осталось проверить, что все неравенства линейной задачи $C'$
сохранятся:

\begin{itemize}
  \item На ребрах из $L_+ \to R_+$ ничего не изменится, так как мы добавили
  $d$ и вычли $d$, в частности жёсткие ребра останутся жёсткими.
  \item На ребрах из $L_- \to R_+$ мы уменьшим $u_i + v_j$ на $d$, но уменьшать
  нам не запрещено, так как неравенство сохранится,
  \item На ребрах из $L_+ \to R_-$ мы по определению не получим противоречия
  в неравенствах $u_i + d + v_j \leqslant a_{ij}$, так как $d \leqslant a_{ij} - u_i - v_j$,
  \item Ребра из $L_- \to R_-$ мы никак не модифицировали, поэтому всё корректно
  будет и тут.
\end{itemize}

Поэтому из $L_+$ после пересчёта переменных будет ещё одно жёсткое ребро, которое
ведёт в $R_-$, а значит алгоритмом Куна мы строго увеличим количество посещенных
вершин.

Бесконечно этот процесс не может происходить, так как мы не можем строго увеличивать
бесконечно количество посещённых вершин. Максимум через $n$ действий мы получим,
что $R_-$ состоит из одной вершины, а все вершины левой доли мы посетили, поэтому
мы сможем найти удлиняющую цепочку.

\subsection{Понижение асимптотики алгоритма до $\O(n^3)$}

Сейчас наш алгоритм работает за $\O(n^4)$. Действительно, мы должны добавить $n$
вершин, каждый раз запуская алгоритм Куна от вершины не более $n$ раз, пересчет
переменных происходит за $\O(n^2)$ и за $\O(n^2)$ будет работать одна фаза алгоритма
Куна, поэтому асимптотика $\O(n^4)$, но, можно и быстрее, давайте поймём как
проапгрейдить наш алгоритм. 

Ключевая идея: теперь мы будем добавлять в рассмотрение строки матрицы одну за 
одной, а не рассматривать их все сразу. Таким образом, описанный выше алгоритм примет вид:

\begin{itemize}
  \item Добавляем в рассмотрение очередную строку матрицы $a$.
  \item Пока нет увеличивающей цепи, начинающейся в этой строке, пересчитываем переменные.
  \item Как только появляется увеличивающая цепь, чередуем паросочетание вдоль неё 
  (включая тем самым последнюю строку в паросочетание), и переходим к началу 
  (к рассмотрению следующей строки).
\end{itemize}

Чтобы достичь требуемой асимптотики, надо реализовать шаги 2-3, выполняющиеся 
для каждой строки матрицы, за время $\O(n^2)$.

Заметим, что если вершина была достижима из $i$-ой, то после пересчётов переменных она
останется достижимой, так как жёсткие ребра остаются жесткими, и пересчет
переменных выполняется $\O(n)$ раз.

Также заметим, что 
для проверки наличия увеличивающей цепочки нет необходимости запускать алгоритм
Куна заново после каждого пересчёта переменной. 
Вместо этого можно оформить light версию обхода Куна: после каждого 
пересчёта переменной мы просматриваем добавившиеся жёсткие рёбра и, если их 
левые концы были достижимыми, помечаем их правые концы также как достижимые и 
продолжаем обход из них.

Введём также величину $c_j$ для всех $j$:

\[
  c_j = \min\limits_{i \in L_+} (a_{ij} - u_i - v_j)
\]

Тогда наша $d = \min\limits_{j \in R_-} c_j$. Это и есть в точности те вершины правой,
которые мы ещё не посетили, соединенных с вершинами левой доли $L_+$. Это делается
за $\O(n)$.

Теперь легко изменять $c_j$ при расстановке потенциалов --- надо просто вычесть
$d$ из всех $c_j, j \in R_-$. Если мы добавляем в левой доли вершину $k$, надо лишь
обновить все $c_j = \min(c_j, a_{kj} - u_k - v_j)$. Это делается за $\O(n)$.

А инициализировать $c_j$ надо
при добавлении $i$-ой вершины, как $c_j = a_{ij} - u_i - v_j$, так как пока
только 1 вершина посещена --- эта вершина на $i$-ой фазе алгоритма.

Теперь поймём, почему теперь алгоритм работает за $\O(n^3)$. Есть, как и в
прошлой версии, внешняя фаза --- добавление вершин левой доли за $\O(n)$. Теперь
обновление переменных выполняется за $\O(n)$, обновление массива $c_j$ происходит
при каждом обновлении за $\O(n)$, что даёт $\O(n^2)$ на каждом шаге. Плюс
ещё мы должны во внешней фазе запустить алгоритм Куна, чтобы найти паросочетание (мы
уже уверены, что оно есть), что тоже $\O(n^2)$. В итоге $\O(n^3)$. Приведем
псевдокод:

\begin{algorithm}
  \caption{Венгерский алгоритм}
    \begin{algorithmic}[1]
      \Let{$M$}{$\varnothing$} \Comment{Поддерживающееся паросочетание на 
      жёстких рёбрах}
      \For{$i \gets 1 \textbf{ to } n$}
        \Let{$c_j$}{$a_{ij} - u_i - v_j$} \Comment{По всем $j$}
        \Let{$flag$}{$true$}
        \While{$flag$}
          \Let{$d$}{$\min{c_j}$} \Comment{По всем $j \in R_-$ --- надо найти это самое $d$}
          \Let{$u_k$}{$u_k + d$} \Comment{По всем $k \in L_+$ --- см. выше}
          \Let{$v_k$}{$v_k - d$} \Comment{По всем $k \in R_+$ --- см. выше}
          \Let{$c_k$}{$c_k - d$} \Comment{По всем $k \in R_-$ --- см. выше}
          \For{$k \in R_-$}
            \If{$c_k = 0$} \Comment{Значит вершина стала достижимой}
              \If{$k \in Right(M)$} \Comment{Вдруг, наша вершина оказалась уже в паросочетании, то мы ещё не можем увеличить}
                \Let{$L_+$}{$L_+ + Left\_neighbour(k)$} \Comment{Добавляем посещенную в $L_+$, которая насыщена паросочетанием $M$}
                \Let{$R_+$}{$R_+ + k$} \Comment{Суммарно добавлений будет не более $\O(n)$}
                \Let{$q$}{$Left\_neighbour(k)$}
                \Let{$c_p$}{$\min(c_p, a_{qp} - u_q - v_p)$} \Comment{надо обновить после добавления по всем $p \in R_-$}
              \Else
                \Let{$flag$}{$false$} \Comment{уже точно знаем, что можем увеличить паросочетание}
                \State{$break$} \Comment{надо выйти вообще, чтобы запустить алгоритм Куна}
              \EndIf
            \EndIf
          \EndFor
        \EndWhile
        \Let{$M$}{$kunh\_algo(i)$}
      \EndFor
    \end{algorithmic}
\end{algorithm}

Действительно, в $L_+$ и $R_+$ мы добавляем не больше, чем $\O(n)$ раз, значит
и 16 строка выполнится не более $\O(n)$ раз, значит в цикле с 5 строки мы
в целом будем выполнять не более $\O(n^2)$ действий. Также
алгоритм Куна будет выполняться не более $\O(n^2)$ действий на каждой итерации.
А значит общая асимптотика равна $\O(n^3)$.

Можно немного сэкономить на практике
и не писать алгоритм Куна, а запоминать, из какой
вершины левой доли мы пошли в правую, а вместо алгоритма Куна делать всего
$\O(n)$ действий, идя обратно по пути.

{\bf Байка от Глебаса:} Была одна команда на ACM, вроде из Саратова. На финал
можно было принести с тобой несколько листов заготовленного кода.
Одна из задач была на венгерский алгоритм, и у команды была распечатка по этому алгоритму.
В итоге она у них не заходила, потому что когда они тестили у себя в контесте,
видимо, тесты были слабые. Мораль: ставьте assertы везде, где можно, чтобы
убедиться, что этот алгоритм работает корректно.

\subsection{Венгерский алгоритм на прямоугольной матрице}

Будем считать, что $n \leqslant m$, иначе просто транспонируем матрицу.

Здесь всё тоже самое, только давайте теперь оценим время работы, если у нас
матрица размера $n \times m$. Заметим, что в алгоритме мы будем добавлять не больше,
чем $\min(n, m) = n$ вершин, так как мы либо ничего не делаем, либо добавляем по
одному элементу к $L_+$ и $R_+$ одновременно. Поэтому тут $\O(n)$, но 16 строку
мы будем обновлять за $\O(m)$, поэтому внешний цикл с 5-ой строки будет
выполняться не более $\O(nm)$ действий, алгоритм Куна тоже будет выполняться
за $\O(nm)$ (максимальное количество рёбер столько). Поэтому асимптотика будет
равна $\O(n^2m)$, что может быть применимо для больших $m$ и
$n$ поменьше.

Также для совсем маленьких $n$, можно оставить лучших $n$ в каждой строке (ведь
если есть решение, что в данной строке кто-то не из $n$ лучших, то остальные
работники забрали не более $n - 1$ столбец в матрице на свои лучшие ответы, а значит
по принципу Дирихле мы можем взять получше ответ для этой строки). Поэтому матрица
осталась максимум $n \times n^2$, значит такой алгоритм будет работать за $\O(n^4 + nm)$.


\end{document}
