\section{Лекция 9 от 28.04.2017}
\subsection{Предельные теоремы}
Рассмотренные центральные статистики --- некоторая экзотика, поскольку обычно сложно догадаться, какую функцию необходимо взять, чтобы она удовлетворяла свойствам центральной статистики.
Более традиционным является использование предельных теорем и неравенств теории вероятностей.
А именно, довольно сложную исходную задачу заменяют на что-то приближенное, но более простое с точки зрения вычислений.
Это так же особенность статистического подхода к решению задач, который
недостаточно строг с точки зрения математики, но показывает хорошие результаты на практике.

Начнем с уже известного нам примера из истории журнала <<Литературный дайджест>>.
\begin{example}
    Пусть имеется выбор между двумя кандидатами --- A и B.
    Генеральная совокупность в данном случае --- все избиратели.
    Обозначим за $p$ долю избирателей, госолующих за A.
    Так как опросить всех избирателей не представляется возможным (это уже похоже на сами выборы), берется некоторая выборка из $n$ человек.
    После опроса получена некоторая последовательность из нулей и единиц, где $i$-й элемент равен единице, если и только если $i$-й респондент сказал, что будет голосовать за A.
    Если рассмотреть случайные величины $X_1, \ldots, X_n$, которые суть элементы выборки, то получим выборку из бернуллиевского распределения с неизвестным параметром $p$ (вероятность того, что $X_i = 1$).
    Нас интересует оценка этого параметра.
    Но если $p^*$ является точечной оценкой $p$, то есть некая функция от выборки, то $\Pr(p^* = p) \longrightarrow 0\ (n \to \infty)$.
    Поэтому при достаточно больших $n$ (а количество опрошенных обычно в районе $1000-2000$, что достаточно велико) вероятность попадания точечной оценки в верное значение параметра $p$ крайне мала, что не вдохновляет.
    Значит, нужно обратиться к интервальным оценкам, отказавшись от точечных.
    Если мы сможем указать достаточно маленький интервал, который накрывает значение $p$ с вероятностью, скажем, $0,99$, то о чем еще можно мечтать?
\end{example}

На самом деле переход от событий, которые происходят с вероятностью 1, к событиям, которые происходят с вероятностью $0,99$ бывает достаточно значительным. Это хорошо видно на следующем примере.

\begin{example}
    Пусть имеется два пункта --- C и D. Каждый день из C в D переезжает 1000 человек.
    Для перевозки имеется две абсолютно одинаковые электрички конкурирующих компаний, поэтому каждый пассажир выбирает электричку равновероятно.
    Если обе компании хотят удовлетворить всех, кто выбрал их электричку, то каждый день из C в D должно уходить две электрички с 1000 мест в каждой.
    Но тогда каждый день из C в D будет отправляться 1000 пустых мест, что достаточно много.
    Если немного ужесточить условия так, что компании готовы предоставить места с вероятностью $0,99$.
    То есть пассажир, выбравший ту или иную электричку с вероятностью $1/2$, получит в ней свободное место с вероятностью $0,99$, а с вероятностью $0,01$ вынужден будет пересесть на другую.
    Оказывается, что в таком случае компаниям достаточно менее 550 мест в каждой электричке.
    Получается, что при малом изменении вероятности ответ меняется очень сильно.
\end{example}

В случае оценок наблюдается аналогичная ситуация.
Одна постановка задачи --- поиск интервала, который накрывает неизвестную вероятность $p$ с вероятностью $0,99$, --- имеет смысл.
Однако если мы захотим найти интервал, который накрывает параметр с вероятностью 1, то получим отрезок $[0, 1]$, и такой ответ вряд ли кого-то устроит.

Вернемся к задаче с избирателями.
Помимо коэффициента надежности, мы можем задать длину интервала $(p_1^*, p_2^*)$, то есть точность: $p_2^* - p_1^*$.
Если длина является случайной величиной, то можно зафиксировать математическое ожидание, например, $\E(p_2^* - p_1^*)/2$.
Третий параметр --- объем выборки $n$.
Все три параметра связаны между собой, то есть по двум данным определяется третий.

Попробуем найти доверительный интервал с помощью неравенства Чебышева.
Мы знаем, что для выборки из бернуллиевского распределения с неизвестным параметра $p$ оптимальной оценкой вероятности является выборочное среднее $p^* = \overline{X}$. То есть $p^*$ --- это доля респондентов, которые собираются голосовать за A.
По закону больших чисел мы знаем, что $\overline{X} \longrightarrow p$ с вероятностью 1. Неравенство Чебышева дает следующий результат:
\[
    \Pr\left(|\overline{X} - p| \leqslant \epsilon\right) \geqslant 1 - \frac{\D X_1}{n \epsilon^2}.
\]
Обозначим $\D X_1 / n \epsilon^2 = 1 / \alpha^2$. Тогда $\epsilon = \alpha \sqrt{\D X_1} / \sqrt{n}$. Неравенство в этом случае переписывается в виде:
\[
    \Pr\left(|\overline{X} - p| \leqslant \frac{\alpha \sqrt{\D X_1}}{\sqrt{n}}\right) \geqslant 1 - \frac{1}{\alpha^2}.
\]
Однако доверительный интервал мы все же не получили, ведь его границы выражены через дисперсию, которая равна $p(1-p)$ и нам не известна.
Один из способов решения этой проблемы заключается в загрублении оценки: $p(1-p) \leqslant 1/4$.
В таком случае имеем
\[
    \Pr\left( p \in \Big( \overline{X} - \frac{\alpha}{2\sqrt{n}},\ \overline{X} + \frac{\alpha}{2\sqrt{n}} \Big) \right) \geqslant 1 - \frac{1}{\alpha^2}.
\]
Мы получили следующие соотношения характеристик: коэффициент надежности равен $1 - 1 / \alpha^2$, длина интервала равна $\alpha / \sqrt{n}$, объем выборки равен $n$.
Снова видим их связь.

Хотя неравенство Чебышева является точным (в том смысле, что можно привести такую случайную величину, при которой оно обращается в равенство), как правило оно дает грубую оценку.
Рассмотрим другой способ --- использование центральной предельной теоремы.

Для того, чтобы применить теорему Муавра-Лапласа, мы должны нормировать случайную величину. Сделаем это:
\[
    \Pr(|\overline{X} - p| \leqslant \epsilon) = \Pr\left( \left|\frac{X_1 + \ldots + X_n - np}{\sqrt{np(1-p)}}\right| \leqslant \frac{\epsilon\sqrt{n}}{\sqrt{p(1-p)}} \right).
\]
По центральной предельной теореме эта вероятность приближенно равна
\[
    \Pr\left(|Z| \leqslant \frac{\epsilon\sqrt{n}}{\sqrt{p(1-p)}} \right), \qquad Z \sim \mathcal{N}(0, 1).
\]
Мы хотим, чтобы она была равна $\gamma$, то есть нужно найти такое $z_{\gamma}$, что $\Pr(|Z| \leqslant z_{\gamma}) = \gamma$.
Но для стандартного нормального распределения это сделать проще (с помощью таблиц или программных пакетов).
После нахождения $z_{\gamma}$ получаем $\epsilon = z_{\gamma}\sqrt{p(1-p) / n}$ и также загрубляем оценку, избавляясь от $p$.
В итоге имеем доверительный интервал длины $z_{\gamma} / \sqrt{n}$.

\begin{remark}
    $z_{\gamma}$ --- такое значение, что площадь под графиком плотности стандартного нормального распределения на отрезке $[-z_{\gamma},\ z_{\gamma}]$ равна $\gamma$. 
\end{remark}

Заметим, что в данном методе мы перешли от случайной величины к ее предельной случайной величине.
Это не обосновывается математически и является особенностью статистического подхода.

Рассмотрим теперь более общую ситуацию, снова применив центральную предельную теорему.
Для начала вспомним, что функция правдоподобия есть произведение плотностей или вероятностей.
Сразу перейдем к частной производной логарифма функции правдоподобия:
\[
    \dlnpn{Y} = \sum_{i=1}^n \underbrace{\dlnpn{X_i}}_{V_i}.
\]
Мы получили сумму независимых одинаково распределенных случайных величин $V_1, \ldots, V_n$. При этом при доказательстве неравенства Рао-Крамера мы пользовались леммой, из которой следует, что математическое ожидание $V_i$ равно нулю в регулярных моделях.
Заметим, что суммарная дисперсия есть информация Фишера:
\[
    \D{\dlnpn{Y}} = \E{\dlnpn{Y}}^2 = I_n(\theta).
\]
Тогда снова можно нормировать сумму случайных величин и воспользоваться сходимостью:
\[
    \frac{V_1 + \ldots + V_n}{\sqrt{I_n(\theta)}} \quad \dto \quad Z \sim \mathcal{N}(0, 1).
\]
А значит, от равенства
\[
    \Pr\left( \left| \frac{V_1 + \ldots + V_n}{\sqrt{I_n(\theta)}} \right| \leqslant \epsilon \right) = \gamma
\]
можно аналогичным образом перейти с помощью центральной предельной теоремы к равенству $\Pr(|Z| \leqslant \epsilon) = \gamma$.
Здесь снова можно легко найти необходимое значение $\epsilon$. Осталось перейти к исходной задаче, что требует разрешимости относительно $\theta$ неравенства под знаком вероятности.
Рассмотрим данный подход на примере.
\begin{example}
    Пусть выборка $Y = (X_1, \ldots, X_n)$ взята из $\mathcal{L}(X)$, где $X \sim Pois(\theta)$. В данном случае вероятность есть
    \[
        p_1(x, \theta) = \frac{\theta^x}{x!}e^{-\theta}, \qquad x \in \Z_{\geqslant 0}.
    \]
    Сразу напишем частную производную логарифма функции правдоподобия и информацию Фишера:
    \[
        \dlnpn{Y} = \frac{n}{\theta}(\overline{X} - \theta), \quad I_n(\theta) = \frac{n}{\theta} \quad \Rightarrow \quad \frac{V_1 + \ldots + V_n}{\sqrt{I_n(\theta)}} = \sqrt{\frac{n}{\theta}}(\overline{X} - \theta).
    \]
    Пусть мы нашли такое значение $z_{\gamma}$, что $\Pr(|Z| \leqslant z_{\gamma}) = \gamma$, где $Z$ --- стандартная нормальная случайная величина.
    Далее необходимо решить неравенство
    \[
        \left|\sqrt{\frac{n}{\theta}}(\overline{X} - \theta)\right| \leqslant z_{\gamma}
    \]
относительно $\theta$. Решением будет следующий интервал:
    \[
        \left(\overline{X} + \frac{z_{\gamma}^2}{2n} - z_{\gamma}\sqrt{\frac{\overline{X}}{n} + \frac{z_{\gamma}^2}{4n^2}},\ \overline{X} + \frac{z_{\gamma}^2}{2n} + z_{\gamma}\sqrt{\frac{\overline{X}}{n} + \frac{z_{\gamma}^2}{4n^2}}\right).
    \]
    Однако в статистике ответ несколько упрощают.
    Поскольку порядок слагаемого $z_{\gamma}^2/2n$ есть $1/n$, а под корнем старшее слагаемое дает порядок $1/\sqrt{n}$, интервал записывают в следующем виде:
    \[
        \left(\overline{X} - \frac{z_{\gamma} \sqrt{\overline{X}}}{\sqrt{n}},\ \overline{X} + \frac{z_{\gamma} \sqrt{\overline{X}}}{\sqrt{n}}\right).
    \]
    Сразу возникает вопрос: имеет ли полученный интервал коэффициент доверия $\gamma$?
    Не ясно, ведь, во-первых, мы сделали загрубление, выбросив из ответа все, кроме старших слагаемых, а во-вторых, мы перешли к предельной случайной величине. Однако, на практике такой метод часто используется.
\end{example}

Важно отметить, что в разобранном примере нам повезло, поскольку искомое неравенство разрешилось относительно $\theta$.


