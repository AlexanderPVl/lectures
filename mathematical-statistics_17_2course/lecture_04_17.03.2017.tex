\section{Лекция 4 от 17.03.2017}
\subsection{Минимизация дисперсии оценки. Информация Фишера}
Как обычно, положим, что у нас есть выборка \((X_{1}, \dots, X_{n})\) из 
распределения \(\mathcal{L}(x) \in \{\Pr_{\theta} \mid \theta \in \Theta\}\), 
где \(\Theta \subseteq \R^{d}\). Теперь у нас есть задача: по выборке построить 
несмещённую оценку \(\theta^{*} = \theta^{*}(X_{1}, \dots, X_{n})\) такую, что 
её дисперсия минимальна. Интуитивно смысл такого ограничения можно понять так: 
при малой дисперсии вычисленное значение будет не слишком сильно отклоняться от 
теоретического значения. Но можно ли найти какое-либо неравенство, которое 
будет ограничивать дисперсию снизу?

Ответ на этот вопрос: и да, и нет. С одной стороны, так называемое 
\emph{неравенство Рао-Крамера} даёт неулучшаемую оценку снизу. С другой 
стороны, оно верно не всегда~--- для него требуется так называемая 
\emph{регулярность} модели. Что понимается под регулярной моделью? Нестрого 
говоря, в таких моделях функция правдоподобия дважды непрерывно 
дифференцируема, то есть существуют моменты от производных. Вообще, это нужно 
для того, чтобы мы могли переносить оператор дифференцирования в интеграл и 
назад.

Теперь докажем одну лемму, которая понадобится для доказательства неравенства 
Рао-Крамера. Но для её формулировки нужно ввести понятие информации Фишера:
\begin{definition}
	Пусть \((X_{1}, \dots, X_{n})\)~--- выборка из распределения 
	\(\mathcal{L}(X) \in \{\Pr_{\theta} \mid \theta \in \Theta\}\), и \(\Theta 
	\subseteq \R\). Тогда \emph{информацией Фишера} называется
	\[
		I_{n}(\theta) = \E{\left(\frac{\partial \ln p_{n}(X_{1}, \dots, X_{n}; 
		\theta)}{\partial \theta}\right)^{2}}
	\]
\end{definition}
\begin{lemma}
	Пусть \((X_{1}, \dots, X_{n})\)~--- выборка из распределения 
	\(\mathcal{L}(X) \in \{\Pr_{\theta} \mid \theta \in \Theta\}\), и \(\Theta 
	\subseteq \R\). Далее, предположим, что существуют непрерывные производные
	\[
		\frac{\partial p_{n}(X_{1}, \dots, X_{n}; \theta)}{\partial \theta} 
		\text{ и } \frac{\partial^{2} p_{n}(X_{1}, \dots, X_{n}; 
		\theta)}{\partial \theta^{2}}
	\]
	и при этом информация Фишера конечна. Тогда
	\[
		\E{\frac{\partial \ln p_{n}(X_{1}, \dots, X_{n}; \theta)}{\partial 
		\theta}} = 0 \text{ и } I_{n}(\theta) = -\E{\frac{\partial^{2} 
		\ln p_{n}(X_{1}, \dots, X_{n}; \theta)}{\partial \theta^{2}}}
	\]
\end{lemma}
\begin{remark}
	Так как по определению функции правдоподобия
	\[
		p_{n}(X_{1}, \dots, X_{n}; \theta) = \prod_{k = 1}^{n} p_{X}(X_{i}) 
		\implies \ln p_{n}(X_{1}, \dots, X_{n}; \theta) = \sum_{k = 1}^{n} \ln 
		p_{X}(X_{i}),
	\]
	то из этой леммы следует, что
	\[
		I_{n}(\theta) = -\E{\frac{\partial^{2} p_{n}(X_{1}, \dots, X_{n}; 
		\theta)}{\partial \theta^{2}}} = -\sum_{k = 1}^{n} 
		\E{\frac{\partial^{2} \ln p (X_{i}; \theta)}{\partial \theta^{2}}} = 
		-n\E{\frac{\partial^{2} \ln p (X_{1}; \theta)}{\partial \theta^{2}}}.
	\]
	
	Последнее равенство означает, что \(I_{n}(\theta) = nI_{1}(\theta)\).
\end{remark}

Теперь приступим к доказательству.

\begin{proof}
	Вспомним, что интегрирование плотности по всему пространству даёт единицу:
	\[
		\int\limits_{\R^{n}} p_{n}(\mathbf{x}; \theta)\diff \mathbf{x} = 1.
	\]
	
	Продифференцируем по \(\theta\) и занесём оператор дифференцирования внутрь 
	(это возможно из-за непрерывной дифференцируемости):
	\[
		0 = \frac{\partial}{\partial \theta} \int\limits_{\R^{n}} 
		p_{n}(\mathbf{x}; \theta)\diff \mathbf{x} = \int\limits_{\R^{n}} 
		\frac{\partial p_{n}(\mathbf{x}; \theta)}{\partial \theta}\diff 
		\mathbf{x}.
	\]
	
	Теперь помножим и поделим частную производную на \(p_{n}(\mathbf{x}; 
	\theta)\) и преобразуем выражение:
	\[
		0 = \int\limits_{\R^{n}} \frac{1}{p_{n}(\mathbf{x}; 
		\theta)}\frac{\partial p_{n}(\mathbf{x}; \theta)}{\partial \theta} 
		p_{n}(\mathbf{x}; \theta)\diff \mathbf{x} = \int\limits_{\R^{n}} 
		\frac{\partial \ln p_{n}(\mathbf{x}; \theta)}{\partial \theta} 
		p_{n}(\mathbf{x}; \theta)\diff \mathbf{x}.
	\]
	
	Осталось заметить, что полученное равенство есть ни что иное, как первый 
	пункт леммы. Теперь перейдём ко второму пункту.
	
	Для него мы распишем вторую производную, как производную от производной:
	\begin{align*}
		\frac{\partial^{2} \ln p_{n}(\mathbf{x}; \theta)}{\partial \theta^{2}} 
		&= \frac{\partial}{\partial \theta}\left(\frac{1}{p_{n}(\mathbf{x};  
		\theta)}\frac{\partial p_{n}(\mathbf{x}; \theta)}{\partial 
		\theta}\right) \\
		&= \frac{1}{(p_{n}(\mathbf{x}; \theta))^{2}}\left(p_{n}(\mathbf{x}; 
		\theta)\frac{\partial^{2} p_{n}(\mathbf{x}; \theta)}{\partial 
		\theta^{2}} - \left(\frac{\partial p_{n}(\mathbf{x}; \theta)}{\partial 
		\theta}\right)^{2}\right).
	\end{align*}
	
	Теперь распишем матожидание этого выражения:
	\[
		\E{\frac{\partial^{2} \ln p_{n}(\mathbf{x}; \theta)}{\partial 
		\theta^{2}}} = \int\limits_{\R^{n}} \frac{1}{(p_{n}(\mathbf{x}; 
		\theta))^{2}}\left(p_{n}(\mathbf{x}; \theta)\frac{\partial^{2} 
		p_{n}(\mathbf{x}; \theta)}{\partial \theta^{2}} - \left(\frac{\partial 
		p_{n}(\mathbf{x}; \theta)}{\partial \theta}\right)^{2}\right)p_{n} 
		(\mathbf{x}; \theta) \diff \mathbf{x}.
	\]
	
	Разложим его по линейности:
	\[
		\E{\frac{\partial^{2} \ln p_{n}(\mathbf{x}; \theta)}{\partial 
		\theta^{2}}} = \int\limits_{\R^{n}} \frac{\partial^{2} 
		p_{n}(\mathbf{x}; \theta)}{\partial \theta^{2}}\diff \mathbf{x} - 
		\int\limits_{\R^{n}} \left(\frac{1}{p_{n}(\mathbf{x}; 
		\theta)}\frac{\partial p_{n}(\mathbf{x}; \theta)}{\partial 
		\theta}\right)^{2}p_{n} (\mathbf{x}; \theta) \diff \mathbf{x}.
	\]
	
	Теперь воспользуемся непрерывностью второй производной и вынесем вторую 
	производную из первого интеграла:
	\[
		\E{\frac{\partial^{2} \ln p_{n}(\mathbf{x}; \theta)}{\partial 
		\theta^{2}}} = \frac{\partial^{2}}{\partial \theta^{2}} 
		\int\limits_{\R^{n}} p_{n}(\mathbf{x}; \theta)\diff 
		\mathbf{x} - \int\limits_{\R^{n}} \left(\frac{\partial \ln 
		p_{n}(\mathbf{x}; \theta)}{\partial \theta}\right)^{2}p_{n} 
		(\mathbf{x}; \theta) \diff \mathbf{x}.
	\]
	
	Осталось вспомнить, что первый интеграл равен единице. Тогда
	\[
		\E{\frac{\partial^{2} \ln p_{n}(\mathbf{x}; \theta)}{\partial 
		\theta^{2}}} = -\E{\left(\frac{\partial \ln p_{n}(X_{1}, \dots, X_{n}; 
		\theta)}{\partial \theta}\right)^{2}}. \qedhere
	\]
\end{proof}

\subsection{Неравенство Рао-Крамера}

Великолепно. Теперь приступим к доказательству самого неравенства Рао-Крамера.
\begin{theorem}[Рао, Крамер]
	Пусть \((X_{1}, \dots, X_{n})\)~--- выборка из распределения 
	\(\mathcal{L}(X) \in \{\Pr_{\theta} \mid \theta \in \Theta\}\), и \(\Theta 
	\subseteq \R\). Далее, пусть \(\tau(\theta)\)~--- непрерывно 
	дифференцируемая функция параметра. Теперь предположим, что выполнены 
	следующие условия:
	\begin{enumerate}
		\item Cуществуют непрерывные производные
		\[
			\frac{\partial p_{n}(X_{1}, \dots, X_{n}; \theta)}{\partial \theta} 
			\text{ и } \frac{\partial^{2} p_{n}(X_{1}, \dots, X_{n}; 
			\theta)}{\partial \theta^{2}}.
		\]
		
		\item Информация Фишера \(I_{n}(\theta)\) конечна.
		
		\item Существует несмещённая оценка \(T(X_{1}, \dots, X_{n})\) 
		параметра \(\tau(\theta)\) такая, что её дисперсия конечна и 
		\[
			\int\limits_{\R^{n}} \left|T(\mathbf{x})\frac{\partial 
			p_{n}(\mathbf{x}; \theta)}{\partial \theta}\right|\diff \mathbf{x} 
			< +\infty.
		\]
	\end{enumerate}
	
	Тогда
	\[
		\D{T(X_{1}, \dots, X_{n})} \geq 
		\frac{(\tau'(\theta))^{2}}{I_{n}(\theta)},
	\]
	причём равенство достигается тогда и только тогда, когда
	\[
		\dlnpn{X_{1}, \dots, X_{n}} \overset{\text{п.н.}}{=} c(\theta)(T(X_{1}, 
		\dots, X_{n}) -\tau(\theta)),
	\]
	где \(c(\theta)\)~--- ограниченная функция от \(\theta\), не зависящая от 
	выборки.
\end{theorem}
\begin{remark}
	Можно ввести так называемый \emph{коэффициент эффективности}, который 
	определяется следующим образом:
	\[
		\eff(\theta) = \frac{(\tau'(\theta))^{2}}{I_{n}(\theta)\D{T(X_{1}, 
		\dots, X_{n})}}.
	\]
\end{remark}
\begin{definition}
	Пусть выполнены условия неравенства Рао-Крамера. Тогда будем называть 
	оценку \(T(X_{1}, \dots, X_{n})\) \emph{эффективной}, если в неравенстве 
	достигается равенство, или, что то же самое, коэффициент эффективности 
	равен 1.
\end{definition}

Ранее у нас вводились оптимальные оценки. Но стоит понимать, что эффективность 
и оптимальность~--- это разные вещи. Понятно, что из эффективности следует 
оптимальность, но в обратную сторону это верно не всегда~--- эффективных оценок 
может просто не быть (из-за нерегулярности модели, например).
\begin{remark}
	Если для \(\tau(\theta)\) существует эффективная оценка, то она существует 
	и для любого линейного преобразования \(\tau(\theta)\).
\end{remark}

Теперь приступим к самому доказательству.

\begin{proof}
	Так как \(T(X_{1}, \dots, X_{n})\) является несмещённой оценкой для 
	\(\tau(\theta)\), то
	\[
		\tau(\theta) = \E{T(X_{1}, \dots, X_{n})} = \int\limits_{\R^{n}} 
		T(\mathbf{x})p_{n}(\mathbf{x}; \theta) \diff \mathbf{x}.
	\]
	
	Распишем \(\tau'(\theta)\):
	\begin{align*}
		\tau'(\theta) = \frac{\partial}{\partial 
		\theta}\int\limits_{\R^{n}} T(\mathbf{x})p_{n}(\mathbf{x}; \theta) 
		\diff \mathbf{x}.
	\end{align*}
	
	Воспользуемся непрерывной дифференцируемостью \(\tau(\theta)\) и занесём 
	производную внутрь интеграла:
	\[
		\tau'(\theta) = \int\limits_{\R^{n}} T(\mathbf{x}) \frac{\partial 
		p_{n}(\mathbf{x}; \theta)}{\partial \theta} \diff \mathbf{x} = 
		\int\limits_{\R^{n}} T(\mathbf{x}) \dlnpn{\mathbf{x}} p_{n}(\mathbf{x}; 
		\theta) \diff \mathbf{x}.
	\]
	
	Заметим, что это есть матожидание от некоторой функции, а точнее:
	\[
		\tau'(\theta) = \E{T(X_{1}, \dots, X_{n}) \dlnpn{X_{1}, \dots, X_{n}}}.
	\]
	
	Теперь вспомним, что матожидание частной производной логарифма функции 
	правдоподобия равно нулю. Тогда
	\[
		\tau'(\theta) = \E{T(X_{1}, \dots, X_{n}) \dlnpn{X_{1}, \dots, X_{n}}} 
		- \E{\tau(\theta) \dlnpn{X_{1}, \dots, X_{n}}}.
	\]
	
	Следовательно,
	\[
		\tau'(\theta) = \E{(T(X_{1}, \dots, X_{n}) - \tau(\theta))\dlnpn{X_{1}, 
		\dots, X_{n}}}
	\]
	
	Но это есть ни что иное, как ковариация:
	\[
		\tau'(\theta) = \cov\left(T(X_{1}, \dots, X_{n}), \dlnpn{X_{1}, \dots, 
		X_{n}}\right).
	\]
	
	Согласно неравенству Коши-Буняковского
	\[
		|\tau'(\theta)| \leq \sqrt{\D{T(X_{1}, \dots, X_{n})}\D{\dlnpn{X_{1}, 
		\dots, X_{n}}}}.
	\]
	
	Теперь заменим дисперию частной производной на второй момент, так как 
	матожидание нулевое:
	\[
		|\tau'(\theta)| \leq \sqrt{\D{T(X_{1}, \dots, 
		X_{n})}\E{\left(\dlnpn{X_{1}, \dots, X_{n}}\right)^{2}}}.
	\]
	
	Возведём в квадрат:
	\[
		(\tau'(\theta))^{2} \leq \D{T(X_{1}, \dots, X_{n})}I_{n}(\theta).
	\]
	
	А это и есть ровно то, что нам нужно.
	
	Теперь посмотрим, что происходит, когда дисперсия минимальна. В таком 
	случае коэффициент корреляции равен по модулю единице:
	\[
		\left|\rho\left(T(X_{1}, \dots, X_{n}), \dlnpn{X_{1}, \dots, 
		X_{n}}\right)\right| = 1.
	\]
	
	Как известно, это выполнено тогда и только тогда, когда между случайными 
	величинами есть линейная зависимость:
	\[
		\dlnpn{X_{1}, \dots, X_{n}} \overset{\text{п.н.}}{=} aT(X_{1}, \dots, 
		X_{n}) + b.
	\]
	
	Надо понять, как устроены \(a\) и \(b\). Возьмём матожидание:
	\[
		0 = \E{\dlnpn{X_{1}, \dots, X_{n}}} = a\E{T(X_{1}, \dots, 
		X_{n})} + b = a\tau(\theta) + b.
	\]
	
	Отсюда видно, что \(\tau(\theta) = -b/a\) и
	\[
		\dlnpn{X_{1}, \dots, X_{n}} \overset{\text{п.н.}}{=} a(T(X_{1}, \dots, 
		X_{n}) - \tau(\theta)).
	\]
	
	Так как на константу нет ограничений, кроме конечности, то можно сказать, 
	что она есть функция от \(\theta\).
\end{proof}

Теперь покажем одно свойство, которое может позволить считать минимальную 
дисперсию.
\begin{lemma}
	Пусть \(T(X_{1}, \dots, X_{n})\)~--- оценка такая, что в условиях 
	неравенства Рао-Крамера
	\[
		\dlnpn{X_{1}, \dots, X_{n}} \overset{\text{п.н.}}{=} c(\theta)(T(X_{1}, 
		\dots, X_{n}) - \tau(\theta)).
	\]
	
	Тогда \(\D{T(X_{1}, \dots, X_{n})} = \tau'(\theta)/c(\theta)\).
\end{lemma}
\begin{proof}
	Продифференцируем данное равенство по \(\theta\):
	\[
		\frac{\partial^{2} \ln p_{n}(X_{1}, \dots, X_{n}; \theta)}{\partial 
		\theta^{2}} = c'(\theta)T(X_{1}, \dots, X_{n}) - c'(\theta)\tau(\theta) 
		- c(\theta)\tau'(\theta).
	\]
	
	Теперь возьмём матожидание от обеих сторон:
	\[
		\E{\frac{\partial^{2} \ln p_{n}(X_{1}, \dots, X_{n}; \theta)}{\partial 
		\theta^{2}}} = c'(\theta)\E{T(X_{1}, \dots, X_{n}) - \tau(\theta)} - 
		\E{c(\theta)\tau'(\theta)}.
	\]
	
	Согласно результату леммы об информации Фишера
	\[
		I_{n}(\theta) = -\E{\frac{\partial^{2} \ln p_{n}(X_{1}, \dots, X_{n}; 
		\theta)}{\partial \theta^{2}}} = c(\theta)\tau'(\theta).
	\]
	
	Тогда
	\[
		\D{T(X_{1}, \dots, X_{n})} = \frac{(\tau'(\theta))^2}{I_{n}(\theta)} = 
		\frac{\tau'(\theta)}{c(\theta)}. \qedhere
	\]
\end{proof}

Из неравенства Рао-Крамера можно сделать одно полезное следствие. Допустим, что 
есть некое событие \(A\) и мы хотим посчитать вероятность этого события \(p\). 
Для этого мы провели \(n\) наблюдений, и из них в \(n_{A}\) экспериментах 
наблюдалось событие \(A\). Тогда верно следующее:
\begin{theorem}
	Схема Бернулли \(p_{n}^{*} = n_{A}/n\) является эффективной оценкой для 
	параметра \(\theta = \Pr{A}\).
\end{theorem}
\begin{proof}
	Пусть \(Z_{i} = \mathrm{I}\{X_{i} \in A\}\). Тогда \(p_{n}^{*} = 
	\overline{Z}\). Запишем её функцию правдоподобия:
	\[
		p_{n}(Z_{1}, \dots, Z_{n}; \theta) = \theta^{Z_{1} + \dots + Z_{n}}(1 - 
		\theta)^{n - Z_{1} - \dots - Z_{n}} = \theta^{n\overline{Z}}(1 
		- \theta)^{n - n\overline{Z}}.
	\]
	
	Теперь прологарифмируем её:
	\[
		\ln p_{n}(Z_{1}, \dots, Z_{n}; \theta) = n\overline{Z}\ln{\theta} + (n 
		- n\overline{Z})\ln(1 - \theta).
	\]
	
	Далее, дифференцируем по \(\theta\):
	\[
		\dlnpn{Z_{1}, \dots, Z_{n}} = \frac{n\overline{Z}}{\theta} - \frac{n - 
		n\overline{Z}}{1 - \theta} = \frac{n}{\theta(1 - \theta)}(\overline{Z} 
		- \theta).
	\]
	
	Тогда по критерию равенства в неравенстве Рао-Крамера получаем, что 
	\(\overline{Z}\) действительно является эффективной оценкой.
\end{proof}
\begin{consequence}
	Эмпирическая функция распределения является эффективной оценкой 
	теоретической функции распределения.
\end{consequence}