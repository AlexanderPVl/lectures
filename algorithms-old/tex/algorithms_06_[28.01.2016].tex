% Размер страницы и шрифта
\documentclass[12pt,a4paper]{article}

%% Работа с русским языком
\usepackage{cmap}                   % поиск в PDF
\usepackage{mathtext}               % русские буквы в формулах
\usepackage[T2A]{fontenc}           % кодировка
\usepackage[utf8]{inputenc}         % кодировка исходного текста
\usepackage[english,russian]{babel} % локализация и переносы

%% Изменяем размер полей
\usepackage[top=0.5in, bottom=0.75in, left=0.625in, right=0.625in]{geometry}

%% Различные пакеты для работы с математикой
\usepackage{mathtools}  % Тот же amsmath, только с некоторыми поправками
\usepackage{amssymb}    % Математические символы
\usepackage{amsthm}     % Пакет для написания теорем
\usepackage{amstext}
\usepackage{array}
\usepackage{amsfonts}
\usepackage{icomma}     % "Умная" запятая: $0,2$ --- число, $0, 2$ --- перечисление

%% Графика
\usepackage[pdftex]{graphicx}
\graphicspath{{images/}}

%% Прочие пакеты
\usepackage{listings}               % Пакет для написания кода на каком-то языке программирования
\usepackage{algorithm}              % Пакет для написания алгоритмов
\usepackage[noend]{algpseudocode}   % Подключает псевдокод, отключает end if и иже с ними
\usepackage{indentfirst}            % Начало текста с красной строки
\usepackage[colorlinks=true, urlcolor=blue]{hyperref}   % Ссылки
\usepackage{pgfplots}               % Графики
\pgfplotsset{compat=1.12}
\usepackage{forest}                 % Деревья
\usepackage{titlesec}               % Изменение формата заголовков
\usepackage[normalem]{ulem}         % Для зачёркиваний
\usepackage[autocite=footnote]{biblatex}    % Кавычки для цитат и прочее
\usepackage[makeroom]{cancel}       % И снова зачёркивание (на этот раз косое)

% Изменим формат \section и \subsection:
\titleformat{\section}
	{\vspace{1cm}\centering\LARGE\bfseries} % Стиль заголовка
	{}                                      % префикс
	{0pt}                                   % Расстояние между префиксом и заголовком
	{}                                      % Как отображается префикс
\titleformat{\subsection}                   	% Аналогично для \subsection
	{\Large\bfseries}
	{}
	{0pt}
	{}

% Поправленный вид lstlisting
\lstset { %
    backgroundcolor=\color{black!5}, % set backgroundcolor
    basicstyle=\footnotesize,% basic font setting
}

% Теоремы и утверждения. В комменте указываем номер лекции, в которой это используется.
\newtheorem*{hanoi_recurrent}{Свойство} % Лекция 1
\let\epsilent\varepsilon                % Лекция 8
\DeclareMathOperator{\rk}{rank}         % Лекция 20

% Информация об авторах
\author{Группа лектория ФКН ПМИ 2015-2016 \\
	Никита Попов \\
	Тамерлан Таболов \\
	Лёша Хачиянц}
\title{Лекции по предмету \\
	\textbf{Алгоритмы и структуры данных}}
\date{2016 год}


\begin{document}
\maketitle

Вход --- множество точек $(x_i, y_i)$. Выход --- $i, j: d((x_i, y_i), (x_j, y_j))$ --- минимально.

``Разделяй и властвуй'':

\begin{lstlisting}
closest_pair_rec(P_x, P_y)
    if n < 4 then
        solve directly
    L_x := P_x[1..ceil(n/2)]. L_y = ...
    R_x := P_x[ceil(n/2)+1..n], R_y = ...
    (l_1, l_2) := closest_pair_rec(L_x, L_y)
    (r_1, r_2) := closest_pair_rec(R_x, R_y)
    delta := min(d(l_1, l_2), d(r_1, r_2))
    
    S_x^l := {(x, y) \in L_x | x^*-x < delta}
    S_x^r := {(x, y) \in L_x | x-x^* < delta}

    S_y := {(x, y) \in P_y | |x-x^*| \leqslant delta}

    for i := 1 to |S_y| do
        c_i := argmin_{i-15\leqslant j\leqslant i+15}(d(S_y[i], S_y[j]))
        if d(S_y[i], s_y[c]) < delta then

        WAAAAAAAAT
\end{lstlisting}

\begin{lstlisting}
closest_pair(P)
    P_x := sort P by x
    P_y := sort P by y
    closest_pair_rec(P_x, P_y)
\end{lstlisting}

Теперь нужно объединить. Сколько времени у нас есть?

Для $n\log n$ $T(n) = 2T(\frac{n}{2}) + O(n)$, то есть у нас остаётся $O(n)$

Теперь нам нужно проверить, нет ли пары ближе, не по рзные стороны от границы. Нам интересна ближайшая такая пара, но только если расстояние меньше delta.

Пусть $x^* = max\left\{ x\mid(x, y)\in L_x \right\}$

\begin{lstlisting}
\end{lstlisting}

Тут картинка

Рассмотрим точки, попавшие в полосу. Для всех точек в $L_x$ нам не обязательно рассматривать все точки в полосе в $R_y$. Рассмотрим только те, что по $y$ лежат в delta-окрестности


Возьмём полосу и поделим на квадраты со стороной $\delta/2$. Сколько точек в каждом таком квадрате? Если там есть две точки, то расстояние между ними меньше $\delta$, а это невозможно.

Понятно, что между точками на расстоянии $<\delta$ не более трёх рядов квадратов. Значит, всегда будет достаточно рассмотреть 15 ближайших точек.
\end{document}
