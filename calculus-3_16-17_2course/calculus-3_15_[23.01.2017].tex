\documentclass[a4paper, 12pt]{article}
\usepackage{header}

\begin{document}
\pagestyle{fancy}
\section{Лекция 15 от 23.01.2017 \\Сжимающие отображения и неравенство КБШ}

\subsection*{Сжимающие отображения}

Пусть  $R = (M, \rho)$ --- метрическое пространство.

\begin{Def}
Отображение  $A$ метрического пространства  $R$ в себя называется сжимающим отображением или сжатием, если существует такое положительное действительное число  $q < 1$, что для любых двух точек  $x,y \in R$ имеет место неравенство:
 $$\rho(Ax, Ay) \le q \rho(x, y).$$
\end{Def}

\begin{Def}
Точка x называется неподвижной точкой отображения $f$, если имеет место равенство
$$f(x) = x$$
\end{Def}


\begin{Theorem}[О сжимающем отображении]
	Всякое сжимающее отображение $f$, заданное на полном метрическом пространстве $M$, имеет одну и только одну неподвижную точку.
\end{Theorem}
\begin{proof}

Пусть  $x_0 $ — произвольная точка пространства  $R $. Положим $x_1 = f(x_0) $, $x_2 = f(x_1) = f(f(x_0))$, \dots $x_{n+1} = f(x_{n})$\dots

Докажем, что последовательность  $\{ x_n \} $ является фундаментальной.
Пусть $r := \rho(x_0, x_1)$.  Тогда, по определению сжимающего отображения мы можем сказать, что  $\rho(x_1, x_2) \leq qr$. Аналогично, $\rho(x_2, x_3) \leq q \rho(x_1, x_2) \leq q^2r $, $\rho(x_3, x_4) \leq q^3r $, \dots  $\rho(x_{n}, x_{n+1}) \leq q^nr$.

Зафиксируем произвольное $\eps>0$. 
Так как  $q < 1 $, $\exists N\ \forall n>N: \frac{rq^n}{1-q} < \eps$.

Возьмём произвольные $n, m >N$ без ограничения общности $n>m$.
Многократно пользуясь неравенством треугольника и один раз --- формулой суммы геометрической прогрессии получим:
\begin{multline}\rho(x_n, x_m) \leq \rho(x_n, x_{m+1})+\rho(x_{m+1}, x_m)\leq\\ \leq \rho(x_n, x_{m+2})+ \rho(x_{m+2}, x_{m+1})+\rho(x_{m+1}, x_m) \leq\\\leq \dots\leq rq^{m} + rq^{m+1}+\dots+rq^{n-1} < \frac{rq^n}{1-q} < \eps \end{multline}	
Таким образом, фундаментальность последовательности  $\{ x_n \} $ доказана. 


Поскольку наше метрическое пространство полное, то последовательность $\{ x_n \}$ будет иметь предел в этом пространстве: $x:= \lim\limits_{n \to \infty}x_n$.
Значит $\rho(x_{n}, x) \xrightarrow[x \to\infty]{} 0$. Зная что $0\leq\rho(f(x_{n}), f(x))\leq q \rho(x_{n}, x)$ делаем вывод что $\rho(f(x_{n}), f(x)) \xrightarrow[x \to\infty]{} 0$ и следовательно $\lim\limits_{n \to \infty}f(x_n) = f(x)$.

Получаем, что  $\lim\limits_{n \to \infty}x_n = x$, и одновременно $\lim\limits_{n \to \infty}f(x_n) = \lim\limits_{n \to \infty}x_n= f(x)$. Откуда, в силу единственности предела последовательности, получили что $f(x) = x$.
Существование неподвижной точки доказано.


Осталось доказать единственность. 
Пусть есть хоть бы две различных неподвижных точки $x$ и $\tilde{x}$.
Тогда $\rho(f(x), f(\tilde{x})) \leq q\rho(x, \tilde{x})$ по определению сжимающего отображения. Однако также $\rho(f(x), f(\tilde{x})) = \rho(x, \tilde{x})$ так как рассматриваемые точки --- неподвижные. С учётом того, что $q<1$, из этого с неизбежностью следует что $\rho(x, \tilde{x}) = 0$, и, соответственно $x = \tilde{x}$.

\end{proof}

Сжимающие отображения оказываются полезны при решении задач самого разного рода. Например, они используются при доказательстве существования и единственности решиния задачи Коши в курсе дифференциальных уравнений. 

\subsection*{Теорема Коши--Буняковского--Шварца}


\begin{Theorem}[Коши--Буняковского--Шварца (КБШ)]
	Пусть $H$ --- линейное пространство со скалярным произведением $( x,\;y)$, и $\|x\|$ — норма, порождённая этим скалярным произведением. Тогда для любых $x,\;y\in L$ имеем: $|( x,\;y | \leqslant \|x\|\cdot\|y\|,$	причём равенство достигается тогда и только тогда, когда векторы $x$ и $y$ линейно зависимы.
\end{Theorem}
\begin{proof}
	Если $y = \0$, то утверждение тривиально верно, так как $$ |( x,\;\0 | = |( x,\;\0+\0)| = |( x,\;\0)|+|( x,\;\0)|,$$
	откуда следует что $|( x,\;\0)| = 0$.
	
	Теперь будем считать, что $y\neq \0$. Введем функцию $f: \R \Rightarrow\R$, $f(t):=(x+ty, x+ty)$. Пользуясь свойствами скалярного произведения можно понять что $$ f(t) =(x+ty, x+ty) = (x, x)+2t(x, y)+t^2(y, y),$$ то есть $f$ --- обычная квадратичная функция от $t$. Заметим, что $\forall t\in\R:$ $f(t)\geq0$, причём $f(t)=0$ возможно только если $x$ и $y$ линейно зависимы. Таким образом заключаем, что дискриминант многочлена $(x, x)+2t(x, y)+t^2(y, y)$ --- неположительный, и равен нулю только при линейной заивимости $x$ и $y$. То есть 
	\[D = 4(x, y)^2-4(x, x)(y, y)\leq 0\]
	\[(x, y)^2\leq(x, x)(y, y)\]
	\[(x, y)\leq \sqrt{(x, x)}\cdot \sqrt{(y, y)},\]
	что нам и тредовалось доказать. Напоследок снова заметим, что равенство достигается при равенстве дискриминанта нулю, то есть при линейной зависимости $x$ и $y$.
\end{proof}
\begin{Task}[(Бонусная задача) ]
	Если норма задана скалярным произведением, то 
	$\ 2\|x\|^2+2\|y\|^2=\|x+y\|^2+\|x-y\|^2$. Это равенство часто называют тождеством параллелограмма, и оно несложно проверяется. А верно ли обратное, что если выполнено равенство, то норма задана скалярным произведением?
\end{Task}
\end{document}