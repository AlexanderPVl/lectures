% \documentclass[a4paper, 12pt]{article}
% \usepackage{header}

% \begin{document}
% \pagestyle{fancy}

\section{Лекция 9 от 14.10.2016. $\Pclass$ vs. $\NPclass$}

\subsection{Класс $\Pclass$ и сведение}

Сейчас мы переходим в теоретический материал, который является основой
классификации алгоритмов. Нас совершенно не интересует эффективность, кроме
как разделение на <<алгоритмы, которые работают за полиномиальное время>> и все
остальные.

Договоримся, что мы определяем <<алгоритм>>, как детерминированную
машину Тьюринга. И будем
использовать тезис Чёрча-Тьюринга о том, что любая вычислимая функция является
вычислимой на машине Тьюринга. Также условимся, что мы рассматриваем только
задачи принятия решения (принадлежит ли слово $w$ языку $L$).

\begin{Def}
  Задача (формально говоря язык) $A \in \Pclass$, если существует машина Тьюринга,
  проверяющая принадлежность
  слова языку,
  время работы $t_A(x)$ которой ограничено полиномом от $P(|x|)$, который является
  фиксированным для всех входов.
\end{Def}

Также, до определения класса $\NPclass$, определим,
что значит, что язык $A$ сводится к языку $B$ за полиномиальное время.

\begin{Def}
    Язык $A$ \textbf{сводится} к языку $B$ за полиномиальное время, если 
    существует функция $f$, вычислимая за полиномиальное время, такая, что 
    $w \in A \Longleftrightarrow f(w) \in B$.
    Обозначение: $A \leqslant_p B$.
\end{Def}
Докажем основное утверждение о сводимости:
\begin{Statement}
    Пусть $A \leqslant_p B$ и $B \in \Pclass$ (существует полиномиальный 
    алгоритм решения задачи). Тогда $A \in \Pclass$.
\end{Statement}

\begin{proof}
 Достаточно построить алгоритм, показывающий, что для языка $A$ тоже существует 
 полиномиальное решение. Пусть у нас имеются МТ $M_A$, допускающая язык $A$, и 
 $M_B$, допускающая язык $B$. Поскольку $A \leqslant_p B$, то $\exists$ 
 полиномиальная функция $f$, удовлетворяющая определению выше. Построим алгоритм вычисления $M_A$:
 \begin{algorithmic}
    \Require{слово $w$}
    \State вычислить $f(w)$
    \State \Return $M_B\big(f(w)\big)$
 \end{algorithmic}
 Поскольку сама функция $f$ является вычислимой за полиномиальное время, то 
 вычисление $f(w)$ потратит полиномиальное время. Кроме того, поскольку 
 $B \in \Pclass$, то вычисление $M_B\big(f(w)\big)$ также займёт полиномиальное 
 время (от размера $f(w)$, но $f(w)$ также вычисляется за полином)! Отсюда 
 $M_A(w)$ также допускает слово $w$ за полиномальное время, что по определению 
 означает, что $A \in \Pclass$.
\end{proof}

\subsection{Класс $\NPclass$}

\begin{Def}
  Язык \(L\) лежит в классе \(\NPclass\), если существует функция 
  от двух аргументов \(A(x, y)\)~--- \textbf{алгоритм верификации}~---  с 
  полиномиальной сложностью от \(x\) такая, что \(x\) лежит в \(L\) 
  тогда и только тогда, когда для него существует \(y\) (его принято называть 
  \textbf{сертификатом}) такой, что \(A(x, y) = 1\). При этом сертификат должен
  быть полиномиально зависим от размера \(x\).
\end{Def}

Ясно, что $\Pclass \subseteq \NPclass$, так как за алгоритм верификации можно
взять просто полиномиальный алгоритм принадлежности слова языку.

Также введём понятие $\coNPclass$ класса:

\begin{Def}
  \[
    \coNPclass = \{\overline{L} = \Sigma^{*} \setminus L \mid L \in \NPclass\}
  \]
\end{Def}

Другими словами, $\coNPclass$~--- множество всех языков, которые могут
верифицировать \textbf{не}принадлежность слово языку. Например, такая задача
--- <<Нет ли гамильтонового цикла в графе?>>. Неизвестно, лежит ли она в $\NPclass$,
но она точно лежит в $\coNPclass$ (предъявление верификатора оставляем, как
упражнение читателю). Также ясно, что $\Pclass \subseteq \coNPclass$.

\begin{Def}
  Язык $A \in \mathbf{PSPACE}$, если существует МТ принимающая данный язык
  с полиномиальным количеством дополнительной памяти.
\end{Def}

\begin{Def}
  Язык $A \in \mathbf{EXPTIME}$, если существует МТ принимающая данный язык
  с экспоненциальным временем работы (а именно $\O(2^{n^k})$).
\end{Def}

Оставим тоже следующую лемму, как упражнение читателю:

\begin{Lemma}
  $\NPclass \subseteq \mathbf{PSPACE} \subseteq \mathbf{EXPTIME}$ 
\end{Lemma}

\subsection{$\NPCclass$ класс и теорема Кука-Левина}

\begin{Def}
  Язык $A$--- $\NPclass$-трудный, если $\forall \ B \in \NPclass : B \leqslant_p A$.
\end{Def}

\begin{Def}
  Язык $A \in \NPCclass$ ($\NPclass$-complete или Nondeterministic Polynomial 
  Complete), если $A \in \NPclass$ и $\forall \ B \in \NPclass \implies B
  \leqslant_p A$.
\end{Def}

Другими словами, $\NPCclass$~--- это в точности те языки, которые
сами лежат в $\NPclass$ и являются $\NPclass$-трудными одновременно.

Докажем следующие простые утверждения:

\begin{Lemma}
  Пусть $L \in \NPCclass$. Тогда, если $L \in \Pclass$, то $\Pclass = \NPclass$.
\end{Lemma}
\begin{proof}
  Пусть $L' \in \NPclass$. Так как $L \in \NPCclass$, то $L' \leqslant_p L$.
  Но, поскольку $L \in \Pclass$, то $L' \in \Pclass$, откуда $\Pclass = \NPclass$.
\end{proof}

\begin{Lemma}
Если $B \in \NPCclass$ и $B \leqslant_p C \in \NPclass$, то $C \in \NPCclass$.
\end{Lemma}
\begin{proof}
  Проверим оба условия, входящих в определение $\NPclass$-полного языка.
  \begin{enumerate}
    \item $C \in \NPclass$ --- по условию.
    \item Пусть $L' \in \NPclass$. Так как $B \in \NPCclass$, то $L' 
    \leqslant_p B$, что по определению означает, что существует некая 
    функция $f_{L'B}$
    (которая сводит одну задачу к другой), вычислимая за полиномиальное 
    время. Кроме того, $B \leqslant_p C$, что по определению означает, что 
    существует некая функция $f_{BC}$, вычислимая за полиномиальное время.
        
    Но это означает, что $L' \leqslant_p C$, так как существует функция 
    $f_{L'C}$, такая, что $f_{L'C}(w) = f_{BC}\big(f_{L'B}(w)\big)$ -- она 
    также является вычислимой за полиномиальное время.
        
    Следовательно, $\forall\ L' \in \NPclass: L' \leqslant_p C$, откуда 
    $C$ -- $\NPclass$-трудный по определению.
  \end{enumerate}
  Оба условия выполняются, значит $C \in \NPCclass$.
\end{proof}

Но всё равно должны же остаться вопросы по тому, зачем всё это надо и как
вообще доказывать, что задача трудна, то есть лежит в классе $\NPclass$. 
Ведь должна же быть какая-то <<первая>> задача из $\NPCclass$. Если вы решаете
задачу и вдруг понимаете, что она эквивалентна какой-то задаче из $\NPCclass$,
то возможно стоит либо перечитать задачу (и понять, каким условием
надо точно пользоваться, если вдруг вы его отбросили),
либо решить и получить $\$1\,000\,000$.

Одна из первых $\NPCclass$ задач была задача $\SAT$ (от англ.
Satisfiability). Можем считать, что нам
дана булева формула в конъюнктивно нормальной форме. Сейчас мы покажем, что
любая $\NPclass$ задача сводится к этой. Вообще, так как если мы хотим сводить
любую $\NPclass$ задачу к $\SAT$, то какие-то обычные рассуждения вряд ли пройдут.
Действительно, мы будем стараться моделировать работу МТ через КНФ.

\begin{Theorem}[Теорема Кука-Левина]
  $\SAT \in \NPCclass$.
\end{Theorem}

\begin{proof}
  Сначала покажем, что $\SAT \in \NPclass$. Действительно, по входу булевой
  формулы и верификатору (значение переменных $x_1, \ldots, x_n$) мы с лёгкостью
  проверим выполнимость формулы. Поэтому есть линейный алгоритм верификации.

  Пусть $\mathcal{P} \in \NPclass$. Надо понять, как полиномиально свести
  эту задачу к $\SAT$. По определению класса $\NPclass$ существует полином
  $p$ и алгоритм $\mathcal{P'}$, который верифицирует задачу $\mathcal{P}$, 
  причём верификатор размера не больше, чем $p(|x|)$.

  Пусть
  \[
    \Phi : \{0, \ldots, N\} \times \overline{A} \to \{-1, \ldots, N\}
    \times \overline{A} \times \{-1, 0, 1\}
  \]
  МТ, работающая за полиномиальное время, отвечающая за задачу $\mathcal{P'}$
  с алфавитом $A$. Добавим пустой символ к $A$, чтобы легче было работать.
  Пусть работа этой МТ ограничена полиномом $q$, то есть time$(\Phi, x\#c)
  \leqslant q(|x\#c|)$. Сейчас мы соорудим набор дизъюнктов или <<клауз>>
  (от англ. Clause) $Z(x)$, который
  ограничен полиномом $Q \gets q(|x\#c|)$, такой, что $Z(x)$ выполняется
  тогда и только тогда, когда слово принадлежит языку $\mathcal{P}$.

  Ясно, что мы не уйдём за пределы ленты от $-Q$ до $Q$.

  Теперь создадим несколько переменных:
  \begin{itemize}
    \item переменные $v_{ij\sigma}$ для всех $0 \leqslant i \leqslant Q$,
    $-Q \leqslant j \leqslant Q$ и $\sigma \in \overline{A}$. Верно ли, что
    в момент $i$ (после $i$ шагов выполнения МТ) позиция с номером $j$ содержала
    символ $\sigma$.

    \item переменные $w_{ijn}$ для всех $0 \leqslant i \leqslant Q$,
    $-Q \leqslant j \leqslant Q$ и $-1 \leqslant n \leqslant N$. Верно ли, что
    в момент $i$ на позиции $j$ МТ была в состоянии $n$. За минус 1 отвечает
    финальное состояние.
  \end{itemize}

  Поэтому, если у нас есть конфигурация МТ на $i$-ом шаге
  с позицией $\pi^{(i)}$, символами $s^{(i)}$ и состоянием $n^{(i)}$, тогда
  мы должны выставить $v_{ij\sigma} := true$ тогда и только тогда, когда
  $s_j^{(i)} = \sigma$ ($s_j$ --- позиция на $j$-ом месте ленты МТ). И мы должны
  выставить $w_{ijn} := true$ тоже тогда и только тогда, когда $\pi^{(i)} = j$ и
  $n^{(i)} = n$.

  Сейчас мы предъявим набор клауз $Z(x)$, который будет выполняться тогда и
  только тогда, когда существует строка $c$ полиномиального размера, что 
  МТ $\Phi$ на входе $x\#c$ выдаёт единицу.

  В каждый момент времени на каждой позиции стоит строго один символ:

  \begin{itemize}
    \item дизъюнкт $(v_{ij\sigma} \mid \sigma \in \overline{A})$ по всем 
    $0 \leqslant i \leqslant Q$ и $-Q \leqslant j \leqslant Q$.
    \item $(\overline{v_{ij\sigma}} \vee \overline{v_{ij\tau}})$ по всем 
    $0 \leqslant i \leqslant Q$, и $-Q \leqslant j \leqslant Q$, и $\sigma \neq
    \tau \in \overline{A}$.
  \end{itemize}

  В каждый момент времени единственная позиция в строке сканируется и
  единственная инструкция выполняется:

  \begin{itemize}
    \item $(w_{ijn} \mid -Q \leqslant j \leqslant Q, -1 \leqslant n \leqslant N)$
    для всех $0 \leqslant i \leqslant Q$.
    \item $(\overline{w_{ijn}} \vee \overline{w_{ij'n'}})$ по всем 
    $0 \leqslant i \leqslant Q$ и $-Q \leqslant j, j' \leqslant Q$ и 
    $-1\leqslant n, n' \leqslant N$ с условием, что $(j, n) \neq (j', n')$.
  \end{itemize}

  Алгоритм корректно начинает свою работу:

  \begin{itemize}
    \item $(v_{0, j, x_j})$ по всем $1 \leqslant j \leqslant |x|$.
    \item $(v_{0, |x| + 1, \#})$ --- разделяющий символ.
    \item $(v_{0, |x |+ 1 + j, 0} \vee v_{0, |x| + 1 + j, 1})$
    по всем $1 \leqslant j \leqslant p(|x|)$.
    \item $(v_{0, j, \sqcup})$ по всем $-Q \leqslant j \leqslant 0$ и
    $|x| + 2 + p(|x|) \leqslant j \leqslant Q$ --- пустые символы не с входа.
    \item $(w_{0,1,0})$ --- начальное положение головки.
  \end{itemize}

  Алгоритм работает корректно:

  \begin{itemize}
    \item $(\overline{v_{ij\sigma}} \vee \overline{w_{ijn}} \vee v_{i + 1, j, \tau})$ и
    $(\overline{v_{ij\sigma}} \vee \overline{w_{ijn}} \vee w_{i + 1, j + \delta, m})$
    по всем $0 \leqslant i < Q, -Q \leqslant j\leqslant Q; \sigma, \tau \in 
    \overline{A}, \delta \in \{-1, 0, 1\}$, где $\Phi(n, \sigma) = 
    (m, \tau, \delta)$ (переход по состоянию $n$, символу $\sigma$ в какое-то
    состояние $m$, символ $\tau$ пишем на $j$-ом месте и сдвиг на $-1, 0$ или $1$).
  \end{itemize}

  Когда алгоритм достигает финального состояния (минус 1 в нашем случае), алгоритм
  останавливается:
  \begin{itemize}
    \item $(\overline{w_{i, j, -1}} \vee w_{i + 1, j, -1})$ и 
    $(\overline{w_{i, j, -1}}\vee \overline{v_{i, j, \sigma}}\vee v_{i + 1, j, \sigma})$
    по всем $0 \leqslant i < Q; -Q \leqslant j \leqslant Q$ и $\sigma \in
    \overline{A}$ --- как раз, если достигли состояния $-1$, тогда все состояния
    в каждое время должны быть минус один и символ меняться не должен.
  \end{itemize}

  Позиции, которые не были просмотрены, должны остаться неизменными:

  \begin{itemize}
    \item $(\overline{v_{ij\sigma}}\vee \overline{w_{ij'n}}\vee v_{i+1, j, \sigma})$
    по всем $0 \leqslant i \leqslant Q; \sigma \in \overline{A}; -1 \leqslant
    n \leqslant N$ и $-Q \leqslant j, j' \leqslant Q$ при условии, что $j \neq j'$.
  \end{itemize}

  Алгоритм выводит единицу:
  \begin{itemize}
    \item $(v_{Q, 1, 1})$ и $(v_{Q, 2, \sqcup})$ --- на первой позиции стоит
    единица, а потом пробел (можно считать, что только один, можно, что все
    последующие, это не играет на роль полиномиальности МТ и полиномиальности
    размера КНФ).
  \end{itemize}

  Заметим, что размер $Z(x)$ будет не больше, чем $\O(Q^3\log Q)$, существует
  $\O(Q^3)$ литералов и нужно $\O(\log Q)$ памяти, чтобы закодировать индексы.

  Осталось показать, что $Z(x)$ выполняется тогда только тогда, когда $x$
  принадлежит языку.

  Если $Z(x)$ выполняется, то пусть $T$ --- это набор переменных, удовлетворяющим
  всем клаузам. Поставим $c_j = 1$ по всем $j$ для которых 
  $T(v_{o, |x| + 1 + j, 1}) = true$ и $c_j = 0$ в ином случае. Сверху мы
  описали работу МТ $\Phi$ на входе $x\#c$. Поэтому мы можем заключить, что
  $\Phi(x\#c) = 1$. Так как $\Phi$ --- алгоритм верификации, то это значит, что
  $x$ принадлежит языку.

  Пусть $x$ принадлежит языку. Тогда пусть $c$ --- сертификат проверки для $x$.
  Тогда пусть конфигурация МТ на входе $x\#c$ на $i$-ом шаге пусть будет равна 
  $(n^{(i)}, s^{(i)}, \pi^{(i)})$. Тогда давайте поставим $T(v_{i,j,\sigma}) = 
  true$ тогда и только тогда, когда $s_j^{(i)} = \sigma$ и $T(w_{i, j, n})=true$
  тогда и только тогда, когда $\pi^{(i)} = j, n^{(i)} = n$ по всем $i \leqslant m$.
  Для б\'{о}льших $i$ выставим $T(v_{i, j, \sigma}) := T(v_{i - 1, j, \sigma})$
  и $T(w_{i, j, n}) := T(w_{i - 1, j, n})$ по всевозможным $j, n, \sigma$. Тогда
  можно убедиться, что формула выполняется, что завершает наше доказательство.
\end{proof}


% \end{document}
