\documentclass[a4paper, 12pt]{article}
\usepackage{header}

\begin{document}
\pagestyle{fancy}
\section{Лекция 12 от 28ю.11.2016 \\ Степенные ряды}
\begin{Def}
	Степенной ряд --- это функциональный ряд вида $\sum\limits_{n = 0}^{\infty}c_n (x-x_0)^n$.
\end{Def}
В этом ряду, $x$ --- переменная, $\{c_n\}$ --- \textit{последовательность коэффициентов}, $x_0 \in \mathbb{R}$ ---  \textit{центр ряда}.

Отметим, что ряд начинается с $n = 0$. Это будет важно в дальнейшем, давая возможность представлять рядами функции в нуле не равные нулю.


В процессе всех дальнейших рассуждений в рамах этой лекции будем считать $x_0 = 0$. Это не умаляет общности, так как фактически это сдвиг по оси (замена переменной $x` = x_0+x$, если угодно).



\begin{Theorem} [Абеля I]
	Пусть ряд $\sum\limits_{n = 0}^{\infty}c_n x^n$ сходится в точке $x_1$. Тогда $\forall x |x| < |x_1|$ этот ряд сходится абсолютно. Более того, $\forall x_2 \in (0, x_1)$ сходимость на $[-x_2, x_2]$ --- равномерная.
\end{Theorem}
\begin{proof}
	
	Возьмем $K_{0}=\left \{ z \right.: z <\left. z_{0} \right \}$
	Пусть $q= \frac{z}{z_{0}} <1$
	Так как $\sum\limits_{0}^{\infty}c_{n}z^{n}$ --- сходится в точке $z_{0}$, то есть  $\sum\limits_{0}^{\infty}c_{n}z_{0}^{n}<\infty$, то можно утверждать, что $\lim\limits_{n\rightarrow\infty}c_{n}z^{n}=0, то есть \left \{ c_{n}z^{n} \right \}$ --- ограничена: $\exists M:\forall n\in N:  c_{n}z_{0}^{n} \leq M
	  c_{n}z^{n} = \left ( c_{n}z_{0}^{n} \right )\left ( \frac{z^{n}}{z_{0}^{n}} \right ) =  c_{n}z_{0}^{n}  \frac{z}{z_{0}} ^{n}\leq M q^{n}\leq M$
	Получили, что $\sum\limits_{0}^{\infty}Mq^{n}$ – сходится.
	
	Значит по признаку сравнения ряд $\sum\limits_{0}^{\infty} c_{n}z^{n}$ – сходится абсолютно $\forall z\in K_{0}$.
	                       
	Для доказательства равномерной сходимости воспользуемся признаком Вейерштрасса.
	$\forall n \in \mathbb{N} \forall x \in [-x_2; x_2] |c_nx^n| \leq C\left|\frac{x}{x_1} \right| < C\left|\frac{x_2}{x_1} \right|  $.
	
	Так как   $\sum\limits_{0}^{\infty} C\left|\frac{x_2}{x_1} \right|$ --- сходится, то $\sum\limits_{n = 0}^{\infty}c_n x^n$ равномерно сходится на $[-x_2; x_2]$           \end{proof}


\begin{Def}
	Радиусом сходимости $R$ степенного ряда  называется точная верхняя грань множества точек, в которых ряд сходится.
\end{Def}


\begin{Consequence}
	Ряд  $\sum\limits_{n = 0}^{\infty}c_n x^n$ сходится абсолютно $\forall x \in (-R; R)$ (этот интервал называется интервалом сходимости ряда), и более того, $\forall r \in (0; r)$ сходимость на $(-r; r)$ --- равномерная.
\end{Consequence}
Всё вышесказанное очевидно следует из определения точной верхней грани и теоремы Абеля.

\subsection{Нахождение радиуса сходимости}

Факт существования у рядов радиуса сходимости --- это прекрасно, но хотелось бы уметь его находить.


\begin{Theorem} [Формула Коши-Адамара]
	Пусть $\sum\limits_{n = 0}^{\infty}c_n x^n$  --- степенной ряд. Тогда радиус сходимости этого ряда $R= \frac{1}{\varlimsup\limits_{n \to \infty} \sqrt[n]{|c_n|}}$ (полагая при $\varlimsup\limits_{n \to \infty} \sqrt[n]{|c_n|} = \infty$ что $R = 0$ и при $\varlimsup\limits_{n \to \infty} \sqrt[n]{|c_n|} = 0$ что $R = \infty$ ).
\end{Theorem}
\begin{proof}
	Как ни странно, прямо следует из признака Коши сходимости числовых рядов.
	
\end{proof}

Зная эту формулу можно легко придумать ряд с любым радиусом сходимости.

\subsection{Поведение в концах интервала сходимости}

В концах интервала сходимости может происходить разное.
Два простых примера:
\\$\sum\limits_{n = 0}^{\infty}\frac{1}{n} x^n$ --- радиус сходимости равен 1, в при $x= \pm 1$ ряд расходится.
\\$\sum\limits_{n = 0}^{\infty}\frac{(-1)^n}{n} x^n$ --- радиус сходимости равен 1, в при $x= \pm 1$ ряд сходится условно.
\\$\sum\limits_{n = 0}^{\infty}\frac{1}{n^2} x^n$ --- радиус сходимости равен 1, в при $x= \pm 1$ ряд сходится абсолютно.




\begin{Theorem} [Абеля II]
	Пусть ряд $\sum\limits_{n = 0}^{\infty}c_n x^n$ сходится в точке $x_1$. Тогда он равномерно сходится на отрезке. $[0, x_1]$.
\end{Theorem}

\begin{proof}
Пусть $x$ не превышает радиуса сходимости ряда. То есть: $0 \leq x\leq R$. Заменим переменную $x$ на $R$ и получим из ряда $\sum\limits_{n=0}^{\infty}a_{n}x_{n}$ ряд, имеющий вид: $\sum\limits_{n=0}^{\infty}a_{n}R_{n}\left ( \frac{x}{R} \right )^{n}$. Видим, что полученный ряд $\sum\limits_{n=0}^{\infty}a_{n}R^{n}$ не зависит от переменной $x$, тогда его сходимость означает и равномерную сходимость. Очевидно, что последовательность $\{ \left ( \frac{x}{R} \right )^{n}\}$ ограничена на отрезке $\left [ 0;R \right ]$, ее члены неотрицательны: $0\leq\left ( \frac{x}{R} \right )^{n}\leq 1$. Эта последовательность убывает в каждой точке (при  $x=R$ она не строго убывает, точнее, является стационарной). Значит выполняются условия признака Абеля равномерной сходимости рядов. То есть ряд $\sum\limits_{n=0}^{\infty}a_{n}z^{n}$ равномерно сходится на отрезке $\left [ 0;R \right ]$.
\end{proof}

\begin{Consequence}
Сумма степенного ряда непрерывна на всём множестве его сходимости.
\end{Consequence}

\begin{proof}
Согласно первой теореме Абеля, ряд равномерно сходится на $\left [ -r,r \right ] \subset \left ( -R, R \right )$, однако про весь интервал это точно утверждать нельзя, так как на интервале $\left ( -R, R \right )$ ряд может сходиться и неравномерно. Пусть $x_{0}\in\left ( -R, R \right )$. Выберем такое $r$, что $x_{0}<r<R$. Так как$ x_{0}$ --- внутренняя точка отрезка $\left [ -r,r \right ]$ и на $\left [ -r,r \right ]$ ряд  сходится равномерно, то, по теореме о непрерывности суммы равномерно сходящегося ряда непрерывных функций, сумма ряда является непрерывной функцией на $\left [ -r,r \right ]$, включая точку $x_{0}$.
Поскольку точку $x_{0}\in\left ( -R, R \right )$ мы взяли произвольную, то сумма ряда непрерывна на интервале $\left ( -R, R \right )$.



\end{proof}
\end{document}