\documentclass[a4paper, 12pt]{article}
\usepackage{header}
\begin{document}
\pagestyle{fancy}
\section{Лекция 07 от 17.09.2016 \\ Функциональные последовательности и ряды}
	\subsection{Поточечная и равномерная сходимость}
	Начиная с этой лекции, будем говорить о функциональных последовательностях и рядах.
	\par Пусть $X$ --- произвольное множество точек, а $\{f(x)\}_n^\infty$ --- последовательность функций, определённых на $X$ или на его подмножествах.
	\begin{Def}
		Будем говорить, что $f_n(x)$ сходится поточечно к $f(x)$, если
		$$
		\forall x \in X\; \forall \varepsilon>0\; \exists N\in \N\colon \; \forall n>N\; |f_n(x) - f(x)| < \varepsilon.
		$$
	\end{Def}
	\begin{Designation}
	$f_n(x) \overset{X}{\underset{n\to\infty}{\longrightarrow}} f(x)$.
	\end{Designation}
	Почему такое определение не совсем удобно для нас? Сходимость в каждой точке может быть своя, произвольная, а хотелось бы, чтобы свойства функций $f_n$ и $f$ были похожи. Приведем пример, когда это не выполняется.
	\begin{Examples}
		Если $f_n(x) = x^n$, $X = [0;1]$, то
		$$
		f_n(x) \overset{[0;1]}{\underset{n\to\infty}{\longrightarrow}}		
		\begin{cases*}
			0,\;x<1\\
			1,\;x = 1
		\end{cases*}.
		$$
	\end{Examples}
	То есть бывает так, что все функции последовательности непрерывны на отрезке и стремятся к разрывной функции.
	\par Для устранения этого недостатка введём другое определение сходимости функциональной последовательности.
	\begin{Def}
		Будем говорить, что последовательность функций $\{f_n(x)\}_{n=1}^\infty$ сходится к $f(x)$ равномерно на множестве $X$, если 
		$$
		\forall \varepsilon > 0\; \exists N\in \N\colon \; \forall n>N\; \forall x \in X\; |f_n(x) - f(x)| < \varepsilon
		$$
	\end{Def}
	\begin{Designation}
		$f_n(x) \overset{X}{\underset{n\to\infty}{\rightrightarrows}} f(x)$.
	\end{Designation}
	Из определений сразу очевидно следует утверждение.
	\begin{Statement}
		Если $f_n(x)\overset{X}{\underset{n\to\infty}{\rightrightarrows}} f(x)$, то $f_n(x) \overset{X}{\underset{n\to\infty}{\longrightarrow}} f(x)$.
	\end{Statement}
	
	А что, если нам даны последовательность $f_n(x)$, функция $f(x)$ и множество $X$, то как нам понять, сходится ли $f_n(x)$ к $f(x)$?
	
	Существует мощный способ. Обозначим $r_n(x) =\underset{x \in X}{\sup}\;|f_n(x) - f(x)|$.
	\begin{Statement}
		$f_n(x)\overset{X}{\underset{n\to\infty}{\rightrightarrows}} f(x) \Leftrightarrow r_n\to 0$.
	\end{Statement}
	\begin{proof}\ \\
		\textbf{Необходимость}. Зафиксируем произвольное $\varepsilon > 0$, положим $\varepsilon_1 = \varepsilon/2$. Тогда
		\begin{gather*}
			\exists N\in \N \colon\; \forall n>N\; \forall x \in X\; |f_n(x) - f(x)| < \varepsilon_1,\\
			\Rightarrow r_n =  \underset{x \in X}{\sup}\; |f_n(x) - f(x)| \leqslant \varepsilon_1 < \varepsilon.
		\end{gather*}
		То есть $r_n \to 0$ при $n\to \infty$.
		\par \textbf{Достаточность}.
		$$
		\forall \varepsilon > 0\; \exists N \in \N\colon \; \forall n>N\; r_n < \varepsilon \Rightarrow \forall x \in X\; |f_n(x) - f(x)| < \varepsilon.
		$$
	\end{proof}
	Часто это утверждение называют \textbf{супремум-критерием}. Для приведённого выше примера $f_n(x) = x^n$.
	\begin{Statement}\ \\		
		\begin{enumerate}
			\item $x^n \overset{[0;1]}{\underset{n\to\infty}{\longrightarrow}} 0$.
			\item $x^n \overset{[0;1]}{\underset{n\to\infty}{\not\rightrightarrows}} 0$.
		\end{enumerate}
	\end{Statement}
	\begin{proof}
		$r_n = \underset{x\in(0;1)}{\sup}\;|x^n - 1| = 1 \not\to 0$.
	\end{proof}
	Есть ещё одна подобного рода последовательность.
	$$
		f_n(x) = \cfrac{1}{0!} + \cfrac{x}{1!} + \cfrac{x^2}{2!} + \ldots + \cfrac{x^n}{n!}
	$$
	В любой точке значение $\lim\limits_{n \to \infty} f_n(x) = e^x$, то есть 	$f_n(x) \overset{\R}{\underset{n\to\infty}{\longrightarrow}} e^x$, но $f_n(x) \overset{\R}{\underset{n\to\infty}{\not\rightrightarrows}} e^x$. Однако, как легко понять, $f_n(x) \overset{(-C;C)}{\underset{n\to\infty}{\rightrightarrows}} e^x$ для всякого $C>0$. Эта последовательность ещё всплывёт в нашем курсе.
	\begin{Statement}
		Если $f_n(x) \overset{X}{\underset{n\to\infty}{\rightrightarrows}} f(x)$ и $g_n(x) \overset{X}{\underset{n\to\infty}{\rightrightarrows}} g(x)$, $\alpha \in \R$, то 
		\begin{enumerate}
			\item $f_n(x) + g_n(x) \overset{X}{\underset{n\to\infty}{\rightrightarrows}} f(x) + g(x)$,
			\item $\alpha f_n(x) \overset{X}{\underset{n\to\infty}{\rightrightarrows}} \alpha f(x)$.
		\end{enumerate}
	\end{Statement}
	\begin{proof}
		Докажем пункт 1, второй доказывается аналогичною. Зафиксируем произвольное $\varepsilon > 0$, положим $\varepsilon_1 = \varepsilon / 2$. Тогда
		\begin{gather*}
			\exists N_1 \in \N\colon \; \forall x\in X\; |f_n(x) - f(x) | < \varepsilon_1,\\
			\exists N_2 \in \N\colon \; \forall x\in X\; |g_n(x) - g(x) | < \varepsilon_1.\\
		\end{gather*}
		Положим $N = \max\{N_1, N_2\}$. Тогда
		$$
			|(f_n(x) + g_n(x)) - (f(x) + g(x))| \leqslant |f_n(x) - f(x)| + |g_n(x) - g(x)| < \varepsilon_1 + \varepsilon_1 = \varepsilon.
		$$
		Получили требуемое.
	\end{proof}
	\begin{Statement}
		Если $f_n(x)\overset{X}{\underset{n\to\infty}{\rightrightarrows}} f(x)$ и $g(x)$ ограничена на множестве $X$, то\\ $f_n(x)g(x) \overset{X}{\underset{n\to\infty}{\rightrightarrows}} f(x) g(x)$.
	\end{Statement}
	\begin{proof}
		$\exists C>0\colon\; \forall x\in X\; |g(x)| < C$. Зафиксируем произвольное $\varepsilon > 0$, положим $\varepsilon_1 = \varepsilon/C$. Найдём такое $N\in \N$, что 
		$\forall n>N,\; \forall x\in X\; |f_n(x) - f(x)| < \varepsilon_1$. Тогда \\$\forall n >N\colon \forall x\in X\; |f_n(x)g(x) - f(x)g(x)| < C\varepsilon_1 = \varepsilon$. 
	\end{proof}
	\begin{Comment}
		Если $f_n(x) \overset{X}{\underset{n\to\infty}{\rightrightarrows}} f(x)$, $g_n(x) \overset{X}{\underset{n\to\infty}{\rightrightarrows}} g(x)$ и $f(x),\; g(x)$ ограничены на множестве $X$, то $f_n(x)g_n(x)\overset{X}{\underset{n\to\infty}{\rightrightarrows}} f(x)g(x)$.
	\end{Comment}
	\begin{Comment}
		Если $f_n(x) \overset{X}{\underset{n\to\infty}{\rightrightarrows}} f(x)$ и $f(x)$ отделена от нуля (т.е. существует такое $\alpha>0$, что для любого элемента множества $X$ $|f(x)| \geqslant \alpha$), то $\cfrac{1}{f_n(x)} \overset{X}{\underset{n\to\infty}{\rightrightarrows}} \cfrac{1}{f(x)}$. 
	\end{Comment}
	Доказательство этих фактов остаётся в качестве управжнения. \textbf{Указание}. Рассмотреть \\$\cfrac{1}{f_n} - \cfrac{1}{f} = (f-f_n)\cfrac{1}{f_n\cdot f}$.

	\subsection{Геометрический смысл равномерной сходимости}
	Несложно понять, что если $f_n(x) \overset{X}{\underset{n\to\infty}{\rightrightarrows}} f(x)$, то для всякого $\varepsilon > 0$, начиная с какого-то $N\in \N$ для всех $n>N$ все графики функций $f_n(x)$ окажется в $\varepsilon$-коридоре функции $f(x)$.
	\subsection{Критерий Коши равномерной сходимости}
	\begin{Theorem}[Критерий Коши равнoмерной сходимости]
		Следующие условия эквивалентны:
		\begin{enumerate}
			\item $f_n(x)$ $\overset{X}{\underset{n\to\infty}{\rightrightarrows}} ???$ (равномерно сходится куда-то);
			\item $\{f_n(x)\}_{n=1}^\infty$ удовлетворяет условию Коши равномерной сходимости на $X$:
			$$
			\forall \varepsilon > 0\; \exists N\in\N\colon \; \forall n,m>N\; \forall x\in X\; |f_n(x) - f_m(x)| < \varepsilon.
			$$
		\end{enumerate}
	\end{Theorem}
	\begin{proof}
		\par \textbf{1 $\Leftarrow$ 2}. Заметим, что для всякого $x\in X$ числовая последовательность $\{f_n(x)\}_{n=1}^\infty$ является фундаментальной. Тогда $\forall x\in X\; \exists \lim\limits_{n\to \infty} f_n(x) = f(x)$. Зафиксируем произвольное $\varepsilon > 0,\; \varepsilon_1 = \varepsilon / 2$. Найдём такое $N \in \N$, что 
		$$
			\forall n,m>N\; \forall x\in X\; |f_n(x) - f_m(x)| < \varepsilon_1.
		$$
		Зафиксировав $n$ перейдём к пределу при $m \to \infty$, получим $|f_n(x) - f(x)| \leqslant \varepsilon_1 < \varepsilon$. Получили требуемое.
		\par \textbf{1 $\Rightarrow$ 2}. Пусть $f_n(x) \overset{X}{\underset{n\to\infty}{\rightrightarrows}} f(x)$.
		\par Зафиксируем произвольное $\varepsilon > 0$, положим $\varepsilon_1 = \varepsilon / 2$. Найдём такое $N\in \N\; \forall n>N\; \forall x\in X$ верно $ |f_n(x) - f(x)| < \varepsilon_1$. Тогда $\forall n,m > N\; \forall x \in X$ выполнено
		$$
			|f_n(x) - f_m(x)| \leqslant |f_n(x) - f(x)| + |f(x) - f_m(x)| < \varepsilon_1 + \varepsilon_1 = \varepsilon.
		$$
	\end{proof}
	\subsection{Функциональные ряды}
	Перейдём к рассмотрению функциональных рядов. Тут все определения и теоремы переносятся с обычных рядов с заменой числовых последовательностей на функциональные.
	\begin{Def}
		Будем говорить, что ряд $\sum\limits_{n=1}^{\infty}f_n(x)$ сходится равномерно к $S(x)$ на множестве $X$, если последовательность его частичных сумм $S_n(x) = f_1(x) + f_2(x) + \ldots + f_n(x)$ сходится равномерно к $S(x)$ на множестве $X$. 
	\end{Def}
	Отсюда же можно сформулировать ряд утверждений, которые по сути мы уже доказали.
	\begin{Statement}
		Пусть $\sum\limits_{n=1}^\infty f_n(x) \overset{X}{\underset{n\to\infty}{\rightrightarrows}} S_1(x)$, 
		$\sum\limits_{n=1}^\infty g_n(x) \overset{X}{\underset{n\to\infty}{\rightrightarrows}} S_2(x)$. Тогда их почленная сумма $\sum\limits_{n=1}^\infty  (f_n(x) + g_n(x) ) \overset{X}{\underset{n\to\infty}{\rightrightarrows}} S_1(x) + S_2(x)$.
	\end{Statement}
	\begin{Statement}
		Если $\sum\limits_{n=1}^\infty f_n(x) \overset{X}{\underset{n\to\infty}{\rightrightarrows}} S(x)$, $\alpha \in \R$, то
		$$
			\sum\limits_{n=1}^\infty \alpha f_n(x) \overset{X}{\underset{n\to\infty}{\rightrightarrows}} \alpha S(x).
		$$
	\end{Statement}
	\begin{Statement}
		Если $\sum\limits_{n=1}^\infty f_n(x) \overset{X}{\underset{n\to\infty}{\rightrightarrows}} S(x)$, а $g(x)$ ограничена на $X$, то 
		$$
		\sum\limits_{n=1}^\infty g(x) f_n(x) \overset{X}{\underset{n\to\infty}{\rightrightarrows}} g(x)S(x).
		$$
	\end{Statement}
	Ну и конечно, мы не обойдёмся без критерия Коши.
	\begin{Statement}[Критерий Коши равномерной сходимости функционального ряда]
		Следующие утверждения эквивалентны.
		\begin{enumerate}
			\item $\sum\limits_{n=1}^\infty f_n(x) \overset{X}{\underset{n\to\infty}{\rightrightarrows}} ???$ (опять же, сходится куда-то).
			\item Выполняется условие Коши 
			$$\forall \varepsilon >0\; \exists N \in \N\colon \; \forall m>N,\; \forall p \in \N,\; \forall x\in X\; \left|\sum_{n=m+1}^{m+p} f_n(x)\right| < \varepsilon.
			$$
		\end{enumerate}
	\end{Statement}
	Отсюда же нахаляву получаем утверждение.
	\begin{Statement}[Необходимое условие равномерной сходимости функционального ряда]
		Если $\sum\limits_{n=1}^\infty f_n(x)  \overset{X}{\underset{n\to\infty}{\rightrightarrows}}$, то $f_n(x) \overset{X}{\underset{n\to\infty}{\rightrightarrows}} 0$.
	\end{Statement}
\end{document}
