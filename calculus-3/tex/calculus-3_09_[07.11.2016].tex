\documentclass[a4paper, 12pt]{article}
\usepackage{header}

\begin{document}
\pagestyle{fancy}
\section{Лекция 09 от 07.11.2016 \\ Предел по базе}
Все пределы, которые раньше возникали в нашем курсе --- это частные случаи предела по базе.

\subsection{Что это такое?}

Пусть $X$ --- произвольное непустое множество.

\begin{Def}
Система подмножеств $\B$ множества $X$ называется базой, если
\begin{enumerate}
\item $\varnothing \not\in \B$;
\item $\forall B_1,\ B_2 \in \B\ \exists B_3 \in \B :\ B_3 \subset B_1 \cap B_2$.
\end{enumerate}
\end{Def}

\begin{Comment}
В классическом понимании символ $\subset$ означает строгое включение, однако в современной математике это также может означать равенство множеств, и мы будем пользоваться именно этим значением. Если хотят подчеркнуть, что множества не равны, то пишут $\varsubsetneq$.
\end{Comment}

Пусть функция $f$ определена на $X$ или части $X$ и принимает действительные значения
(впрочем, действительность не принципиальна).
\begin{Def}
Число $A$ называют пределом функции $f$ по базе $\B$, если
\begin{itemize}
\item[0.] $\exists B \in \B: f$ определена на $B$;
\item[1.] $\forall \eps > 0\ \exists B \in \B: \forall x \in B\ |f(x) - A| < \eps$.
\end{itemize}
\end{Def}
Вообще говоря, нулевое условие можно опустить, так как оно следует из первого, но исторически сложилось, что его все-таки пишут --- на практике гораздо удобней сначала проверить, определена ли функция хоть где-то.

\begin{Examples}\ 
\begin{itemize}
\item $X = \N$, $\B = \{B_n\}_{n=1}^{\infty}$, $B_n = \{n, n+1, n+2, \ldots  \}$. Тогда $B_n \cap B_m = B_{\max(n, m)}$. Такая база задает предел числовой последовательности.
\item $X = \R$, $\B = \{B_\delta\}_{\delta>0}$, $B_\delta = (-\delta, \delta)\setminus \{0\}$. Такая база задает двусторонний предел функции при $x\to 0$. Аналогично можно задать односторонние пределы.
\item Пусть зафиксирован отрезок $[a, b]$ и $f:[a, b] \to \R$.

Пусть $X$ --- множество всех отмеченных разбиений $[a, b]$ (то есть это разбиения с зафиксированной точкой на каждом отрезке). Тогда базой Римана называется база\\ $\B = \{B_\delta\}_{\delta>0}$, где $B_\delta$ это совокупность всех отмеченных разбиений с диаметром меньше $\delta$. Соответственно, интеграл Риман является пределом по 
этой базе интегральных сумм Римана, рассматриваемых как функция от отмеченных разбиений при фиксированной функции $f$:
$$
\sigma(f, (\tau, \xi)) = \sum\limits_{j=1}^{n}f(\xi_j)|\Delta_j|.
$$
\end{itemize}
\end{Examples}

\subsection{Ключевые свойства}
Пусть $\B$ --- база $X$.

\begin{Statement}
Если $\lim\limits_{\B} f = A_1$ и $\lim\limits_{\B} f = A_2$, то $A_1 = A_2$.
\end{Statement}
\begin{proof}
Пусть $A_1 \neq A_2$. Положим $\eps = \frac{|A_1 - A_2|}{2}$. Тогда:
\begin{align*}
& \exists B_1 \in \B: \forall x \in B_1\ |f(x) - A_1| < \eps; \\
& \exists B_2 \in \B: \forall x \in B_2\ |f(x) - A_2| < \eps.
\end{align*}
Тогда существует $B_3 \in \B$ такой, что $B_3 \subset B_1 \cap B_2$. Для него будет верно, что
$$
\forall x \in B_3:\ |A_1 - A_2| = |A_1 - f(x) + f(x) - A_2| \leq |A_1 - f(x)| + |A_2 - f(x)| < 2\eps = |A_1 - A_2|.
$$
При этом важно понимать, что $B_3 \neq \varnothing$, просто по определению.

Получили противоречие.
\end{proof}

Давно знакомое всем доказательство, но зато оно показывает, почему база определена именно так.

\begin{Statement}
Пусть $\lim\limits_{\B} f(x) = A$, $\lim\limits_{\B}g(x) = B$ и $\alpha \in \R$. Тогда:
\begin{enumerate}
\item $\lim\limits_{\B}(f + g) = A + B$;
\item $\lim\limits_{\B}(fg) = AB$;
\item $\lim\limits_{\B}(\alpha f) = \alpha A$;
\item $\lim\limits_{\B}\dfrac{f}{g} = \dfrac{A}{B}$, если $B \neq 0$.
\end{enumerate} 
\end{Statement}

\begin{proof}
Это тоже почти школьный материал, так что докажем только один пункт. Пусть это будет последний.

Немного преобразуем:
$$
\frac{f}{g} - \frac{A}{B} = \frac{Bf - Ag}{gB} = \frac{Bf - BA + BA - Ag}{gB} = \frac{B(f - A) + A(B - g)}{gB}.
$$
Возьмем произвольное $\eps > 0$. Положим $\eps_1 = \min\left( \dfrac{\eps B^2}{100(|A| + |B|) + 1}; \dfrac{|B|}{2} \right)$. Найдем такие $B_1,\ B_2 \in \B$, что:
\begin{gather}
\forall x \in B_1: |f(x) - A| < \eps_1;\\
\forall x \in B_2: |g(x) - B| < \eps_1.
\end{gather}
Найдем такое $B_3 \in \B$, что $B_3 \subset B_1 \cap B_2$. Тогда для всех $x \in B_3$ верно, что:
$$
\left| \frac{f(x)}{g(x)} - \frac{A}{B} \right| \leq \frac{|B|\eps_1 + |A|\eps_1}{B^2 / 2} = \eps_1 \frac{2(|A| + |B|)}{B^2} < \eps.
$$
\end{proof}

\begin{Statement}
Если существует предел $\lim\limits_{\B}f$ и функция $f$ неотрицательна хотя бы на одном элементе $B$ базы $\B$, то $\lim\limits_{\B}f \geq 0$.
\end{Statement}
\begin{proof}
Пусть $\lim\limits_{\B}f = A < 0$. Тогда для $\eps = \frac{|A|}{2}$ существует такой $\widetilde{B} \in \B$, что $\forall x \in \widetilde{B}:\ |f(x) - A| < \eps.$

Но существует $x \in B \cap \widetilde{B}$, и тогда для него одновременно будет верно, что $f(x) \geq 0$ и\\ $f(x) \leq \frac{A}{2} < 0$. Противоречие.
\end{proof}

\begin{Consequence}
Пусть $f \geq g$ на некотором элементе $B \in \B$ и существуют пределы $\lim\limits_{\B}f = A$ и $\lim\limits_{\B}g = \widetilde{A}$. Тогда $A \geq \widetilde{A}$.
\end{Consequence}
\begin{proof}
$\lim\limits_{\B}(f - g) = A - \widetilde{A}$ и одновременно, $(f - g) \geq 0$ на $\B$.
\end{proof}

Пусть $\B$ и $\widetilde{\B}$ --- базы на $X$.
\begin{Statement}
Пусть $\lim\limits_{\B}f = A$ и для каждого элемента $B \in \B$ существует элемент $\widetilde{B} \in \widetilde{\B}$ такой, что $\widetilde{B} \subset B$. Тогда $\lim\limits_{\widetilde{\B}}f = A$.
\end{Statement}

\begin{proof}
Зафиксируем произвольное $\eps > 0$. Найдем $B \in \B$ такое, что $\forall x \in B: |f(x) - A| < \eps$.
Теперь найдем $\widetilde{B} \in \widetilde{\B}$ такое, что $\widetilde{B} \subset B$. Тогда $\forall x \in \widetilde{B}: |f(x) - A| < \eps$. Получили по определению предела, что $\lim\limits_{\widetilde{\B}}f = A$.
\end{proof}

Фактически это обобщение утверждения, что любая подпоследовательность сходится туда же, куда и вся последовательность, и что если есть предел, то есть и оба односторонних предела, и они все равны.

Ну а где есть предел, там есть и критерий Коши!

\subsection{Критерий Коши}

\begin{Def}
Функция $f$ удовлетворяет условию Коши по базе $\B$, если:
\begin{itemize}
\item[0.] $\exists B_0 \in \B: f$ определена на $B_0$;
\item[1.] $\forall \eps > 0: \exists B \in \B: \forall x, \widetilde{x} \in B\ |f(x) - f(\widetilde{x})| < \eps$.
\end{itemize}
\end{Def}

\begin{Theorem}[Критерий Коши]
Следующие условия эквивалентны:
\begin{enumerate}
\item существует предел $\lim\limits_\B f$;
\item функция $f$ удовлетворяет условию Коши по базе $\B$.
\end{enumerate}
\begin{proof}
Напоминаем, что мы пока определили только конечные пределы по базе.

$(1) \Rightarrow (2)$. Доказываем как обычно, через $\eps/2$ и прочее.

$(2) \Rightarrow (1)$. Построим последовательность $B_1 \supset B_2 \supset B_3 \supset \ldots$:
\begin{align}
\text{для }\eps = 1, &\ \exists B_1 \in \B: \forall x, \widetilde{x} \in B_1\ |f(x) - f(\widetilde{x})| < 1;\\
\text{для }\eps = 1/2,&\ \exists \widetilde{B_2} \in \B: \forall x, \widetilde{x} \in \widetilde{B_2}\ |f(x) - f(\widetilde{x})| < 1/2,\\
&\ \exists B_2 \subset B_1 \cap \widetilde{B_2}:\text{ тогда }\forall x, \widetilde{x} \in B_2\ |f(x) - f(\widetilde{x})| < 1/2;\\ 
\text{для }\eps = 1/3,&\ \exists \widetilde{B_3} \in \B: \forall x, \widetilde{x} \in \widetilde{B_3}\ |f(x) - f(\widetilde{x})| < 1/3, \\
&\ \exists B_3 \subset B_2 \cap \widetilde{B_3}:\text{ тогда } \forall x, \widetilde{x} \in B_3\ |f(x) - f(\widetilde{x})| < 1/3;\\ 
\ldots
\end{align}

Для каждого элемента $B_n$ выберем точку $x_n \in B_n$. Заметим, что $\{ f(x_n) \}_{n=1}^\infty$ --- функциональная последовательность, так как
\begin{align}
&\forall \eps > 0\ \exists N \in \N,:\ \frac{1}{N} < \eps \Rightarrow \forall n, m > N \ |f(x_n) - f(x_m)| < \frac{1}{N} < \eps.
\end{align}
Это верно, так как $x_n \in B_n \subset B_N$ и аналогично для $x_m$.

Значит, существует предел $\lim\limits_{n\to \infty}f(x_n) = A$. Покажем, что $\lim\limits_{\B}f = A$.

Зафиксируем произвольное $\eps > 0$.  Тогда
$$
\exists N \in \N: \forall n > N\ |f(x_n) - A| < \eps / 2.
$$
При этом, существует такое $n>N$, что $\dfrac{1}{n} < \dfrac{\eps}{2}$. Значит,
$$
\forall x \in B_n \in \B: |f(x) - A| \leq |f(x) - f(x_n)| + |f(x_n) - A| < \frac{1}{n} + \frac{\eps}{2} < \eps.
$$
\end{proof}
\end{Theorem}

\end{document}