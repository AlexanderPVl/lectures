\documentclass[a4letter,12pt]{article}
\usepackage{amsmath}
\usepackage{subfiles}
\usepackage{algo-header}

\usepackage{epigraph}

\newcommand{\doverline}[1]{\overline{\overline{#1}}}

\begin{document}
	\title{Разбор примерной контрольной работы по алгоритмам}
	\author{Орлов Никита}
	\maketitle
	
	\subsection*{Задача 1}	
	Приведите примеры функций $f(n)$ и $g(n)$, таких что $f(n) = \Theta(g(n))$ и одновременно $2^{f(n)} = \doverline{o}(2^{g(n)})$, или докажите, что таких функций не существует.
	
	\subsubsection*{Решение}
	Допустим, что $f(n) = \Theta(g(n))$. Распишем $\Theta$ по определению:
	\[
	f(n) = \Theta(g(n))
	\Leftrightarrow
	\exists c_1, c_2,n_0 > 0 \ \forall n\geqslant n_0: \
	0\leqslant c_1\cdot g(n)\leqslant f(n) \leqslant c_2 \cdot g(n)
	\]
	
	Возведем 2 в степень всех частей неравенства, при этом неравенство останется верным, поскольку $a < b \Rightarrow 2^a < 2^b$:
	\[
	2^0 \leqslant 2^{c_1g(n)} \leqslant 2^{f(n)} \leqslant 2^{c_2g(n)}
	\]
	Получили, что $2^{f(n)} = \Theta(2^{g(n)})$, противоречие условию, а значит таких функций не существует.
	
	\subsection*{Задача 2}
	
	Решите рекуррентное соотношение в терминах $\Theta$:
	\[
	T(n) = T\left(\frac n3\right) + T\left(\frac n4\right) + 5n
	\]
	
	\subsubsection*{Решение}
	Оценим сумму сверху и снизу. Так как
	\[
	2T\left(\frac n3\right)+5n \leqslant  T\left(\frac n3\right) + T\left(\frac n4\right) + 5n \leqslant 2T\left(\frac n4\right) + 5n
	\]
	
	По основной теореме о рекуррентных соотношениях,
	\[
	O(n) \leqslant T\left(\frac n3\right) + T\left(\frac n4\right) + 5n \leqslant O(n) \Rightarrow T(n) = \Theta(n)
	\]
	\subsection*{Задача 3}
	Медиана упорядоченного по возрастанию множества из $k$ элементов -- это $\frac k2$-ый по порядку элемент этого множества в предположении, что $k$ -- четно. Пусть $X[1..n]$ и $Y[1..n]$ -- два отсортированных массива, каждый из которых содержит по $n$ элементов. Опишите алгоритм, в котором поиск медианы всех $2n$ элементов, содержащихся в массивах $X, \ Y$, выполнялся бы за время $O(\log n)$.
	
	\subsubsection*{Решение}
	Будем поступать следующим образом. Возьмем медианы в этих массивах, пусть их индексы в массивах равны соответственно $m_1$ и $m_2$. Если $X[m_1] < Y[m_2]$, то будем искать медиану в массивах $X[:m_1]$ и $Y[m_2:]$, если $X[m_1] > Y[m_2]$, то будем искать медиану в массивах $X[m_1:]$ и $Y[:m_2]$ рекурсивно. Если же медианы равны, то выдаем одну из них. Этот алгоритм на каждом шаге уменьшает в два раза размер задачи, а значит сложность - сумма убывающей геометрической прогрессии - равна $O(\log n)$
	
	\subsection*{Задача 4}
	Опишите, как реализовать структуру данных, поддерживающую два стека, используя один массив $A[1..n]$ из $n$ элементов. Опишите реализацию процедур $IsEmpty1(), \ Push1(v), \ Pop1()$ и аналогичные для второго стека. Переполнение должно происходить при попытке вставки нового элемента в один из стеков в ситуации, когда суммарное число элементов в обоих стеках равно $n$.
	
	\subsubsection*{Решение}
	
	Так как нам известен размер массива, будем заполнять один стек с начала массива, а другой с конца. Будем хранить индексы последних элементов каждого стека. Примерная реализация структуры:
	
	\begin{struct}{double-ended stack}[H]
		\State array $A[1..n]$
		\State integer $end1 = 1, \ end2 = n$
		
		\State 
		
		\Function{$Push1$}{$v$}
			\If{$end1 + 1 == end2}$
				\State \Return \textbf{error}
			\EndIf
			\State $end1 := end1 + 1$
			\State $A[end1] = v$	
		\EndFunction
		
		\State 
		
		\Function{$Pop1$}{$ $}
			\If{$end1 - 1 == -1$}
				\State \Return \textbf{error}
			\EndIf
			\State $res := A[end1]$
			\State $end1 := end1 - 1$
			\State \Return $res$
		\EndFunction
		
		\State
		
		\Function{$IsEmpty1$}{$ $}
			\State \Return $end1 == 0$
		\EndFunction
		
		\State
		
		\Function{$Push2$}{$v$}
			\If{$end2 - 1 == end2}$
				\State \Return \textbf{error}
			\EndIf
			\State $end2 := end2 - 1$
			\State $A[end2] = v$	
		\EndFunction
		
		\State
		
		\Function{$Pop2$}{$ $}
			\If{$end2 + 1 == n + 1$}
				\State \Return \textbf{error}
			\EndIf
			\State $res := A[end2]$
			\State $end2 := end2 + 1$
			\State \Return $res$
		\EndFunction
		
		\State
		
		\Function{$IsEmpty2$}{$ $}
			\State \Return $end2 == n + 1$
		\EndFunction
	\end{struct}
	
	
	\subsection*{Задача 5}
	Структура данных поддерживает операцию, такую что последовательность из $n$ вызовов той операции занимает время $\Theta(n\log n)$ в худшем случае. Какова амортизированная сложность этой операции? Какой может быть сложность этой операции в худшем случае? Приведите пример структуры данных и операции, для которых достигается этот худший случай?
	
	\subsubsection*{Решение}
	
	Получить амортизированную стоимость просто: поделим время последовательности вызовов на число вызовов. Получим:
	\[
	\frac{n\log n}{n} = \log n
	\]
	
	\textit{TO BE CONTINUED. ETA: 22:00 28.02.2017}
	

	
	
	
	
	
	
	
	
	
	
	
	
\end{document}
