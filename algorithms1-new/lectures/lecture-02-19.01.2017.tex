\documentclass[../book.tex]{subfiles}


\begin{document}
	\section{Лекция 2. Merge sort, Binary search, рекуррентные соотношения}
	
	\textit{Лекция будет дополнена примерами решения рекуррентных соотношений без основной теоремы, а так же доказательством теоремы. Текущая версия вычитана лектором.}

	Поговорим еще немного про сортировки. Сортировка вставками имеет квадратичную сложность, что не оптимально. Есть более быстрые алгоритмы, один из них называется \textit{сортировкой слиянием}.
	
	\subsection{Сортировка слиянием}
	
	Суть сортировки достаточно проста: если у нас есть две отсортированные последовательности, мы можем их объединить в одну отсортированную, а последовательность длины один уже отсортирована, а значит мы можем разбить наш массив на блоки одинаковой длины, и рекурсивно отсортировать.
	
	
	Опишем функцию слияния двух массивов разной длины.
	\begin{algorithm}[H]
		\caption{Функция слияния отсортированных массивов}
		\begin{algorithmic}[1]
			\Require Отсортированные массивы $A, B$.
			\Ensure Отсортированный массив $C$
			\Function{merge}{$A, B$}
				\State $ i := 0 $
				\State $ j := 0 $
				\State vector $C$
				\While{$ i < len(A) $ \textbf{and} $ j < len(B)$}		
					\If{$ A[i] \leqslant B[j]$}
						\State $C.push\_back(A[i])$
						\State $i := i + 1$
					\Else
						\State $C.push\_back(B[i])$
						\State $j := j + 1$
					\EndIf
				\EndWhile
				\State $ C := C + A[i:len(A)] + B[j:len(B)]$
				\State \Return $C$
			\EndFunction
		\end{algorithmic}
	\end{algorithm}

		
	\begin{proof_cor}
		На входе отсортированные массивы. На каждой итерации цикла выбирается наименьший из еще не выбранных элементов.
	\end{proof_cor}
	\begin{time}
		Алгоритм смотрит на каждый элемент каждого массива, тогда его сложность получается $\Theta(len(A) + len(B))$.
	\end{time}	

	Теперь сам алгоритм сортировки слиянием:
	\begin{algorithm}[H]
		\caption{Merge Sort}
		\begin{algorithmic}[1]
			\Require Массив $A$
			\Ensure Отсортированный массив $C$
			\Function{merge\_sort}{$A$}
				\If{$ len(A) < 2$}
					\State \Return A
				\Else
					\State $n := len(A)$
					\State $A_1 := merge\_sort(A[0: n/2])$
					\State $A_2 := merge\_sort(A[n/2: n])$
					\State \Return $merge(A_1, A_2)$
				\EndIf
		
			\EndFunction
		\end{algorithmic}
	\end{algorithm}
	\begin{remark}
		Сортировку слиянием можно написать разными способами. Например, изначально разбить массив на куски длиной 10, каждую из них отсортировать вставками, а потом уже последовательно слить в один отсортированный массив.
	\end{remark}
	\begin{proof_cor}
		Пока корректна функция \textit{merge}, весь алгоритм корректен, так как вся задача разбивается на меньшие подзадачи, а рекурсия остановится на массивах размера 1.
	\end{proof_cor}
	\begin{time}
		Время работы данного алгоритма $T(n) = \Theta(n\cdot \log(n))$: У нас $\log(n)$ - глубина рекурсии, а на каждом шаге мы пройдемся в сумме по всему массиву. 
		
		%время работы описывается реккурентным соотношением. через теорему
	\end{time} 
	
	
	Время работы сортировки слиянием можно записать и по-другому, с помощью \textit{рекуррентной формулы}. В нашем случае она будет иметь вид
	\[
	T(n) = 
	\begin{cases}
		c, & n = 1; \\
		2T(\frac{n}{2}) + O(n), & n > 1;
	\end{cases}
	\]
	где $2T(\frac{n}{2})$ - сложность рекурсивных вызовов, $O(n)$ - сложность слияния. Для таких соотношений хочется получить правило их раскрытия в явную формулу. О таком преобразовании говорит \textit{основная теорема о рекуррентных соотношениях}.
	
	\subsection{Основная теорема}
	
	\begin{theorem}
		Пусть у нас есть рекуррентное соотношение, записанное в виде
		\[
		T(n) \leqslant
		\begin{cases}
		c, & n = 2\\
		aT(\frac{n}{b}) + cn^d, & n > 2
		\end{cases}
		\]
		Тогда его можно представить в следующем виде:
		\[
		T(n) = 
		\begin{cases}
		O(n^d \log(n)), & a = b^d \\
		O(n^d), & a < b^d \\
		O(n^{\log_b(a)}), & a > b^d
		\end{cases}
		\]
	\end{theorem}
	
	\TODO{Доказательство}
	\begin{proof}
		Рассмотрим алгоритм, который нашу задачу делит на каждом шаге на $b$ подзадач.
		\begin{center}
			\begin{tikzpicture}
				\begin{scope}[every node/.style={circle,thick,draw}]
					\node[label=\large$ cn^d $] (top) at (10, 10) {};
					
					\node[label=left:\Large $ \frac{cn^d}{b^d}$] (mid-left) at (7, 8) {};
					\node (mid-right) at (13, 8) {};
					
					\node (bot-left) at (5, 6) {};
					\node[label=right:\Large $ \frac{cn^d}{b^{2d}} $] (bot-right) at (9, 6) {};
				\end{scope}
				
				\begin{scope}[>={Stealth[black]}, every edge/.style={draw=black,very thick}]
					\path [->] (top) edge (mid-left);
					\path [->] (top) edge (mid-right);
					
					\path [->] (mid-left) edge (bot-left);
					\path [->] (mid-left) edge (bot-right);
					
					\draw[dashed] (mid-left) -- node[above] {$a$} (mid-right);
					\draw[dashed] (bot-left) -- node[above] {$a$} (bot-right);
				\end{scope}
			\end{tikzpicture}
		\end{center}
		На каждом уровне мы делим задачу на a подзадач, размер каждой отличается от другого уровня в b раз.  
		\begin{comment}
					на i уровне, n/b^i -- размер подзадачи
					всего на каждом уровне a^i подзадач.
					на каждом уровне a_i * (n/b^i)^d * c
					отсюда основная теорема.
		\end{comment}

	\end{proof}
	%Эта теорема может раскрыть не все рекуррентные соотношения, поэтому покажем еще несколько способов решения.
	
	\subsection{Binary Search}
	
	Теперь давайте разберем другой рекуррентный алгоритм - \textit{бинарный поиск}. Это алгоритм ищет в отсортированном элементе некоторый элемент и возвращает его индекс.
		
	\begin{algorithm}[H]
		\caption{Binary Search}
		\begin{algorithmic}[1]
			\Require Отсортированный массив $A$, элемент $x$
			\Ensure Индекс элемента $e$ если он есть в массиве, -1 в противном случае.
			\Function{binary\_search}{$A$, $x$}
				\State $b := 0$
				\State $e := len(A)$
				\While {$b < e$}
					\State $m = round(\frac{b + e}{2})$
					\If {$A[m] == x$}
						\State \Return m
					\ElsIf {$A[m] < x$}
						\State $b := m + 1$
					\Else
						\State $e := m$
					\EndIf
				\EndWhile
				\State \Return -1
			\EndFunction
		\end{algorithmic}
	\end{algorithm}
	
	Словами: смотрим в середину, сравниваем, сдвигаем границы поиска в сторону, где элемент может быть, повторяем пока не нашли или пока $b != e$.
	
	
	\begin{time}
		Формула времени нашего алгоритма:
		\[
		T(n) = 
		\begin{cases}
			c, & n = 2 \\
			T(\frac{n}{2}) + c, & n > 2
		\end{cases}
		\]
		По основной теореме 
		\[
		a = 1, \ b = 2, \ d = 0
		\]
		\[
		T(n) = O(\log(n))
		\]
	\end{time}
	
	
	
	\pagebreak
	
\end{document}