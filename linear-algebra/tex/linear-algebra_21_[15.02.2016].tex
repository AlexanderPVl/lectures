% Размер страницы и шрифта
\documentclass[12pt,a4paper]{article}

%% Работа с русским языком
\usepackage{cmap}                   % поиск в PDF
\usepackage{mathtext}               % русские буквы в формулах
\usepackage[T2A]{fontenc}           % кодировка
\usepackage[utf8]{inputenc}         % кодировка исходного текста
\usepackage[english,russian]{babel} % локализация и переносы

%% Изменяем размер полей
\usepackage[top=0.5in, bottom=0.75in, left=0.625in, right=0.625in]{geometry}

%% Различные пакеты для работы с математикой
\usepackage{mathtools}  % Тот же amsmath, только с некоторыми поправками
\usepackage{amssymb}    % Математические символы
\usepackage{amsthm}     % Пакет для написания теорем
\usepackage{amstext}
\usepackage{array}
\usepackage{amsfonts}
\usepackage{icomma}     % "Умная" запятая: $0,2$ --- число, $0, 2$ --- перечисление

%% Графика
\usepackage[pdftex]{graphicx}
\graphicspath{{images/}}

%% Прочие пакеты
\usepackage{listings}               % Пакет для написания кода на каком-то языке программирования
\usepackage{algorithm}              % Пакет для написания алгоритмов
\usepackage[noend]{algpseudocode}   % Подключает псевдокод, отключает end if и иже с ними
\usepackage{indentfirst}            % Начало текста с красной строки
\usepackage[colorlinks=true, urlcolor=blue]{hyperref}   % Ссылки
\usepackage{pgfplots}               % Графики
\pgfplotsset{compat=1.12}
\usepackage{forest}                 % Деревья
\usepackage{titlesec}               % Изменение формата заголовков
\usepackage[normalem]{ulem}         % Для зачёркиваний
\usepackage[autocite=footnote]{biblatex}    % Кавычки для цитат и прочее
\usepackage[makeroom]{cancel}       % И снова зачёркивание (на этот раз косое)

% Изменим формат \section и \subsection:
\titleformat{\section}
	{\vspace{1cm}\centering\LARGE\bfseries} % Стиль заголовка
	{}                                      % префикс
	{0pt}                                   % Расстояние между префиксом и заголовком
	{}                                      % Как отображается префикс
\titleformat{\subsection}                   	% Аналогично для \subsection
	{\Large\bfseries}
	{}
	{0pt}
	{}

% Поправленный вид lstlisting
\lstset { %
    backgroundcolor=\color{black!5}, % set backgroundcolor
    basicstyle=\footnotesize,% basic font setting
}

% Теоремы и утверждения. В комменте указываем номер лекции, в которой это используется.
\newtheorem*{hanoi_recurrent}{Свойство} % Лекция 1
\let\epsilent\varepsilon                % Лекция 8
\DeclareMathOperator{\rk}{rank}         % Лекция 20

% Информация об авторах
\author{Группа лектория ФКН ПМИ 2015-2016 \\
	Никита Попов \\
	Тамерлан Таболов \\
	Лёша Хачиянц}
\title{Лекции по предмету \\
	\textbf{Алгоритмы и структуры данных}}
\date{2016 год}


\begin{document}

\section{Лекция 21 от 15.02.2016}

\subsection*{Инвариантность и обратимость}
Пусть $\phi\colon V \rightarrow V$ --- линейный оператор, и $\mathbb{e}$ --- базис в $V$. 

\begin{Designation}
    $A(\phi,\;\mathbb{e})$ --- матрица линейного оператора $\phi$ в базисе $\mathbb{e}$.
\end{Designation}

Если $\mathbb{e}' = (e_1', \ldots, e_n')$ --- ещё один базис, причём $(e_1', \ldots, e_n') = (e_1, \ldots, e_n)C$, где $C$ --- матрица перехода, $A = A(\phi,\; \mathbb{e})$ и  $A' = A(\phi,\; \mathbb{e}')$.
В прошлый раз мы доказали, что $A' = C^{-1}AC$.

\begin{Consequence}
    Величина $\det A$ не зависит от выбора базиса. Обозначение: $\det\phi$.
\end{Consequence}

\begin{proof}
    Пусть $A'$ --- матрица $\phi$ в другом базисе. Тогда получается, что:
    \begin{gather*}
        \det A' = \det \left(C^{-1}AC\right) = \det C^{-1} \det A \det C = \det A \det C \cfrac{1}{\det C} = \det A.
    \end{gather*}
\end{proof}

Заметим, что $\det A$ --- инвариант самого $\phi$. 

\begin{Def}
    Две матрицы $A', A \in M_n(F)$ называются подобными, если существует такая матрица $C \in M_n(F), \det C \neq 0$, что $A' = C^{-1}AC$.
\end{Def}

\begin{Comment}
    Отношение подобия на $M_n$ является отношением эквивалентности. 
\end{Comment}

\begin{Suggestion}
    Пусть $\phi \in L(V)$. Тогда эти условия эквивалентны:
    \begin{enumerate}
        \item $\Ker\phi = \{0\}$;
        \item $\Im \phi = V$;
        \item $\phi$ обратим (то есть это биекция, изоморфизм);
        \item $\det \phi \neq 0$.
    \end{enumerate}
\end{Suggestion}

\begin{proof}\ 
    \begin{enumerate}
        \item $\Leftrightarrow$ 2 --- следует из формулы $\dim V = \dim \Ker \phi + \dim \Im \phi$.
        \item $\Leftrightarrow$ 3 --- уже было.
        \item $\Leftrightarrow$ 4 --- уже было.
    \end{enumerate}
\end{proof}

\begin{Def}
    Линейный оператор $\phi$ называется вырожденным, если $\det \phi = 0$, и невырожденным, если $\det \phi \neq 0$.
\end{Def}

\begin{Def} 
    Подпространство $U \subseteq V$ называется инвариантным относительно $\phi$ (или $\phi$-инвариантным), если $\phi(U)\subseteq U$. То есть $\forall u\in U \colon \phi(u)\in U$. 
\end{Def}

\begin{Examples}\
    \begin{enumerate}
        \item $\{0\}, V$ --- они инвариантны для любого $\phi$.
        \item $\Ker\phi$ $\phi$-инвариантно, $\phi(\Ker\phi) = \{0\} \subset \Ker\phi$
        \item $\Im\phi$ тоже $\phi$-инвариантно, $\phi(\Im\phi)\subset \phi(V) = \Im \phi$.
    \end{enumerate}
\end{Examples}

Пусть $U\subset V$ --- $\phi$-инвариантное подпространство. Также пусть $(e_1, \ldots, e_k)$ --- базис в $U$. Дополним его до базиса $V\colon$ $\mathbb{e} = (e_1, \ldots, e_n)$. 
\begin{gather}
    \underbrace{A(\phi,\;\mathbb{e})}_{\text{Матрица с углом нулей}} = \begin{pmatrix}
    B& C \\
    0& D
    \end{pmatrix}, \quad\text{где $B\in M_k$}
\end{gather}
Это нетрудно понять, если учесть, что $\phi(e_i)\in \langle e_1, \ldots, e_k\rangle,\ i=1,\dots, k$.
Если $U = \Ker \phi$, то $B = 0$. Если $U = \Im \phi$, то $D = 0$. 

Обратно, если матрица $A$ имеет в базисе $\mathbb{e}$ такой вид, то $U = \langle e_1, \ldots e_k\rangle$ --- инвариантное подпространство. 

\begin{Generalization}
    Пусть $V = U \oplus W$, где $U,\ W$ --- инвариантные подпространства, и $(e_1, \ldots, e_k)$ --- базис $W$. Тогда $\mathbb{e} = (e_1, \dots, e_n)$ --- базис $V$.
    \[  
        A(\phi,\; \mathbb{e}) = \begin{pmatrix}
            *& 0 \\
            0& *
        \end{pmatrix}
    \]
\end{Generalization}

\begin{Generalization}
  \[
  A(\phi, \mathbb{e}) = 
  \begin{array}{cc}
  \arraycolsep=1.4pt
  \begin{pmatrix}
  * &0 &0 &\ldots &0\\
  0 &* &0 &\ldots &0 \\
  0 &0 &* &\ldots &0 \\
  \vdots &\vdots &\vdots &\ddots &\vdots\\
  0 &0 &0 &\ldots &*
  \end{pmatrix}
  \begin{matrix}
  k_1 \\ k_2 \\ k_3 \\ \vdots \\ k_s
  \end{matrix}
  \end{array}\]
  Здесь $k_1, \ldots, k_s$ --- размеры квадратных блоков блочно-диагональной матрицы. Матрица $A(\phi,\; \mathbb{e})$ имеет такой вид тогда и только тогда, когда:
  \begin{gather*}
      U_1 = \langle e_1, \ldots, e_{k_1}\rangle \\
      U_2 = \langle e_{k_1+1}, \ldots, e_{k_2} \rangle \\
      \vdots\\
      U_{k_s} = \langle e_{n-k_s+1}, \ldots, e_n \rangle
  \end{gather*}
\end{Generalization}

\begin{Thedream}
    Найти такой базис, в котором матрица линейного оператора была бы диагональной. Но такое возможно не всегда.
\end{Thedream}

\subsection*{Собственные векторы и собственные значения}

Пусть $\phi\in L(V)$.

\begin{Def}
    Ненулевой вектор $v\in V$ называется \textit{собственным} для $V$, если $\phi(v) = \lambda v$ для некоторго $\lambda \in F$. При этом число $\lambda$ называется собственным значением линейного оператора $\phi$, отвечающим собственному вектору $v$.
\end{Def}

\begin{Suggestion}
    Вектор $v \in V,\ v \neq 0$ --- собственный вектор в $V$ тогда и только тогда, когда линейная оболочка $\langle v \rangle$ является $\phi$-инвариантным подпространством
\end{Suggestion}

\begin{proof}\
\begin{itemize}
\item $[\Rightarrow]$ $\phi(v) = \lambda v \Rightarrow \langle v \rangle = \{kv\ |\ k\in F\}$. Тогда $\phi(kv) = \lambda k v \in \langle v \rangle.$

\item $[\Leftarrow]$ $\phi(V) \in \langle v \rangle \Rightarrow \exists \lambda \in F\colon \phi(v) = \lambda v$.
\end{itemize}
\end{proof}

\begin{Examples}
    \begin{enumerate}
        \item $V = \mathbb{R}^2$, $\phi$ --- ортогональная проекция на прямуую $l$. \\
        \definecolor{uuuuuu}{rgb}{0.26666666666666666,0.26666666666666666,0.26666666666666666}
\definecolor{ffqqqq}{rgb}{1.,0.,0.}
\definecolor{qqqqff}{rgb}{0.,0.,1.}
\definecolor{xdxdff}{rgb}{0.49019607843137253,0.49019607843137253,1.}
\begin{tikzpicture}
[line cap=round,line join=round,x=1.0cm,y=1.0cm]
\draw[->,color=black] (-6.325793656583929,0.) -- (8.562522298306716,0.);
\foreach \x in {-6.,-5.,-4.,-3.,-2.,-1.,1.,2.,3.,4.,5.,6.,7.,8.}
\draw[shift={(\x,0)},color=black] (0pt,2pt) -- (0pt,-2pt) node[below] {\footnotesize $\x$};
\draw[->,color=black] (0.,-4.268699038293166) -- (0.,4.869752361413149);
\foreach \y in {-4.,-3.,-2.,-1.,1.,2.,3.,4.}
\draw[shift={(0,\y)},color=black] (2pt,0pt) -- (-2pt,0pt) node[left] {\footnotesize $\y$};
\draw[color=black] (0pt,-10pt) node[right] {\footnotesize $0$};
\clip(-6.325793656583929,-4.268699038293166) rectangle (8.562522298306716,4.869752361413149);
\draw [domain=-6.325793656583929:8.562522298306716] plot(\x,{(-0.--2.3*\x)/2.22});
\draw [domain=-6.325793656583929:8.562522298306716] plot(\x,{(-0.--2.22*\x)/-2.3});
\draw [->,color=ffqqqq] (0.,0.) -- (1.62,3.18);
\draw (1.62,3.18)-- (2.3703405621232294,2.4557582400375795);
\draw [->,color=ffqqqq] (0.,0.) -- (2.3703405621232294,2.4557582400375795);
\begin{scriptsize}
\draw [fill=xdxdff] (0.,0.) circle (2.5pt);
\draw[color=black] (4.330621985691042,4.747088584235883) node {$l$};
\draw [fill=qqqqff] (1.62,3.18) circle (2.5pt);
\draw [fill=uuuuuu] (2.3703405621232294,2.4557582400375795) circle (1.5pt);
\end{scriptsize}
\end{tikzpicture}\\
        $0\neq v \in l \Rightarrow \phi(v) = 1\cdot v,\ \lambda =1$ \\
        $0 \neq v \perp l \Rightarrow \phi(v) = 0 = 0 \cdot v,\ \lambda = 1$
        \item Поворот на угол $\phi$ вокруг нуля на угол $\alpha$. 
        \begin{itemize}
            \item $\alpha = 0 + 2\pi k$. Любой ненулевой вектор собственный. $\lambda = 1$.
            \item $\alpha = \pi + 2\pi k$. Любой ненулевой вектор собственный. $\lambda = -1$.
            \item $\alpha \neq \pi k$. Собственных векторов нет.
        \end{itemize}
        \item $V = P_n(F)$ --- многочлены степени  $n$, $\phi = \Delta\colon f \rightarrow f'$. Тогда $0 \neq f$ --- собственный вектор тогда, и только тогда, когда $f = const$.
    \end{enumerate}
\end{Examples}

\subsection*{Диагонализуемость}
\begin{Def}
    Линейный оператор $\phi$ называется диагонализуемым, если существует базис $\mathbb{e}$ в $V$ такой, что $A(\phi, \mathbb{e})$ диагональна. 
\end{Def}
\begin{Suggestion}[Критерий диагонализуемости]
    Отображение $\phi$ диагонализуемо тогда и только тогда, когда в $V$ существует базис из собственных векторов.
\end{Suggestion}

\begin{proof}
    Пусть $\mathbb{e}$ --- базис $V$. Тогда $A = \diag(\lambda_1, \ldots, \lambda_n)$, что равносильно $\phi(e_i) = \lambda_i e_i$. Это и означает, что все векторы собственные.
\end{proof}
В примерах выше:
\begin{enumerate}
    \item $\phi$ диагонализуем. $e_1 \in l,\ e_2 \perp l$. Тогда матрица примет вид $A = \begin{pmatrix}
        1 &0 \\
        0 &0
    \end{pmatrix}$.
    \item Если $\alpha = \pi k$, то $\phi$ диагонализуем ($\phi = \id$ или $\phi = - \id$). Не диагонализуем в других случаях.
    \item $\phi$ диагонализуем тогда и только тогда, когда $n = 0$. При $n > 0$ собственных векторов \textbf{МАЛО}.
\end{enumerate}

\subsection*{Собственное подпространство}
Пусть $\phi\in L(V)$, $\lambda\in F$. 

\begin{Def}
    Множество $V_{\lambda}(\phi) = \{v\in V\ |\ \phi(v) = \lambda v\}$ называется собственным подпространством линейного оператора, отвечающим собственному значению $\lambda$.
\end{Def}

\begin{Task}
Доказать, что $V_\lambda(\phi)$ --- действительно подпространство.
\end{Task}

\begin{Suggestion}
$V_\lambda(\phi) = \Ker(\phi - \lambda\id)$.
\end{Suggestion}
\begin{proof}
    $$
    v \in V_{\lambda}(\phi) \Leftrightarrow \phi(v) = \lambda v \Leftrightarrow \phi(v) - \lambda v = 0 \Leftrightarrow (\phi - \lambda \id)(v) = 0  \Leftrightarrow v \in \Ker(\phi - \lambda \id)
    $$
\end{proof}
\begin{Consequence}
    Собственное подпространство $V_{\lambda}(\phi) \neq \{0\}$ тогда и только тогда, когда \\$\det(\phi - \lambda \id) = 0$.
\end{Consequence}
\begin{Def}
    Многочлен $\chi_{\phi}(t) = (-1)^n\det(\phi - t \id)$ называется характеристическим.
\end{Def}
\end{document}
