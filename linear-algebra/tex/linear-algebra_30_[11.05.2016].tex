% Размер страницы и шрифта
\documentclass[12pt,a4paper]{article}

%% Работа с русским языком
\usepackage{cmap}                   % поиск в PDF
\usepackage{mathtext}               % русские буквы в формулах
\usepackage[T2A]{fontenc}           % кодировка
\usepackage[utf8]{inputenc}         % кодировка исходного текста
\usepackage[english,russian]{babel} % локализация и переносы

%% Изменяем размер полей
\usepackage[top=0.5in, bottom=0.75in, left=0.625in, right=0.625in]{geometry}

%% Различные пакеты для работы с математикой
\usepackage{mathtools}  % Тот же amsmath, только с некоторыми поправками
\usepackage{amssymb}    % Математические символы
\usepackage{amsthm}     % Пакет для написания теорем
\usepackage{amstext}
\usepackage{array}
\usepackage{amsfonts}
\usepackage{icomma}     % "Умная" запятая: $0,2$ --- число, $0, 2$ --- перечисление

%% Графика
\usepackage[pdftex]{graphicx}
\graphicspath{{images/}}

%% Прочие пакеты
\usepackage{listings}               % Пакет для написания кода на каком-то языке программирования
\usepackage{algorithm}              % Пакет для написания алгоритмов
\usepackage[noend]{algpseudocode}   % Подключает псевдокод, отключает end if и иже с ними
\usepackage{indentfirst}            % Начало текста с красной строки
\usepackage[colorlinks=true, urlcolor=blue]{hyperref}   % Ссылки
\usepackage{pgfplots}               % Графики
\pgfplotsset{compat=1.12}
\usepackage{forest}                 % Деревья
\usepackage{titlesec}               % Изменение формата заголовков
\usepackage[normalem]{ulem}         % Для зачёркиваний
\usepackage[autocite=footnote]{biblatex}    % Кавычки для цитат и прочее
\usepackage[makeroom]{cancel}       % И снова зачёркивание (на этот раз косое)

% Изменим формат \section и \subsection:
\titleformat{\section}
	{\vspace{1cm}\centering\LARGE\bfseries} % Стиль заголовка
	{}                                      % префикс
	{0pt}                                   % Расстояние между префиксом и заголовком
	{}                                      % Как отображается префикс
\titleformat{\subsection}                   	% Аналогично для \subsection
	{\Large\bfseries}
	{}
	{0pt}
	{}

% Поправленный вид lstlisting
\lstset { %
    backgroundcolor=\color{black!5}, % set backgroundcolor
    basicstyle=\footnotesize,% basic font setting
}

% Теоремы и утверждения. В комменте указываем номер лекции, в которой это используется.
\newtheorem*{hanoi_recurrent}{Свойство} % Лекция 1
\let\epsilent\varepsilon                % Лекция 8
\DeclareMathOperator{\rk}{rank}         % Лекция 20

% Информация об авторах
\author{Группа лектория ФКН ПМИ 2015-2016 \\
	Никита Попов \\
	Тамерлан Таболов \\
	Лёша Хачиянц}
\title{Лекции по предмету \\
	\textbf{Алгоритмы и структуры данных}}
\date{2016 год}


\begin{document}
\section{Лекция 30 от 11.05.2016}

\subsection*{$N$-мерные параллелепипеды в евклидовых пространствах}

Пусть $\E$ --- евклидово пространство. Вспомним, что такое расстояния в нем.

Для векторов $x,\ y \in E$ расстояние это $\rho(x, y):= |x - y|$.

Для подмножеств $P, Q \subseteq \E$ расстояние это $\rho(P, Q) := \inf\limits_{x \in P,\ y \in Q} \rho(x, y)$.

Для подпространства $U \subseteq \E$ и вектора $x \in \E$ известны следующие вещи:
\begin{enumerate}
\item $\rho(x, U) = |\ort_Ux|$
\item $\rho(x, U)^2 = \frac{\det G(e_1, \ldots, e_k, x)}{\det G(e_1, \ldots, e_k)}$, где $e_1, \ldots, e_k$ --- базис в $U$.
\end{enumerate}

Рассмотрим теперь векторы $a_1, \ldots, a_n \in \E$, причем $n$ --- необязательно размерность $\E$.
\begin{Def}
$N$-мерным параллелепипедом, натянутым на векторы $a_1, \ldots, a_n$ называется подмножество 
$$
P(a_1, \ldots, a_n) := \left\{ x = \sum_{i = 1}^n x_ia_i \mid 0 \leqslant x_i \leqslant 1 \right\}.
$$  
\end{Def}

\begin{Examples}\
\begin{enumerate}
\item При $n = 2$ это обычный двухмерный параллелограмм.
\item При $n = 3$ это трехмерный параллелепипед.
\end{enumerate}
\end{Examples}

\begin{Def}
Для параллелепипеда $P(a_1, \ldots, a_{n})$ основание это $P(a_1, \ldots, a_{n-1})$, а высота --- $h = \ort_{\langle a_1, \ldots, a_{n-1}\rangle}a_n$.
\end{Def}

\begin{Def}
Объем $n$-мерного параллелепипеда $P(a_1, \ldots, a_{n})$ --- это число $\vol P(a_1, \ldots, a_{n})$, определяемое рекурсивно следующим образом:
\begin{align*}
n = 1 \quad& \vol P(a_1) = |a_1| \\
n > 1 \quad& \vol P(a_1, \ldots, a_n) = \vol P(a_1, \ldots, a_{n-1})\cdot |h|
\end{align*}
\end{Def}

\begin{Theorem}
$\vol P\vector{a}^2 = \det G\vector{a}$.
\end{Theorem}

\begin{proof}
Докажем это утверждение по индукции.

База: при $n = 1$ имеем $\vol P(a_1)^2 = |a_1|^2 = (a_1, a_1) = \det G(a_1)$.

Теперь пусть утверждение доказано для всех меньших значений. Докажем для $n$.
\begin{gather*}
\vol P\vector{a}^2 = \vol P(a_1, \ldots, a_{n-1})^2 \cdot |h|^2 = 
\det G(a_1, \ldots, a_{n-1}) \cdot |\ort_{\langle a_1, \ldots, a_{n-1} \rangle}a_n|^2 = \\
= \begin{cases}
	0 = \det G\vector{a}, & \text{если $a_1, \ldots a_{n-1}$ линейно зависимы} \\
	\det G(a_1, \ldots, a_n)\frac{\det G(a_1, \ldots, a_n)}{\det G(a_1, \ldots, a_{n-1})} = \det G\vector{a}, & \text{если $a_1, \ldots, a_{n-1}$ линейно независимы}
  \end{cases}
\end{gather*}
\end{proof}

\begin{Consequence}
Объем параллелепипеда не зависит от выбора основания.
\end{Consequence}

\begin{Theorem}
Пусть $\vector{e}$ --- ортогональный базис в $\E$, и $\vector{a} = \vector{e}A$ для некоторой матрицы $A \in M_n(\R)$. Тогда $\vol P\vector{a} = |\det A|$.
\end{Theorem}

\begin{Comment}
Это --- геометрический смысл определителя!
\end{Comment}

\begin{proof}
Вспомним, что матрица Грама ортогонального базиса равна единичной матрице:
$$
G\vector{a} =A^TG\vector{e}A = A^TA.
$$
Тогда для определителя справедливо следующее:
$$
\det G\vector{a} = \det(A^TA) = (\det A)^2.
$$
Осталось воспользоваться предыдущей теоремой:
$$
\vol P\vector{a} = \sqrt{\det G \vector{a}} = |\det A|.
$$
\end{proof}

\subsection*{Изоморфизм евклидовых пространств}

Теперь рассмотрим два евклидовых пространства, $\E$ и $\E'$.
\begin{Def}
Изоморфизм евклидовых пространств $\E$ и $\E'$  --- это биективное отображение $\phi: \E \rightarrow \E'$ такое, что
\begin{enumerate}
\item $\phi$ --- изоморфизм векторных пространств.
\item $(\phi(x), \phi(y))' = (x, y)$ для любых $x \in \E$ и $y \in \E'$. Здесь через $()'$ обозначается скалярное произведение в $\E'$.
\end{enumerate}
\end{Def}

\begin{Def}
Евклидовы пространства $\E$ и $\E'$ называются изоморфными, если между ними существует изоморфизм. Обозначение: $\E \simeq \E'$.
\end{Def}

\begin{Theorem}
Два конечномерных евклидовых пространства $\E$ и $\E'$ изоморфны тогда и только тогда, когда их размерности совпадают.
\end{Theorem}

\begin{proof}\ \\
$[\Rightarrow]$ Очевидно из первого пункта определения изоморфизма евклидовых пространств.

$[\Leftarrow]$ Зафиксируем ортонормированные базисы $\e = \vector{e}$ в $\E$ и $\e' = \vector{e'}$ в $\E'$, где $n = \dim \E = \dim \E'$. Зададим изоморфизм $\phi: \E \rightarrow \E'$ \textit{векторных пространств} по формуле
$$
\phi(e_i) = e_i'\quad \forall i = 1,\ldots, n.
$$
Тогда имеем следующее (напомним, что $\delta_{ij}$ это символ Кронекера):
$$
(\phi(e_i), \phi(e_j))' = (e_i', e_j')' = \delta_{ij} = (e_i, e_j).
$$
Теперь рассмотрим векторы $x = \sum_{_i = 1}^{n}x_ie_i$ и $y = \sum_{j = 1}^{n}y_je_j$ и проверим второй пункт определения изоморфизма евклидовых пространств. Будем пользоваться билинейностью скалярного произведения.
\begin{gather*}
(\phi(x), \phi(y))' = \left(\phi\left(\sum_{i = 1}^{n}x_ie_i\right), \phi\left(\sum_{j = 1}^{j = n}y_je_j\right)\right)' = \left(\sum_{i = 1}^{n}x_i\phi(e_i), \sum_{j = 1}^{n}y_j\phi(e_j)\right)' = \\
= \sum_{i = 1}^{n}\sum_{j = 1}^{n}x_iy_j\phi(e_i, e_j)' = \sum_{i = 1}^{n}\sum_{j = 1}^{n}x_iy_j(e_i, e_j) = (x, y)
\end{gather*}
\end{proof}

\subsection*{Линейные операторы в евклидовых пространствах}

Пусть $\E$ --- евклидово пространство, $\phi$ --- его линейный оператор. Тогда ему можно сопоставить две билинейные функции на $\E$:
\begin{gather*}
\beta_\phi(x, y) = (x, \phi(y)) \\
\beta^T_\phi(x, y) = (\phi(x), y)
\end{gather*}
Введем базис $\e = \vector{e}$ в $\E$, матрицу Грама $G = G\vector{e}$, матрицу оператора $A_\phi \hm= A(\phi, \e)$, а также два вектора $x = \sum_{i = 1}^{n}x_ie_i$ и $y = \sum_{j = 1}^{n}y_je_j$. Тогда имеем следующее:
\begin{gather*}
\phi(x) = A_\phi\vvector{x} \qquad \phi(y) = A_\phi \vvector{y} \\ 
\beta_\phi(x, y) = \vector{x} GA_\phi \vvector{y} \qquad
\beta^T_\phi(x, y) = \vector{x}A_\phi^TG\vvector{y}
\end{gather*}
Отсюда мы можем вывести матрицы данных билинейных форм:
\begin{gather*}
B(\beta_\phi, \e) = GA_\phi \\
B(\beta^T_\phi, \e) = A^T_\phi G
\end{gather*}

\begin{Comment}
Отображения $\phi \mapsto \beta_\phi$ и $\phi \mapsto \beta^T_\phi$ являются биекциями между $L(\E)$ и пространством всех билинейных форм на $\E$.
\end{Comment}

\begin{Def}
Линейный оператор $\psi \in L(\E)$ называется сопряженным к $\phi$, если для всех векторов $x,\ y \in \E$ верно, что $(\psi(x), y)\hm = (x, \phi(y))$. Это также равносильно тому, что $\beta_\psi^T = \beta_\phi$. Обозначение: $\psi = \phi^*$.
\end{Def}

\begin{Suggestion}\ 
\begin{enumerate}
\item $\phi^*$ существует и единственен.
\item $A_{\phi^*} = G^{-1}A_\phi^TG$, где $A_{\phi^*} = A(\phi^*, \e)$, а все остальные обозначения прежние. В частности, если $\e$ --- ортонормированный базис, то $A_{\phi^*} = A_{\phi}^T$.
\end{enumerate}
\end{Suggestion}

\begin{proof}
Снова обозначим $\phi^*$ как $\psi$. Мы уже знаем, что $B(\beta_\psi^T, \e) = A_\psi^TG$ и $B(\beta_\phi, \e) \hm= GA_\phi$. Мы хотим, чтобы эти две матрицы были равны. Транспонируем их и, воспользовавшись тем, что $G = G^T$, получаем:
$$
GA_\psi = A_\phi^TG.
$$
Выразив $A_\psi$, получаем, что такая матрица (и, соответственно, оператор) единственная: 
$$A_\psi = G^{-1}A_\phi^TG.$$ 
Существование же напрямую следует из того, что линейный оператор с матрицей $G^{-1}A_\phi^TG$ обладает нужными свойствами.
\end{proof}

\begin{Def}
Линейный оператор $\phi$ называется самосопряженным (симметрическим), если $\phi^* = \phi$. Это равносильно тому, что $(\phi(x), y) = (x, \phi(y)))$ для любых векторов $x,\ y \in \E$.
\end{Def}

\begin{Comment}
В случае, когда $\e$ --- ортонормированный базис в $\E$ и $A_\phi = A(\phi, \e)$, то самосопряженность линейного оператора $\phi$ равносильно тому, что $A_\phi = A_\phi^T$. Отсюда и второе название таких операторов --- симметрические. 
\end{Comment}

Здесь важно, что мы работаем  именно над евклидовым пространством, так как мы использовали скалярное произведение для проведения биекции с билинейными формами.

\begin{Examples}
Пусть $U \subseteq \E$ --- подпространство. Отображение $\phi : x \mapsto \pr_Ux$ является самосопряженным. 

\begin{proof}\ \\
\textbf{I способ (координатный).}

Пусть $(e_1, \ldots, e_k)$ --- ортонормированный базис в $U$, а $(e_{k+1}, \ldots, e_n)$ --- ортонормированный базис в $U^T$. Тогда $\e = (e_1, \ldots, e_n)$ --- ортонормированный базис в $\E$. А значит, матрица $\phi$ будет иметь в таком базисе следующий вид:
$$
A(\phi, \e) = \diag(\underbrace{1, \ldots, 1}_{k},\underbrace{0, \ldots, 0}_{n - k})
$$ 
При транспонировании диагональная матрица не меняется, следовательно, $A(\phi, \e)^T = A(\phi, \e)$. Что и означает, что $\phi = \phi^*$.

\textbf{II способ (бескоординатный).}

Проверим условие $(x, \phi(y)) = (\phi(x), y)$:
\begin{align*}
(\phi(x), y) =& (\pr_Ux, \pr_Uy + \ort_Uy) = (\pr_Ux, \pr_Uy) + \underbrace{(\pr_Ux, \ort_Uy)}_{=0} = \\
=& (\pr_Ux, \pr_Uy) + \underbrace{(\ort_Ux, \pr_Uy)}_{=0} = (\pr_Ux + \ort_Ux, \ort_Uy) = (x, \phi(y)).
\end{align*}
\end{proof}
\end{Examples}






\end{document}