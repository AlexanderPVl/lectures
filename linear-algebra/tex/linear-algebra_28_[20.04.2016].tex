% Размер страницы и шрифта
\documentclass[12pt,a4paper]{article}

%% Работа с русским языком
\usepackage{cmap}                   % поиск в PDF
\usepackage{mathtext}               % русские буквы в формулах
\usepackage[T2A]{fontenc}           % кодировка
\usepackage[utf8]{inputenc}         % кодировка исходного текста
\usepackage[english,russian]{babel} % локализация и переносы

%% Изменяем размер полей
\usepackage[top=0.5in, bottom=0.75in, left=0.625in, right=0.625in]{geometry}

%% Различные пакеты для работы с математикой
\usepackage{mathtools}  % Тот же amsmath, только с некоторыми поправками
\usepackage{amssymb}    % Математические символы
\usepackage{amsthm}     % Пакет для написания теорем
\usepackage{amstext}
\usepackage{array}
\usepackage{amsfonts}
\usepackage{icomma}     % "Умная" запятая: $0,2$ --- число, $0, 2$ --- перечисление

%% Графика
\usepackage[pdftex]{graphicx}
\graphicspath{{images/}}

%% Прочие пакеты
\usepackage{listings}               % Пакет для написания кода на каком-то языке программирования
\usepackage{algorithm}              % Пакет для написания алгоритмов
\usepackage[noend]{algpseudocode}   % Подключает псевдокод, отключает end if и иже с ними
\usepackage{indentfirst}            % Начало текста с красной строки
\usepackage[colorlinks=true, urlcolor=blue]{hyperref}   % Ссылки
\usepackage{pgfplots}               % Графики
\pgfplotsset{compat=1.12}
\usepackage{forest}                 % Деревья
\usepackage{titlesec}               % Изменение формата заголовков
\usepackage[normalem]{ulem}         % Для зачёркиваний
\usepackage[autocite=footnote]{biblatex}    % Кавычки для цитат и прочее
\usepackage[makeroom]{cancel}       % И снова зачёркивание (на этот раз косое)

% Изменим формат \section и \subsection:
\titleformat{\section}
	{\vspace{1cm}\centering\LARGE\bfseries} % Стиль заголовка
	{}                                      % префикс
	{0pt}                                   % Расстояние между префиксом и заголовком
	{}                                      % Как отображается префикс
\titleformat{\subsection}                   	% Аналогично для \subsection
	{\Large\bfseries}
	{}
	{0pt}
	{}

% Поправленный вид lstlisting
\lstset { %
    backgroundcolor=\color{black!5}, % set backgroundcolor
    basicstyle=\footnotesize,% basic font setting
}

% Теоремы и утверждения. В комменте указываем номер лекции, в которой это используется.
\newtheorem*{hanoi_recurrent}{Свойство} % Лекция 1
\let\epsilent\varepsilon                % Лекция 8
\DeclareMathOperator{\rk}{rank}         % Лекция 20

% Информация об авторах
\author{Группа лектория ФКН ПМИ 2015-2016 \\
	Никита Попов \\
	Тамерлан Таболов \\
	Лёша Хачиянц}
\title{Лекции по предмету \\
	\textbf{Алгоритмы и структуры данных}}
\date{2016 год}


\begin{document}
\renewcommand{\f}{\mathbb{f}}

\section{Лекция 28 от 19.04.2016}

\subsection*{Ортогонализация}
Пусть $V$ --- векторное пространство над полем $F$ размерности $n$, и $\mathbb{e} = (e_1, \ldots, e_n)$ --- его базис. Пусть также $Q\colon \; V \to F$ --- квадратичная форма, $\beta\colon\; V\times V \to F$ --- соответствующая билинейная функция и $B = B(\beta, \mathbb{e})$ --- ее матрица.
$$
B = \begin{pmatrix}
	b_{11}& b_{12} & b_{13} & \vdots\\
	b_{21}& b_{22} & b_{23} & \vdots\\
	b_{31}& b_{32}& b_{33} & \vdots\\
	\cdots& \cdots& \cdots& \ddots
\end{pmatrix}
$$

Рассмотрим $B_i$ --- левые верхние $i\times i$-подматрицы. Например, $B_1 = (b_{11})$, $B_2 = \begin{pmatrix}
	b_{11}& b_{12}\\
	b_{21}& b_{22}
\end{pmatrix}$
и так далее.
\par Матрица $B_i$ --- это матрица ограничения билинейной функции $\beta$ на подпространство, натянутое на векторы $(e_1, \ldots, e_i)$. Назовем \textit{верхним угловым минором} число $\delta_i = \det(B_i)$. Также будем считать, что $\delta_0 = 1$. 
\begin{Def}
	Базис $\mathbb{e}$ называется ортогональным (по отношению к $\beta$), если $\beta(e_i, e_j) = 0$ для любых $i \neq j$. В ортогональном базисе матрица квадратичной формы имеет канонический вид.
\end{Def}
\begin{Theorem}[Метод ортогонализации Грама -- Шмидта]
		Предположим, что $\delta_i \neq 0$ для всех $i$. Тогда существует единственный  базис $\mathbb{e}' = (e_1', \ldots, e_n')$ в $V$ такой, что
		\begin{enumerate}
			\item $\mathbb{e}'$ --- ортогональный
			\item $e_1' = e_1,\\ e_2' \in e_2 + \langle e_1'\rangle,\\ e_3' \in  e_3 + \langle e_1', e_2' \rangle,\\ \ldots\\ e_n' \in  e_n + \langle e_1', \ldots, e_{n-1}'\rangle$
			\item $Q(e_i') = \cfrac{\delta_i}{\delta_{i-1}}$ для всех $i$.
		\end{enumerate}	
\end{Theorem}
\begin{proof}
	Индукция по $n$. База для $n = 1$ очевидна. 
	\par Теперь пусть всё доказано для всех $k<n$. Докажем для $n$. По предположению индукции, существует единственный базис $(e_1', e_2', \ldots, e_n')$ с требуемыми свойствами.
	
	Наблюдение: $\langle e_i, \ldots, e_n\rangle = \langle e_i', \ldots, e_n'\rangle$.
	
	Ищем $e_n'$ в виде $e_n' = e_n + \lambda_1 e_1' + \ldots + \lambda_{n-1}e_{n-1}'$. Тогда для всех $i$:
	\[
 \beta(e_n', e_i') = \beta(e_n, e_i') + \sum\limits^{n-1}_{j = 1} \lambda_j\beta (e_j', e_i')
	\]
	Чтобы выполнялись требуемые условия, необходимо, чтобы эта сумма равнялась нулю.
	
	Заметим,что последнее слагаемое обращается в нуль при $i \neq j$ по свойству выбранного базиса. Тогда остается только следующее:
	\[
		0 = \beta(e_n, e_i') + \lambda_i \beta(e_i', e_i')= \beta(e_n, e_i') +\l_i Q(e_i') = \beta(e_n, e_i') + \l_i\underbrace{\cfrac{\delta_i}{\delta_{i - 1}}}_{\neq 0}.
	\]
	Выбирая $\lambda_i = -\cfrac{\beta(e_n, e_i')}{\beta(e_i', e_i')}$, получаем нужное равенство и  однозначность разложения. Таким образом, условия 1 и 2 выполнены.
	
	Проверим условие 3. Пусть $C$ --- матрица перехода от $\mathbb{e}$ к $\mathbb{e}'$. Тогда легко понять, что $C$ --- верхнетреугольная с единицами на главной диагонали. Значит, матрица $B' = C^{T}BC$ тоже диагональна. Заметим также, что $C_i$ (та самая верхняя $i\times i$-подматрица) является матрицей перехода от $(e_1, \ldots, e_i)$ к $(e_1', \ldots, e_i')$. Тогда:
	\[
		B_i' = C_i^TB_iC_i \Rightarrow \det B_i' = 1\cdot \det(B_i) \cdot 1 = \delta_i.
	\]
	Но поскольку $B' = \begin{pmatrix}
		Q(e_1')& & 0 \\
		 & \ddots&  \\
		 0 && Q(e_n')
	\end{pmatrix}$, то $\delta_n = Q(e_1')\cdot\ldots\cdot Q(e_n')$. Отсюда и получаем, что $\cfrac{\delta_n}{\delta_{n-1}} = Q(e_n')$.
\end{proof}

\begin{Examples}
Пусть $V = \R^2$. Тогда $e_1' = e_1$, а $e_2'$ получается, если спроецировать вектор $e_2$ на прямую, ортогональную $e_1$. Если $V = \R^3$, то $e_3'$ является проекцией на прямую, ортогональную плоскости $(e_1', e_2')$.
\end{Examples}

\subsection*{Теорема Якоби и критерий Сильвестра}

Рассмотрим следствия данной теоремы для случая, когда $F = \mathbb{R}$.
\begin{Theorem}[Якоби]
	Пусть $\delta_i \neq 0$ для всех $i$. Тогда $\rk Q = n$ и $i_{-}(Q)$ равен числу перемен знака последовательности $\delta_0, \delta_1, \ldots, \delta_n$ (напомним, что $\delta_0 = 1$).
\end{Theorem}
\begin{proof}
	Применим процесс ортогонализации. Получим базис $(e_1', \ldots, e_n')$, в котором $ Q(y_1, \ldots, y_n) = \cfrac{\delta_1}{\delta_0}y_1^2 +\ldots + \cfrac{\delta_n}{\delta_{n-1}}y_n^2$, где $y_1, \ldots, y_n$ --- координаты некоторого вектора в данном базисе. Если для некоторого $i$ выполняется, что $\cfrac{\delta_i}{\delta_{i-1}} < 0$, то значит, $\sgn \delta_i \neq \sgn \delta_{i - 1}$. Что и означает, что отрицательный индекс равен количеству перемен знака в последовательности $\delta_0, \delta_1, \ldots, \delta_n$.
	
	Что касательно определителя, то условие  $\rk Q = n$ равносильно условию $\det B \neq 0$. Но $\det B = \delta_n \neq 0$, а значит, все верно.
\end{proof}
\begin{Theorem}[Критерий Сильвестра]
	$Q > 0$ тогда и только тогда, когда $\delta_i > 0$  для всех $i$.
\end{Theorem}
\begin{proof}\ \\
	$[\Leftarrow]$ Следует из предыдущей теоремы. \\
	$[\Rightarrow]$ Докажем, чтo $\delta_i = \det(B_i) > 0$. Действительно, $B_i$ --- это матрица ограничения $Q\bigr|_{\langle e_1, \ldots, e_i\rangle}$. Оно так же будет строго положительным, следовательно,  существует матрица $C_i \in M_n(\mathbb{R}),\\ \det(C_i)\neq 0$, такая, что $C_i^TBC_i = E$. Но тогда
	$\det C^T_i\det B_i \det C_i = \det E = 1$. Следовательно, $\det B_i = \cfrac{1}{(\det C_i)^2} > 0$, что и требовалось.
\end{proof}
\begin{Theorem}
	 $Q < 0 \Leftrightarrow \begin{cases}
		\delta_i < 0, & 2 \nmid i \\
		\delta_i > 0, & 2 \mid i
	\end{cases}$
\end{Theorem}
\begin{proof}
	Применяя критерий Сильвестра для $B(Q, \mathbb{e}) = - B(-Q, \mathbb{e})$, получаем требуемое.
\end{proof}

\subsection*{Евклидовы пространства. Основные понятия}
\begin{Def}
	Евклидово пространство --- это векторное пространство $\mathbb{E}$ над полем $\mathbb{R}$, на котором задана положительно определённая симметрическая билинейная функция $(\cdot, \cdot)$, которую мы будем называть скалярным произведением.
\end{Def}
\begin{Examples}\
\begin{enumerate}
\item	$\mathbb{R}^n,\ x = \begin{pmatrix}x_1\\ \vdots\\ x_n\end{pmatrix}$, $y = \begin{pmatrix}y_1\\ \vdots\\ y_n\end{pmatrix}, 
		(x,y) = \sum\limits_{i = 1}^n x_iy_i$.
\item $\E = C[0, 1]$, $(f, g) = \int_{0}^{1}f(x)g(x)dx$, $(f, f) = \int_{0}^{1}f^2(x)dx > 0$.
\end{enumerate}
\end{Examples}
\begin{Comment}
	Важно отметить, что евклидово пространство можно определить только над полем $\mathbb{R}$.
\end{Comment}

\begin{Def}
	Пусть $x\in \mathbb{E}$. Тогда длиной вектора называют величину $|x| = \sqrt{(x,x)}$.
\end{Def}

Очевидно, что $|x| \geqslant 0$, причем $|x| = 0$ тогда и только тогда, когда $x = 0$.

\begin{Suggestion}[Неравенство Коши-Буняковского]
	Пусть $x, y \in \E$. Тогда $|(x,y)| \leqslant |x||y|$, причём знак равенства возможен тогда и только тогда, когда $x$ и $y$ пропорциональны.
\end{Suggestion}
\begin{proof}\
	\begin{enumerate}
		\item $x,y$ пропорциональны, т.е. $x = \lambda y$ для некоторого $\lambda$. Тогда:
		\[
			|(x,y)| = |(x,\lambda x)| = \lambda|(x,x)| = |x|\lambda|x| = |x||y|.
		\]
		\item $x,y$ линейно независимы. Тогда они будут базисом своей линейной оболочки. Тогда матрица $B$ билинейной функции $(\cdot, \cdot)\bigr|_{\langle x, y\rangle}$ равна:
		\begin{gather*}
		B = \begin{pmatrix}
			(x,x)& (x,y)\\
			(y,x)& (y,y)
		\end{pmatrix}
		\end{gather*}
	Так как $\det B > 0$, то $(x,x)(y,y) - (x,y)^2 > 0$. Следовательно:
	\begin{gather*}
			|(x,y)|^2 < |x|^2|y|^2\\
			|(x,y)| < |x||y|
	\end{gather*}
	\end{enumerate}
\end{proof}

\begin{Def}
	Углом между векторами $x$ и $y$ называют такой $\alpha$, что $ \cos \alpha = \cfrac{(x,y)}{|x||y|}$.
\end{Def}

Рассмотрим систему векторов $(v_1, \ldots, v_k)$, где $v_i \in \E$.
\begin{Def}
Матрица Грама системы $v_1, \ldots, v_k$ это
	$$G(v_1,\ldots, v_k) := (g_{ij}),\quad g_{ij} = (v_i,v_j).$$
\end{Def}

\begin{Suggestion}\
	\begin{enumerate}
		\item $\det G(v_1, \ldots, v_k) \geqslant 0$
		\item $\det G(v_1, \ldots, v_k) = 0$ тогда и только тогда, когда $v_1, \ldots, v_k$ линейно зависимы.
	\end{enumerate}
\end{Suggestion}

\begin{proof}\
	\begin{enumerate}
		\item $v_1, \ldots, v_k$ линейно независимы. Следовательно, матрица $G(v_1, \ldots, v_k)$ является матрицей ограничения $(\cdot, \cdot)$ на $\langle v_1, \ldots, v_k\rangle$, базисом в котором является $(v_1, \ldots, v_k)$. А значит, $\det G(v_1, \ldots, v_k) > 0$.
		\item $v_1, \ldots, v_k$  линейно зависимы. Значит, существуют коэффициенты $(\l_1, \ldots, \l_k) \neq (0,  \ldots, 0)$ такие, что $\l_1v_1 + \ldots + \l_kv_k = 0$. Если обозначить матрицу Грама $G(v_1, \ldots, v_k)$ за $G$, то тогда
		\begin{gather*}
		\l_1 G_{(1)} + \ldots + \l_kG_{(k)} = \\ =
		(\l_1v_1 + \ldots + \l_kv_k, v_1) + (\l_1v_1 + \ldots + \l_kv_k, v_2) + \ldots + (\l_1v_1 + \ldots + \l_kv_k, v_k) =\\= 0 + 0 + \ldots + 0
		\end{gather*}
		То есть строки линейно зависимы и $\det G = 0$.
	\end{enumerate}
\end{proof}
\end{document}