\documentclass[a4paper,10pt]{amsart}

\usepackage[T2A]{fontenc}
\usepackage[utf8x]{inputenc}
\usepackage{amssymb}
\usepackage[russian]{babel}
\usepackage{geometry}
\usepackage{hyperref}
\usepackage{enumitem}


\geometry{a4paper,top=2cm,bottom=2cm,left=2cm,right=2cm}

\setlength{\parindent}{0pt}
\setlength{\parskip}{\medskipamount}

\newcommand{\Ker}{\mathop{\mathrm{Ker}}}
\renewcommand{\Im}{\mathop{\mathrm{Im}}}
\DeclareMathOperator{\Tor}{\mathrm{Tor}}
\newcommand{\xar}{\mathop{\mathrm{char}}}

%\def\Ker{{\rm Ker}}%
%\def\Im{{\rm Im}}%
\def\Mat{{\rm Mat}}%
\def\cont{{\rm cont}}%
%\def\Tor{{\rm Tor}}%
\def\Char{{\rm Char}}%
\def\signum{{\rm sig}}%
\def\Sym{{\rm Sym}}%
\def\St{{\rm St}}%
\def\Aut{{\rm Aut}}%
\def\Chi{{\mathbb X}}%
\def\Tau{{\rm T}}%
\def\Rho{{\rm R}}%
\def\rk{{\rm rk}}%
\def\ggT{{\rm ggT}}%
\def\kgV{{\rm kgV}}%
\def\Div{{\rm Div}}%
\def\div{{\rm div}}%
\def\quot{/\!\!/}%
\def\mal{\! \cdot \!}%
\def\Of{{\mathcal{O}}}
%
\def\subgrpneq{\le}%
\def\subgrp{\le}%
\def\ideal#1{\le_{#1}}%
\def\submod#1{\le_{#1}}%
%
\def\Bild{{\rm Bild}}%
\def\Kern{{\rm Kern}}%
\def\bangle#1{{\langle #1 \rangle}}%
\def\rq#1{\widehat{#1}}%
\def\t#1{\widetilde{#1}}%
\def\b#1{\overline{#1}}%
%
\def\abs#1{{\vert #1 \vert}}%
\def\norm#1#2{{\Vert #1 \Vert}_{#2}}%
\def\PS#1#2{{\sum_{\nu=0}^{\infty} #1_{\nu} #2^{\nu}}}%
%
\def\C{{\rm C}}%
\def\O{{\rm O}}%
\def\HH{{\mathbb H}}%
\def\LL{{\mathbb L}}%
\def\FF{{\mathbb F}}%
\def\CC{{\mathbb C}}%
\def\KK{{\mathbb K}}%
\def\TT{{\mathbb T}}%
\def\ZZ{{\mathbb Z}}%
\def\RR{{\mathbb R}}%
\def\SS{{\mathbb S}}%
\def\NN{{\mathbb N}}%
\def\QQ{{\mathbb Q}}%
\def\PP{{\mathbb P}}%
\def\AA{{\mathbb A}}%
%
\def\eins{{\mathbf 1}}%
%
\def\AG{{\rm AG}}%
\def\Aut{{\rm Aut}}%
\def\Hol{{\rm Hol}}%
\def\GL{{\rm GL}}%
\def\SL{{\rm SL}}%
\def\SO{{\rm SO}}%
\def\Sp{{\rm Sp}}%
\def\gl{\mathfrak{gl}}%
\def\rg{{\rm rg}}%
\def\sl{\mathfrak{sl}}%
\def\HDiv{{\rm HDiv}}%
\def\CDiv{{\rm CDiv}}%
\def\Res{{\rm Res}}%
\def\Pst{{\rm Pst}}%
\def\Nst{{\rm Nst}}%
\def\rad{{\rm rad}}%
\def\GL{{\rm GL}}%
\def\Tr{{\rm Tr}}%
\def\Pic{{\rm Pic}}%
\def\Hom{{\rm Hom}}%
\def\hom{{\rm hom}}%
\def\Mor{{\rm Mor}}%
\def\codim{{\rm codim}}%
\def\Supp{{\rm Supp}}%
\def\Spec{{\rm Spec}}%
\def\Proj{{\rm Proj}}%
\def\Maps{{\rm Maps}}%
\def\cone{{\rm cone}}%
\def\ord{{\rm ord}}%
\def\pr{{\rm pr}}%
\def\id{{\rm id}}%
\def\mult{{\rm mult}}%
\def\inv{{\rm inv}}%
\def\neut{{\rm neut}}%
%
\def\AAA{\mathcal{A}}
\def\BBB{\mathcal{B}}
\def\CCC{\mathcal{C}}
\def\EEE{\mathcal{E}}
\def\FFF{\mathcal{F}}

\def\CF{{\rm CF}}
\def\GCD{{\rm GCD}}
\def\Mat{{\rm Mat}}
\def\End{{\rm End}}
\def\cont{{\rm cont}}
\def\Kegel{{\rm Kegel}}
\def\Char{{\rm Char}}
\def\Der{{\rm Der}}
\def\signum{{\rm sg}}
\def\grad{{\rm grad}}
\def\Spur{{\rm Spur}}
\def\Sym{{\rm Sym}}
\def\Alt{{\rm Alt}}
\def\Abb{{\rm Abb}}
\def\Chi{{\mathbb X}}
\def\Tau{{\rm T}}
\def\Rho{{\rm R}}
\def\ad{{\rm ad}}
\def\Frob{{\rm Frob}}
\def\Rang{{\rm Rang}}
\def\SpRang{{\rm SpRang}}
\def\ZRang{{\rm ZRang}}
\def\ggT{{\rm ggT}}
\def\kgV{{\rm kgV}}
\def\Div{{\rm Div}}
\def\div{{\rm div}}
\def\quot{/\!\!/}
\def\mal{\! \cdot \!}
\def\add{{\rm add}}
\def\mult{{\rm mult}}
\def\smult{{\rm smult}}

\def\subgrpneq{\le}
\def\subgrp{\le}
\def\ideal#1{\unlhd_{#1}}
\def\submod#1{\le_{#1}}

\def\Bild{{\rm Bild}}
\def\Kern{{\rm Kern}}
\def\Kon{{\rm Kon}}
\def\bangle#1{{\langle #1 \rangle}}
\def\rq#1{\widehat{#1}}
\def\t#1{\widetilde{#1}}
\def\b#1{\overline{#1}}

\def\abs#1{{\vert #1 \vert}}
\def\norm#1#2{{\Vert #1 \Vert}_{#2}}
\def\PS#1#2{{\sum_{\nu=0}^{\infty} #1_{\nu} #2^{\nu}}}


\def\eins{{\mathbf 1}}

\def\ElM{{\rm ElM}}
\def\ZOp{{\rm ZOp}}
\def\SpOp{{\rm SpOp}}
\def\Gal{{\rm Gal}}
\def\Def{{\rm Def}}
\def\Fix{{\rm Fix}}
\def\ord{{\rm ord}}
\def\Aut{{\rm Aut}}
\def\Hol{{\rm Hol}}
\def\GL{{\rm GL}}
\def\SL{{\rm SL}}
\def\SO{{\rm SO}}
\def\Sp{{\rm Sp}}
\def\Spann{{\rm Spann}}
\def\Lin{{\rm Lin}}
\def\gl{\mathfrak{gl}}
\def\rg{{\rm rg}}
\def\sl{\mathfrak{sl}}
\def\so{\mathfrak{so}}
\def\sp{\mathfrak{sp}}
\def\gg{\mathfrak{g}}
\def\HDiv{{\rm HDiv}}
\def\CDiv{{\rm CDiv}}
\def\Res{{\rm Res}}
\def\Pst{{\rm Pst}}
\def\Nst{{\rm Nst}}
\def\WDiv{{\rm WDiv}}
\def\GL{{\rm GL}}
\def\Tr{{\rm Tr}}
\def\Pic{{\rm Pic}}
\def\Hom{{\rm Hom}}
\def\hom{{\rm hom}}
\def\Mor{{\rm Mor}}
\def\codim{{\rm codim}}
\def\Supp{{\rm Supp}}
\def\Spec{{\rm Spec}}
\def\Proj{{\rm Proj}}
\def\Maps{{\rm Maps}}
\def\cone{{\rm cone}}
\def\ord{{\rm ord}}
\def\pr{{\rm pr}}
\def\id{{\rm id}}
\def\mult{{\rm mult}}
\def\inv{{\rm inv}}
\def\neut{{\rm neut}}
\def\trdeg{{\rm trdeg}}
\def\sing{{\rm sing}}
\def\reg{{\rm reg}}


%%%%%%%%%%%%%%%%%%%%%%%%%%%

\newtheorem{theorem}{Теорема}
\newtheorem{proposition}{Предложение}
\newtheorem{lemma}{Лемма}
\newtheorem{corollary}{Следствие}
\theoremstyle{definition}
\newtheorem{definition}{Определение}
\newtheorem{problem}{Задача}
%
\theoremstyle{remark}
\newtheorem{exercise}{Упражнение}
\newtheorem{exc}{Упражнение}
\newtheorem{remark}{Замечание}
\newtheorem{example}{Пример}

\renewcommand{\theenumi}{\textup{(\alph{enumi})}}
\renewcommand{\labelenumi}{\theenumi}
\newcounter{property}
\renewcommand{\theproperty}{\textup{(\arabic{property})}}
\newcommand{\property}{\refstepcounter{property}\item}
\newcounter{prooperty}
\renewcommand{\theprooperty}{\textup{(\arabic{prooperty})}}
\newcommand{\prooperty}{\refstepcounter{prooperty}\item}

\makeatletter
\def\keywords#1{{\def\@thefnmark{\relax}\@footnotetext{#1}}}
\let\subjclass\keywords
\makeatother
%


\begin{document}
	%
	\sloppy
	s\thispagestyle{empty}
	%
	\centerline{\large \bf Лекции курса \guillemotleft
		Алгебра\guillemotright{}, лекторы И.\,В.~Аржанцев и Р.\,С.~Авдеев}

\smallskip

\centerline{\large ФКН НИУ ВШЭ, 1-й курс ОП ПМИ, 4-й модуль,
2015/2016 учебный год}


\tableofcontents

\newpage

\section*{Лекция 1}


\medskip

{\it Полугруппы и группы: основные определения и примеры. Группы
подстановок и группы матриц. Подгруппы. Порядок элемента и
циклические подгруппы. Смежные классы и индекс подгруппы. Теорема
Лагранжа.}

\medskip

\begin{definition}
{\it Множество с бинарной операцией}~--- это множество $M$ с
заданным отображением
$$
M\times M \to M, \quad (a,b) \mapsto a\circ b.
$$
\end{definition}

Множество с бинарной операцией обычно обозначают $(M,\circ)$.

\begin{definition}
Множество с бинарной операцией $(M,\circ)$ называется {\it
полугруппой}, если данная бинарная операция {\it ассоциативна},
т.\,е.
$$
a\circ (b \circ c) = (a\circ b)\circ c \quad \text{для всех} \ a,b,c\in M.
$$
\end{definition}

Не все естественно возникающие операции ассоциативны. Например, если
$M=\NN$ и $a\circ b:=a^b$, то
$$
2^{\left(1^2\right)}=2\ne (2^1)^2=4.
$$

Другой пример неассоциативной бинарной операции: $M = \ZZ$ и $a
\circ b := a - b$ (проверьте!).

Полугруппу обычно обозначают $(S,\circ)$.

\begin{definition}
Полугруппа $(S,\circ)$ называется {\it моноидом}, если в ней есть
{\it нейтральный элемент}, т.\,е. такой элемент $e\in S$, что
$e\circ a=a\circ e=a$ для любого $a\in S$.
\end{definition}

Во Франции полугруппа $(\NN,+)$ является моноидом, а в России нет.

\begin{remark}
Если в полугруппе есть нейтральный элемент, то он один. В самом
деле, $e_1\circ e_2=e_1=e_2$.
\end{remark}

\begin{definition}
Моноид $(S,\circ)$ называется {\it группой}, если для каждого
элемента $a\in S$ найдется {\it обратный элемент}, т.\,е. такой
$b\in S$, что $a\circ b = b\circ a= e$.
\end{definition}

\begin{exc}
Докажите, что если обратный элемент существует, то он один.
\end{exc}

Обратный элемент обозначается $a^{-1}$. Группу принято обозначать
$(G,\circ)$ или просто $G$, когда понятно, о какой операции идёт
речь. Обычно символ $\circ$ для обозначения операции опускают и
пишут просто $ab$.

\begin{definition}
Группа $G$ называется {\it коммутативной} или {\it абелевой}, если
групповая операция {\it коммутативна}, т.\,е. $ab=ba$ для любых
$a,b\in G$.
\end{definition}

Если в случае произвольной группы $G$ принято использовать
мультипликативные обозначения для групповой операции~--- $gh$, $e$,
$g^{-1}$, то в теории абелевых групп чаще используют аддитивные
обозначения, т.\,е. $a+b$, $0$, $-a$.

\begin{definition}
{\it Порядок} группы $G$~--- это число элементов в~$G$. Группа
называется {\it конечной}, если её порядок конечен, и {\it
бесконечной} иначе.
\end{definition}

Порядок группы $G$ обозначается $|G|$.

\smallskip

Приведём несколько серий примеров групп.

\smallskip

1) Числовые аддитивные группы: \ $(\ZZ,+)$, $(\QQ,+)$, $(\RR,+)$,
$(\CC,+)$, $(\ZZ_n,+)$.

\smallskip

2) Числовые мультипликативные группы: \
$(\QQ\setminus\{0\},\times)$, $(\RR\setminus\{0\},\times)$,
$(\CC\setminus\{0\},\times)$,
$(\ZZ_p\setminus\{\overline{0}\},\times)$, $p$~--- простое.

3) Группы матриц: \ $\GL_n(\RR)=\{A\in\Mat(n\times n, \RR) \mid
\det(A)\ne 0\}$;  \ $\SL_n(\RR)=\{A\in\Mat(n\times n, \RR) \mid
\det(A)=1\}$.

4) Группы подстановок: \ симметрическая группа $S_n$~--- все
подстановки длины $n$, $|S_n|=n!$;

знакопеременная группа $A_n$~--- чётные подстановки длины $n$,
$|A_n|=n!/2$.

\begin{exc}
Докажите, что группа $S_n$ коммутативна $\Leftrightarrow$ $n
\leqslant 2$, а $A_n$ коммутативна $\Leftrightarrow$ $n \leqslant
3$.
\end{exc}

\begin{definition}
Подмножество $H$ группы $G$ называется {\it подгруппой}, если выполнены следующие три условия: (1) $e \in H$; \quad (2) $ab\in H$ для любых $a,b
\in H$; \quad (3) $a^{-1}\in H$ для любого
$a\in H$.
\end{definition}

\begin{exc}
Проверьте, что $H$ является подгруппой тогда и только тогда, когда
 $H$ непусто и $ab^{-1}\in H$ для любых $a,b\in H$.
\end{exc}

В каждой группе $G$ есть {\it несобственные} подгруппы $H=\{e\}$ и
$H=G$. Все прочие подгруппы называются {\it собственными}. Например,
чётные числа $2\ZZ$ образуют собственную подгруппу в $(\ZZ,+)$.

\begin{proposition} \label{sbgrz}
Всякая подгруппа в $(\ZZ,+)$ имеет вид $k\ZZ$ для некоторого целого
неотрицательного $k$.
\end{proposition}

\begin{proof}
Пусть $H$~--- подгруппа в $\ZZ$. Если $H=\{0\}$, положим $k=0$.
Иначе пусть $k$~--- наименьшее натуральное число, лежащее в~$H$
(почему такое есть?). Тогда $k\ZZ \subseteq H$. С другой стороны,
если $a\in H$ и $a=qk+r$~--- результат деления $a$ на $k$ с
остатком, то $0 \leqslant r \leqslant k-1$ и $r = a - qk \in H$.
Отсюда $r=0$ и $H=k\ZZ$.
\end{proof}

\begin{definition}
Пусть $G$~--- группа и $g\in G$. {\it Циклической подгруппой},
порождённой элементом~$g$, называется подмножество $\{g^n \mid
n\in\ZZ\}$ в $G$.
\end{definition}

Циклическая подгруппа, порождённая элементом $g$, обозначается
$\langle g\rangle$. Элемент $g$ называется {\it порождающим} или
{\it образующим} для подгруппы $\langle g\rangle$. Например,
подгруппа $2\ZZ$ в $(\ZZ,+)$ является циклической, и в качестве
порождающего элемента в ней можно взять $g=2$ или $g=-2$. Другими
словами, $2\ZZ=\langle 2\rangle=\langle -2\rangle$.

\begin{definition}
Пусть $G$~--- группа и $g\in G$. {\it Порядком} элемента $g$
называется такое наименьшее натуральное число~$m$, что $g^m=e$. Если
такого натурального числа $m$ не существует, говорят, что порядок
элемента $g$ равен бесконечности.
\end{definition}

Порядок элемента обозначается $\ord(g)$. Заметим, что $\ord(g)=1$ тогда и только тогда, когда $g=e$.

Следующее предложение объясняет, почему для порядка группы и порядка элемента используется одно и то же слово.

\begin{proposition} \label{p1}
Пусть $G$~--- группа и $g\in G$. Тогда $\ord(g)=|\langle g\rangle|$.
\end{proposition}

\begin{proof}
Заметим, что если $g^k=g^s$, то $g^{k-s}=e$. Поэтому если элемент
$g$ имеет бесконечный порядок, то все элементы $g^n$, $n\in\ZZ$,
попарно различны, и подгруппа $\langle g\rangle$ содержит бесконечно
много элементов. Если же порядок элемента $g$ равен $m$, то из
минимальности числа $m$ следует, что элементы $e=g^0, g=g^1,
g^2,\ldots,g^{m-1}$ попарно различны. Далее, для всякого $n\in\ZZ$
мы имеем $n=mq+r$, где $0 \leqslant r \leqslant m-1$, и
$$
g^n=g^{mq+r}=(g^m)^qg^r=e^qg^r=g^r.
$$
Следовательно, $\langle g\rangle=\{e,g,\ldots, g^{m-1}\}$ и
$|\langle g\rangle|=m$.
\end{proof}

\begin{definition}
Группа $G$ называется {\it циклической}, если найдётся такой элемент
$g\in G$, что $G=\langle g\rangle$.
\end{definition}

Ясно, что любая циклическая группа коммутативна и не более чем
счётна. Примерами циклических групп являются группы $(\ZZ,+)$ и
$(\ZZ_n,+)$, $n \ge 1$.

Перейдем ещё к одному сюжету, связанному с парой группа--подгруппа.

\begin{definition}
Пусть $G$~--- группа, $H\subseteq G$~--- подгруппа и $g\in G$. {\it
Левым смежным классом} элемента $g$ группы $G$ по подгруппе $H$
называется подмножество
$$
gH=\{gh \mid h\in H\}.
$$
\end{definition}

\begin{lemma} \label{l1}
Пусть $G$~--- группа, $H\subseteq G$~--- её подгруппа и $g_1,g_2\in
G$. Тогда либо $g_1H=g_2H$, либо $g_1H\cap g_2H=\varnothing$.
\end{lemma}

\begin{proof}
Предположим, что $g_1H\cap g_2H\ne\varnothing$, т.\,е.
$g_1h_1=g_2h_2$ для некоторых $h_1,h_2\in H$. Нужно доказать, что
$g_1H=g_2H$. Заметим, что $g_1H=g_2h_2h_1^{-1}H\subseteq g_2H$.
Обратное включение доказывается аналогично.
\end{proof}

\begin{lemma} \label{l2}
Пусть $G$~--- группа и $H\subseteq G$~--- конечная подгруппа. Тогда
$|gH|=|H|$ для любого $g\in G$.
\end{lemma}

\begin{proof}
Поскольку $gH=\{gh \mid h\in H\}$, в $|gH|$ элементов не больше, чем
в~$H$. Если $gh_1=gh_2$, то домножаем слева на $g^{-1}$ и получаем
$h_1=h_2$. Значит, все элементы вида $gh$, где $h\in H$, попарно
различны, откуда $|gH|=|H|$.
\end{proof}

\begin{definition}
Пусть $G$~--- группа и $H\subseteq G$~--- подгруппа. {\it Индексом}
подгруппы $H$ в группе $G$ называется число левых смежных классов
$G$ по~$H$.
\end{definition}

Индекс группы $G$ по подгруппе $H$ обозначается $[G:H]$.

\smallskip

{\bf Теорема Лагранжа}.\ Пусть $G$~--- конечная группа и $H\subseteq
G$~--- подгруппа. Тогда
$$
|G| = |H| \cdot [G:H].
$$

\begin{proof}
Каждый элемент группы $G$ лежит в (своём) левом смежном классе по
подгруппе $H$, разные смежные классы не пересекаются
(лемма~\ref{l1}) и каждый из них содержит по $|H|$ элементов
(лемма~\ref{l2}).
\end{proof}

На следующей лекции мы обсудим следствия из данной теоремы.


\newpage

\section*{Лекция 2}

\medskip

{\it Следствия из теоремы Лагранжа. Нормальные подгруппы. Факторгруппы и теорема о гомоморфизме. Прямое произведение групп. Разложение конечной циклической группы.}

\medskip

Рассмотрим некоторые следствия из теоремы Лагранжа.

\begin{corollary} \label{c1}
Пусть $G$~--- конечная группа и $H\subseteq G$~--- подгруппа. Тогда
$|H|$ делит $|G|$.
\end{corollary}

\begin{corollary} \label{c2}
Пусть $G$~--- конечная группа и $g\in G$. Тогда $\ord(g)$ делит
$|G|$.
\end{corollary}

\begin{proof}
Это вытекает из следствия~\ref{c1} и предложения~2 прошлой лекции.
\end{proof}

\begin{corollary} \label{c3}
Пусть $G$~--- конечная группа и $g\in G$. Тогда $g^{|G|}=e$.
\end{corollary}

\begin{proof}
Согласно следствию~\ref{c2}, мы имеем $|G|=\ord(g) \cdot s$, откуда
$g^{|G|}=(g^{\ord(g)})^s=e^s=e$.
\end{proof}

\begin{corollary} \label{c5}
Пусть $G$~--- группа. Предположим, что $|G|$~--- простое число.
Тогда $G$~--- циклическая группа, порождаемая любым своим
неединичным элементом.
\end{corollary}

\begin{proof}
Пусть $g\in G$~--- произвольный неединичный элемент. Тогда
циклическая подгруппа $\langle g\rangle$ содержит более одного
элемента и $|\langle g\rangle|$ делит $|G|$ по следствию~\ref{c1}.
Значит, $|\langle g\rangle|=|G|$, откуда $G=\langle g\rangle$.
\end{proof}

Наряду с левым смежным классом можно определить {\it правый смежный
класс} элемента $g$ группы $G$ по подгруппе $H$:
$$
Hg=\{hg \mid h\in H\}.
$$

Повторяя доказательство теоремы Лагранжа для правых смежных классов,
мы получим, что для конечной группы $G$ число правых смежных классов
по подгруппе $H$ равно числу левых смежных классов и равно
$|G|/|H|$. В то же время равенство $gH=Hg$ выполнено не всегда.
Разумеется, оно выполнено, если группа $G$ абелева. Подгруппы $H$
(неабелевых) групп $G$, для которых $gH=Hg$ выполнено для любого
$g\in G$, будут изучаться в следующей лекции.

\begin{definition}
Подгруппа $H$ группы $G$ называется {\it нормальной}, если $gH=Hg$
для любого $g\in G$.
\end{definition}

\begin{proposition}
Для подгруппы $H \subseteq G$ следующие условия эквивалентны:

\vspace{-2mm}
\begin{enumerate}
\item[(1)]
$H$ нормальна;

\item[(2)]
$gHg^{-1} \subseteq H$ для любого $g \in G$;

\item[(3)]
$gHg^{-1}=H$ для любого $g\in G$.
\end{enumerate}
\end{proposition}

\vspace{-6mm}

\begin{proof}
(1)$\Rightarrow$(2) Пусть $h \in H$ и $g \in G$. Поскольку $gH =
Hg$, имеем $gh = h'g$ для некоторого $h' \in H$. Тогда $ghg^{-1} =
h'gg^{-1} = h' \in H$.\\
(2)$\Rightarrow$(3) Так как $gHg^{-1} \subseteq H$, остаётся
проверить обратное включение. Для $h \in H$ имеем $h = gg^{-1} h g
g^{-1} = g(g^{-1}hg)g^{-1} \subseteq gHg^{-1}$, поскольку $g^{-1}hg
\in H$ в силу пункта~(2), где вместо $g$ взято $g^{-1}$.

(3)$\Rightarrow$(1) Для произвольного $g \in G$ в силу (3) имеем $gH
= gHg^{-1} g \subseteq Hg$, так что $gH \subseteq Hg$. Аналогично
проверяется обратное включение.
\end{proof}
\vspace{-1mm}

Условие (2) в этом предложении кажется излишним, но именно его
удобно проверять при доказательстве нормальности подгруппы~$H$.

Обозначим через $G/H$ множество (левых) смежных классов группы $G$
по нормальной подгруппе~$H$. На $G/H$ можно определить бинарную
операцию следующим образом:
$$
(g_1H)(g_2H):=g_1g_2H.
$$

Зачем здесь нужна нормальность подгруппы $H$? Для проверки
корректности: заменим $g_1$ и $g_2$ другими представителями $g_1h_1$
и $g_2h_2$ тех же смежных классов. Нужно проверить, что
$g_1g_2H=g_1h_1g_2h_2H$. Это следует из того, что
$g_1h_1g_2h_2=g_1g_2(g_2^{-1}h_1g_2)h_2$ и $g_2^{-1}h_1g_2$ лежит в
$H$.

Ясно, что указанная операция на множестве $G/H$ ассоциативна,
обладает нейтральным элементом $eH$ и для каждого элемента $gH$ есть
обратный элемент $g^{-1}H$.

\begin{definition}
Множество $G/H$ с указанной операцией называется {\it факторгруппой}
группы $G$ по нормальной подгруппе $H$.
\end{definition}

\begin{example}
Если $G=(\ZZ,+)$ и $H=n\ZZ$, то $G/H$~--- это в точности группа
вычетов $(\ZZ_n,+)$.
\end{example}

Как представлять себе факторгруппу? В этом помогает теорема о
гомоморфизме. Но прежде чем её сформулировать, обсудим ещё несколько
понятий.

\begin{definition}
Пусть $G$ и $F$~--- группы. Отображение $\varphi\colon G\to F$
называется {\it гомоморфизмом}, если
$\varphi(ab)=\varphi(a)\varphi(b)$ для любых $a,b\in G$.
\end{definition}

\begin{remark}
Подчеркнём, что в этом определении произведение $ab$ берётся в
группе~$G$, в то время как произведение $\varphi(a) \varphi(b)$~---
в группе~$F$.
\end{remark}

\begin{lemma}
Пусть $\varphi \colon G \to F$~--- гомоморфизм групп, и пусть $e_G$
и $e_F$~--- нейтральные элементы групп $G$ и $F$ соответственно.
Тогда:

\vspace{-2mm}
\begin{enumerate}
\item[(а)]
$\varphi(e_G) = e_F$;

\item[(б)]
$\varphi(a^{-1})=\varphi(a)^{-1}$ для любого $a\in G$.
\end{enumerate}
\end{lemma}

\vspace{-5mm}

\begin{proof}
(а) Имеем $\varphi(e_G)=\varphi(e_Ge_G)=\varphi(e_G)\varphi(e_G)$.
Теперь умножая крайние части этого равенства на $\varphi(e_G)^{-1}$
(например, слева), получим $e_F = \varphi(e_G)$.

(б) Имеем $\varphi(a^{-1}) \varphi(a) = \varphi(a^{-1}a) =
\varphi(e_G) = e_F$, откуда $\varphi(a^{-1}) = \varphi(a)^{-1}$.
\end{proof}

\begin{definition}
Гомоморфизм групп $\varphi\colon G\to F$ называется {\it
изоморфизмом}, если отображение $\varphi$ биективно.
\end{definition}

\begin{exc}
Пусть $\varphi\colon G\to F$~--- изоморфизм групп. Проверьте, что
обратное отображение $\varphi^{-1}\colon F \to G$ также является
изоморфизмом.
\end{exc}

\begin{definition}
Группы $G$ и $F$ называют {\it изоморфными}, если между ними
существует изоморфизм.

Обозначение: $G\cong F$ (или $G \simeq F$).
\end{definition}

В алгебре группы рассматривают с точностью до изоморфизма:
изоморфные группы считаются \guillemotleft
одинаковыми\guillemotright{}.

\begin{definition}
С каждым гомоморфизмом групп $\varphi\colon G\to F$ связаны его {\it
ядро}
$$
\Ker(\varphi)=\{g\in G \mid \varphi(g)=e_F\}
$$
и {\it образ}
$$
\Im(\varphi)=\{a\in F \mid \exists g\in G : \varphi(g)=a\}.
$$
\end{definition}

Ясно, что $\Ker(\varphi)\subseteq G$ и $\Im(\varphi)\subseteq F$~---
подгруппы.

\begin{lemma}
Гомоморфизм групп $\varphi \colon G \to F$ инъективен тогда и только
тогда, когда $\Ker(\varphi) = \{e_G\}$.
\end{lemma}
\vspace{-3mm}
\begin{proof}
Ясно, что если $\varphi$ инъективен, то $\Ker(\varphi) = \lbrace e_G
\rbrace$. Обратно, пусть $g_1, g_2 \in G$ и $\varphi(g_1) =
\varphi(g_2)$. Тогда $g_1^{-1} g_2 \in \Ker(\varphi)$, поскольку
$\varphi(g_1^{-1} g_2) = \varphi(g_1^{-1}) \varphi(g_2) =
\varphi(g_1)^{-1} \varphi (g_2) = e_F$. Отсюда $g_1^{-1}g_2 = e_G$ и
$g_1 = g_2$.
\end{proof}

\begin{corollary}
Гомоморфизм групп $\varphi\colon G\to F$ является изоморфизмом тогда
и только тогда, когда $\Ker(\varphi)=\{e_G\}$ и $\Im(\varphi)=F$.
\end{corollary}

\begin{proposition}
Пусть $\varphi \colon G \to F$~--- гомоморфизм групп. Тогда
подгруппа $\Ker(\varphi)$ нормальна в~$G$.
\end{proposition}
\vspace{-3mm}
\begin{proof}
Достаточно проверить, что $g^{-1}hg \in \Ker(\varphi)$ для любых
$g\in G$ и $h \in \Ker(\varphi)$. Это следует из цепочки равенств
$$
\varphi(g^{-1}hg) = \varphi(g^{-1}) \varphi(h) \varphi(g) =
\varphi(g^{-1}) e_F \varphi(g) = \varphi(g^{-1}) \varphi(g) =
\varphi(g)^{-1} \varphi (g) = e_F.
$$
\end{proof}

{\bf Теорема о гомоморфизме}. Пусть $\varphi\colon G\to F$~---
гомоморфизм групп. Тогда группа $\Im(\varphi)$ изоморфна
факторгруппе $G/\Ker(\varphi)$.

\begin{proof}
Рассмотрим отображение $\psi \colon G / \Ker(\varphi) \to F$,
заданное формулой $\psi(g\Ker(\varphi)) = \varphi(g)$. Проверка
корректности: равенство $\varphi(gh_1)=\varphi(gh_2)$ для любых
$h_1,h_2\in\Ker(\varphi)$ следует из цепочки
$$
\varphi(gh_1)=\varphi(g)\varphi(h_1)=\varphi(g)=\varphi(g)\varphi(h_2)=\varphi(gh_2).
$$
Отображение $\psi$ сюръективно по построению и инъективно в силу
того, что $\varphi(g) = e_F$ тогда и только тогда, когда $g \in \Ker
(\varphi)$ (т.\,е. $g\Ker(\varphi) = \Ker(\varphi)$). Остаётся
проверить, что $\psi$~--- гомоморфизм:
$$
\psi((g\Ker(\varphi))(g'\Ker(\varphi))) = \psi(gg'\Ker(\varphi)) =
\varphi(gg') = \varphi(g)\varphi(g') =
\psi(g\Ker(\varphi))\psi(g'\Ker(\varphi)).
$$
\end{proof}

Тем самым, чтобы удобно реализовать факторгруппу $G/H$, можно найти
такой гомоморфизм $\varphi\colon G\to F$ в некоторую группу~$F$, что
$H = \Ker(\varphi)$, и тогда $G/H \cong \Im(\varphi)$.

\begin{example}
Пусть $G=(\RR,+)$ и $H=(\ZZ,+)$. Рассмотрим группу
$F=(\CC\setminus\{0\},\times)$ и гомоморфизм
$$
\varphi\colon G\to F, \quad a\mapsto e^{2\pi\imath a} = \cos (2\pi
a) + i \sin (2\pi a).
$$
Тогда $\Ker(\varphi)=H$ и факторгруппа $G/H$ изоморфна окружности
$S^1$, рассматриваемой как подгруппа в~$F$, состоящая из комплексных
чисел с модулем~$1$.
\end{example}

Определим ещё одну важную конструкцию, позволяющую строить новые
группы из имеющихся.

\begin{definition}
{\it Прямым произведением} групп $G_1, \ldots, G_m$ называется
множество
$$
G_1\times\ldots\times G_m=\{(g_1,\ldots,g_m) \mid g_1\in G_1,\ldots,
g_m\in G_m\}
$$
с операцией
$(g_1,\ldots,g_m)(g_1',\ldots,g_m')=(g_1g_1',\ldots,g_mg_m')$.
\end{definition}

Ясно, что эта операция ассоциативна, обладает нейтральным элементом
$(e_{G_1},\ldots,e_{G_m})$ и для каждого элемента $(g_1,\ldots,g_m)$
есть обратный элемент $(g_1^{-1},\ldots,g_m^{-1})$.

\begin{remark}
Группа $G_1\times\ldots\times G_m$ коммутативна в точности тогда,
когда коммутативна каждая из групп $G_1,\ldots, G_m$.
\end{remark}

\begin{remark}
Если все группы $G_1, \ldots, G_m$ конечны, то $|G_1 \times \ldots
\times G_m| = |G_1| \cdot \ldots \cdot |G_m|$.
\end{remark}

\begin{definition}
Группа $G$ \textit{раскладывается в прямое произведение} своих подгрупп $H_1, \ldots, H_m$, если отображение $H_1 \times \ldots \times H_M \rightarrow G$, $(h_1, \ldots, h_m) \mapsto h_1 \cdot \ldots \cdot h_m$ является изоморфизмом.
\end{definition}


\newpage

\section*{Лекция 3}

\medskip

{\it Факторизация по
сомножителям. Конечно порождённые и свободные абелевы группы. Подгруппы
свободных абелевых групп.}

\medskip

Следующий результат связывает конструкции факторгруппы и прямого
произведения.

{\bf Теорема о факторизации по сомножителям}. \ Пусть $H_1, \ldots,
H_m$~--- нормальные подгруппы в группах $G_1, \ldots, G_m$
соответственно. Тогда $H_1 \times \ldots \times H_m$~--- нормальная
подгруппа в $G_1 \times \ldots \times G_m$ и имеет место изоморфизм
групп
$$
(G_1 \times \ldots \times G_m) / (H_1 \times \ldots \times H_m)
\cong G_1 / H_1 \times \ldots \times G_m / H_m.
$$

\begin{proof}
Прямая проверка показывает, что $H_1\times\ldots\times H_m$~---
нормальная подгруппа в $G_1\times\ldots\times G_m$. Требуемый
изоморфизм устанавливается отображением
$$
(g_1,\ldots,g_m)(H_1\times\ldots\times H_m)\mapsto
(g_1H_1,\ldots,g_mH_m).
$$
\end{proof}

\begin{theorem}
Пусть $n=ml$~--- разложение натурального числа $n$ на два взаимно
простых множителя. Тогда имеет место изоморфизм групп
$$
\ZZ_n\cong \ZZ_m\times\ZZ_l.
$$
\end{theorem}

\begin{proof}
Рассмотрим отображение
$$
\varphi\colon \ZZ_n\to \ZZ_m\times\ZZ_l, \quad k \ (\text{mod}\ n)
\mapsto (k\ (\text{mod}\ m), k\ (\text{mod}\  l)).
$$
Поскольку $m$ и $l$ делят~$n$, отображение $\varphi$ определено
корректно. Ясно, что $\varphi$~--- гомоморфизм. Далее, если $k$
переходит в нейтральный элемент $(0,0)$, то $k$ делится и на $m$, и
на $l$, а~значит, делится на $n$ в силу взаимной простоты $m$ и~$l$.
Отсюда следует, что гомоморфизм $\varphi$ инъективен. Поскольку
множества $\ZZ_n$ и $\ZZ_m\times\ZZ_l$ содержат одинаковое число
элементов, отображение $\varphi$ биективно.
\end{proof}

\begin{corollary} \label{corpr}
Пусть $n \geqslant 2$~--- натуральное число и $n=p_1^{k_1}\ldots
p_s^{k_s}$~--- его разложение в произведение простых множителей
\textup(где $p_i \ne p_j$ при $i \ne j$\textup). Тогда имеет место
изоморфизм групп
$$
\ZZ_n\cong\ZZ_{p_1^{k_1}}\times\ldots\times\ZZ_{p_s^{k_s}}.
$$
\end{corollary}

Всюду в этой и следующей лекции $(A,+)$~--- абелева группа с
аддитивной формой записи операции. Для произвольного элемента $a\in
A$ и целого числа $s$ положим
$$
sa =
\begin{cases}
\underbrace{a + \ldots + a}_s, & \text{ если } s > 0; \\
0, & \text{ если } s = 0; \\
\underbrace{(-a) + \ldots + (-a)}_{|s|}, & \text{ если } s < 0.
\end{cases}
$$

\begin{definition}
Абелева группа $A$ называется {\it конечно порождённой}, если
найдутся такие элементы $a_1,\ldots,a_n\in A$, что всякий элемент
$a\in A$ представим в виде $a=s_1a_1 + \ldots + s_na_n$ для
некоторых целых чисел $s_1, \ldots, s_n$. При этом элементы $a_1,
\ldots, a_n$ называются {\it порождающими} или {\it образующими}
группы~$A$.
\end{definition}

\begin{remark}
Всякая конечно порождённая группа конечна или счётна.
\end{remark}

\begin{remark}
Всякая конечная группа является конечно порождённой.
\end{remark}

\begin{definition}
Конечно порождённая абелева группа $A$ называется {\it свободной},
если в ней существует {\it базис}, т.\,е. такой набор элементов
$a_1,\ldots, a_n$, что каждый элемент $a\in A$ единственным образом
представим в виде $a=s_1a_1 + \ldots + s_na_n$, где $s_1, \ldots,
s_n \in \ZZ$. При этом число $n$ называется {\it рангом} свободной
абелевой группы $A$ и обозначается $\rk\,A$.
\end{definition}

\begin{example}
Абелева группа $\ZZ^n:=\{(c_1,\ldots,c_n) \mid c_i\in\ZZ\}$ является
свободной с базисом
$$
\begin{aligned}
e_1 &= (1,0,\ldots,0), \\
e_2 &= (0,1,\ldots,0),\\
 &\ldots \\
e_n &= (0,0,\ldots,1).
\end{aligned}
$$
Этот базис называется {\it стандартным}. В группе $\ZZ^n$ можно
найти и много других базисов. Ниже мы все их опишем.
\end{example}

\begin{proposition}
Ранг свободной абелевой группы определён корректно, т.\,е. любые два
её базиса содержат одинаковое число элементов.
\end{proposition}

\begin{proof}
Пусть $a_1, \ldots, a_n$ и $b_1, \ldots, b_m$~--- два базиса
группы~$A$. Предположим, что $n < m$. Элементы $b_1, \ldots, b_m$
однозначно разлагаются по базису $a_1, \ldots, a_n$, поэтому мы
можем записать
$$
\begin{aligned}
b_1 &= s_{11}a_1 + s_{12}a_2 + \ldots + s_{1n}a_n, \\
b_2 &= s_{21}a_1 + s_{22}a_2 + \ldots + s_{2n}a_n,\\
 &\ldots \\
b_m &= s_{m1}a_1 + s_{m2}a_2 + \ldots + s_{mn}a_n,
\end{aligned}
$$
где все коэффициенты $s_{ij}$~--- целые числа. Рассмотрим
прямоугольную матрицу $S = (s_{ij})$ размера $m \times n$. Так как
$n < m$, то ранг этой матрицы не превосходит~$n$, а~значит, строки
этой матрицы линейно зависимы над~$\QQ$. Домножая коэффициенты этой
зависимости на наименьшее общее кратное их знаменателей, мы найдём
такие целые $s_1, \ldots, s_m$, из которых не все равны нулю, что
$s_1 b_1 + \ldots + s_m b_m=0$. Поскольку $0 = 0b_1 + \ldots +
0b_m$, это противоречит однозначной выразимости элемента $0$ через
базис $b_1, \ldots, b_m$.
\end{proof}

\begin{proposition}
Всякая свободная абелева группа ранга $n$ изоморфна группе $\ZZ^n$.
\end{proposition}

\begin{proof}
Пусть $A$~--- свободная абелева группа, и пусть $a_1,\ldots,a_n$~---
её базис. Рассмотрим отображение
$$\varphi \colon \ZZ^n \to A, \quad (s_1, \ldots, s_n)
\mapsto s_1a_1 + \ldots + s_na_n.
$$
Легко видеть, что $\varphi$~--- гомоморфизм. Так как всякий элемент
$a \in A$ представим в виде $s_1a_1 + \ldots + s_na_n$, где $s_1,
\ldots, s_n \in \ZZ$, то $\varphi$ сюръективен. Из единственности
такого представления следует инъективность~$\varphi$. Значит,
$\varphi$~--- изоморфизм.
\end{proof}

Пусть $e_1', \ldots, e_n'$~--- некоторый набор элементов из $\ZZ^n$.
Выразив эти элементы через стандартный базис $e_1, \ldots, e_n$, мы
можем записать
$$
(e_1', \ldots, e_n') = (e_1, \ldots, e_n)C,
$$
где $C$~--- целочисленная квадратная матрица порядка~$n$.

\begin{proposition}
Элементы $e_1', \ldots, e_n'$ составляют базис группы $\ZZ^n$ тогда
и только тогда, когда $\det C = \pm 1$.
\end{proposition}

\begin{proof}
Предположим сначала, что $e'_1, \ldots, e'_n$~--- базис. Тогда
элементы $e_1, \ldots, e_n$ через него выражаются, поэтому $(e_1,
\ldots, e_n) = (e'_1, \ldots, e'_n) D$ для некоторой целочисленной
квадратной матрицы $D$ порядка~$n$. Но тогда $(e_1, \ldots, e_n) =
(e_1, \ldots, e_n)CD$, откуда $CD = E_n$, где $E_n$~--- единичная
матрица порядка~$n$. Значит, $(\det C)(\det D) = 1$. Учитывая, что
$\det C$ и $\det D$~--- целые числа, мы получаем $\det C = \pm 1$.

Обратно, пусть $\det C = \pm 1$. Тогда матрица $C^{-1}$ является
целочисленной, а соотношение $(e_1, \ldots, e_n) = (e'_1, \ldots,
e'_n)C^{-1}$ показывает, что элементы $e_1, \ldots, e_n$ выражаются
через $e'_1, \ldots, e'_n$. Но $e_1, \ldots, e_n$~--- базис, поэтому
элементы $e'_1, \ldots, e'_n$ порождают группу~$\ZZ^n$. Осталось
доказать, что всякий элемент из $\ZZ^n$ однозначно через них
выражается. Предположим, что $s'_1e'_1 + \ldots + s'_ne'_n =
s''_1e'_1 + \ldots + s''_n e'_n$ для некоторых целых чисел $s'_1,
\ldots, s'_n, s''_1, \ldots, s''_n$. Мы можем это переписать в
следующем виде:
$$
(e'_1, \ldots, e'_n)
\begin{pmatrix} s'_1 \\ \vdots \\ s'_n \end{pmatrix} =
(e'_1, \ldots, e'_n)
\begin{pmatrix} s''_1 \\ \vdots \\ s''_n \end{pmatrix}.
$$
Учитывая, что $(e'_1, \ldots, e'_n) = (e_1, \ldots, e_n)C$ и что
$e_1, \ldots, e_n$~--- это базис, получаем
$$
C \begin{pmatrix} s'_1 \\ \vdots \\ s'_n \end{pmatrix} = C
\begin{pmatrix} s''_1 \\ \vdots \\ s''_n \end{pmatrix}.
$$
Домножая это равенство слева на~$C^{-1}$, окончательно получаем
$$
\begin{pmatrix} s'_1 \\ \vdots \\ s'_n \end{pmatrix} =
\begin{pmatrix} s''_1 \\ \vdots \\ s''_n \end{pmatrix}.
$$
\end{proof}

\begin{theorem}
Всякая подгруппа $N$ свободной абелевой группы $L$ ранга $n$
является свободной абелевой группой ранга $\leqslant n$.
\end{theorem}

\begin{proof}
Воспользуемся индукцией по $n$. При $n=0$ доказывать нечего. Пусть
$n>0$ и $e_1,\ldots,e_n$~--- базис группы $L$. Рассмотрим в $L$
подгруппу
$$
L_1 = \langle e_1,\ldots,e_{n-1}\rangle : = \ZZ e_1 + \ldots + \ZZ
e_{n-1}.
$$
Это свободная абелева группа ранга $n-1$. По предположению индукции
подгруппа $N_1:=N\cap L_1 \subseteq L_1$ является свободной абелевой
группой ранга $m \leqslant n-1$. Зафиксируем в $N_1$ базис $f_1,
\ldots, f_m$.

Рассмотрим отображение
$$
\varphi \colon N \to \ZZ, \quad s_1e_1 + \ldots + s_ne_n \mapsto
s_n.
$$
Легко видеть, что $\varphi$~--- гомоморфизм и что $\Ker \varphi =
N_1$. Далее, $\Im \varphi$~--- подгруппа в~$\ZZ$, по предложению~1
из лекции~1 она имеет вид $k \ZZ$ для некоторого целого $k \geqslant
0$. Если $k=0$, то $N \subseteq L_1$, откуда $N = N_1$ и всё
доказано. Если $k>0$, то пусть $f_{m+1}$~--- какой-нибудь элемент из
$N$, для которого $\varphi(f_{m+1}) = k$. Докажем, что $f_1, \ldots,
f_m, f_{m+1}$~--- базис в~$N$. Пусть $f \in N$~--- произвольный
элемент, и пусть $\varphi(f) = sk$, где $s \in \ZZ$. Тогда
$\varphi(f - sf_{m+1}) = 0$, откуда $f - sf_{m+1} \in N_1$ и,
следовательно, $f - sf_{m+1} = s_1 f_1 + \ldots + s_m f_m$ для
некоторых $s_1, \ldots, s_m \in \ZZ$. Значит, $f = s_1 f_1 + \ldots
+ s_m f_m + s f_{m+1}$ и элементы $f_1, \ldots, f_m, f_{m+1}$
порождают группу~$N$. Осталось доказать, что они образуют базис
в~$N$. Предположим, что
$$
s_1 f_1 + \ldots + s_m f_m + s_{m+1} f_{m+1} = s'_1 f_1 + \ldots +
s'_m f_m + s'_{m+1} f_{m+1}
$$
для некоторых целых чисел $s_1, \ldots, s_m, s_{m+1}, s'_1, \ldots,
s'_m, s'_{m+1}$. Рассмотрев образ обеих частей этого равенства при
гомоморфизме~$\varphi$, получаем $s_{m+1} k = s'_{m+1} k$, откуда
$s_{m+1} = s'_{m+1}$ и
$$
s_1 f_1 + \ldots + s_m f_m = s'_1 f_1 + \ldots + s'_m f_m.
$$
Но $f_1, \ldots, f_m$~--- базис в~$N_1$, поэтому $s_1 = s'_1$,
\ldots, $s_m = s'_m$.
\end{proof}



\newpage

\section*{Лекция 4}

\medskip

{\it Теорема о согласованных базисах. Алгоритм
приведения целочисленной матрицы к диагональному виду. Строение конечно порождённых абелевых групп. Конечные абелевы группы. } % Экспонента конечной абелевой группы.}

В~теории абелевых групп операция прямого произведения конечного
числа групп обычно называется \textit{прямой суммой} и обозначается
символом~$\oplus$, так что пишут $A_1 \oplus A_2 \oplus \ldots
\oplus A_n$ вместо $A_1 \times A_2 \times \ldots \times A_n$.

Дадим более точное описание подгрупп свободных абелевых групп.

\smallskip

{\bf Теорема о согласованных базисах.}\ Для всякой подгруппы $N$
свободной абелевой группы $L$ ранга $n$ найдётся такой базис $e_1,
\ldots, e_n$ группы $L$ и такие натуральные числа $u_1, \ldots,
u_m$, $m \leqslant n$, что $u_1 e_1, \ldots, u_m e_m$~--- базис
группы $N$ и $u_i | u_{i+1}$ при $i = 1, \ldots, m-1$.

\smallskip

\begin{remark}
Числа $u_1, \ldots, u_p$, фигурирующие в теореме о согласованных
базисах, называются {\it инвариантными множителями} подгруппы $N
\subseteq L$. Можно показать, что они определены по подгруппе
однозначно.
\end{remark}

\begin{corollary}
В~условиях теоремы о согласованных базисах имеет место изоморфизм
$$
L / N \cong \ZZ_{u_1} \times \ldots \times \ZZ_{u_m} \times
\underbrace{\ZZ \times \ldots \times \ZZ}_{n - m}.
$$
\end{corollary}

\begin{proof}
Рассмотрим изоморфизм $L \cong \ZZ^n = \underbrace{\ZZ \times \ldots
\times \ZZ}_n$, сопоставляющий произвольному элементу $s_1 e_1 +
\ldots + s_n e_n \in L$ набор $(s_1, \ldots, s_n) \in \ZZ^n$. При
этом изоморфизме подгруппа $N \subseteq L$ отождествляется с
подгруппой
$$
u_1 \ZZ \times \ldots \times u_m \ZZ \times \underbrace{\lbrace 0
\rbrace \times \ldots \times \lbrace 0 \rbrace}_{n-m} \subseteq
\ZZ^n.
$$
Теперь требуемый результат получается применением теоремы о
факторизации по сомножителям.
\end{proof}

Теперь вернемся к доказательству теоремы о согласованных базисов.
Однако это требует некоторой подготовки.

\begin{definition}
{\it Целочисленными элементарными преобразованиями строк} матрицы
называются преобразования следующих трёх типов:

1) прибавление к одной строке другой, умноженной на целое число;

2) перестановка двух строк;

3) умножение одной строки на $-1$.

Аналогично определяются {\it целочисленные элементарные
преобразования столбцов} матрицы.
\end{definition}

Прямоугольную матрицу $C=(c_{ij})$ размера $n\times m$ назовём {\it
диагональной} и обозначим $\text{diag}(u_1,\ldots,u_p)$, если
$c_{ij}=0$ при $i\ne j$ и $c_{ii}=u_i$ при $i=1,\ldots,p$, где
$p=\text{min}(n,m)$.

\begin{proposition} \label{palg}
Всякую прямоугольную целочисленную матрицу $C=(c_{ij})$ с помощью
элементарных преобразований строк и столбцов можно привести к виду
$\text{diag}(u_1,\ldots,u_p)$, где $u_1,\ldots,u_p \geqslant 0$ и
$u_i|u_{i+1}$ при $i=1,\ldots,p-1$.
\end{proposition}

\begin{proof}
Если $C=0$, то доказывать нечего. Если $C\ne 0$, но $c_{11}=0$, то
переставим строки и столбцы и получим $c_{11}\ne 0$. Умножив, если
нужно, первую строку на $-1$, добьёмся условия $c_{11}>0$. Теперь
будем стремиться уменьшить~$c_{11}$.

Если какой-то элемент $c_{i1}$ не делится на $c_{11}$, то разделим с
остатком: $c_{i1}=qc_{11}+r$. Вычитая из $i$-й строки $1$-ю строку,
умноженную на~$q$, и затем переставляя $1$-ю и $i$-ю строки,
уменьшаем~$c_{11}$. Повторяя эту процедуру, в~итоге добиваемся, что
все элементы $1$-й строки и $1$-го столбца делятся на $c_{11}$.

Если какой-то $c_{ij}$ не делится на $c_{11}$, то поступаем
следующим образом. Вычтя из $i$-й строки $1$-ю строку с подходящим
коэффициентом, добьёмся $c_{i1}=0$. После этого прибавим к $1$-й
строке $i$-ю строку. При этом $c_{11}$ не изменится, а $c_{1j}$
перестанет делиться на $c_{11}$, и мы вновь сможем уменьшить
$c_{11}$.

В~итоге добьёмся того, что все элементы делятся на~$c_{11}$. После
этого обнулим все элементы $1$-й строки и $1$-го столбца, начиная со
вторых, и продолжим процесс с меньшей матрицей.
\end{proof}

Теперь мы готовы доказать теорему о согласованных базисах.

\begin{proof}[Доказательство теоремы о согласованных базисах]
Мы знаем, что $N$ является свободной абелевой группой ранга $m
\leqslant n$. Пусть $e_1, \ldots, e_n$~--- базис в $L$ и $f_1,
\ldots, f_m$~--- базис в~$N$. Тогда $(f_1, \ldots, f_m) = (e_1,
\ldots, e_n)C$, где $C$~--- целочисленная матрица размера $n \times
m$ и ранга~$m$. Покажем, что целочисленные элементарные
преобразования строк (столбцов) матрицы $C$~--- это в точности
элементарные преобразования над базисом в~$L$ (в~$N$). Для этого
рассмотрим сначала случай строк. Заметим, что каждое из
целочисленных элементарных преобразований строк реализуется при
помощи умножения матрицы $C$ слева на квадратную матрицу~$P$
порядка~$n$, определяемую следующим образом:

(1) в случае прибавления к $i$-й строке $j$-й, умноженной на целое
число~$z$, в матрице~$P$ на диагонали стоят единицы, на $(ij)$-м
месте~--- число~$z$, а на остальных местах~--- нули;

(2) в случае перестановки $i$-й и $j$-й строк имеем $p_{ij} = p_{ji}
= 1$, $p_{kk} = 1$ при $k \ne i,j$, а на остальных местах стоят
нули;

(3) в случае умножения $i$-й строки на $-1$ имеем $p_{ii} = -1$,
$p_{jj} = 1$ при $j \ne i$, а на остальных местах стоят нули.

Теперь заметим, что равенство $(f_1, \ldots, f_m) = (e_1, \ldots,
e_n)C$ эквивалентно равенству $(f_1, \ldots, f_m) = (e_1, \ldots,
e_n)P^{-1} PC$. Таким образом, базис $(f_1, \ldots, f_m)$ выражается
через новый базис $(e'_1, \ldots, e'_n) := (e_1, \ldots, e_n)P^{-1}$
при помощи матрицы~$PC$.

В случае столбцов всё аналогично: каждое из целочисленых
элементарных преобразований столбцов реализуется при помощи
умножения матрицы $C$ справа на некоторую квадратную матрицу $Q$
порядка~$m$ (определяемую почти так же, как~$P$). В~этом случае
имеем $(f_1, \ldots, f_m)Q = (e_1, \ldots, e_n)CQ$, так что новый
базис $(f'_1, \ldots, f'_m) := (f_1, \ldots, f_m)Q$ выражается через
$(e_1, \ldots, e_n)$ при помощи матрицы $CQ$.

Воспользовавшись предложением~\ref{palg}, мы можем привести матрицу
$C$ при помощи целочисленных элементарных преобразований строк и
столбцов к диагональному виду~$C'' = \text{diag}(u_1, \ldots, u_m)$,
где $u_i | u_{i+1}$ для всех $i = 1, \ldots, m-1$. С~учётом
сказанного выше это означает, что для некоторого базиса $e''_1,
\ldots, e''_n$ в~$L$ и некоторого базиса $f''_1, \ldots, f''_m$
в~$N$ справедливо соотношение $(f''_1, \ldots, f''_m) = (e''_1,
\ldots, e''_n) C''$. Иными словами, $f''_i = u_i e''_i$ для всех $i
= 1, \ldots, m$, а~это и требовалось.
\end{proof}

\begin{definition}
Конечная абелева группа $A$ называется {\it примарной}, если её
порядок равен $p^k$ для некоторого простого числа~$p$.
\end{definition}

\begin{remark}
В общем случае (когда группы не предполагаются коммутативными)
конечная группа $G$ с~условием $|G| = p^k$ ($p$~--- простое)
называется {\it $p$-группой}.
\end{remark}

Следствие~1 лекции~3 показывает, что каждая конечная циклическая
группа разлагается в прямую сумму примарных циклических подгрупп.

\begin{theorem} \label{traz}
Всякая конечно порождённая абелева группа $A$ разлагается в прямую
сумму примарных и бесконечных циклических подгрупп, т.\,е.
\begin{equation} \label{eqn}
A \cong \ZZ_{p_1^{k_1}} \oplus \ldots \oplus \ZZ_{p_s^{k_s}} \oplus
\ZZ \oplus \ldots \oplus \ZZ,
\end{equation}
где $p_1, \ldots, p_s$~--- простые числа \textup(не обязательно
попарно различные\textup) и $k_1, \ldots, k_s \in \NN$. Кроме того,
число бесконечных циклических слагаемых, а~также число и порядки
примарных циклических слагаемых определено однозначно.
\end{theorem}

Сразу выделим некоторые следствия из этой теоремы.

\begin{corollary}
Абелева группа $A$ является конечно порождённой тогда и только тогда, когда $A$ разлагается в прямую сумму циклических подгрупп. 
\end{corollary}

\begin{proof}
В одну сторону следует из теоремы. В другую сторону: пусть $A = A_1 \oplus \ldots \oplus A_m$, где $A_i$~--- циклическая подгруппа, то есть $A_i = \langle a_i \rangle$, $a_i \in A$. Тогда $\{a_1, \ldots, a_m \}$ --- набор порождающих элементов для группы $A$.
\end{proof}

\begin{corollary}
Всякая конечная абелева группа разлагается в прямую сумму примарных
циклических подгрупп, причём число и порядки примарных циклических
слагаемых определено однозначно.
\end{corollary}

Теперь преступим к доказательству самой теоремы.

\begin{proof}
Пусть $a_1,\ldots,a_n$~--- конечная система порождающих группы $A$.
Рассмотрим гомоморфизм
$$
\varphi \colon \ZZ^n \to A, \quad (s_1, \ldots, s_n) \mapsto s_1 a_1
+ \ldots + s_n a_n.
$$
Ясно, что $\varphi$ сюръективен. Тогда по теореме о гомоморфизме
получаем $A \cong \ZZ^n / N$, где $N = \Ker \varphi$. По теореме о
согласованных базисах существует такой базис $e_1, \ldots, e_n$
группы $\ZZ^n$ и такие натуральные числа $u_1, \ldots, u_m$, $m
\leqslant n$, что $u_1 e_1, \ldots, u_m e_m$~--- базис группы~$N$.
Тогда имеем
$$
\begin{array}{ccccccccccccc}
L &=& \langle e_1 \rangle &\oplus & \ldots & \oplus & \langle e_m
\rangle & \oplus & \langle e_{m+1} \rangle & \oplus & \ldots &
\oplus & \langle e_n \rangle, \\
N &=& \langle u_1e_1 \rangle & \oplus & \ldots & \oplus & \langle
u_m e_m \rangle &\oplus & \lbrace 0 \rbrace & \oplus & \ldots &
\oplus & \lbrace 0 \rbrace.
\end{array}
$$
Применяя теорему о факторизации по сомножителям, мы получаем
$$
\ZZ^n / N \cong \ZZ / u_1 \ZZ \oplus \ldots \oplus \ZZ / u_m \ZZ
\oplus \underbrace{\ZZ / \lbrace 0 \rbrace \oplus \ldots \oplus \ZZ
/ \lbrace 0 \rbrace}_{n-m} \cong \ZZ_{u_1} \oplus \ldots \oplus
\ZZ_{u_m} \oplus \underbrace{\ZZ \oplus \ldots \oplus \ZZ}_{n-m}.
$$
Чтобы добиться разложения~(\ref{eqn}), остаётся представить каждое
из циклических слагаемых $\ZZ_{u_i}$ в виде прямой суммы примарных
циклических подгрупп, воспользовавшись следствием~1 из лекции~3.

Перейдём к доказательству единственности разложения~(\ref{eqn}).
Пусть $\langle c \rangle_q$ обозначает циклическую группу порядка
$q$ с порождающей~$c$. Пусть имеется разложение
\begin{equation} \label{eqn2}
A = \langle c_1\rangle_{p_1^{k_1}} \oplus \ldots \oplus \langle c_s
\rangle_{p_s^{k_s}} \oplus \langle c_{s+1} \rangle_{\infty} \oplus
\ldots \oplus \langle c_{s+t} \rangle_{\infty}
\end{equation}
(заметьте, что мы просто переписали в другом виде правую часть
соотношения~(\ref{eqn})). Рассмотрим в~$A$ так называемую {\it
подгруппу кручения}
$$
\Tor A := \{ a \in A \mid ma=0 \ \text{для некоторого} \ m \in \NN
\}.
$$
Иными словами, $\Tor A$~--- это подгруппа в~$A$, состоящая из всех
элементов конечного порядка. Выделим эту подгруппу в
разложении~(\ref{eqn2}). Рассмотрим произвольный элемент $a \in A$.
Он представим в виде
$$
a = r_1c_1 + \ldots + r_m c_m + r_{m+1} c_{m+1} + \ldots + r_n c_n
$$
для некоторых целых чисел $r_1, \ldots, r_n$. Легко видеть, что $a$
имеет конечный порядок тогда и только тогда, когда $r_{m+1} = \ldots
= r_m = 0$. Отсюда получаем, что
\begin{equation} \label{eqn3}
\Tor A = \langle c_1 \rangle_{p_1^{k_1}} \oplus \ldots \oplus
\langle c_s \rangle_{p_s^{k_s}}.
\end{equation}
Применяя опять теорему о факторизации по сомножителям, мы получаем
$A / \Tor A \cong \ZZ^t$, где $t$ --- количество бесконечных 
циклических подгрупп в разложении~(\ref{eqn}). Отсюда следует, 
что число $t$ однозначно выражается в терминах самой группы~$A$ 
(как ранг свободной абелевой группы $A / \Tor A$). Значит, $t$ 
не зависит от разложения~(\ref{eqn2}).

Однозначность числа и порядков примарных циклических
групп будет доказана на следующей лекции.

\end{proof}

%Далее, для каждого простого числа $p$ определим в $A$ {\it подгруппу
%$p$-кручения}
%\begin{equation} \label{eqn4}
%\Tor_p A := \{ a\in A \mid p^ka=0 \ \text{для некоторого} \ k \in
%\NN \}.
%\end{equation}
%Ясно, что $\Tor_p A \subset \Tor A$. Выделим подгруппу $\Tor_p A$ в
%разложении~(\ref{eqn3}). Легко видеть, что $\langle c_i
%\rangle_{p_i^{k_i}} \subseteq \Tor_p A$ для всех $i$ с условием $p_i
%= p$. Если же $p_i \ne p$, то по следствию~2 из теоремы Лагранжа
%(см. лекцию~1) порядок любого ненулевого элемента $x \in \langle c_i
%\rangle_{p_i^{k_i}}$ является степенью числа~$p_i$, а~значит, $p^k x
%\ne 0$ для всех $k \in \NN$. Отсюда следует, что $\Tor_p A$ является
%суммой тех конечных слагаемых в разложении~(\ref{eqn3}), порядки
%которых суть степени~$p$. Поэтому доказательство теперь сводится к
%случаю, когда $A$~--- примарная группа.
%
%Пусть $|A|=p^k$ и
%$$
%A = \langle c_1\rangle_{p^{k_1}}\oplus\ldots\oplus\langle
%c_r\rangle_{p^{k_r}}, \quad k_1+\ldots+k_r=k.
%$$
%Докажем индукцией по~$k$, что набор чисел $k_1, \ldots, k_r$ не
%зависит от разложения.
%
%Если $k = 1$, то $|A| = p$, но тогда $A \cong \ZZ_p$ по следствию~5
%из теоремы Лагранжа (см. лекцию~1). Пусть теперь $k > 1$. Рассмотрим
%подгруппу $pA: = \{ pa \mid a \in A \}$. В~терминах
%равенства~(\ref{eqn4}) имеем
%$$
%pA = \langle pc_1 \rangle_{p^{k_1-1}} \oplus \ldots \oplus \langle
%pc_r\rangle_{p^{k_r-1}}.
%$$
%В частности, при $k_i = 1$ соответствующее слагаемое равно $\lbrace
%0 \rbrace$ (и тем самым исчезает). Так как $|pA| = p^{k - r} < p^k$,
%то по предположению индукции группа $pA$ разлагается в прямую сумму
%примарных циклических подгрупп однозначно с точностью до порядка
%слагаемых. Следовательно, ненулевые числа в наборе $k_1 - 1, \ldots,
%k_r-1$ определены однозначно (с точностью до перестановки). Отсюда
%мы находим значения $k_i$, отличные от~$1$. Количество тех~$k_i$,
%которые равны~$1$, однозначно восстанавливается из условия $k_1 +
%\ldots + k_r = k$.
%\end{proof}

%Заметим, что теорема о согласованных базисах даёт нам другое
%разложение конечной абелевой группы~$A$:
%\begin{equation} \label{eqn5}
%A=\ZZ_{u_1}\oplus\ldots\oplus\ZZ_{u_m}, \quad \text{где} \
%u_i|u_{i+1} \ \text{при} \ i = 1, \ldots, m-1.
%\end{equation}
%Числа $u_1, \ldots, u_m$ называют {\it инвариантными множителями}
%конечной абелевой группы~$A$.
%
%\begin{definition}
%{\it Экспонентой} конечной абелевой группы $A$ называется число
%$\exp A$, равное наименьшему общему кратному порядков элементов
%из~$A$.
%\end{definition}
%
%\begin{remark}
%Легко видеть, что $\exp A = \min \lbrace n \in \NN \mid ma = 0 \
%\text{для всех} \ a \in A \rbrace$.
%\end{remark}
%
%\begin{proposition}
%Экспонента конечной абелевой группы~$A$ равна её последнему
%инвариантному множителю~$u_m$.
%\end{proposition}
%
%\begin{proof}
%Обратимся к разложению~(\ref{eqn5}). Так как $u_i | u_m$ для всех $i
%= 1, \ldots, m$, то $u_ma=0$ для всех $a \in A$. Это означает, что
%$\exp A \leqslant u_m$ (и тем самым $\exp A \, | u_m$). С~другой
%стороны, в $A$ имеется циклическая подгруппа порядка $u_m$. Значит,
%$\exp A \geqslant u_m$.
%\end{proof}
%
%\begin{corollary}
%Конечная абелева группа $A$ является циклической тогда и только
%тогда, когда $\exp A =\nobreak |A|$.
%\end{corollary}
%
%\begin{proof}
%Группа $A$ является циклической тогда и только тогда, когда в
%разложении~(\ref{eqn5}) присутствует только одно слагаемое, т.\,е.
%$A = \ZZ_{u_m}$ и $|A| = u_m$.
%\end{proof}

\newpage

\section*{Лекция 5}

\medskip

{\it Строение конечно порождённых абелевых груп (продолжение). Экспонента
конечной абелевой группы. Действие группы на множестве. Орбиты и стабилизаторы.}
%Транзитивные и свободные действия. Три действия группы на себе.
%Теорема Кэли. Классы сопряжённости.}

Продолжим доказательство теоремы с прошлой лекции.

\begin{theorem} \label{traz}
Всякая конечно порождённая абелева группа $A$ разлагается в прямую
сумму примарных и бесконечных циклических подгрупп, т.\,е.
\begin{equation} \label{eqn}
A \cong \ZZ_{p_1^{k_1}} \oplus \ldots \oplus \ZZ_{p_s^{k_s}} \oplus
\ZZ \oplus \ldots \oplus \ZZ,
\end{equation}
где $p_1, \ldots, p_s$~--- простые числа \textup(не обязательно
попарно различные\textup) и $k_1, \ldots, k_s \in \NN$. Кроме того,
число бесконечных циклических слагаемых, а~также число и порядки
примарных циклических слагаемых определено однозначно.
\end{theorem}

\begin{proof}
На прошлой лекции мы доказали существование разложения и то, что количество
бесконечных циклических групп $\ZZ$ определено однозначно. Для этого мы вводили
понятие \textit{подгруппы кручения}:
\begin{equation} \label{eqn3}
\Tor A = \langle c_1 \rangle_{p_1^{k_1}} \oplus \ldots \oplus
\langle c_s \rangle_{p_s^{k_s}}.
\end{equation}
Далее, для каждого простого числа $p$ определим в $A$ {\it подгруппу
$p$-кручения}
\begin{equation} \label{eqn4}
\Tor_p A := \{ a\in A \mid p^ka=0 \ \text{для некоторого} \ k \in
\NN \}.
\end{equation}
Ясно, что $\Tor_p A \subset \Tor A$. Выделим подгруппу $\Tor_p A$ в
разложении~(\ref{eqn3}). Легко видеть, что $\langle c_i
\rangle_{p_i^{k_i}} \subseteq \Tor_p A$ для всех $i$ с условием $p_i
= p$. Если же $p_i \ne p$, то по следствию~2 из теоремы Лагранжа
(см. лекцию~2) порядок любого ненулевого элемента $x \in \langle c_i
\rangle_{p_i^{k_i}}$ является степенью числа~$p_i$, а~значит, $p^k x
\ne 0$ для всех $k \in \NN$. Отсюда следует, что $\Tor_p A$ является
суммой тех конечных слагаемых в разложении~(\ref{eqn3}), порядки
которых суть степени~$p$. Поэтому доказательство теперь сводится к
случаю, когда $A$~--- примарная группа.

Пусть $|A|=p^k$ и
$$
A = \langle c_1\rangle_{p^{k_1}}\oplus\ldots\oplus\langle
c_r\rangle_{p^{k_r}}, \quad k_1+\ldots+k_r=k.
$$
Докажем индукцией по~$k$, что набор чисел $k_1, \ldots, k_r$ не
зависит от разложения.

Если $k = 1$, то $|A| = p$, но тогда $A \cong \ZZ_p$ по следствию~5
из теоремы Лагранжа (см. лекцию~2). Пусть теперь $k > 1$. Рассмотрим
подгруппу $pA: = \{ pa \mid a \in A \}$. В~терминах
равенства~(\ref{eqn4}) имеем
$$
pA = \langle pc_1 \rangle_{p^{k_1-1}} \oplus \ldots \oplus \langle
pc_r\rangle_{p^{k_r-1}}.
$$
В частности, при $k_i = 1$ соответствующее слагаемое равно $\lbrace
0 \rbrace$ (и тем самым исчезает). Так как $|pA| = p^{k - r} < p^k$,
то по предположению индукции группа $pA$ разлагается в прямую сумму
примарных циклических подгрупп однозначно с точностью до порядка
слагаемых. Следовательно, ненулевые числа в наборе $k_1 - 1, \ldots,
k_r-1$ определены однозначно (с точностью до перестановки). Отсюда
мы находим значения $k_i$, отличные от~$1$. Количество тех~$k_i$,
которые равны~$1$, однозначно восстанавливается из условия $k_1 +
\ldots + k_r = k$.
\end{proof}

Заметим, что теорема о согласованных базисах даёт нам другое
разложение конечной абелевой группы~$A$:
\begin{equation} \label{eqn5}
A=\ZZ_{u_1}\oplus\ldots\oplus\ZZ_{u_m}, \quad \text{где} \
u_i|u_{i+1} \ \text{при} \ i = 1, \ldots, m-1.
\end{equation}
Числа $u_1, \ldots, u_m$ называют {\it инвариантными множителями}
конечной абелевой группы~$A$.

\begin{definition}
{\it Экспонентой} конечной абелевой группы $A$ называется число
$\exp A$, равное наименьшему общему кратному порядков элементов
из~$A$. Легко заметить, что это равносильно следующему условию:
$$
\exp A = \min \lbrace n \in \NN \mid na = 0 \
\text{для всех} \ a \in A \rbrace
$$
\end{definition}

\begin{proposition}
Экспонента конечной абелевой группы~$A$ равна её последнему
инвариантному множителю~$u_m$.
\end{proposition}

\begin{proof}
Обратимся к разложению~(\ref{eqn5}). Так как $u_i | u_m$ для всех $i
= 1, \ldots, m$, то $u_ma=0$ для всех $a \in A$. Это означает, что
$\exp A \leqslant u_m$ (и тем самым $\exp A \, | u_m$). С~другой
стороны, в $A$ имеется циклическая подгруппа порядка $u_m$. Значит,
$\exp A \geqslant u_m$.
\end{proof}

\begin{corollary}
Конечная абелева группа $A$ является циклической тогда и только
тогда, когда $\exp A =\nobreak |A|$.
\end{corollary}

\begin{proof}
Группа $A$ является циклической тогда и только тогда, когда в
разложении~(\ref{eqn5}) присутствует только одно слагаемое, т.\,е.
$A = \ZZ_{u_m}$ и $|A| = u_m$.
\end{proof}


Пусть $G$~--- произвольная группа и $X$~--- некоторое множество.

\begin{definition}
\textit{Действием} группы $G$ на множестве $X$ называется
отображение $G\times X\to X$, $(g,x)\mapsto gx$, удовлетворяющее
следующим условиям:

1) $ex=x$ для любого $x\in X$ ($e$~--- нейтральный элемент
группы~$G$);

2) $g(hx)=(gh)x$ для всех $g,h\in G$ и $x\in X$.

Обозначение: $G:X$.
\end{definition}

Если задано действие группы $G$ на множестве~$X$, то каждый элемент
$g \in G$ определяет биекцию $a_g \colon X \to\nobreak X$ по правилу
$a_g(x) = gx$ (обратным отображением для $a_g$ будет $a_{g^{-1}}$).
Обозначим через $S(X)$ группу всех биекций (перестановок) множества
$X$ с операцией композиции. Тогда отображение $a \colon G \to S(X)$,
$g \mapsto a_g$, является гомоморфизмом групп. Действительно, для
произвольных элементов $g,h \in G$ и $x \in X$ имеем
$$
a_{gh}(x) = (gh)x = g(hx) = g a_h(x) = a_g (a_h(x)) = (a_g a_h)(x).
$$
Можно показать, что задание действия группы $G$ на множестве $X$
равносильно заданию соответствующего гомоморфизма $a \colon G \to
S(X)$.

\begin{example}
Симметрическая группа $S_n$ естественно действует на множестве $X =
\lbrace 1, 2, \ldots, n \rbrace$ по формуле $\sigma x =\nobreak
\sigma (x)$ ($\sigma \in S_n$, $x \in X$). Условие~1) здесь
выполнено по определению тождественной подстановки, условие~2)
выполнено по определению композиции подстановок.
\end{example}

Пусть задано действие группы $G$ на множестве~$X$.

\begin{definition}
{\it Орбитой} точки $x\in X$ называется подмножество
$$
Gx = \lbrace x' \in X \mid x' = gx \ \text{для некоторого} \ g \in G
\rbrace = \{ gx \mid g\in G\}.
$$
\end{definition}

\begin{remark}
Для точек $x, x' \in X$ отношение \guillemotleft$x'$ лежит в орбите
$Gx$\guillemotright{} является отношением эквивалентности:

(1) (рефлексивность) $x \in Gx$ для всех $x \in X$: это верно, так
как $x = ex \in Gx$ для всех $x \in X$;

(2) (симметричность) если $x' \in Gx$, то $x \in Gx'$: это верно,
так как из условия $x' = gx$ следует $x = ex = (g^{-1}g)x =
g^{-1}(gx) = g^{-1}x' \in Gx'$;

(3) (транзитивность) если $x' \in Gx$ и $x'' \in Gx'$, то $x'' \in
Gx$: это верно, так как из условий $x' = gx$ и $x'' = hx'$ следует
$x'' = hx' = h(gx) = (hg)x \in Gx$.

Отсюда вытекает, что множество $X$ разбивается в объединение попарно
непересекающихся орбит действия группы~$G$.
\end{remark}

\begin{definition}
{\it Стабилизатором \textup(стационарной подгруппой\textup)} точки
$x \in X$ называется подгруппа $\St(x) := \{ g \in G \mid gx = x
\}$.
\end{definition}

\begin{exercise}
Проверьте, что множество $\St(x)$ действительно является подгруппой
в~$G$.
\end{exercise}

%\begin{example}
%Рассмотрим действие группы $\SL_n(\RR)$, $n \geqslant 2$ на
%множестве~$\RR^n$, заданное формулой $(A, v) \mapsto A \cdot v$, где
%в правой части вектор $v$ рассматривается как столбец своих
%координат. Оказывается, что для этого действия имеется всего две
%орбиты $\lbrace 0 \rbrace$ и $\RR^n \setminus \lbrace 0 \rbrace$.
%Чтобы показать, что $\RR^n \setminus\nobreak \lbrace 0 \rbrace$
%действительно является одной орбитой, достаточно проверить, что
%всякий ненулевой вектор можно получить, подействовав на элемент
%$e_1$ (первый базисный вектор) подходящей матрицей из
%группы~$\SL_n(\RR)$. Пусть $v \in \RR^n$~--- произвольный вектор с
%координатами $(x_1, \ldots, x_n)$. Покажем, что существует
%матрица~$A \in \SL_n(\RR)$, для которой $Ae_1 = v$ или,
%эквивалентно,
%\begin{equation} \label{eqn1}
%A\begin{pmatrix} 1 \\ 0 \\ \vdots \\ 0 \end{pmatrix} =
%\begin{pmatrix} x_1\\ x_2 \\ \vdots \\ x_n \end{pmatrix}.
%\end{equation}
%Из уравнения~(\ref{eqn1}) следует, что в первом столбце матрицы~$A$
%должны стоять в точности числа $x_1, \ldots, x_n$. Как мы знаем из
%линейной алгебры, вектор $v$ можно дополнить до базиса $v, v_2,
%\ldots, v_n$ пространства~$\RR^n$. Пусть $A'$~--- квадратная матрица
%порядка~$n$, в которой по столбцам записаны координаты векторов $v,
%v_2, \ldots, v_n$. Эта матрица невырожденна и удовлетворяет условию
%$A'e_1 = v$ (а~также $A'e_i = v_i$ для всех $i = 2, \ldots, n$).
%Однако её определитель может быть отличен от~$1$. Поделив все
%элементы последнего столбца матрицы $A'$ на $\det A'$, мы получим
%искомую матрицу~$A$ с определителем~$1$. Итак, мы показали, что
%$\RR^n \setminus \lbrace 0 \rbrace$~--- одна орбита для нашего
%действия. Легко видеть, что стабилизатор точки $e_1$ при этом будет
%состоять из всех матриц в $\SL_n(\RR)$, у которых первый столбец
%равен $\begin{pmatrix} 1 \\ 0 \\ \vdots \\ 0 \end{pmatrix}$. (У
%любой другой точки стабилизатор будет другим!)
%\end{example}

\begin{lemma}
Пусть конечная группа $G$ действует на множестве~$X$. Тогда для
всякого элемента $x\in X$ справедливо равенство
$$
|Gx| = |G| / |\St(x)|.
$$
В~частности, число элементов в \textup(любой\textup) орбите делит
порядок группы~$G$.
\end{lemma}

\begin{proof}
Рассмотрим множество\footnote{Это множество может не быть
факторгруппой, так как подгруппа $\St(x)$ не обязана быть нормальной
в~$G$.} $G / \St(x)$ левых смежных классов группы $G$ по подгруппе
$\St(x)$ и определим отображение $\psi \colon G / \St(x) \to Gx$ по
формуле $g\St(x) \mapsto gx$. Это определение корректно, поскольку
для любого другого представителя $g'$ левого смежного класса
$g\St(x)$ имеем $g' = g h$, где $h \in \St(x)$, и тогда $g'x = (gh)x
= g(hx) = gx$. Сюръективность отображения $\psi$ следует из
определения орбиты $Gx$. Проверим инъективность. Предположим, что
$g_1 \St(x) = g_2 \St(x)$ для некоторых $g_1, g_2 \in G$. Тогда
$g_1x = g_2x$. Подействовав на левую и правую части элементом
$g_2^{-1}$, получим $(g_2^{-1}g_1)x = x$, откуда $g_2^{-1}g_1 \in
\St(x)$. Последнее и означает, что $g_1 \St(x) = g_2 \St(x)$. Итак,
мы показали, что отобржание $\psi$ является биекцией. Значит, $|Gx|
= |G / \St(x)| = [G : \St(x)]$ и требуемое равенство вытекает из
теоремы Лагранжа (см. лекцию~1).
\end{proof}

\begin{example}
Рассмотрим действие группы $S^1 = \lbrace z \in \CC \mid |z| = 1
\rbrace$ на множестве~$\CC$, заданное
формулой $(z,w) \mapsto zw$, где $z \in S^1$, $w \in \CC$,
а~$zw$~--- обычное произведение комплексных чисел. Для этого
действия орбитами будут множества вида $|z| = c$, где $c \in
\RR_{\geqslant 0}$,~--- это всевозможные окружности с центром в
нуле, а также отдельная орбита, состоящая из нуля. Имеем
$$
\St(z) =
\begin{cases}
\lbrace 1 \rbrace, & \text{если} \ z \ne 0;\\
S^1, & \text{если} \ z = 0.
\end{cases}
$$
\end{example}

%
%Пусть снова группа $G$ действует на множестве~$X$.
%
%\begin{definition}
%Действие $G$ на $X$ называется {\it транзитивным}, если для любых
%$x, x' \in X$ найдётся такой элемент $g \in G$, что $x' = gx$.
%(Иными словами, все точки множества $X$ образуют одну орбиту.)
%\end{definition}
%
%\begin{definition}
%Действие $G$ на $X$ называется {\it свободным}, если для любой точки
%$x \in X$ условие $gx=x$ влечёт $g=e$. (Иными словами, $\St(x) =
%\lbrace e \rbrace$ для всех $x \in X$.)
%\end{definition}
%
%\begin{definition}
%Действие $G$ на $X$ называется {\it эффективным}, если условие
%$gx=x$ для всех $x\in X$ влечёт $g=e$. (Иными словами, $\bigcap
%\limits_{x \in X} \St(x) = \lbrace e \rbrace$.)
%\end{definition}
%
%\begin{remark}
%Из определений следует, что всякое свободное действие эффективно.
%Обратное утверждение неверно, как показывает пример~1 при $n
%\geqslant 3$, см. ниже.
%\end{remark}
%
%В~примерах 1--3 все действия эффективны. В~примере~1 действие
%транзитивно, свободно при $n \leqslant 2$ и не свободно при $n
%\geqslant 3$. В~примере~2 действие не транзитивно и не свободно; но
%если его ограничить на подмножество $\CC \setminus \lbrace 0
%\rbrace$ (то есть выбросить из $\CC$ точку~$0$), то оно станет
%свободным. В примере~3 действие не транзитивно и не свободно; но
%если его ограничить на подмножество $\RR^n \setminus \lbrace 0
%\rbrace$, то оно станет транзитивным.
%
%\begin{remark}
%Действие $G$ на $X$ эффективно тогда и только тогда, когда
%определяемый им гомоморфизм $a \colon G \to S(X)$ инъективен.
%\end{remark}
%
%\begin{definition}
%{\it Ядром неэффективности} действия группы $G$ на множестве~$X$
%называется подгруппа $K = \{ g\in G \mid gx = x \ \text{для всех} \
%x\in X\}$.
%\end{definition}
%
%Легко проверить, что $K = \Ker a$, где $a \colon G \to S(X)$~---
%определяемый действием гомоморфизм. Отсюда следует, что $K$~---
%нормальная подгруппа в~$G$. Рассмотрим факторгруппу $G/K$ и
%определим её действие на множестве $X$ по формуле $(gK)x = gx$.
%Поскольку $kx = x$ для всех $k \in K$ и $x \in X$, действие
%определено корректно.
%
%\begin{lemma}
%Определённое выше действие группы $G/K$ на множестве $X$ является
%эффективным.
%\end{lemma}
%
%\begin{proof}
%Пусть элемент $g \in G$ таков, что $(gK)x = x$ для всех $x \in X$.
%Тогда $gx = x$ для всех $x \in X$, откуда $g \in K$ и $gK = K$.
%\end{proof}
%
%Пусть $G$~--- произвольная группа. Рассмотрим три действия $G$ на
%самой себе, т.\,е. положим $X=G$:
%
%1) действие {\it умножениями слева}: $(g,h)\mapsto gh$;
%
%2) действие {\it умножениями справа}: $(g,h)\mapsto hg^{-1}$;
%
%3) действие {\it сопряжениями}: $(g,h)\mapsto ghg^{-1}$.
%
%Непосредственно проверяется, что первые два действия свободны и
%транзитивны. Орбиты третьего действия называются {\it классами
%сопряжённости} группы~$G$. Например, $\{e\}$~--- класс сопряжённости
%в любой группе. В~частности, для нетривиальных групп действие
%сопряжениями не является транзитивным.
%
%\begin{definition}
%Два действия группы $G$ на множествах $X$ и $Y$ называются {\it
%изоморфными}, если существует такая биекция $\varphi\colon X\to Y$,
%что
%\begin{equation} \label{eqn2}
%\varphi(gx)=g\varphi(x) \ \text{для любых} \ g\in G, x\in X.
%\end{equation}
%\end{definition}
%
%\begin{proposition}
%Всякое свободное транзитивное действие группы $G$ на множестве $X$
%изоморфно действию группы $G$ на себе левыми сдвигами.
%\end{proposition}
%
%\begin{proof}
%Зафиксируем произвольный элемент $x\in X$. Покажем, что отображение
%$\varphi \colon G\to X$, заданное формулой $\varphi(h) = hx$,
%является искомой биекцией. Сюръективность (соответственно
%инъективность) отображения $\varphi$ следует из транзитивности
%(соответственно свободности) действия $G$ на~$X$.
%Условие~(\ref{eqn2}) следует из цепочки равенств $\varphi(gh) =
%(gh)x = g(hx) = g(\varphi(h))$.
%\end{proof}
%
%\begin{corollary}
%Действия группы $G$ на себе правыми и левыми сдвигами изоморфны.
%\end{corollary}
%
%\smallskip
%
%{\bf Теорема Кэли.} Всякая конечная группа $G$ порядка $n$ изоморфна
%подгруппе симметрической группы~$S_n$.
%
%\begin{proof}
%Рассмотрим действие группы $G$ на себе левыми сдвигами. Как мы
%знаем, это действие свободно, поэтому соответствующий гомоморфизм $a
%\colon G \to S(G) \simeq\nobreak S_n$ инъективен, т.\,е. $\Ker a =
%\lbrace e \rbrace$. Учитывая, что $G / \lbrace e \rbrace \cong G$,
%по теореме о гомоморфизме получаем $G \cong \Im a$.
%\end{proof}


\newpage

\section*{Лекция 6}

\medskip

{\it Три действия группы на себе. Теорема Кэли. Классы сопряжённости.
Кольца. Делители нуля, обратимые элементы, нильпотенты. Поля и алгебры.
Идеалы. }
% и факторкольца. Теорема о
%гомоморфизме. Центр алгебры матриц над полем. Простота алгебры
%матриц над полем.}

\medskip

Пусть $G$~--- произвольная группа. Рассмотрим три действия $G$ на
самой себе, т.\,е. положим $X=G$:

1) действие {\it умножениями слева (левыми сдвигами)}: $(g,h)\mapsto gh$;

2) действие {\it умножениями справа (правыми сдвигами)}: $(g,h)\mapsto hg^{-1}$;

3) действие {\it сопряжениями}: $(g,h)\mapsto ghg^{-1}$.

\begin{remark}
Для действий левыми и правыми сдвигами есть ровно одна орбита (сама $G$) и
стабилизатор любой точки тривиален, то есть $\St(x) = \{0\}$.
\end{remark}

\begin{definition}
Орбитой действия сопряжениями называются \textit{классами сопряженности}
\end{definition}

\begin{example}
В любой группе $G$ есть класс сопряженности $\{e\}$. \\Также, если $G$ коммутативна, то $\{x\}$ является классом сопряженности для всех $x$ из $G$.
\end{example}

\smallskip

{\bf Теорема Кэли.} Всякая конечная группа $G$ порядка $n$ изоморфна
подгруппе симметрической группы~$S_n$.

\begin{proof}
Рассмотрим действие группы $G$ на себе левыми сдвигами. Как мы
знаем, это действие свободно, поэтому соответствующий гомоморфизм $a
\colon G \to S(G) \simeq\nobreak S_n$ инъективен, т.\,е. $\Ker a =
\lbrace e \rbrace$. Учитывая, что $G / \lbrace e \rbrace \cong G$,
по теореме о гомоморфизме получаем $G \cong \Im a$.
\end{proof}

\medskip

Теперь приступим к изучению колец.
\begin{definition}
{\it Кольцом} называется множество $R$ с двумя бинарными операциями
\guillemotleft $+$\guillemotright{}~(сложение) и \guillemotleft
$\times$\guillemotright{}~(умножение), обладающими следующими
свойствами:

1) $(R,+)$ является абелевой группой (называемой {\it аддитивной
группой} кольца $R$);

2) выполнены {\it левая и правая дистрибутивности}, т.е.
$$
a(b+c)=ab+ac \quad \text{и} \quad (b+c)a=ba+ca \quad \text{для всех}
\ a,b,c\in R.
$$

В этом курсе мы рассматриваем только ассоциативные кольца с
единицей, поэтому дополнительно считаем, что выполнены ещё два
свойства:

3) $a(bc)=(ab)c$ для всех $a,b,c\in R$ (\textit{ассоциативность
умножения});

4) существует такой элемент $1\in R$ (называемый \textit{единицей}),
что
\begin{equation} \label{eq1}
a1 = 1a = a \ \text{для всякого} \ a \in R.
\end{equation}
\end{definition}

\begin{remark}
В произвольном кольце $R$ выполнены равенства
\begin{equation} \label{eq2}
a0 = 0a = 0 \ \text{для всякого} \ a \in R.
\end{equation}
В самом деле, имеем $a0 = a(0 + 0) = a0 + a0$, откуда $0 = a0$.
Аналогично устанавливается равенство $0a = 0$.
\end{remark}

\begin{remark}
Если кольцо $R$ содержит более одного элемента, то $0\ne 1$. Это
следует из соотношений~(\ref{eq1}) и~(\ref{eq2}).
\end{remark}

\textbf{Примеры колец:}
\begin{enumerate}[label=\textup{(\arabic*)},ref=\textup{\arabic*}]
\item \label{ex_num}
числовые кольца $\ZZ$, $\QQ$, $\RR$, $\CC$;

\item
кольцо $\ZZ_n$ вычетов по модулю~$n$;

\item \label{ex_mat}
кольцо $\Mat(n\times n, \RR)$ матриц с коэффициентами из~$\RR$;

\item \label{ex_pol}
кольцо $\RR[x]$ многочленов от переменной $x$ с коэффициентами
из~$\RR$;

\item
кольцо $\RR[[x]]$ \textit{формальных степенных рядов} от переменной
$x$ с коэффициентами из~$\RR$:
$$
\RR[[x]] := \lbrace \sum \limits_{i = 0}^\infty a_i x^i \mid a_i \in
\RR \rbrace;
$$

\item \label{ex_func}
кольцо $\FFF(M, \RR)$ всех функций из множества $M$ во
множество~$\RR$ с операциями поточечного сложения и умножения:
$$
(f_1 + f_2)(m) := f_1(m) + f_2(m); \quad (f_1f_2)(m) := f_1(m)
f_2(m) \quad \text{для всех} \quad f_1,f_2 \in \FFF(M, \RR), m \in
M.
$$
\end{enumerate}

\begin{remark}
В примерах (\ref{ex_mat})--(\ref{ex_func}) вместо $\RR$ можно брать
любое кольцо, в частности $\ZZ$, $\QQ$, $\CC$, $\ZZ_n$.
\end{remark}

\begin{remark}
Обобщая пример~(\ref{ex_pol}), можно рассматривать кольцо $\RR[x_1,
\ldots, x_n]$ многочленов от нескольких переменных $x_1, \ldots,
x_n$ с коэффициентами из~$\RR$.
\end{remark}

\begin{definition}
Кольцо $R$ называется {\it коммутативным}, если $ab=ba$ для всех
$a,b\in R$.
\end{definition}

Все перечисленные в примерах (\ref{ex_num})--(\ref{ex_func}) кольца,
кроме $\Mat(n\times n, \RR)$ при $n \geqslant 2$, коммутативны.

Пусть $R$~--- кольцо.

\begin{definition}
Элемент $a\in R$ называется {\it обратимым}, если найдётся такой
$b\in R$, что $ab=ba=1$. Такой элемент $b$ обозначается классическим образом как $a^{-1}$.
\end{definition}

\begin{remark}
Все обратимые элементы кольца $R$ образуют группу относительно
операции умножения.
\end{remark}

\begin{definition}
Элемент $a\in R$ называется \textit{левым} (соответственно
\textit{правым}) \textit{делителем нуля}, если $a \ne 0$ и найдётся
такой $b \in R$, $b\ne 0$, что $ab=0$ (соответственно $ba = 0$).
\end{definition}

\begin{remark}
В~случае коммутативных колец понятия левого и правого делителей нуля
совпадают, поэтому говорят просто о делителях нуля.
\end{remark}

\begin{remark}
Все делители нуля в $R$ необратимы: если $ab = 0$, $a \ne 0$, $b \ne
0$ и существует $a^{-1}$, то получаем $a^{-1}ab = a^{-1}0$, откуда
$b = 0$~--- противоречие.
\end{remark}

\begin{definition}
Элемент $a\in R$ называется {\it нильпотентом}, если $a \ne 0$ и
найдётся такое $m \in \NN$, что $a^m=0$.
\end{definition}

\begin{remark}
Всякий нильпотент в $R$ является делителем нуля: если $a \ne 0$,
$a^m = 0$ и число $m$ наименьшее с таким свойством, то $m \geqslant
2$ и $a^{m-1} \ne 0$, откуда $aa^{m-1} = a^{m-1}a = 0$.
\end{remark}

%\begin{definition}
%Элемент $a\in R$ называется {\it идемпотентом}, если $a^2=a$.
%\end{definition}

\begin{definition}
{\it Полем} называется коммутативное ассоциативное кольцо $K$ с
единицей, в котором всякий ненулевой элемент обратим.
\end{definition}

\begin{remark}
Тривиальное кольцо $\lbrace 0 \rbrace$ полем не считается, поэтому
$0 \ne 1$ в любом поле.
\end{remark}

\textbf{Примеры полей:} $\QQ$, $\RR$, $\CC$, $\ZZ_2$.

\begin{proposition}
Кольцо вычетов $\ZZ_n$ является полем тогда и только тогда, когда
$n$~--- простое число.
\end{proposition}

\begin{proof}
Если число $n$ составное, то $n = m k$, где $1 < m, k < n$. Тогда
$\overline{m} \overline{k} = \overline{n} = \overline{0}$.
Следовательно, $\overline k$ и $\overline m$~--- делители нуля в
$\ZZ_n$, ввиду чего не все ненулевые элементы там обратимы.

Если $n = p$~--- простое число, то возьмём произвольный ненулевой
вычет $\overline{a} \in \ZZ_p$ и покажем, что он обратим. Рассмотрим
вычеты
\begin{equation} \label{eq3}
\overline{1} \overline{a}, \overline{2} \overline{a}, \ldots,
\overline{(p-1)} \overline{a}.
\end{equation}
Если $\overline{r} \overline{a} = \overline{s} \overline{a}$ при $1
\leqslant r,s \leqslant p-1$, то число $(r - s)a$ делится на~$p$. В
силу взаимной простоты чисел $a$ и $p$ получаем, что число $r - s$
делится на~$p$. Тогда из условия $|r-s| \leqslant p - 2$ следует,
что $r = s$. Это рассуждение показывает, что все вычеты~(\ref{eq3})
попарно различны. Поскольку все они отличны от нуля, среди них
должна найтись единица: существует такое $b \in \lbrace 1, \ldots,
p-1 \rbrace$, что $\overline{b} \overline{a}=\overline{1}$. Это и
означает, что вычет $\overline{a}$ обратим.
\end{proof}

\begin{definition}
{\it Алгеброй} над полем $K$ (или кратко \textit{$K$-алгеброй})
называется множество $A$ с операциями сложения, умножения и
умножения на элементы поля $K$, обладающими следующими свойствами:

1) относительно сложения и умножения $A$ есть кольцо;

2) относительно сложения и умножения на элементы из $K$ множество
$A$ есть векторное пространство;

3 $(\lambda a)b=a(\lambda b)=\lambda(ab)$ для любых $\lambda\in K$ и
$a,b\in A$.

{\it Размерностью} алгебры $A$ называется её размерность как
векторного пространства над~$K$. (Обозначение: $\dim_K A$.)
\end{definition}

\textbf{Примеры.}

1) Алгебра матриц $\Mat(n\times n, K)$ над
произвольным полем~$K$. Её размерность равна $n^2$.

2) Алгебра $K[x]$ многочленов от переменной $x$ над произвольным
полем~$K$. Её размерность равна~$\infty$.

3) $K, F$ --- поля, $K \subset F$, $F$ --- алгебра над $K$. \\
Если это $\RR \subset \CC$, то $\dim_\RR\CC = 2$.\\
Если это $\QQ \subset \RR$, то $\dim_\QQ\RR = \infty$.

\begin{definition}
\textit{Подкольцом} кольца $R$ называется всякое подмножество $R'
\subseteq R$, замкнутое относительно операций сложения и умножения
(т.\,е. $a + b \in R'$ и $ab \in R'$ для всех $a,b \in R'$) и
являющееся кольцом относительно этих операций. \textit{Подполем}
называется всякое подкольцо, являющееся полем.
\end{definition}

Например, $\ZZ$ является подкольцом в~$\QQ$, а~скалярные матрицы
образуют подполе в кольце $\Mat(n \times n, \RR)$.

\begin{remark}
Если $K$~--- подполе поля~$F$, то $F$ является алгеброй над~$K$.
Так, поле $\CC$ является бесконечномерной алгеброй над~$\QQ$, тогда
как над $\RR$ имеет размерность~$2$.
\end{remark}

\begin{definition}
\textit{Подалгеброй} алгебры $A$ (над полем~$K$) называется всякое
подмножество $A' \subseteq A$, замкнутое относительно всех трёх
имеющихся в $A$ операций (сложения, умножения и умножения на
элементы из~$K$) и являющееся алгеброй (над~$K$) относительно этих
операций.
\end{definition}

Легко видеть, что подмножество $A' \subseteq A$ является алгеброй
тогда и только тогда, когда оно является одновременно подкольцом и
векторным подпространством в~$A$.

Гомоморфизмы колец, алгебр определяются естественным образом как
отображения, сохраняющие все операции.

\begin{exercise}
Сформулируйте точные определения гомоморфизма колец и гомоморфизма
алгебр.
\end{exercise}

\begin{definition}
\textit{Изоморфизмом} колец, алгебр называется всякий гомоморфизм,
являющийся биекцией.
\end{definition}

В теории групп нормальные подгруппы обладают тем свойством, что по
ним можно \guillemotleft факторизовать\guillemotright{}. В~этом
смысле аналогами нормальных подгрупп в теории колец служат идеалы.

\begin{definition}
Подмножество $I$ кольца $R$ называется (\textit{двусторонним}) {\it
идеалом}, если оно является подгруппой по сложению и $ra\in I$,
$ar\in I$ для любых $a\in I$, $r\in R$.
\end{definition}

\begin{remark}
В~некоммутативных кольцах рассматривают также левые и правые идеалы.
\end{remark}

В каждом кольце $R$ есть {\it несобственные} идеалы $I=0$ и $I=R$.
Все остальные идеалы называются {\it собственными}.

\begin{exercise}
Пусть $R$~--- коммутативное кольцо. С~каждым элементом $a \in R$
связан идеал $(a) := \{ ra \mid r \in R \}$.
\end{exercise}

\begin{definition}
Идеал $I$ называется {\it главным}, если существует такой элемент
$a\in R$, что $I=(a)$. (В~этой ситуации говорят, что $I$ порождён
элементом~$a$.)
\end{definition}

\textbf{Пример.} В~кольце $\ZZ$ подмножество $k \ZZ$ ($k \in \ZZ$)
является главным идеалом, порождённым элементом~$k$. Более того, все
идеалы в $\ZZ$ являются главными.

\begin{remark}
Главный идеал $(a)$ является несобственным тогда и только тогда,
когда $a=0$ или $a$ обратим.
\end{remark}

Более общо, с каждым подмножеством $S \subseteq R$ связан идеал
$$
(S) := \{ r_1 a_1 + \ldots + r_k a_k \mid a_i \in S, r_i \in R,
k\in\NN\}.
$$
(Проверьте, что это действительно идеал!) Это наименьший по
включению идеал в~$R$, содержащий подмножество~$S$. В~этой ситуации
говорят, что идеал $I=(S)$ порождён подмножеством~$S$.

%
%Вернёмся к случаю произвольного кольца $R$. Поскольку любой идеал
%$I$ является подгруппой абелевой группы $(R,+)$, мы можем
%рассмотреть факторгруппу $R/I$. Введём на ней умножение по формуле
%$$
%(a+I)(b+I) := ab + I.
%$$
%Покажем, что это определение корректно. Пусть элементы $a',b' \in R$
%таковы, что $a' + I = a + I$ и $b' + I = b + I$. Проверим, что $a'b'
%+ I = ab + I$. Заметим, что $a' = a + x$ и $b' = b + y$ для
%некоторых $x, y \in I$. Тогда
%$$
%a'b' + I = (a + x)(b + y) + I = ab + ay + xb + xy + I = ab + I,
%$$
%поскольку $ay, xb, xy \in I$ в силу определения идеала.
%
%\begin{exercise}
%Проверьте, что множество $R/I$ является кольцом относительно
%имеющейся там операции сложения и только что введённой операции
%умножения.
%\end{exercise}
%
%\begin{definition}
%Кольцо $R/I$ называется {\it факторкольцом} кольца $R$ по
%идеалу~$I$.
%\end{definition}
%
%\textbf{Пример.} $\ZZ / n \ZZ = \ZZ_n$.
%
%Пусть $\varphi\colon R\to R'$~--- гомоморфизм колец. Тогда
%определены его ядро $\Ker \varphi = \lbrace r \in R \mid \varphi(r)
%= 0 \rbrace$ и образ $\Im \varphi = \lbrace \varphi(r) \mid r \in R
%\rbrace \subseteq R'$.
%
%\begin{lemma}
%Ядро $\Ker \varphi$ является идеалом в~$R$.
%\end{lemma}
%
%\begin{proof}
%Так как $\varphi$~--- гомоморфизм абелевых групп, то $\Ker \varphi$
%является подгруппой в $R$ по сложению. Покажем теперь, что $ra \in
%\Ker \varphi$ и $ar \in \Ker \varphi$ для произвольных элементов $a
%\in \Ker \varphi$ и $r \in R$. Имеем $\varphi(ra) = \varphi(r)
%\varphi(a) = \varphi(r) 0 = 0$, откуда $ra \in \Ker \varphi$.
%Аналогично получаем $ar \in \Ker \varphi$.
%\end{proof}
%
%\begin{exercise}
%Проверьте, $\Im \varphi$~--- подкольцо в~$R'$.
%\end{exercise}
%
%\smallskip
%
%{\bf Теорема о гомоморфизме для колец.}\ Пусть $\varphi\colon R\to
%R'$~-- гомоморфизм колец. Тогда имеет место изоморфизм
%$$
%R/\Ker\,\varphi\cong\Im\varphi.
%$$
%
%\smallskip
%
%\begin{proof}
%Положим для краткости $I = \Ker \varphi$ и рассмотрим отображение
%$$
%\pi \colon R/I \to \Im \varphi, \quad a+I \mapsto \varphi(a).
%$$
%Из доказательства теоремы о гомоморфизме для групп следует, что
%отображение $\pi$ корректно определено и является изоморфизмом
%абелевых групп (по сложению). Покажем, что $\pi$~--- изоморфизм
%колец. Для этого остаётся проверить, что $\pi$ сохраняет операцию
%умножения:
%$$
%\pi((a+I)(b+I)) = \pi(ab+I) = \varphi(ab) = \varphi(a) \varphi(b) =
%\pi(a+I) \pi(b+I).
%$$
%\end{proof}
%
%\begin{example}
%Пусть $R = \FFF(M, \RR)$. Зафиксируем произвольную точку $m_0 \in M$
%и рассмотрим гомоморфизм $\varphi \colon R \to \RR$, $f \mapsto
%f(m_0)$. Ясно, что гомоморфизм $\varphi$ сюръективен. Его ядром
%является идеал $I$ всех функций, обращающихся в нуль в точке $m_0$.
%По теореме о гомоморфизме получаем $R / I \cong \RR$.
%\end{example}
%
%\begin{definition}
%Кольцо $R$ называется {\it простым}, если в нём нет собственных
%(двусторонних) идеалов.
%\end{definition}
%
%\textbf{Пример.} Всякое поле является простым кольцом.
%
%\begin{definition}
%\textit{Центром} алгебры $A$ над полем $K$ называется её
%подмножество
%$$
%Z(A) = \{ a \in A \mid ab = ba \ \text{для всех} \ b \in A \}.
%$$
%\end{definition}
%
%\begin{theorem}
%Пусть $K$~--- поле, $n$~--- натуральное число и $A = \Mat(n \times
%n, K)$~--- алгебра квадратных матриц порядка~$n$ над полем~$K$.
%
%\textup{(1)} $Z(A) = \lbrace \lambda E \mid \lambda \in K \rbrace$,
%где $E$~--- единичная матрица \textup(в частности, $Z(A)$~---
%одномерное подпространство в~$A$\textup);
%
%\textup{(2)} алгебра $A$ проста \textup(как кольцо\textup).
%\end{theorem}
%
%\begin{proof}
%Для каждой пары индексов $i,j \in \lbrace 1, \ldots, n \rbrace$
%обозначим через $E_{ij}$ соответствующую \textit{матричную
%единицу}~--- такую матрицу, в которой на $(i,j)$-месте стоит
%единица, а на всех остальных местах~--- нули. Непосредственная
%проверка показывает, что
%$$
%E_{ij}E_{kl} =
%\begin{cases}
%E_{il}, & \ \text{если} \ j = k;\\
%0, & \ \text{если} \ j \ne k.
%\end{cases}
%$$
%Заметим, что матричные единицы образуют базис в~$A$ и всякая матрица
%$X = (x_{kl})$ представима в виде $X = \sum \limits_{k,l = 1}^n
%x_{kl} E_{kl}$.
%
%(1) Пусть матрица $X = \sum \limits_{k,l = 1}^n x_{kl} E_{kl}$ лежит
%в $Z(A)$. Тогда $X$ коммутирует со всеми матричными единицами.
%Выясним, что означает условие $XE_{ij} = E_{ij}X$. Имеем
%$$
%XE_{ij} = (\sum \limits_{k,l = 1}^n x_{kl} E_{kl})E_{ij} = \sum
%\limits_{k = 1}^n x_{ki}E_{kj}; \qquad E_{ij}X = E_{ij}(\sum
%\limits_{k,l = 1}^n x_{kl} E_{kl}) = \sum \limits_{l = 1}^n
%x_{jl}E_{il}.
%$$
%Сравнивая правые части двух равенств, получаем $x_{ii} = x_{jj}$,
%$x_{ki}=0$ при $k \ne i$ и $x_{jl}=0$ при $j \ne l$. Поскольку эти
%равенства имеют место при любых значениях $i,j$, мы получаем, что
%матрица $X$ скалярна, т.\,е. $X = \lambda E$ для некоторого $\lambda
%\in K$. С~другой стороны, ясно, что всякая скалярная матрица лежит в
%$Z(A)$.
%
%(2) Пусть $I$~--- двусторонний идеал алгебры~$A$. Если $I \ne
%\lbrace 0 \rbrace$, то $I$ содержит ненулевую матрицу~$X$. Покажем,
%что тогда $I = A$. Пусть индексы $k,l$ таковы, что $x_{kl} \ne 0$.
%Тогда
%$$
%E_{ik} X E_{lj} = E_{ik}(\sum_{p,q = 1}^n x_{pq} E_{pq}) E_{lj} =
%E_{ik} \sum \limits_{p = 1}^n x_{pl}E_{pj} = x_{kl} E_{ij} \in I.
%$$
%Домножая $x_{kl}E_{ij}$ на скалярную матрицу $(x_{kl})^{-1}E$, мы
%получаем, что $E_{ij} \in I$. Из произвольности выбора $i,j$
%следует, что все матричные единицы лежат в~$I$. Отсюда $I = A$, что
%и требовалось.
%\end{proof}


\newpage

\section*{Лекция~7}

\medskip

{\it Факторкольца. Теорема о
гомоморфизме колец. Евклидовы кольца, кольца главных идеалов и факториальные
кольца.}

\medskip



Вернёмся к случаю произвольного кольца $R$. Поскольку любой идеал
$I$ является подгруппой абелевой группы $(R,+)$, мы можем
рассмотреть факторгруппу $R/I$. Введём на ней умножение по формуле
$$
(a+I)(b+I) := ab + I.
$$
Покажем, что это определение корректно. Пусть элементы $a',b' \in R$
таковы, что $a' + I = a + I$ и $b' + I = b + I$. Проверим, что $a'b'
+ I = ab + I$. Заметим, что $a' = a + x$ и $b' = b + y$ для
некоторых $x, y \in I$. Тогда
$$
a'b' + I = (a + x)(b + y) + I = ab + ay + xb + xy + I = ab + I,
$$
поскольку $ay, xb, xy \in I$ в силу определения идеала.

\begin{exercise}
Проверьте, что множество $R/I$ является кольцом относительно
имеющейся там операции сложения и только что введённой операции
умножения.
\end{exercise}

\begin{definition}
Кольцо $R/I$ называется {\it факторкольцом} кольца $R$ по
идеалу~$I$.
\end{definition}

\textbf{Пример.} $\ZZ / n \ZZ = \ZZ_n$.

Пусть $\varphi\colon R\to R'$~--- гомоморфизм колец. Тогда
определены его ядро $\Ker \varphi = \lbrace r \in R \mid \varphi(r)
= 0 \rbrace$ и образ $\Im \varphi = \lbrace \varphi(r) \mid r \in R
\rbrace \subseteq R'$.

\begin{lemma}
Ядро $\Ker \varphi$ является идеалом в~$R$.
\end{lemma}

\begin{proof}
Так как $\varphi$~--- гомоморфизм абелевых групп, то $\Ker \varphi$
является подгруппой в $R$ по сложению. Покажем теперь, что $ra \in
\Ker \varphi$ и $ar \in \Ker \varphi$ для произвольных элементов $a
\in \Ker \varphi$ и $r \in R$. Имеем $\varphi(ra) = \varphi(r)
\varphi(a) = \varphi(r) 0 = 0$, откуда $ra \in \Ker \varphi$.
Аналогично получаем $ar \in \Ker \varphi$.
\end{proof}

\begin{exercise}
Проверьте, $\Im \varphi$~--- подкольцо в~$R'$.
\end{exercise}

\smallskip

{\bf Теорема о гомоморфизме для колец.}\ Пусть $\varphi\colon R\to
R'$~-- гомоморфизм колец. Тогда имеет место изоморфизм
$$
R/\Ker\,\varphi\cong\Im\varphi.
$$

\smallskip

\begin{proof}
Положим для краткости $I = \Ker \varphi$ и рассмотрим отображение
$$
\pi \colon R/I \to \Im \varphi, \quad a+I \mapsto \varphi(a).
$$
Из доказательства теоремы о гомоморфизме для групп следует, что
отображение $\pi$ корректно определено и является изоморфизмом
абелевых групп (по сложению). Покажем, что $\pi$~--- изоморфизм
колец. Для этого остаётся проверить, что $\pi$ сохраняет операцию
умножения:
$$
\pi((a+I)(b+I)) = \pi(ab+I) = \varphi(ab) = \varphi(a) \varphi(b) =
\pi(a+I) \pi(b+I).
$$
\end{proof}

\begin{example}\ 
\begin{enumerate}
\item Пусть $R = \FFF(M, \RR)$. Зафиксируем произвольную точку $m_0 \in M$
и рассмотрим гомоморфизм $\varphi \colon R \to \RR$, $f \mapsto
f(m_0)$. Ясно, что гомоморфизм $\varphi$ сюръективен. Его ядром
является идеал $I$ всех функций, обращающихся в нуль в точке $m_0$.
По теореме о гомоморфизме получаем $R / I \cong \RR$.
\item Рассмотрим отображение $\varphi \colon \RR[x] \to \CC$, $f \mapsto f(i)$. Очевидно, что $\varphi$ --- гомоморфизм, причем сюръективный. Если функция принадлежит ядру $\varphi$, то есть $f(i) = 0$, то $(x - i) \mid f$ в кольце $\CC[x]$. Но и сопряженный к корню также будет являться корнем многочлена, так что дополнительно $(x + i) \mid f$. Итого, получаем, что $f \in (x - i)(x + i) = (x^2 + 1)$ и, соответственно, $\Ker \varphi  \subseteq (x^2 + 1)$. В обратную сторону включение тем более очевидно. Далее, по теореме о гомоморфизме получаем $\RR[x] / (x^2 + 1) \cong \CC$.
\end{enumerate}
\end{example}

%\begin{definition}
%Кольцо $R$ называется {\it простым}, если в нём нет собственных
%(двусторонних) идеалов.
%\end{definition}
%
%\textbf{Пример.} Всякое поле является простым кольцом.
%
%\begin{definition}
%\textit{Центром} алгебры $A$ над полем $K$ называется её
%подмножество
%$$
%Z(A) = \{ a \in A \mid ab = ba \ \text{для всех} \ b \in A \}.
%$$
%\end{definition}
%
%\begin{theorem}
%Пусть $K$~--- поле, $n$~--- натуральное число и $A = \Mat(n \times
%n, K)$~--- алгебра квадратных матриц порядка~$n$ над полем~$K$.
%
%\textup{(1)} $Z(A) = \lbrace \lambda E \mid \lambda \in K \rbrace$,
%где $E$~--- единичная матрица \textup(в частности, $Z(A)$~---
%одномерное подпространство в~$A$\textup);
%
%\textup{(2)} алгебра $A$ проста \textup(как кольцо\textup).
%\end{theorem}
%
%\begin{proof}
%Для каждой пары индексов $i,j \in \lbrace 1, \ldots, n \rbrace$
%обозначим через $E_{ij}$ соответствующую \textit{матричную
%единицу}~--- такую матрицу, в которой на $(i,j)$-месте стоит
%единица, а на всех остальных местах~--- нули. Непосредственная
%проверка показывает, что
%$$
%E_{ij}E_{kl} =
%\begin{cases}
%E_{il}, & \ \text{если} \ j = k;\\
%0, & \ \text{если} \ j \ne k.
%\end{cases}
%$$
%Заметим, что матричные единицы образуют базис в~$A$ и всякая матрица
%$X = (x_{kl})$ представима в виде $X = \sum \limits_{k,l = 1}^n
%x_{kl} E_{kl}$.
%
%(1) Пусть матрица $X = \sum \limits_{k,l = 1}^n x_{kl} E_{kl}$ лежит
%в $Z(A)$. Тогда $X$ коммутирует со всеми матричными единицами.
%Выясним, что означает условие $XE_{ij} = E_{ij}X$. Имеем
%$$
%XE_{ij} = (\sum \limits_{k,l = 1}^n x_{kl} E_{kl})E_{ij} = \sum
%\limits_{k = 1}^n x_{ki}E_{kj}; \qquad E_{ij}X = E_{ij}(\sum
%\limits_{k,l = 1}^n x_{kl} E_{kl}) = \sum \limits_{l = 1}^n
%x_{jl}E_{il}.
%$$
%Сравнивая правые части двух равенств, получаем $x_{ii} = x_{jj}$,
%$x_{ki}=0$ при $k \ne i$ и $x_{jl}=0$ при $j \ne l$. Поскольку эти
%равенства имеют место при любых значениях $i,j$, мы получаем, что
%матрица $X$ скалярна, т.\,е. $X = \lambda E$ для некоторого $\lambda
%\in K$. С~другой стороны, ясно, что всякая скалярная матрица лежит в
%$Z(A)$.
%
%(2) Пусть $I$~--- двусторонний идеал алгебры~$A$. Если $I \ne
%\lbrace 0 \rbrace$, то $I$ содержит ненулевую матрицу~$X$. Покажем,
%что тогда $I = A$. Пусть индексы $k,l$ таковы, что $x_{kl} \ne 0$.
%Тогда
%$$
%E_{ik} X E_{lj} = E_{ik}(\sum_{p,q = 1}^n x_{pq} E_{pq}) E_{lj} =
%E_{ik} \sum \limits_{p = 1}^n x_{pl}E_{pj} = x_{kl} E_{ij} \in I.
%$$
%Домножая $x_{kl}E_{ij}$ на скалярную матрицу $(x_{kl})^{-1}E$, мы
%получаем, что $E_{ij} \in I$. Из произвольности выбора $i,j$
%следует, что все матричные единицы лежат в~$I$. Отсюда $I = A$, что
%и требовалось.
%\end{proof}


Далее в этой лекции всюду предполагается, что $R$~--- коммутативное кольцо
без делителей нуля.

\begin{definition}
Говорят, что элемент $b \in R$ {\it делит} элемент $a\in R$ ($b$~---
\textit{делитель}~$a$, $a$ \textit{делится} на~$b$; пишут $b \,|\,
a$) если существует элемент $c\in R$, для которого $a=bc$.
\end{definition}

\begin{definition}
Два элемента $a, b \in R$ называются {\it ассоциированными}, если
$a=bc$ для некоторого обратимого элемента $c$ кольца~$R$.
\end{definition}

\begin{remark}
Легко видеть, что отношение ассоциированности является отношением
эквивалентности на кольце~$R$.
\end{remark}

\begin{definition}
Кольцо $R$ без делителей нуля, не являющееся полем, называется {\it
евклидовым}, если существует функция
$$
N\colon R\setminus\{0\} \to \ZZ_{\geqslant 0}
$$
(называемая {\it нормой}), удовлетворяющая следующим условиям:

1) $N(ab) \geqslant N(a)$ для всех $a, b \in R \setminus \{0\}$;

2) для любых $a, b \in R$, $b \ne 0$, существуют такие $q,r\in R$,
что $a = qb + r$ и либо $r = 0$, либо $N(r) < N(b)$.
\end{definition}

Неформально говоря, условие 2) означает возможность \guillemotleft
деления с остатком\guillemotright{} в кольце~$R$.

\textbf{Примеры евклидовых колец:}

1) $\ZZ$ с нормой $N(a) = |a|$;

2) $K[x]$ (где $K$~--- произвольное поле) с нормой $N(f) = \deg f$.

\begin{lemma} \label{lemma_first}
Пусть $R$~--- евклидово кольцо и $a,b \in R \setminus \lbrace 0
\rbrace$. Равенство $N(ab) = N(a)$ выполнено тогда и только тогда,
когда $b$ обратим.
\end{lemma}

\begin{proof}
Если $b$ обратим, то $N(a)\leqslant N(ab)\leqslant
N(abb^{-1})=N(a)$, откуда $N(ab)=N(a)$.

Пусть теперь $N(ab) = N(a)$. Разделим $a$ на $ab$ с остатком: $a =
qab + r$, где либо $r = 0$, либо $N(r) < N(ab)$. Если $r \ne 0$, то
с учётом равенства $r = a(1-qb)$ имеем $N(a) \leqslant N(a(1-qb)) =
N(r) < N(ab) = N(a)$~--- противоречие. Значит, $r = 0$ и $a = qab$,
откуда $a(1 - qb) = 0$. Так как в $R$ нет делителей нуля и $a \ne
0$, то $1 - qb = 0$, откуда $qb = 1$, т.\,е. $b$ обратим.
\end{proof}


\begin{definition}
Кольцо $R$ называется \textit{кольцом главных идеалов}, если всякий
идеал в $R$ является главным.
\end{definition}

\begin{theorem} \label{thm_er_rpi}
Всякое евклидово кольцо $R$ является кольцом главных идеалов.
\end{theorem}

\begin{proof}
Пусть $I$~--- произвольный идеал в~$R$. Если $I = \lbrace 0
\rbrace$, то $I = (0)$ и поэтому $I$ является главным. Далее
считаем, что $I \ne \lbrace 0 \rbrace$. Пусть $a \in I \setminus
\lbrace 0 \rbrace$~--- элемент с наименьшей нормой. Тогда главный
идеал $(a)$ содержится в~$I$. Предположим, что какой-то элемент $b
\in I$ не лежит в~$(a)$, т.\,е. не делится на~$a$. Тогда разделим
$b$ на $a$ с остатком: $b = qa + r$, где $r \ne 0$ и $N(r) < N(a)$.
Так как $r = b - aq$, то $r \in I$, что в силу неравенства $N(r) <
N(a)$ противоречит нашему выбору элемента~$a$.
\end{proof}

\begin{definition}
{\it Наибольшим общим делителем} элементов $a$ и $b$ кольца $R$
называется их общий делитель, который делится на любой другой их
общий делитель. Он обозначается $(a,b)$.
\end{definition}

\begin{remark}
Если наибольший общий делитель двух элементов $a,b \in R$
существует, то он определён однозначно с точностью до
ассоциированности, т.\,е. умножения на обратимый элемент кольца~$R$.
\end{remark}

\begin{theorem} \label{thm_lcd}
Пусть $R$~--- евклидово кольцо и $a,b$~--- произвольные элементы.
Тогда:

\textup{(1)} существует наибольший общий делитель $(a,b)$;

\textup{(2)} существуют такие элементы $u,v \in R$, что $(a,b) = ua
+ vb$.
\end{theorem}

\begin{proof}\ \\
\underline{Способ 1}:  утверждение (1) получается применением
(прямого хода) алгоритма Евклида, а~утверждение~(2)~--- применением
обратного хода в алгоритме Евклида.

\underline{Способ 2}: рассмотрим идеал $I = (a, b)$. Так как $R$ --- кольцо главных идеалов, то существует такой элемент $d \in R$, что $I = (d)$ и существуют $x, y \in R$ такие, что 
$$
d = ax + dy. \qquad (*)
$$
Покажем, что $d = (a, b)$. Для начала, так как $a$ и $b$ лежат в идеале $I = (d)$, то они оба делятся на $d$, то есть $d$ является одним из их делителей. А из равенства $(*)$ ясно, что любой другой общий делитель $a$ и $b$ будет также делиться на $d$. Итого, $d$ --- наибольший общий делитель.
\end{proof}

\begin{definition}
Ненулевой необратимый элемент $p$ кольца $R$ называется {\it
простым}, если он не может быть представлен в виде $p = a b$, где
$a, b \in R$~--- необратимые элементы.
\end{definition}

\begin{remark}
Простые элементы в кольце многочленов $K[x]$ над полем $K$ принято
называть {\it неприводимыми многочленами}.
\end{remark}

\begin{lemma} \label{ll}
Если простой элемент $p$ евклидова кольца $R$ делит произведение
$a_1a_2\ldots a_n$, то он делит один из сомножителей.
\end{lemma}

\begin{proof}
Индукция по~$n$. Пусть $n=2$ и предположим, что $p$ не делит~$a_1$.
Тогда $(p, a_1) = 1$ и по утверждению~(2) теоремы~\ref{thm_lcd}
найдутся такие элементы $u, v \in R$, что $1 = up + v a_1$. Умножая
обе части этого равенства на~$a_2$, получаем
$$
a_2 = upa_2 + v a_1a_2.
$$
Легко видеть, что $p$ делит правую часть последнего равенства,
поэтому $p$ делит и левую часть, т.\,е.~$a_2$.

При $n > 2$ применяем предыдущее рассуждение к $(a_1 \ldots
a_{n-1})a_n$ и пользуемся предположением индукции.
\end{proof}

\begin{definition}
Кольцо $R$ называется {\it факториальным}, если всякий его ненулевой
необратимый элемент \guillemotleft разложим на простые
множители\guillemotright{}, т.\,е. представим в виде произведения
(конечного числа) простых элементов, причём это представление
единственно с точностью до перестановки множителей и
ассоциированности.
\end{definition}

Более формально единственность разложения на простые множители
следует понимать так: если для элемента $a \in R$ есть два
представления
$$
a = p_1 p_2 \ldots p_n = q_1q_2 \ldots q_m,
$$
где все элементы $p_i, q_j$ простые, то $n = m$ и существует такая
подстановка $\sigma \in S_n$, что для каждого $i = 1,\ldots, n$
элементы $p_i$ и $q_{\sigma(i)}$ ассоциированы.

\begin{theorem} \label{thm_er_ufd}
Всякое евклидово кольцо $R$ является факториальным.
\end{theorem}

\begin{proof}[Доказательство \textup{состоит из двух шагов.}]~

\textit{Шаг}~1. Сначала докажем, что всякий ненулевой необратимый
элемент из $R$ разложим на простые множители. Предположим, что это
не так, и среди всех элементов, не разложимых на простые множители,
выберем элемент $a$ с наименьшей нормой. Тогда $a$ не может быть
простым (иначе он разложим в произведение, состоящее из одного
простого множителя), поэтому существует представление вида $a = bc$,
где $b,c \in R$~--- ненулевые необратимые элементы. Но тогда в силу
леммы~\ref{lemma_first} имеем $N(b) < N(a)$ и $N(c) < N(a)$, поэтому
элементы $b$ и $c$ разложимы на простые множители. Но тогда и $a$
разложим~--- противоречие.

\textit{Шаг}~2. Докажем теперь индукцией по~$n$, что если для
некоторого элемента $a \in R$ имеются два разложения
$$
a = p_1p_2\ldots p_n=q_1q_2\ldots q_m,
$$
где все элемнты $p_i$ и $q_j$ простые, то $m=n$ и после подходящей
перенумерации элементов $q_j$ окажется, что при всех $i = 1,\ldots,
n$ элемент $p_i$ ассоциирован с~$q_i$.

Если $n=1$, то $a = p_1$; тогда из определения простого элемента
следует, что $m = 1$ и тем самым $q_1 = p_1$. Пусть теперь $n > 1$.
Тогда элемент $p_1$ делит произведение $q_1 q_2 \ldots q_m$. По
лемме~\ref{ll} этот элемент делит некоторый~$q_i$, а значит,
ассоциирован с ним. Выполнив перенумерацию, можно считать, что $i =
1$ и $q_1 = cp_1$ для некоторого обратимого элемента $c \in R$. Так
как в $R$ нет делителей нуля, то мы можем сократить на~$p_1$, после
чего получится равенство
$$
p_2 p_3 \ldots p_n = (cq_2)q_3 \ldots q_m
$$
(заметьте, что элемент $cq_2$ прост!). Дальше используем
предположение индукции.
\end{proof}

Можно показать (см. листок с задачами к лекции~6), что при $n
\geqslant 2$ кольцо многочленов $K[x_1, \ldots, x_n]$ над
произвольным полем~$K$ не является кольцом главных идеалов, а
значит, по теореме~\ref{thm_er_rpi} это кольцо не является
евклидовым. Тем не менее, наша цель в оставшейся части этой
лекции~--- доказать, что кольцо $K[x_1, \ldots, x_n]$ факториально.

Начнём издалека. С~каждым (коммутативным) кольцом $R$ (без делителей
нуля) связано его \textit{поле отношений}~$K$. Элементами этого поля
являются дроби вида $\frac{a}{b}$, где $a,b \in R$ и $b\ne 0$, со
стандартными правилами отождествления ($\frac{a}{b} = \frac{c}{d}
\Leftrightarrow ad = bc$), сложения ($\frac{a}{b} + \frac{c}{d} =
\frac{ad+bc}{bd}$) и умножения ($\frac{a}{b}\frac{c}{d} =
\frac{ac}{bd}$). Кольцо $R$ реализуется как подкольцо в~$K$,
состоящее из всех дробей вида $\frac{a}{1}$.

\textbf{Модельный пример:} $\QQ$ есть поле отношений кольца~$\ZZ$.

Всякий гомоморфизм колец $\varphi \colon R \to R'$ индуцирует
гомоморфизм $\widetilde \varphi \colon R[x] \to R'[x]$
соответствующих колец многочленов, задаваемый по правилу
$$
a_n x^n + a_{n-1} x^{n-1} + \ldots a_1 x + a_0 \mapsto \varphi(a_n)
x^n + \varphi(a_{n-1}) x^{n-1} + \ldots \varphi(a_1) x +
\varphi(a_0).
$$
Вспомнив, как определяется умножение в кольце многочленов, легко
показать, что $\widetilde \varphi$ действительно является
гомоморфизмом.

В~частности, если $R$~--- кольцо и $K$~--- его поле частных, то
вложение $R \hookrightarrow K$ индуцирует вложение $R[x]
\hookrightarrow K[x]$, так что всякий многочлен с коэффициентами из
$R$ можно рассматривать как многочлен с коэффициентами из~$K$.

Пусть $R$~--- кольцо.

\begin{definition}
Многочлен $f(x)\in R[x]$ называется {\it примитивным}, если в $R$
нет необратимого элемента, который делит все коэффициенты многочлена
$f(x)$.
\end{definition}

{\bf Лемма Гаусса.}\ Если $R$~--- факториальное кольцо c полем
отношений $K$ и многочлен $f(x) \in R[x]$ разлагается в произведение
двух многочленов в кольце $K[x]$, то он разлагается в произведение
двух пропорциональных им многочленов в кольце~$R[x]$.

В доказательстве леммы Гаусса нам потребуются следующие факты.

\begin{exercise} \label{ex_1}
Пусть $R$~--- факториальное кольцо и $p \in R$~--- простой элемент.
Тогда в факторкольце $R/(p)$ нет делителей нуля.
\end{exercise}

\begin{exercise} \label{ex_2}
Пусть $R$~--- (коммутативное) кольцо (без делителей нуля). Тогда в
кольце многочленов $R[x]$ также нет делителей нуля.
\end{exercise}

\begin{proof}[Доказательство леммы Гаусса]
Пусть $f(x) = g(x)h(x)$, где $g(x), h(x)\in K[x]$. Так как кольцо
$R$ факториально, то для любого набора элементов из $R$ определены
наибольший общий делитель и наименьшее общее кратное. С~учётом этого
приведём все коэффициенты многочлена $g(x)$ к общему знаменателю,
после чего вынесем за скобку этот общий знаменатель и наибольший
общий делитель всех числителей. В результате в скобках останется
примитивный многочлен $g_1(x) \in R[x]$, а за скобками~--- некоторый
элемент из поля~$K$. Аналогичным образом найдём примитивный
многочлен $h_1(x) \in R[x]$, который пропорционален
многочлену~$h(x)$. Теперь мы можем записать $f(x)=\frac{u}{v}g_1(x)
h_1(x)$, где $u,v \in R$, $v \ne 0$ и без ограничения общности можно
считать $(u,v)=1$. Для завершения доказательства достаточно
показать, что элемент $v$ обратим (и тогда разложение $f(x) =
(uv^{-1}g_1(x))h_1(x)$ будет искомым).

Предположим, что $v$ необратим. Тогда найдётся простой элемент $p
\in R$, который делит~$v$. Рассмотрим гомоморфизм факторизации
$\varphi \colon R \to R/(p)$, $a \mapsto a + (p)$, и соответствующий
ему гомоморфизм колец многочленов $\widetilde \varphi \colon R[x]
\to (R/(p))[x]$. В кольце $R[x]$ у нас имеется равенство $vf(x) =
ug_1(x)h_1(x)$. Взяв образ обеих частей этого равенства при
гомоморфизме $\widetilde \varphi$, мы получим следующее равенство в
кольце $(R / (p))[x]$:
\begin{equation} \label{eqn}
\widetilde \varphi(v) \widetilde \varphi(f(x)) = \widetilde
\varphi(u) \widetilde \varphi(g_1(x)) \widetilde \varphi(h_1(x)).
\end{equation}
Поскольку $p$ делит~$v$, имеем $\widetilde \varphi(v) = 0$, поэтому
левая часть равенства~(\ref{eqn}) равна нулю. С другой стороны, из
условия $(u,v) = 1$ следует, что $\widetilde \varphi(u) \ne 0$, а из
примитивности многочленов $g_1(x)$ и $h_1(x)$ вытекает, что
$\widetilde \varphi(g_1(x)) \ne 0$ и $\widetilde \varphi(h_1(x)) \ne
0$. Таким образом, все три множителя в правой части
равенства~(\ref{eqn}) отличны от нуля. Из упражнений~\ref{ex_1}
и~\ref{ex_2} вытекает, что в кольце $(R / (p))[x]$ нет делителей
нуля, поэтому правая часть равенства~(\ref{eqn}) отлична от нуля, и
мы пришли к противоречию.
\end{proof}

\begin{corollary} \label{cc}
Если многочлен $f(x)\in R[x]$ может быть разложен в произведение
двух многочленов меньшей степени в кольце $K[x]$, то он может быть
разложен и в произведение двух многочленов меньшей степени в кольце
$R[x]$.
\end{corollary}

\begin{theorem}
Если кольцо $R$ факториально, то кольцо многочленов $R[x]$ также
факториально.
\end{theorem}

\begin{proof}
Следствие~\ref{cc} показывает, что простые элементы кольца
$R[x]$~--- это в точности элементы одного из следующих двух типов:

1) простые элементы кольца $R$ (рассматриваемые как многочлены
степени~$0$ в $R[x]$);

2) примитивные многочлены из $R[x]$, неприводимые над полем
отношений~$K$.

Ясно, что каждый многочлен из $R[x]$ разлагается в произведение
таких многочленов. Предположим, что какой-то элемент из $R[x]$ двумя
способами представим в виде такого произведения:
$$
a_1 \ldots a_n b_1(x) \ldots b_m(x) = a'_1 \ldots a'_k b'_1(x)
\ldots b'_l(x),
$$
где $a_i, a'_j$~--- простые элементы типа~1 и $b_i(x), b'_j(x)$~---
простые элементы типа~2.

Рассмотрим эти разложения в кольце $K[x]$. Как мы уже знаем из
теоремы~\ref{thm_er_ufd}, кольцо $K[x]$ факториально. Отсюда
следует, что $m = l$ и после подходящей перенумерации элементов
$b'_j(x)$ получается, что при всех $j = 1, \ldots, m$ элементы
$b_j(x)$ и $b'_j(x)$ ассоциированы в $K[x]$, а в силу примитивности
они ассоциированы и в $R[x]$. После сокращения всех таких элементов
у нас останутся два разложения на простые множители (какого-то)
элемента из~$R$. Но кольцо $R$ факториально, поэтому эти два
разложения совпадают с точностью до перестановки множителей и
ассоциированности.
\end{proof}

\begin{theorem}
Пусть $K$~--- произвольное поле. Тогда кольцо многочленов
$K[x_1,\ldots,x_n]$ факториально.
\end{theorem}

\begin{proof}
Воспользуемся индукцией по~$n$. При $n=1$ наше кольцо евклидово и по
теореме~\ref{thm_er_ufd} факториально. При $n > 1$ имеем $K[x_1,
\ldots, x_n] = K[x_1, \ldots, x_{n-1}][x_n]$, кольцо $K[x_1, \ldots,
x_{n-1}]$ факториально по предположению индукции и требуемый
результат следует из предыдущей теоремы.
\end{proof}

\begin{remark}
Несмотря на естественность условия единственности разложения на
простые множители, большинство колец не являются факториальными.
Например, таковым не является кольцо $\ZZ[\sqrt{-5}]$, состоящее из
всех комплексных чисел вида $a + b \sqrt{-5}$, где $a,b \in \ZZ$: в
этом кольце число $6$ разлагается на простые множители двумя
различными способами: $6 = 2 \cdot 3 = (1 + \sqrt{-5})(1 -
\sqrt{-5})$.
\end{remark}

\newpage

\section*{Лекция~8}

\medskip

{\it Элементарные симметрические многочлены. Основная теорема о
	симметрических многочленах. Лексикографический порядок. Теорема
	Виета. Дискриминант многочлена.}

\medskip

Вернемся ненадолго к теме прошлой лекции. Рассмотрим кольцо $R = K[x_1, \ldots, x_n]$, где 	$K$ --- поле. 
На семинарах разбиралось, что оно не является кольцом главных идеалов и, соответственно, евклидовым кольцом. Однако несмотря на это:

\textbf{Теорема.} Кольцо $R$ факториально.

Впрочем, доказывать эту теорему мы не будем.

Вернемся теперь к теме текущей лекции. Пусть $K$~--- произвольное поле.

\begin{definition}
	Многочлен $f(x_1,\ldots,x_n)\in K[x_1,\ldots,x_n]$ называется {\it
		симметрическим}, если
	$f(x_{\tau(1)},\ldots,x_{\tau(n)})=f(x_1,\ldots,x_n)$ для всякой
	перестановки $\tau \in S_n$.
\end{definition}

\textbf{Примеры:}

1) Многочлен $x_1x_2 + x_2x_3$ не является симметрическим, а вот многочлен $x_1x_2 + x_2x_3 + x_1x_3$ --- является.

2) {\it Степенные суммы} $s_k(x_1, \ldots, x_n) = x_1^k + x_2^k +
\ldots + x_n^k$ являются симметрическими многочленами.

3) {\it Элементарные симметрические многочлены}
$$
\sigma_1(x_1, \ldots, x_n) = x_1 + x_2 + \ldots + x_n;
$$
$$
\sigma_2(x_1, \ldots, x_n) = \sum \limits_{1 \leqslant i < j
	\leqslant n} x_i x_j;
$$
$$
..................................................
$$
$$
\sigma_k(x_1, \ldots, x_n) = \sum \limits_{1 \leqslant i_1 < i_2 <
	\ldots < i_k \leqslant n} x_{i_1} x_{i_2} \ldots x_{i_k};
$$
$$
..................................................
$$
$$
\sigma_n(x_1, \ldots, x_n) = x_1 x_2 \ldots x_n
$$
являются симметрическими.

5) Определитель Вандермонда
$$
V(x_1, \ldots, x_n) =
\begin{vmatrix}
1 & x_1 & x_1^2 & \ldots & x_1^{n-1} \\
1 & x_2 & x_2^2 & \ldots & x_2^{n-1} \\
\ldots & \ldots & \ldots & \ldots & \ldots \\
1 & x_n & x_n^2 & \ldots & x_n^{n-1}
\end{vmatrix} =
\prod \limits_{1 \leqslant i < j \leqslant n} (x_j - x_i)
$$
симметрическим многочленом не является (при перестановке индексов
умножается на её знак), а вот его квадрат уже является.

Основная цель этой лекции~--- понять, как устроены все
симметрические многочлены.

Легко видеть, что все симметрические многочлены образуют подкольцо
(и даже подалгебру) в $K[x_1, \ldots, x_n]$. В~частности, если
$F(y_1, \ldots, y_k)$~--- произвольный многочлен и $f_1(x_1, \ldots,
x_n)$, $\ldots$, $f_k(x_1, \ldots, x_n)$~--- симметрические
многочлены, то многочлен
$$
F(f_1(x_1, \ldots, x_n), \ldots, f_k(x_1, \ldots, x_n)) \in K[x_1,
\ldots, x_n]
$$
также является симметрическим. Мы покажем, что всякий симметрический
многочлен однозначно выражается через элементарные симметрические
многочлены.

\medskip

{\bf Основная теорема о симметрических многочленах.}\ Для всякого
симметрического многочлена $f(x_1, \ldots, x_n)$ существует и
единственен такой многочлен $F(y_1, \ldots, y_n)$, что
$$
f(x_1, \ldots, x_n) = F(\sigma_1(x_1, \ldots, x_n), \ldots,
\sigma_n(x_1, \ldots, x_n)).
$$

\textbf{Пример.} $s_2(x_1, \ldots, x_n) = x_1^2 + \ldots + x_n^2 =
(x_1 + \ldots + x_n)^2 - 2\sum \limits_{1 \leqslant i < j \leqslant
	n} x_i x_j = \sigma_1^2 - 2\sigma_2$, откуда $F(y_1, \ldots, y_n) =
y_1^2 - 2y_2$.


Доказательство этой теоремы потребует некоторой подготовки. Начнём с
того, что определим старший член многочлена от многих переменных.

Пусть $M_n$~--- множество всех одночленов от переменных $x_1,
\ldots, x_n$. Определим на $M_n$ {\it лексикографический порядок}
следующим образом:
$$
ax_1^{i_1}x_2^{i_2}\ldots x_n^{i_n} \prec bx_1^{j_1}x_2^{j_2}\ldots
x_n^{j_n} \quad \Leftrightarrow \quad \exists k: \: i_1=j_1,\ldots,
i_{k-1}=j_{k-1}, i_k<j_k.
$$
Например, $x_1^2x_3^9 \prec x_1^2x_2$.

\ \\
\textbf{Свойства:}

1) Лексикографический порядок обладает свойством
транзитивности: если $u,v,w \in M_n$, $u \prec v$ и $v \prec w$, то
$u \prec w$ (докажите это).

2) Если $u,v,w \in M_n$ и $u \prec v$, то $uw \prec vw$.

Свойство транзитивности лексикографического порядка позволяет
корректно определить следующее понятие.

\begin{definition}
	{\it Старшим членом} ненулевого многочлена $f(x_1,\ldots,x_n)$
	называется наибольший в лексикографическом порядке встречающий в нём
	одночлен. Обозначение: $L(f)$.
\end{definition}

\textbf{Примеры:}

1) $L(s_k) = x_1^k$;

2) $L(\sigma_k) = x_1x_2 \ldots x_k$.


{\bf Лемма о старшем члене.}\, Пусть $f(x_1, \ldots, x_n), g(x_1,
\ldots, x_n) \in K[x_1, \ldots, x_n]$~--- произвольные ненулевые
многочлены. Тогда $L(f g ) = L(f) L(g)$.

\begin{proof}
	Пусть $u$~--- какой-то одночлен многочлена~$f$ и $v$~--- какой-то
	одночлен многочлена~$g$. По определению старшего члена имеем
	\begin{equation} \label{inequalities}
		u \preccurlyeq L(f), \quad v \preccurlyeq L(g).
	\end{equation}
	Тогда $uv \preccurlyeq uL(g) \preccurlyeq L(f)L(g)$, т.\,е. $uv
	\preccurlyeq L(f) L(g)$. Более того, легко видеть, что $uv \prec
	L(f) L(g)$ тогда и только тогда, когда хотя бы одно из
	\guillemotleft неравенств\guillemotright{} (\ref{inequalities})
	является строгим. Отсюда следует, что после раскрытия скобок в
	произведении $fg$ одночлен $L(f)L(g)$ будет старше всех остальных
	возникающих одночленов. Ясно, что после приведения подобных членов
	этот одночлен сохранится и будет по-прежнему старше всех остальных
	одночленов, поэтому $L(f)L(g) = L(fg)$.
\end{proof}

\begin{lemma} \label{lemma_1}
	Если $ax_1^{k_1}x_2^{k_2}\ldots x_n^{k_n}$~--- старший член
	некоторого симметрического многочлена~$f(x_1, \ldots, x_n)$, то $k_1
	\geqslant k_2 \geqslant \ldots \geqslant k_n$.
\end{lemma}

\begin{proof}
	От противного. Пусть $k_i < k_{i+1}$ для некоторого~$i \in \lbrace
	1, \ldots, n-1 \rbrace$. Тогда, будучи симметрическим, многочлен $f$
	содержит одночлен $ax_1^{k_1} \ldots
	x_{i-1}^{k_{i-1}}x_i^{k_{i+1}}x_{i+1}^{k_i}x_{i+2}^{k_{i+2}} \ldots
	x_n^{k_n}$, который старше $L(f)$. Противоречие.
\end{proof}

\begin{lemma} \label{lemma_2}
	Пусть $k_1, \ldots, k_n$~--- целые неотрицательные числа. Если $k_1
	\geqslant k_2 \geqslant \ldots \geqslant k_n$, то существуют и
	единственны такие целые неотрицательные числа $l_1, l_2, \ldots,
	l_n$, что
	$$
	x_1^{k_1}x_2^{k_2}\ldots x_n^{k_n}=
	L(\sigma_1(x_1,\ldots,x_n)^{l_1}\sigma_2(x_1,\ldots,x_n)^{l_2}\ldots\sigma_n(x_1,\ldots,x_n)^{l_n}).
	$$
\end{lemma}

\begin{proof}
	С учётом леммы о старшем члене требуемое условие означает, что
	искомые числа $l_1, \ldots, l_n$ удовлетворяют системе
	$$
	\begin{cases}
	l_1 + l_2 + \ldots + l_n = k_1; \\
	\phantom{l_1 + {}} l_2 + \ldots + l_n = k_2;\\
	\phantom{l_1 + l_2 + .} \ldots\ldots\ldots\ldots\\
	\phantom{l_1 + l_s + \ldots + {}} l_n = k_n,
	\end{cases}
	$$
	из которой они легко находятся:
	\begin{align*}
		l_i &= k_i - k_{i+1} \quad \text{при} \ 1 \leqslant i \leqslant n-1; \\
		l_n &= k_n.
	\end{align*}
\end{proof}

\begin{proof}[Доказательство основной теоремы о симметрических многочленах]
	Пусть $f(x_1, \ldots, x_n)$~--- произвольный симметрический
	многочлен.
	
	Сначала докажем существование искомого многочлена~$F(y_1, \ldots,
	y_n)$. Если $f(x_1, \ldots, x_n)$~--- нулевой многочлен, то можно
	взять $F(y_1,\ldots, y_n) = 0$. Далее считаем, что $f(x_1, \ldots,
	x_n) \ne 0$. Пусть $L(f) = ax_1^{k_1} \ldots x_n^{k_n}$, $a \ne 0$.
	Тогда $k_1 \geqslant k_2 \geqslant \ldots \geqslant k_n$ в силу
	леммы~\ref{lemma_1}. По лемме~\ref{lemma_2} найдётся одночлен от
	элементарных симметрических многочленов
	$a\sigma_1^{l_1}\ldots\sigma_n^{l_n}$, старший член которого
	совпадает с~$L(f)$. Положим $f_1 := f -
	a\sigma_1^{l_1}\ldots\sigma_n^{l_n}$. Если $f_1 = 0$, то $f = a
	\sigma_1^{l_1} \ldots \sigma_n^{l_n}$ и искомым многочленом будет
	$F(y_1, \ldots, y_n) = ay_1^{l_1} \ldots y_n^{l_n}$. Если же $f_1
	\ne 0$, то $L(f_1) \prec L(f)$. Повторим ту же процедуру: вычтя из
	$f_1$ подходящий одночлен от $\sigma_1, \ldots, \sigma_n$, мы
	получим новый многочлен~$f_2$ со следующим свойством: либо $f_2 = 0$
	(и тогда мы получаем выражение $f$ через элементерные симметрические
	многочлены), либо $L(f_2) \prec L(f_1)$. Многократно повторяя эту
	процедуру, мы получим последовательность многочленов $f, f_1, f_2,
	\ldots$ со свойством $L(f) \succ L(f_1) \succ L(f_2) \succ \ldots$.
	Покажем, что процесс закончится, т.\,е. найдётся такое~$m$, что $f_m
	= 0$ (и тогда мы получим представление $f$ в виде многочлена от
	$\sigma_1, \ldots, \sigma_n$). Для этого заметим, что переменная
	$x_1$ входит в старший член каждого из многочленов $f_1, f_2,
	\ldots$ в степени, не превышающей $k_1$. Но в силу
	леммы~\ref{lemma_1} одночленов с таким условием имеется лишь
	конечное число, поэтому процесс не может продолжаться бесконечно.
	
	Теперь докажем единственность многочлена $F(y_1, \ldots, y_n)$.
	Предположим, что
	$$
	f(x_1, \ldots, x_n) = F(\sigma_1(x_1, \ldots, x_n), \ldots,
	\sigma_n(x_1, \ldots, x_n) = G(\sigma_1(x_1, \ldots, x_n), \ldots,
	\sigma_n(x_1, \ldots, x_n))
	$$
	для двух различных многочленов $F(y_1, \ldots, y_n), G(y_1, \ldots,
	y_n) \in K[y_1, \ldots, y_n]$. Тогда многочлен $$H(y_1, \ldots, y_n)
	:= F(y_1, \ldots, y_n) - G(y_1, \ldots, y_n)$$ является ненулевым,
	но $H(\sigma_1(x_1, \ldots, x_n), \ldots, \sigma_n(x_1, \ldots,
	x_n)) = 0$. Покажем, что такое невозможно. Пусть $H_1, \ldots,
	H_s$~--- все ненулевые одночлены в~$H$. Обозначим через $w_i$
	старший член многочлена $$H_i(\sigma_1(x_1, \ldots, x_n), \ldots,
	\sigma_n(x_1, \ldots, x_n)) \in K[x_1, \ldots, x_n].$$ В~силу
	леммы~\ref{lemma_2} среди одночленов $w_1, \ldots, w_s$ нет
	пропорциональных. Выберем из них старший в лексикографическом
	порядке. Он не может сократиться ни с одним членов в выражении
	$$
	H_1(\sigma_1(x_1, \ldots, x_n), \ldots, \sigma_n(x_1, \ldots, x_n))
	+ \ldots + H_s(\sigma_1(x_1, \ldots, x_n), \ldots, \sigma_n(x_1,
	\ldots, x_n)),
	$$
	поэтому $H(\sigma_1(x_1, \ldots, x_n), \ldots, \sigma_n(x_1, \ldots,
	x_n)) \ne 0$, и мы пришли к противоречию.
\end{proof}

На практике многочлен $F(y_1, \ldots, y_n)$ можно искать, повторяя
описанный в доказательстве алгоритм, однако он может потребовать
много вычислений. Более эффективным для нахождения многочлена
$F(y_1, \ldots, y_n)$ является метод неопределённых коэффициентов,
который планируется разобрать на семинарах.

{\bf Теорема Виета}. Пусть $\alpha_1,\ldots,\alpha_n$~--- корни
многочлена $x^n + a_{n-1}x^{n-1} + \dots + a_1x + a_0$. Тогда
$$
\sigma_k(\alpha_1,\ldots,\alpha_n)=(-1)^k a_{n-k}, \quad k = 1,
\ldots, n.
$$

\begin{proof}
	Достаточно приравнять коэффициенты при $x^{n-k}$ в левой и правой
	частях равенства
	$$
	x^n+a_{n-1}x^{n-1}+\dots+a_1x+a_0=(x-\alpha_1)(x-\alpha_2)\ldots
	(x-\alpha_n).
	$$
\end{proof}

Из теоремы Виета и основной теоремы о симметрических многочленах
следует, что мы можем выразить значение любого симметрического
многочлена от корней данного многочлена через коэффициенты, не
находя самих корней.

\begin{definition}
	{\it Дискриминантом} многочлена $h(x)=a_nx^n+\ldots+a_1x+a_0$ с
	корнями $\alpha_1, \ldots, \alpha_n$ называется выражение
	$$
	D(h) = a_n^{2n-2} \prod \limits_{1 \leqslant i < j \leqslant n}
	(\alpha_i - \alpha_j)^2.
	$$
\end{definition}

\begin{remark}
	Дискриминант $D(h)$ является симметрическим многочленом от
	$\alpha_1, \ldots, \alpha_n$, а значит, в соответствии с
	вышесказанным он является многочленом от коэффициентов $a_n,
	a_{n-1}, \ldots, a_0$.
\end{remark}

\begin{remark}
	Непосредствено из определения следует, что $D(h) = 0$ тогда и только
	тогда, когда многочлен $h$ имеет кратный корень.
\end{remark}
%
%\begin{example}
%Пусть $h(x)=ax^2+bx+c$. Тогда
%$$
%D(h)=a^2(\alpha_2-\alpha_1)^2=a^2((\alpha_1+\alpha_2)^2-4\alpha_1\alpha_2)=
%a^2((-b/a)^2-4c/a)=b^2-4ac.
%$$
%\end{example}
%
%\bigskip
%
%\textbf{Понятие о базисе Грёбнера\footnote{Это необязательный
%материал, в программу экзамена он не войдёт.}.} Рассмотрим в кольце
%$K[x_1, \ldots, x_n]$ идеал~$I$, порождённый многочленами $f_1,
%\ldots, f_k$. Как выяснить алгоритмически, принадлежит ли данный
%многочлен $f \in K[x_1, \ldots, x_n]$ идеалу~$I$? Другими словами,
%представим ли многочлен $f$ в виде $f_1h_1+\ldots+f_kh_k$ для
%некоторых многочленов $h_1, \ldots, h_k \in K[x_1, \ldots, x_n]$?
%При $k=1$ или $n=1$ ответить на этот вопрос легко, в общем случае
%сложнее.
%
%{\it Базисом Грёбнера} идеала $I$ в кольце $K[x_1,\ldots,x_n]$
%называется такой набор многочленов $g_1, \ldots, g_m \in I$, что для
%всякого $g \in I$ старший член $g$ делится на старший член одного из
%$g_i$. Оказывается, базис Грёбнера данного идеала всегда существует
%и его можно эффективно построить, исходя из набора порождающих $f_1,
%\ldots, f_k$ (алгоритм Бухбергера и его модификации). Имея такой
%базис, мы можем проводить редукции, т.\,е. вычитать из данного
%многочлена $f$ один из элементов базиса Грёбнера, умноженный на
%некоторый многочлен так, чтобы старший член сократился. Осуществляя
%редукции, мы за конечное число шагов выясним, лежит ли $f$ в идеале.
%
%Базисы Грёбнера позволяют алгоритмически решать и многие другие
%задачи, связанные с системами полиномиальных уравнений.


\newpage

\section*{Лекция~9}

\medskip

{\it Примеры полей. Характеристика поля. Расширения полей,
	алгебраические и трансцендентные элементы. Минимальные многочлен.
	Конечное расширение и его степень. Присоединение корня многочлена.
	Поле разложения многочлена: существование и единственность.}

\medskip

Мы знаем не так много примеров полей. Это бесконечные поля $\QQ$,
$\RR$, $\CC$ и конечные поля $\ZZ_p$, где $p$~--- простое число.
Конструкция поля отношений позволяет строить новые поля из уже
имеющихся. А именно, если $K$~--- произвольное поле, то можно
рассмотреть поле отношений $K(x)$ кольца многочленов $K[x]$ (это
поле называется \textit{полем рациональных дробей} над~$K$).
Элементами поля $K(x)$ являются дроби $f(x)/g(x)$, где $f(x), g(x)
\in K[x]$ и $g(x) \ne 0$.

Несколько других примеров полей:
$$
\QQ(\sqrt{2}) = \{a + b \sqrt{2} \mid a, b \in \QQ\}, \quad
\QQ(\sqrt[3]{2}) = \{a + b \sqrt[3]{2} + c\sqrt[3]{4} \mid a, b,
c\in\QQ\}, \quad \QQ(\sqrt{-1}) = \{a + b\sqrt{-1} \mid a, b \in \QQ
\}.
$$

\begin{definition}
	Пусть $K$~--- произвольное поле. {\it Характеристикой} поля $K$
	называется такое наименьшее натуральное число $p$, что
	$\underbrace{1+\ldots+1}_p = 0$. Если такого натурального $p$ не
	существует, говорят, что характеристика поля равна нулю.
	Обозначение: $\xar K$.
\end{definition}

Например, $\xar \QQ = \xar \RR = \xar \CC = 0$ и $\xar \ZZ_p = \xar
\ZZ_p(x) = p$.

Из определения следует, что всякое поле характеристики нуль
бесконечно. Примером бесконечного поля характеристики $p > 0$
является поле $\ZZ_p(x)$.

\begin{proposition}
	Характеристика произвольного поля $K$ либо равна нулю, либо является
	простым числом.
\end{proposition}

\begin{proof}
	Положим $p = \xar K$ и предположим, что $p > 0$. Так как $0 \ne 1$
	в~$K$, то $p \geqslant 2$. Если число $p$ не является простым, то $p
	= mk$ для некоторых $m,k \in \NN$, $1 < m,k < p$. Тогда в $K$ верно
	равенство
	$$
	0 = \underbrace{1 + \ldots + 1}_{mk} = (\underbrace{1 + \ldots +
		1}_m)(\underbrace{1 + \ldots + 1}_k).
	$$
	В~силу минимальности числа~$p$ в последнем выражении обе скобки
	отличны от нуля, но такое невозможно, так как в поле нет делителей
	нуля.
\end{proof}

\begin{exercise}
	Пересечение любого семейства подполей фиксированного поля~$K$
	является подполем в~$K$. В частности, для всякого подмножества $S
	\subseteq K$ существует наименьшее по включению подполе в~$K$,
	содержащее~$S$. Это подполе совпадает с пересечением всех подполей
	в~$K$, содержащих~$S$.
\end{exercise}

Из приведённого выше упражнения следует, что в каждом поле
существует наименьшее по включению подполе, оно называется {\it
	простым подполем}.

\begin{proposition}
	Пусть $K$~--- поле и $K_0$~--- его простое подполе. Тогда:
	
	\textup{(1)} если $\xar K = p > 0$, то $K_0 \cong \ZZ_p$;
	
	\textup{(2)} если $\xar K = 0$, то $K_0 \cong \QQ$.
\end{proposition}

\begin{proof}
	Пусть $\langle 1 \rangle \subseteq K$~--- циклическая подгруппа по
	сложению, порождённая единицей. Заметим, что $\langle 1 \rangle$~---
	подкольцо в~$K$. Поскольку всякое подполе поля $K$ содержит единицу,
	оно содержит и множество~$\langle 1 \rangle$. Следовательно,
	$\langle 1 \rangle \subseteq K_0$.
	
	Если $\xar K = p > 0$, то мы имеем изоморфизм колец $\langle 1
	\rangle \simeq \ZZ_p$. Но, как мы уже знаем из лекции~6,
	кольцо~$\ZZ_p$ является полем, поэтому $K_0 = \langle 1 \rangle
	\simeq \ZZ_p$.
	
	Если же $\xar K = 0$, то мы имеем изоморфизм колец $\langle 1
	\rangle \cong \ZZ$. Тогда $K_0$ содержит все дроби вида $a/b$, где
	$a,b \in \langle 1 \rangle$ и $b \ne 0$. Ясно, что все такие дроби
	образуют поле, изоморфное полю~$\QQ$.
\end{proof}

\begin{definition}
	Если $K$~--- подполе поля $F$, то говорят, что $F$~--- {\it
		расширение} поля~$K$.
\end{definition}

Например, всякое поле есть расширение своего простого подполя.

\begin{definition}
	{\it Степенью} расширения полей $K \subseteq F$ называется
	размерность поля $F$ как векторного пространства над полем~$K$.
	Обозначение $[F : K]$.
\end{definition}

Например, $[\CC : \RR] = 2$ и $[\RR : \QQ] = \infty$.

\begin{definition}
	Расширение полей $K\subseteq F$ называется {\it конечным}, если $[F
	: K] < \infty$.
\end{definition}

\begin{proposition}
	Пусть $K\subseteq F$ и $F\subseteq L$~--- конечные расширения полей.
	Тогда расширение $F\subseteq L$ также конечно и $[L:K]=[L:F][F:K]$.
\end{proposition}

\begin{proof}
	Пусть $e_1,\ldots,e_n$~--- базис $F$ над $K$ и $f_1,\ldots,f_m$~---
	базис $L$ над $F$. Достаточно доказать, что множество
	\begin{equation} \label{eqn_basis}
	\lbrace e_i f_j \mid i = 1,\ldots,n;\, j = 1,\ldots, m \rbrace
	\end{equation}
	является базисом $L$ над~$K$. Для этого сначала покажем, что
	произвольный элемент $a\in L$ представим в виде линейной комбинации
	элементов (\ref{eqn_basis}) с коэффициентами из~$K$. Поскольку $f_1,
	\ldots, f_m$~--- базис $L$ над~$F$, имеем $a = \sum \limits_{j=1}^m
	\alpha_j f_j$ для некоторых $\alpha_j \in F$. Далее, поскольку $e_1,
	\ldots, e_n$~--- базис $F$ над~$K$, для каждого $j = 1, \ldots, m$
	имеем $\alpha_j = \sum \limits_{i = 1}^n \beta_{ij} e_i$ для
	некоторых $\beta_{ij}\in K$. Отсюда получаем, что $a = \sum
	\limits_{i=1}^n \sum \limits_{j=1}^n \beta_{ij} (e_if_j)$.
	
	Теперь проверим линейную независимость элементов~(\ref{eqn_basis}).
	Пусть $\sum \limits_{i=1}^n \sum \limits_{j=1}^n \gamma_{ij}
	(e_jf_i) = 0$, где $\gamma_{ij} \in K$. Переписав это равенство в
	виде $\sum \limits_{j=1}^m (\sum \limits_{i=1}^n \gamma_{ij}e_i)f_j
	= 0$ и воспользовавшись тем, что элементы $f_1, \ldots, f_m$ линейно
	независимы над~$F$, мы получим $\sum \limits_{i=1}^n \gamma_{ij}e_i
	= 0$ для каждого $j = 1, \ldots, m$. Теперь из линейной
	независимости элементов $e_1, \ldots, e_n$ над~$K$ вытекает, что
	$\gamma_{ij} = 0$ при всех $i,j$. Таким образом,
	элементы~(\ref{eqn_basis}) линейно независимы.
\end{proof}

Пусть $K\subseteq F$~--- расширение полей.

\begin{definition}
	Элемент $\alpha \in F$ называется {\it алгебраическим} над подполем
	$K$, если существует ненулевой многочлен $f(x)\in K[x]$, для
	которого $f(\alpha) = 0$. В~противном случае $\alpha$ называется
	{\it трансцендентным} элементом над~$K$.
\end{definition}

\begin{definition}
	{\it Минимальным многочленом} алгебраического элемента $\alpha \in
	F$ над подполем $K$ называется ненулевой многочлен $h_\alpha(x)$
	наименьшей степени, для которого $h_\alpha(\alpha) = 0$.
\end{definition}

\begin{lemma} \label{lemma_min_pol}
	Пусть $\alpha \in F$~--- алгебраический элемент над~$K$ и
	$h_\alpha(x)$~--- его минимальный многочлен. Тогда:
	
	\textup{(а)} $h_\alpha(x)$ определён однозначно с точностью до
	пропорциональности;
	
	\textup{(б)} $h_\alpha(x)$ является неприводимым многочленом над
	полем~$K$;
	
	\textup{(в)} для произвольного многочлена $f(x)\in K[x]$ равенство
	$f(\alpha)=0$ имеет место тогда и только тогда, когда $h_\alpha(x)$
	делит~$f(x)$.
\end{lemma}

\begin{proof}
	(а) Пусть $h'_\alpha(x)$~--- ещё один минимальный многочлен элемента
	$\alpha$ над~$K$. Тогда $\deg h_\alpha(x) = \deg h'_\alpha(x)$.
	Умножив многочлены $h_\alpha(x)$ и $h'_\alpha(x)$ на подходящие
	константы, добьёмся того, чтобы их старшие коэффициенты стали равны
	единице. После этого положим $g(x) = h_\alpha(x) - h'_\alpha(x)$.
	Тогда $g(\alpha) = 0$ и $\deg g(x) < \deg h_\alpha(x)$. Учитывая
	определение минимального многочлена, мы получаем $g(x) = 0$.
	
	(б) Пусть $h_\alpha(x) = h_1(x) h_2(x)$ для некоторых $h_1(x),
	h_2(x) \in K[x]$, причём $0 < \deg h_i(x) < \deg h_\alpha(x)$ при $i
	= 1,2$. Так как $h_\alpha(\alpha) = 0$, то либо $h_1(\alpha)=0$,
	либо $h_2(\alpha)=0$, что противоречит минимальности~$h_\alpha(x)$.
	
	(в) Очевидно, что если $h_\alpha(x)$ делит $f(x)$, то $f(\alpha) =
	0$. Докажем обратное утверждение. Разделим $f(x)$ на $h_\alpha(x)$ с
	остатком: $f(x) = q(x)h_\alpha(x) + r(x)$, где $q(x), r(x) \in K[x]$
	и $\deg r(x) < \deg h_\alpha(x)$. Тогда условие $f(\alpha)=0$ влечёт
	$r(\alpha) = 0$. Из минимальности многочлена $h_\alpha(x)$ получаем
	$r(x)=0$.
\end{proof}

Для каждого элемента $\alpha \in F$ обозначим через $K(\alpha)$
наименьшее подполе в~$F$, содержащее $K$ и~$\alpha$.

\begin{proposition}
	Пусть $\alpha \in F$~--- алгебраический элемент над~$K$ и $n$~---
	степень его минимального многочлена над~$K$. Тогда
	$$
	K(\alpha) = \{\beta_0 + \beta_1 \alpha + \ldots + \beta_{n-1}
	\alpha^{n-1} \mid \beta_0, \ldots, \beta_{n-1} \in K\}.
	$$
	Кроме того, элементы $1, \alpha, \alpha^2, \ldots, \alpha^{n-1}$
	линейно независимы над~$K$. В~частности, $[K(\alpha) : K] = n$.
\end{proposition}

\begin{proof}
	Легко видеть, что
	$$
	K(\alpha) = \lbrace \frac{f(\alpha)}{g(\alpha)} \mid f(x), g(x) \in
	K[x], f(\alpha) \ne 0 \rbrace.
	$$
	Действительно, такие элементы лежат в любом подполе поля~$F$,
	содержащем~$K$ и~$\alpha$, и сами образуют поле. Теперь возьмём
	произвольный элемент $\frac{f(\alpha)}{g(\alpha)} \in K(\alpha)$ и
	покажем, что он представим в виде, указанном в условии. Пусть
	$h_\alpha(x) \in K[x]$~--- минимальный многочлен элемента~$\alpha$
	над~$K$. Поскольку $g(\alpha) \ne 0$, в силу
	леммы~\ref{lemma_min_pol}(в) многочлен $h_\alpha(x)$ не
	делит~$g(x)$. Но $h_\alpha(x)$ неприводим по
	лемме~\ref{lemma_min_pol}(б), поэтому $(g(x), h_\alpha(x)) = 1$.
	Значит, существуют такие многочлены $u(x), v(x) \in K[x]$, что $u(x)
	g(x) + v(x) h_\alpha(x) = 1$. Подставляя в последнее равенство $x =
	\alpha$, мы получаем $u(\alpha) g(\alpha) = 1$. Отсюда
	$\frac{f(\alpha)}{g(\alpha)} = f(\alpha) u(\alpha)$, и мы избавились
	от знаменателя. Теперь уменьшим степень числителя. Пусть $r(x)$~---
	остаток от деления $f(x)u(x)$ на~$h_\alpha(x)$. Тогда $f(\alpha)
	u(\alpha) = r(\alpha)$ и, значит, $\frac{f(\alpha)}{g(\alpha)} =
	r(\alpha)$, что показывает представимость элемента
	$\frac{f\alpha)}{g(\alpha)}$ в требуемом виде.
	
	Остаётся показать, что элементы $1, \alpha, \ldots, \alpha^{n-1}$
	поля $F$ линейно независимы над~$K$. Если $$\gamma_0 + \gamma_1
	\alpha + \ldots + \gamma_{n-1} \alpha^{n-1} = 0$$ для некоторых
	$\gamma_0, \gamma_1, \ldots, \gamma_{n-1} \in K$, то для многочлена
	$w(x) = \gamma_0 + \gamma_1x + \ldots + \gamma_{n-1}x^{n-1} \in
	K[x]$ получаем $w(\alpha) = 0$. Тогда из
	леммы~\ref{lemma_min_pol}(в) и условия $\deg w(x) < \deg
	h_\alpha(x)$ вытекает, что $w(x) = 0$, то есть $\gamma_0 = \gamma_1
	= \ldots = \gamma_{n-1} = 0$.
\end{proof}

\begin{theorem}
	Пусть $K$~--- произвольное поле и $f(x)\in K[x]$~--- многочлен
	положительной степени. Тогда существует конечное расширение
	$K\subseteq F$, в котором многочлен $f(x)$ имеет корень.
\end{theorem}

\begin{proof}
	Достаточно построить конечное расширение, в~котором имеет корень
	один из неприводимых делителей $p(x)$ многочлена~$f(x)$.
	
	Покажем сначала, что факторкольцо $K[x]/(p(x))$ является полем. В
	самом деле, если многочлен $g(x) \in\nobreak K[x]$ не делится
	на~$p(x)$, то $(g(x), p(x)) = 1$, и тогда существуют многочлены
	$u(x), v(x) \in K[x]$, для которых $u(x) g(x) + v(x) p(x) = 1$. Взяв
	образ последнего равенства в факторкольце $K[x] / (p(x))$, мы
	получим
	$$(u(x) + (p(x))) (g(x) + (p(x))) = 1 + (p(x)),$$ т.\,е. элемент
	$u(x)+(p(x))$ является обратным к $g(x)+(p(x))$. Значит, $K[x] /
	(p(x))$~--- поле, и мы возьмём его в качестве~$F$.
	
	Заметим теперь, что расширение $K \subseteq F$ является конечным.
	Действительно, для всякого многочлена $g(x) \in K[x]$ в поле $F =
	K[x]/(p(x))$ имеем $g(x) + (p(x)) = r(x) + (p(x))$, где $r(x)$~---
	остаток от деления $g(x)$ на $p(x)$. Отсюда следует, что $F$
	порождается как векторное пространство над~$K$ элементами
	$$
	1 + (p(x)), x + (p(x)), \ldots, x^{n-1} + (p(x)),
	$$
	где $n = \deg p(x)$. (Так же легко показать, что эти элементы
	образуют базис в $F$ над~$K$.)
	
	Остаётся показать, что в поле $F$ многочлен $p(x)$ имеет корень. Это
	похоже на обман, но корнем будет... $x + (p(x))$. Действительно,
	пусть $p(x) = a_n x^n + a_{n-1} x^{n-1} + a_1x + a_0$, где $a_0,
	a_1, \ldots, a_n \in K$. Тогда
	\begin{multline*}
		p(x + (p(x))) = a_n(x + (p(x)))^n + a_{n-1} (x + (p(x)))^{n-1} +
		\ldots + a_1 (x + (p(x)) + a_0 =\\
		= (a_nx^n + a_{n-1}x^{n-1} + \ldots a_1 x + a_0) + (p(x)) = p(x) +
		(p(x)) = (p(x)),
	\end{multline*}
	а $(p(x))$ есть не что иное, как нуль в~$F$.
\end{proof}

Говорят, что поле $K[x] / (p(x))$ получено из поля $K$ {\it
	присоединением корня} неприводимого многочлена~$p(x)$. Нетрудно
проверить, что если $\alpha$~--- некоторый корень многочлена $p(x)$
в $K[x]/(p(x))$, то поле $K[x]/(p(x))$ совпадает с подполем
$K(\alpha)$.

\begin{definition}
	Пусть $K$~--- некоторое поле и $f(x)\in K[x]$~--- многочлен
	положительной степени. {\it Полем разложения} многочлена $f(x)$
	называется такое расширение $F$ поля~$K$, что
	
	(1) многочлен $f(x)$ разлагается над $F$ на линейные множители;
	
	(2) корни многочлена $f(x)$ не лежат ни в каком собственном подполе
	поля~$F$, содержащем~$K$.
\end{definition}

\begin{example}
	Рассмотрим многочлен $f(x) = x^4+x^3+x^2+x+1$ над $\QQ$. Так как
	$(x-1)f(x) = x^5-1$, корнями многочлена $f(x)$ являются все корни
	степени $5$ из единицы, отличные от единицы. Если присоединить к
	$\QQ$ один из корней $\epsilon$ многочлена~$f$, то его остальные
	корни можно получить, возводя число $\epsilon$ в натуральные
	степени. Таким образом, присоединение одного корня сразу приводит к
	полю разложения многочлена.
\end{example}


\begin{example}
	Многочлен $f(x)=x^3-2$ неприводим над полем $\QQ$. Присоединение к
	полю $\QQ$ корня этого многочлена приводит к полю $\QQ[x]/(x^3-2)
	\cong \QQ(\sqrt[3]{2})$. Данное поле не является полем разложения
	многочлена~$f(x)$, поскольку в нём $f(x)$ имеет только один корень и
	не имеет двух других корней. Поскольку корнями данного многочлена
	являются числа
	$$
	\sqrt[3]{2}, \quad \sqrt[3]{2}(-\frac{1}{2} + \cfrac{\sqrt{-3}}{2}),
	\quad \sqrt[3]{2}(-\frac{1}{2} - \frac{\sqrt{-3}}{2}),
	$$
	полем разложения многочлена $f(x)$ является поле
	$$
	F = \{\alpha_0 + \alpha_1 \sqrt[3]{2} + \alpha_2 \sqrt[3]{4} +
	\alpha_3 \sqrt{-3} + \alpha_4 \sqrt[3]{2} \sqrt{-3} + \alpha_5
	\sqrt[3]{4} \sqrt{-3} \mid \alpha_i \in \QQ\},
	$$
	которое имеет над $\QQ$ степень~$6$.
\end{example}

Пусть $F$ и $F'$~--- два расширения поля~$K$. Говорят, что
изоморфизм $F \xrightarrow{\sim} F'$ является \textit{тождественным
	на~$K$}, если при этом изоморфизме каждый элемент поля $K$ переходит
в себя.

\begin{theorem}
	Поле разложения любого многочлена $f(x) \in K[x]$ существует и
	единственно с точностью до изоморфизма, тождественного на~$K$.
\end{theorem}

Доказательство этой теоремы можно найти, например, в книге
Э.\,Б.~Винберга \guillemotleft Курс алгебры\guillemotright{}. Мы не
включаем это доказательство в программу нашего курса.


\newpage

\section*{Лекция~10}

\medskip

{\it Конечные поля. Простое подполе и порядок конечного поля.
	Автоморфизм Фробениуса. Теорема существования и единственности для
	конечных полей. Поле из четырех элементов. Цикличность
	мультипликативной группы. Неприводимые многочлены над конечным
	полем. Подполя конечного поля.}

\medskip

В этой лекции будем использовать следующее обозначение: $K^\times =
K \setminus \lbrace 0 \rbrace$~--- мультипликативная группа
поля~$K$.

Пусть $K$~--- конечное поле. Тогда его характеристика отлична от
нуля и потому равна некоторому простому числу~$p$. Значит, $K$
содержит поле $\ZZ_p$ в качестве простого подполя.

\begin{theorem} \label{thm1}
	Число элементов конечного поля равно $p^n$ для некоторого простого
	$p$ и натурального $n$.
\end{theorem}

\begin{proof}
	Пусть $K$~--- конечное поле характеристики~$p$, и пусть размерность
	$K$ над простым подполем $\ZZ_p$ равна~$n$. Выберем в $K$ базис
	$e_1, \ldots, e_n$ над $\ZZ_p$. Тогда каждый элемент из $K$
	однозначно представляется в виде $\alpha_1 e_1 + \ldots + \alpha_n
	e_n$, где $\alpha_1, \ldots, \alpha_n$ пробегают~$\ZZ_p$.
	Следовательно, в $K$ ровно $p^n$ элементов.
\end{proof}

Пусть $K$~--- произвольное поле характеристики $p > 0$. Рассмотрим
отображение
$$
\varphi \colon K \to K, \quad a \mapsto a^p.
$$
Покажем, что $\varphi$~--- гомоморфизм. Для любых $a,b \in K$ по
формуле бинома Ньютона имеем
$$
(a + b)^p = a^p + C_p^1 a^{p-1}b + C_p^2 a^{p-2}b^2 + \ldots +
C_p^{p-1} a b^{p-1} + b^p.
$$
Так как $p$~--- простое число, то все биномиальные коэффициенты
$C_p^i$ при $1 \leqslant i \leqslant p-1$ делятся на~$p$. Это
значит, что в нашем поле характеристики $p$ все эти коэффициенты
обнуляются, в результате чего получаем $(a + b)^p = a^p + b^p$.
Ясно, что $(ab)^p = a^p b^p$, так что $\varphi$~--- гомоморфизм.
Ядро любого гомоморфизма колец является идеалом, поэтому $\Ker
\varphi$~--- идеал в~$K$. Но в поле нет собственных идеалов, поэтому
$\Ker \varphi = \lbrace 0 \rbrace$, откуда $\varphi$ инъективен.

Если поле $K$ конечно, то инъективное отображение из $K$ в $K$
автоматически биективно. В этой ситуации $\varphi$ называется {\it
	автоморфизмом Фробениуса} поля $K$.

\begin{remark}
	Пусть $K$~--- произвольное поле и $\psi$~--- произвольный
	автоморфизм (т.\,е. изоморфизм на себя) поля~$K$. Легко видеть, что
	множество неподвижных точек $K^{\psi} = \{ a \in K \mid \psi(a) =
	a\}$ является подполем в~$K$.
\end{remark}

Прежде чем перейти к следующей теореме, обсудим понятие формальной
производной многочлена. Пусть $K[x]$~--- кольцо многочленов над
произвольным полем~$K$. Формальной производной называется
отображение $K[x] \to K[x]$, которое каждому многочлену $f(x) =
a_nx^n + a_{n-1}x^{n-1} + \ldots + a_1 x + a_0$ сопоставляет
многочлен $f'(x) = na_n x^{n-1} + (n-1)a_{n-1}x^{n-2} + \ldots +
a_1$. Из определения следует, что это отображение линейно. Легко
проверить, что для любых $f,g \in K[x]$ справедливо привычное нам
равенство $(fg)' = f'g + fg'$ (в~силу дистрибутивности умножения
проверка этого равенства сводится к случаю, когда $f,g$~---
одночлены). В~частности, $(f(x)^m)' = mf(x)^{m-1}$ для любых $f(x)
\in K[x]$ и $m \in \NN$.

\begin{theorem} \label{thm2}
	Для всякого простого числа $p$ и натурального числа $n$ существует
	единственное \textup(с точностью до изоморфизма\textup) поле из
	$p^n$ элементов.
\end{theorem}

\begin{proof}
	Положим $q = p^n$ для краткости.
	
	{\it Единственность.}\ Пусть поле $K$ содержит $q$ элементов. Тогда
	мультипликативная группа $K^{\times}$ имеет порядок $q-1$. По
	следствию~3 из теоремы Лагранжа мы имеем $a^{q-1}=1$ для всех $a \in
	K \setminus \{0\}$, откуда $a^q - a = 0$ для всех $a\in K$. Это
	значит, что все элементы поля $K$ являются корнями многочлена $x^q -
	x \in \ZZ_p[x]$. Отсюда следует, что $K$ является полем разложения
	многочлена $x^q - x$ над $\ZZ_p$. Из теоремы о полях разложения,
	формулировавшейся на прошлой лекции, следует, что поле $K$
	единственно с точностью до изоморфизма.
	
	\smallskip
	
	{\it Существование.} Пусть $K$~--- поле разложения многочлена $f(x)
	= x^q - x \in \ZZ_p[x]$. Тогда имеем $f'(x)= qx^{q-1} - 1 =\nobreak
	-1$ ($qx^{q-1}$ обнуляется, так как $q$ делится на~$p$, а $p$~---
	характеристика поля~$\ZZ_p$). Покажем, что многочлен $f(x)$ не имеет
	кратных корней в~$K$. Действительно, если $\alpha$~--- корень
	кратности $m \geqslant 2$, то $f(x) = (x - \alpha)^m g(x)$ для
	некоторого многочлена $g(x) \in \ZZ_p[x]$. Но тогда $f'(x) =
	m(x-\alpha)^{m-1} g(x) + (x - \alpha)^m g'(x)$, откуда видно, что
	$f'(x)$ делится на $(x - \alpha)$. Но последнее невозможно, ибо
	$f'(x) = -1$~--- многочлен нулевой степени. Итак, многочлен $f(x)$
	имеет ровно $q$ различных корней в поле~$K$. Заметим, что эти
	корни~--- в точности неподвижные точки автоморфизма $\varphi^n =
	\underbrace{\varphi \circ \ldots \circ \varphi}_n$, где
	$\varphi$~--- автоморфизм Фробениуса. В~самом деле, для элемента $a
	\in K$ равенство $a^q - a = 0$ выполнено тогда и только тогда, когда
	$a^{p^n} = a$, т.\,е. $\varphi^n(a) = a$. Значит, корни многочлена
	$x^q-x$ образуют подполе в~$K$, которое по определению поля
	разложения совпадает с~$K$. Следовательно, в поле $K$ ровно $q$
	элементов.
\end{proof}

Конечныe поля еще называют {\it полями Галуа}. Поле из $q$ элементов
обозначают $\FF_q$. Например, $\FF_p \cong \ZZ_p$.

\begin{example}
	Построим явно поле из четырёх элементов. Многочлен $x^2+x+1$
	неприводим над $\ZZ_2$. Значит, факторкольцо $\ZZ_2[x]/(x^2+x+1)$
	является полем и его элементы~--- это классы $\overline{0},
	\overline{1}, \overline{x}, \overline{x+1}$ (запись $\overline a$
	означает класс элемента $a$ в факторкольце $\ZZ_2[x]/(x^2+x+1)$).
	Например, произведение $\overline{x} \cdot \overline{x+1}$~--- это
	класс элемента $x^2+x$, который равен $\overline{1}$.
\end{example}

\begin{proposition}
	Мультипликативная группа конечного поля $\FF_q$ является
	циклической.
\end{proposition}

\begin{proof}
	Заметим, что $\FF_q^\times$~--- конечная абелева группа, и обозначим
	через $m$ её экспоненту (см. конец лекции~4). Предположим, что
	группа $\FF_q^{\times}$ не является циклической. Тогда $m < q-1$ по
	следствию~2 лекции~4. По определению экспоненты это значит, что $a^m
	= 1$ для всех $a \in \FF_q^{\times}$. Но тогда многочлен $x^m-1$
	имеет в поле $\FF_q$ больше корней, чем его степень,~---
	противоречие.
\end{proof}

\begin{theorem}
	Конечное поле $\FF_q$, где $q=p^n$, можно реализовать в виде
	$\ZZ_p[x]/(h(x))$, где $h(x)$~--- неприводимый многочлен степени $n$
	над $\ZZ_p$. В~частности, для всякого $n \in \NN$ в кольце
	$\ZZ_p[x]$ есть неприводимый многочлен степени~$n$.
\end{theorem}

\begin{proof}
	Пусть $\alpha$~--- порождающий элемент группы $\FF_q^{\times}$.
	Тогда минимальное подполе $\ZZ_p(\alpha)$ поля~$\FF_q$, содержащее
	$\alpha$, совпадает с~$\FF_q$. Значит, поле $\FF_q$ изоморфно полю
	$\ZZ_p[x]/(h(x))$, где $h(x)$~--- минимальный многочлен элемента
	$\alpha$ над $\ZZ_p$. Из результатов прошлой лекции следует, что
	многочлен $h(x)$ неприводим. Поскольку степень расширения $[\FF_q :
	\ZZ_p]$ равна~$n$, этот многочлен имеет степень~$n$.
\end{proof}

\begin{theorem}
	Всякое подполе поля $\FF_q$, где $q=p^n$, изоморфно $\FF_{p^m}$, где
	$m$~--- делитель числа $n$. Обратно, для каждого делителя $m$ числа
	$n$ в поле $\FF_q$ существует ровно одно подполе из $p^m$ элементов.
\end{theorem}

\begin{proof}
	Пусть $F$~--- подполе поля $\FF_q$. По определению простого подполя
	имеем $F \supset \ZZ_p$, откуда $\xar F = p$. Тогда
	теорема~\ref{thm1} нам сообщает, что $|F| = p^m$ для некоторого $m
	\in \NN$. По теореме~\ref{thm2} имеем $F \cong\nobreak \FF_{p^m}$.
	Обозначим через $s$ степень (конечного) расширения $F \subset
	\FF_q$. Рассуждая так же, как в доказательстве теоремы~\ref{thm1},
	мы получим $p^n = (p^m)^s$, откуда $p^n = p^{ms}$ и $m$ делит~$n$.
	
	Пусть теперь $m$~--- делитель числа~$n$, т.\,е. $n = ms$ для
	некоторого $s \in \NN$. Рассмотрим многочлены $f(x) = x^{p^n} - x$ и
	$g(x) = x^{p^m} - x$ над $\ZZ_p$. Заметим, что для элемента $a \in
	\FF_q$ равенства $a^{p^m} = a$ следует
	$$
	a^{p^n} = a^{p^{ms}} = a^{(p^m)^s} =
	(\ldots((a^{p^m})^{p^m})^{p^m}\ldots)^{p^m} \ \text{($s$ раз возвели
		в степень $p^m$)} = a.
	$$
	Поэтому каждый корень многочлена $g(x)$ является и корнем многочлена
	$f(x)$. Отсюда поле разложения многочлена $f(x)$ лежит в поле
	разложения многочлена $g(x)$. Значит, $\FF_{p^m}$ содержится в
	$\FF_{p^n}$.
	
	Наконец, все элементы подполя из $p^m$ элементов неподвижны при
	автоморфизме $\psi = \underbrace{\varphi \circ \ldots \circ
		\varphi}_m \colon x \mapsto x^{p^m}$ ($\varphi$~--- автоморфизм
	Фробениуса). Поскольку число корней многочлена $x^{p^m}-x$ в поле
	$\FF_q$ не превосходит~$p^m$, множество элементов данного подполя
	совпадает с множеством неподвижных точек автоморфизма~$\psi$.
	Значит, такое подполе единственно.
\end{proof}

\end{document}
