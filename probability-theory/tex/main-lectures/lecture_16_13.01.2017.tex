\section{Лекция от 13.01.2017}
\subsection{Задача математической статистики}

Сегодняшняя лекция будет посвящена такой науке, как математическая статистика, 
а точнее, тому, чем она занимается и в чём её отличие от теории вероятностей. 

Для начала вспомним, чем вообще занимается теория вероятностей. Вообще, 
основной вопрос теории вероятности состоит в следующем: пусть есть случайная 
величина \(\xi\). Какова вероятность того, что \(\xi \in [a, b]\)? Чему равно 
её математическое ожидание и дисперсия?

Математическая статистика же занимается обратным вопросом: допустим, что нам 
известны некоторые наблюдения случайной величины \(\xi\). Как она себя ведёт?

Рассмотрим несколько задач математической статистики:

\begin{example}[Вопрос эмпирического выбора]
	Пусть в городе проживает \(N\) человек, среди которых \(M\) больно гриппом. 
	
	Примером задачи \textbf{теории вероятности} может служить следующий вопрос: 
	допустим, мы выбрали \(n\) человек. Какова вероятность того, что среди них 
	будет \(m\) заболевших? Стоит заметить, что для этой задачи \(M\) 
	предполагается известным.
	
	Задачей же \textbf{математической статистики} можно назвать следующую: 
	допустим, что на осмотр к врачу пришло \(n\) человек и оказалось, что из 
	них \(m\) больны. Сколько человек в городе болеет гриппом?
	
	Посмотрим на возможный путь решения. Можно сказать, что выполнена следующая 
	сходимость: \(\frac{m}{n}N \prto M\) при \(n \to \infty\). Далее, можно 
	найти \(\epsilon > 0\) такой, что \(\Pr{|M - \frac{m}{n}N| >  \epsilon} 
	\leq 0.05\). В таком случае интервал \((\frac{m}{n}N - \epsilon,  
	\frac{m}{n}N + \epsilon)\) называется \emph{доверительным}.
\end{example}

\begin{example}[Регрессионная модель]
	Пусть была запущена ракета, и мы хотим оценить траекторию. При этом у нас 
	есть данные вида
	\[
		x_{i} = at_{i}^{2} + bt_{i} + \epsilon_{i},\quad i \in \{1, 2, \dots, 
		N\}
	\]
	где \(x_{i}\)~--- положение ракеты в момент времени \(t_{i}\), а 
	\(\epsilon_{i} \sim \mathcal{N}(0, \sigma^2)\)\footnote{Такое предположение 
	делается для упрощения вычислений}~--- ошибка вычислений. По этим данным 
	нужно оценить \(a\) и \(b\). Однако никто не обещал, что данные будут 
	однородны (т.е. распределение не обязано совпадать).
\end{example}

\begin{example}[Проверка однородности]
	Иногда мы хотим просто сказать нечто про распределение, а не искать его. 
	Допустим, что есть два набора \((x_{1}, \dots, x_{N})\) и \((y_{1}, \dots, 
	y_{N})\). Можно ли сказать, что для всех \(i\) выполнено, что \(x_{i} 
	\eqdist y_{i}\)?
\end{example}

\begin{example}[Проверка независимости]
	Опять же, предположим, что у нас есть два набора \((x_{1}, \dots, x_{N})\) 
	и \((y_{1}, \dots, y_{N})\). Можно ли сказать, что для всех \(i\) 
	выполнено, что \(x_{i}\) независимо с \(y_{i}\)? На практике примером такой 
	задачи служит связь цвета волос и цвета глаз.
\end{example}

\subsection{И снова характеристические функции}
Теперь же вернёмся к теории вероятностей, а точнее~--- к характеристическим 
функциям. С ними связано пара интересных теорем.\
\begin{problem}[Вейерштрасс]
	Пусть \(f(x)\)~--- непрерывная на \([-a, a]\) функция (\(a > 0\)), причём 
	\(f(-a) = f(a)\). Тогда для любого \(\epsilon > 0\) существует линейная 
	тригонометрическая комбинация вида
	\[
		g_{\epsilon}(x) = \sum_{k = 1}^{K} \left(a_{k}\cos\frac{\pi k x}{a} + 
		b_{k}\sin \frac{\pi k x}{a}\right)
	\]
	такая, что для любого \(x \in [-a, a]\) \(|f(x) - g_{\epsilon}(x)| < 
	\epsilon\).
\end{problem}
\begin{proof}
	Доказательство можно найти в курсе матанализа.
\end{proof}
\begin{theorem}[единственности]
	Пусть \(\xi\) и \(\eta\)~--- это случайные величины. Тогда \(\xi \eqdist 
	\eta\) тогда и только тогда, когда \(\phi_{\xi}(t) = \phi_{\eta}(t)\) для 
	всех \(t \in \R\).
\end{theorem}
\begin{proof}
	Несложно понять, что если \(\xi \eqdist \eta\), то для любой борелевской 
	функции \(f\) \(\E{f(\xi)} = \E{f(\eta)}\) и \(\phi_{\xi}(x) = 
	\phi_{\eta}(x)\) для всех \(x \in \R\). Поэтому сосредоточимся на 
	доказательстве обратного утверждения.
	
	Зафиксируем произвольные \(a < b\) и для любого \(\epsilon > 0\) введём 
	функцию \(f^{\epsilon}(x)\), устроенную следующим образом:
	\[
		f^{\epsilon}(x) = \begin{cases}
		0,&, x < a \\
		\frac{x - a}{\epsilon},& a \leq x \leq a + \epsilon \\
		1,& a + \epsilon \leq x < b \\
		\frac{b - x + \epsilon}{\epsilon},& b \leq x < b + \epsilon \\
		0, & x \geq b + \epsilon
		\end{cases}
	\]
	
	По сути, \(f^{\epsilon}(x)\)~--- это просто непрерывное приближение 
	\(\mathbf{1}_{[a, b]}(x)\) с точностью до \(\epsilon\). Теперь возьмём 
	натуральное \(n\) такое, что \([a, b + \epsilon] \subset [-n, n]\). Тогда 
	по теореме Вейерштрасса существует функция \(f_{n}^{\epsilon}(x)\) такая, 
	что
	\begin{gather*}
		f_{n}^{\epsilon}(x) = \sum_{k = 1}^{K} c_{k}e^{i\frac{\pi k}{n}x},\quad 
		c_{k} \in \mathbb{C} \\
		\forall x \in [-n, n]\ |f^{\epsilon}(x) - f_{n}^{\epsilon}(x)| < 
		\frac{1}{n}
	\end{gather*}

	Так как \(f_{n}^{\epsilon}(x)\) периодична, то \(|f_{n}^{\epsilon}(x)| \leq 
	1 + 1/n \leq 2\) для всех \(x \in \R\).
	
	Теперь пришло время оценок. Так как \(\phi_{\xi}(x) = \phi_{\eta}(x)\) для 
	всех \(x \in \R\), то \(\E{f_{n}^{\epsilon}(\xi)} = \E{ 
	f_{n}^{\epsilon}(\eta)}\). Тогда 
	\begin{align*}
		|\E{f^{\epsilon}(\xi)} - \E{f^{\epsilon}(\eta)}| &= 
		|\E{f^{\epsilon}(\xi)} - \E{f_{n}^{\epsilon}(\xi)} + \E{ 
		f_{n}^{\epsilon}(\eta)} - \E{f^{\epsilon}(\eta)}| \\
		&\leq |\E{f^{\epsilon}(\xi)} - \E{f_{n}^{\epsilon}(\xi)}| + 
		|\E{f^{\epsilon}(\eta)} - \E{ f_{n}^{\epsilon}(\eta)}|
	\end{align*}

	Осталось оценить \(|\E{f^{\epsilon}(\xi)} - \E{f_{n}^{\epsilon}(\xi)}|\). 	
	Заметим, что
	\begin{align*}
		|\E{f^{\epsilon}(\xi)} - \E{f_{n}^{\epsilon}(\xi)}| &= 
		\left|\E{(f^{\epsilon}(\eta) - f_{n}^{\epsilon}(\eta)) 
		(\mathbf{1}_{\{|\xi| \leq n\}} + \mathbf{1}_{\{|\xi| > n\}})}\right|\\
		&\leq \E{|f^{\epsilon}(\xi) - f_{n}^{\epsilon}(\xi)| 
		\mathbf{1}_{\{|\xi| \leq n\}}} + \E{|f^{\epsilon}(\xi) - 
		f_{n}^{\epsilon}(\xi)| \mathbf{1}_{\{|\xi| > n\}}}
	\end{align*}

	Осталось заметить, что оба матожидания можно ограничить сверху:
	\begin{gather*}
		\E{|f^{\epsilon}(\xi) - f_{n}^{\epsilon}(\xi)| \mathbf{1}_{\{|\xi| \leq 
		n\}}} \leq \frac{1}{n}\Pr{|\xi| \leq n} \leq \frac{1}{n}\\
		\E{|f^{\epsilon}(\xi) - f_{n}^{\epsilon}(\xi)| \mathbf{1}_{\{|\xi| > 
		n\}}} \leq 2\Pr{|\xi| > n}
	\end{gather*}

	В итоге несложно понять, что при \(n \to \infty\) \(|\E{f^{\epsilon}(\xi)} 
	- \E{f^{\epsilon}(\eta)}| \to 0\). Следовательно, для любого \(\epsilon > 
	0\) \(\E{f^{\epsilon}(\xi)} = \E{f^{\epsilon}(\eta)}\). 
	
	Осталось дело за малым: скажем, что \(|f^{\epsilon}(x)| \leq 1\) и 
	воспользуемся теоремой Лебега о мажорируемой сходимости:
	\[
		\lim\limits_{\epsilon \to 0} \E{f^{\epsilon}(\xi)} = \E{\mathbf{1}_{[a, 
		b]}(\xi)} = F_{\xi}(b) - F_{\xi}(a)
	\]
	
	Тогда для любых \(a < b\) выполнено, что
	\[
		F_{\xi}(b) - F_{\xi}(a) = F_{\eta}(b) - F_{\eta}(a)
	\]
	
	Устремляя \(a\) к \(-\infty\), получаем, что для любого \(b \in \R\) 
	\(F_{\xi}(b) = F_{\eta}(b)\), что и требовалось получить.
\end{proof}

Теперь что-нибудь посчитаем.
\begin{problem}
	Найдите характеристическую функцию стандартного нормального распределения.
\end{problem}
\begin{proof}[Решение]
	Для начала сразу скажем, что распределение симметрично относительно нуля, 
	поэтому характеристическая функция действительнозначна.
	
	Посчитаем характеристическую функцию \(\xi \sim \mathcal{N}(0, 1)\) по 
	определению:
	\[
		\phi_{\xi}(t) = \E{e^{it\xi}} = \int\limits_{-\infty}^{+\infty} 
		\frac{1}{\sqrt{2\pi}}e^{itx}e^{-\frac{x^2}{2}}\diff x = 
		\frac{1}{\sqrt{2\pi}}\int\limits_{-\infty}^{+\infty} 
		\cos(tx)e^{-\frac{x^2}{2}}\diff x
	\]
	
	Продифференцируем обе части:
	\[
		\phi'_{\xi}(t) = \E{(i\xi)e^{it\xi}} = 
		\frac{1}{\sqrt{2\pi}}\int\limits_{-\infty}^{+\infty} 
		ixe^{itx}e^{-\frac{x^2}{2}}\diff x = 
		-\frac{1}{\sqrt{2\pi}}\int\limits_{-\infty}^{+\infty} 
		x\sin(tx)e^{-\frac{x^2}{2}}\diff x
	\]
	
	Полученный интеграл возьмём по частям:
	\[
		-\frac{1}{\sqrt{2\pi}}\int\limits_{-\infty}^{+\infty} 
		x\sin(tx)e^{-\frac{x^2}{2}}\diff x = 
		\left. \frac{1}{\sqrt{2\pi}} \sin(tx) e^{-\frac{x^2}{2}} 
		\right|_{-\infty}^{+\infty} - 
		\frac{t}{\sqrt{2\pi}}\int\limits_{-\infty}^{+\infty} 
		\cos(tx)e^{-\frac{x^2}{2}}\diff x
	\]
	
	Несложно заметить, что мы получили дифференциальное уравнение:
	\[
		\phi'_{\xi}(t) = -t\phi_{\xi}(t)
	\]
	
	Теперь заметим, что
	\[
		\left(\ln \phi_{\xi}(t)\right)' = \frac{\phi'_{\xi}(t)}{\phi_{\xi}(t)} 
		= -t \implies \ln \phi_{\xi}(t) = -\frac{t^2}{2} + C
	\]
	
	Пользуясь тем, что \(\phi_{\xi}(0) = 1\), получаем, что \(\phi_{\xi}(t) = 
	e^{-\frac{t^2}{2}}\).
\end{proof}

\begin{consequence}
	Если \(\xi \sim \mathcal{N}(a, \sigma^2)\), то \(\phi_{\xi}(t) = \exp\{iat 
	- \frac{1}{2}\sigma^2 t^2\}\).
\end{consequence}
\begin{proof}
	Как известно, \(\xi = a + \sigma\eta\), где \(\eta \sim \mathcal{N}(0, 
	1)\). Тогда \(\phi_{a + \sigma \eta}(t) = e^{ita}\phi_{\eta}(\sigma t) = 
	\exp\{iat - \frac{1}{2}\sigma^2 t^2\}\).
\end{proof}
\begin{consequence}
	Пусть \(\xi_1 \sim \mathcal{N}(a_1, \sigma_1^2)\) и \(\xi_2 \sim 
	\mathcal{N}(a_2, \sigma_2^2)\)~--- независимые случайные величины. Тогда 
	\(\xi_1 + \xi_2 \sim \mathcal{N}(a_1 + a_2, \sigma_1^2 + \sigma_2^2)\).
\end{consequence}
\begin{proof}
	Пользуясь независимостью, распишем характеристическую функцию суммы:
	\[
		\phi_{\xi_1 + \xi_2}(t) = \phi_{\xi_1}(t)\phi_{\xi_2}(t) = 
		\exp\left\{i(a_1 + a_2)t - \frac{1}{2}(\sigma_1^2 
		+\sigma_2^2)t^2\right\}
	\]
	
	По теореме единственности \(\xi_1 + \xi_2 \sim \mathcal{N}(a_1 + a_2, 
	\sigma_1^2 + \sigma_2^2)\).
\end{proof}

Теперь введём без доказательства одну весьма важную теорему:
\begin{theorem}[Леви об обращении]
	Пусть \(\xi\)~--- случайная величина с функцией распределения \(F_{\xi}\) и 
	характеристической функцией \(\phi_{\xi}\). Тогда верны два утверждения:
	\begin{enumerate}
		\item Для любых точек непрерывности \(F_{\xi}\) \(a < b\) верно, что
		\[
			F_{\xi}(b) - F_{\xi}(a) = \frac{1}{2\pi}\lim\limits_{c \to +\infty} 
			\int\limits_{-c}^{c} \frac{e^{-ita} - e^{-itb}}{it} 
			\phi_{\xi}(t)\diff t
		\]
		
		\item Если
		\[
			\int\limits_{-\infty}^{+\infty} |\phi_{\xi}(t)|\diff t < +\infty,
		\]
		то \(\xi\) имеет плотность \(p_{\xi}\), которая равна
		\[
			p_{\xi}(x) = \frac{1}{2\pi}\int\limits_{-\infty}^{+\infty} 
			e^{-itx}\phi_{\xi}(t)\diff t
		\]
	\end{enumerate}
\end{theorem}