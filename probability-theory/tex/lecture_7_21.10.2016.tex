\section{Лекция от 21.10.2016}

Будем дальше двигаться в сторону непрерывных вероятностей. Наша задача состоит в определении вероятностного пространства в общем случае.

\subsection{Понятие вероятностного пространства в общем случае}
\begin{definition}
	\emph{Вероятностным пространством} (или \emph{тройкой Колмогорова}) называют тройку \((\Omega, \F, \Pr)\), где \(\Omega\)~--- множество элементарных исходов, \(\F\)~--- множество событий, а \(\Pr\)~--- вероятностная мера.
\end{definition}

Разберем по отдельности каждый символ из этой ``тройки''.
\begin{itemize}
	\item \(\Omega\)~--- произвольное множество элементарных исходов. Как и в дискретном случае, во время эксперимента возможно получить один и только один элементарный исход.
	
	Например, если событие~--- это выстрел в мишень, то множество элементарных исходов будет задаваться плоскостью: \(\Omega = \R^2\).
	
	\item \(\F\)~--- это совокупность подмножеств \(\Omega\), называемых \emph{событиями}. В дискретном случае множество событий содержало все подмножества \(\Omega\) (\(\F = 2^{\Omega}\)) и не было никаких проблем. Однако в общем случае возникают некоторые ограничения на \(\F\). Обсудим, какие же. Перед этим напомним определение \emph{алгебры}:
	\begin{definition}
		Пусть задано некоторое множество \(\Omega\). Система \(\F\) подмножеств \(\Omega\) называется \emph{алгеброй}, если выполняются следующие требования:
		\begin{enumerate}
			\item \(\emptyset \in \F\).
			\item Если \(A \in \F\), то и его дополнение \(\overline{A} = \Omega \setminus A \in \F\).
			\item Если \(A, B \in \F\), то их пересечение и объединение тоже лежит в \(\F\).
		\end{enumerate}
	\end{definition}
	Вообще говоря, достаточно требовать либо пересечение, либо объединение, так как по закону де Моргана \(A \cup B = \overline{\overline{A} \cap \overline{B}}\).
	\begin{example}
		Конечные объединения множеств вида \((-\infty, a]\), \((b, c]\), \((d, +\infty)\) образуют алгебру.
	\end{example}
	Далее, несколько усилим это определение, добавив дополнительное условие:
	\begin{definition}
		Пусть задано некоторое множество \(\Omega\). Система \(\F\) подмножеств \(\Omega\) называется \emph{\(\sigma\)-алгеброй}, если выполняются следующие требования:
		\begin{enumerate}
			\item \(\F\) является алгеброй.
			\item Для любой последовательности событий \(\{A_n\}_{n = 1}^{\infty}\) их пересечение
                \(\bigcap\limits_{n = 1}^{\infty} A_n\) и объединение \(\bigcup\limits_{n = 1}^{\infty} A_n\) лежат в \(\F\).
		\end{enumerate}
	\end{definition}
	В данном случае опять же можно обойтись одним из двух~--- либо пересечением, либо объединением.
	
	В общем случае полагается, что \(\F\) обязана быть \(\sigma\)-алгеброй.\footnote{Вообще, это требуется для того, чтобы всё не поломалось: \url{http://stats.stackexchange.com/questions/199280/why-do-we-need-sigma-algebras-to-define-probability-spaces}}
	
	Покажем, что среди \(\sigma\)-алгебр над одним множеством (их, вообще говоря, может быть много) есть минимальная.
	\begin{lemma}
		Пусть \(M\)~--- некоторая система подмножеств \(\Omega\). Тогда существует минимальная (по включению) \(\sigma\)-алгебра, содержащая в себе все подмножества из \(M\).
	\end{lemma}
	\begin{proof}[Идея доказательства]
		Очевидно, что существуют \(\sigma\)-алгебры, содержащие \(M\) (та же \(2^{\Omega}\)). Тогда рассмотрим все \(\sigma\)-алгебры, содержащие \(M\). Их пересечение тоже является \(\sigma\)-алгеброй и, при этом, минимально по включению.
	\end{proof}
	Минимальную \(\sigma\)-алгебру, содержащую \(M\), принято обозначать через \(\sigma(M)\).
	
	Над \(\R\) принято вводить особую \(\sigma\)-алгебру.
	\begin{definition}
		\emph{Борелевской \(\sigma\)-алгеброй} на множестве \(\R\) называется минимальная \(\sigma\)-алгебра, содержащая все полуинтервалы вида \((a, b]\), где \(a, b \in \R\):
		\[
		\B(\R) = \sigma\left\{(a, b] \mid a, b \in \R, a < b\right\}.
		\]
	\end{definition}
	\begin{exercise}
		\(\B(\R)\) можно определить через интервалы, лучи, отрезки, открытые и замкнутые множества~--- это даст одну и ту же \(\sigma\)-алгебру.
	\end{exercise}
	Можно показать, что \(\B(\R) \neq 2^{\R}\) (например, множество Витали\footnote{\url{https://en.wikipedia.org/wiki/Vitali_set}} не входит в \(\B(\R)\)).
	
	\begin{definition}
		Пара \((\Omega, \F)\), где \(\F\)~--- \(\sigma\)-алгебра подмножеств \(\Omega\), называется \emph{измеримым пространством}.
	\end{definition}
	
	\item Наконец, мы подошли к рассмотрению последнего члена ``тройки Колмогорова''.
	
	\begin{definition}
		Отображение $ \Pr $ из $ \F $ в [0; 1] \(\left( (\Pr: \F \to [0; 1]) \right)\) называется \emph{вероятностной мерой} на \( (\Omega, \F) \), если оно удовлетворяет следующим двум условиям:
		\begin{enumerate}
			\item \(\Pr{\Omega} = 1\).
			\item \emph{Счётная аддитивность:} для любой последовательности попарно непересекающихся событий \(\{A_n\}_{n = 1}^{\infty}\) выполнено, что
			\[
			\Pr{\bigcup\limits_{n = 1}^{\infty}A_n} = \sum\limits_{n = 1}^{\infty}\Pr{A_n}.
			\]
		\end{enumerate}
	\end{definition}
	В дискретных вероятностных пространствах свойство счётной аддитивности доказывалось. Здесь его сразу постулируют.
	
	\begin{lemma}[Свойства вероятности]
		Вероятностные меры обладают следующими свойствами:
		\begin{enumerate}
			\item \(\Pr{\emptyset} = 0\).
			
			\item Конечная аддитивность. Если \(\{A_1, A_2, \dots, A_n\}\)~--- попарно не перескающиеся события, то
			\[
			\Pr{\bigcup\limits_{i = 1}^{n} A_i} = \sum\limits_{i = 1}^{n}\Pr{A_i}.
			\]
			
			\item Если \(A, B \in \F\) и \(A \subseteq B\), то \(\Pr{A} \leq \Pr{B} \).
			
			\item \(\Pr(A) + \Pr(\overline{A}) = 1\).
			
			\item \(\Pr(A \cup B) = \Pr(A) + \Pr(B) - \Pr(A \cap B)\).
			\item Для любого набора событий \(A_1, A_2, \ldots, A_n\) \(\Pr\left(\bigcup\limits_{n = 1}^{m} A_n\right) \leq \sum\limits_{n = 1}^{m} \Pr(A_n)\).
		\end{enumerate}
	\end{lemma}
	
	\begin{proof}
		Докажем только первые два свойства, так как остальные доказываеются ровно так же, как и в дискретном случае. За этими доказательствами можно обратиться к первой лекции.
		
		\begin{enumerate}
			\item Рассмотрим последовательность \(\{A_n\}_{n = 1}^{\infty}\), где для любого \(n \in \N\) \(A_n = \emptyset\). Тогда \(A_i \cap A_j = \emptyset\) для любых \(i, j \in \N\) и по свойству счётной аддитивности
			\[
			\Pr{\bigcup\limits_{n = 1}^{\infty}A_n} = \begin{cases}
			\Pr{\bigcup\limits_{n = 1}^{\infty} \emptyset} = \Pr{\emptyset} \\
			\sum\limits_{n = 1}^{\infty}\Pr{A_n} = \sum\limits_{n = 1}^{\infty}\Pr{\emptyset}
			\end{cases}
			\]
			
			Отсюда получаем, что
			\[
			\Pr{\emptyset} = \sum\limits_{n = 1}^{\infty}\Pr{\emptyset}.
			\]
			
			У этого уравнения есть два решения~--- \(0\) и \(\infty\). Но так как \(\Pr{\emptyset} \in [0, 1]\), то \(\Pr{\emptyset} = 0\).
			
			\item Положим \(A_n = \emptyset\) при \(n > m\). Тогда по свойству счётной аддитивности:
			\[
			\Pr{\bigcup\limits_{n = 1}^{\infty}A_n} = \sum\limits_{n = 1}^{\infty}\Pr{A_n} = \sum\limits_{n = 1}^{m}\Pr{A_n} + \sum\limits_{n = m + 1}^{\infty}\Pr{\emptyset} = \sum\limits_{n = 1}^{m}\Pr{A_n}.\qedhere
			\]
		\end{enumerate}
	\end{proof}
\end{itemize}

\begin{theorem}[О непрерывности вероятностной меры]
	Пусть $ \Pr $ --- конечно-аддитивная функция на $ \sigma $-алгебре событий $ \F $, \(\Pr: \F \to [0; 1],\ \Pr(\Omega) = 1 \). Тогда следующие четыре условия эквивалентны:
	\begin{enumerate}[label = (\alph*)]
		\item $ \Pr $ является счетно-аддитивной функцией.
		
		\item $ \Pr $ непрерывна в ``нуле'', то есть для любых множеств \(A_1, A_2, \ldots \in \F \) таких, что \(A_{n + 1} \subset A_n,\ \bigcap\limits_{n = 1}^{\infty}A_n = \emptyset \), выполняется:
		\[
		\lim\limits_{n}\Pr{A_n} = 0.
		\]  
		
		\item $ \Pr $ непрерывна сверху, то есть для любых множеств \(A_1, A_2, \ldots \in \F \) таких, что \(A_{n} \subset A_{n + 1},\ \bigcup\limits_{n = 1}^{\infty}A_n = A \in \F,\  (A_n \uparrow A) \), выполняется:
		\[
		\lim\limits_{n}\Pr{A_n} = \Pr{\bigcup\limits_{n = 1}^{\infty}A_n}.
		\]  
		
		\item $ \Pr $ непрерывна снизу, то есть для любых множеств \(A_1, A_2, \ldots \in \F \) таких, что \(A_{n + 1} \subset A_{n},\ \bigcap\limits_{n = 1}^{\infty}A_n = A \in \F,\  (A_n \downarrow A) \), выполняется:
		\[
		\lim\limits_{n}\Pr{A_n} = \Pr{\bigcap\limits_{n = 1}^{\infty}A_n}.
		\]  
		
	\end{enumerate}
\end{theorem}
\begin{proof}
	Докажем эквивалентность всех этих утверждений в несколько шагов:
	
	\begin{itemize}
		\item[{\([a \Rightarrow c]\)}] Покажем, что счётная аддитивность влечёт непрерывность сверху. Заметим, что \(\bigcup_{n = 1}^{\infty} A_n\) можно представить в виде счётного объединения попарно непересекающихся множеств:
		\[
		\bigcup\limits_{n = 1}^{\infty} A_n = A_1 \cup (A_2\setminus A_1) \cup (A_3\setminus A_2)\ldots.
		\]
		Действительно, \(A_{i+1} \setminus A_i\) содержит элементы, лежащие только в \(A_{i+1}\). Тогда по счётной аддитивности
		\begin{align}
			\Pr{\bigcup\limits_{n = 1}^{\infty} A_n} &=
			\Pr{A_1} + \Pr{A_2\setminus A_1} + \Pr{A_3\setminus A_2} + \ldots =\\
			&= \Pr{A_1} + \Pr{A_2} - \Pr{A_1} + \Pr{A_3} -\Pr{A_2} + \ldots =
			\lim\limits_{n \to \infty} \Pr{A_n}.
		\end{align}
		
		\item[{\([c \Rightarrow d]\)}] Покажем, что непрерывность сверху влечёт непрерывность снизу. Рассмотрим некоторую последовательность \(A_n \downarrow
		A\). Тогда для любого \(n \geq 1\) выполнено, что \(A_n \subseteq A_1\) и \(A_n = A_1 \setminus (A_1 \setminus A_n)\). Следовательно,
		\[
		\Pr{A_n} = \Pr{A_1 \setminus(A_1\setminus A_n)} = \Pr{A_1} - \Pr{A_1\setminus A_n}.
		\]
		
		Заметим, что последовательность \(\{B_n\}_{n = 1}^{\infty}\), где \(B_n = A_1 \setminus A_n\) является неубывающей и \(B_n \uparrow A_1 \setminus A\). Тогда, согласно пункту \(c\),
		\[
		\lim\limits_{n \to \infty}\Pr{A_1\setminus A_n} = \Pr{\bigcup\limits_{n=1}^{\infty}(A_1\setminus A_n)}.
		\]
		А это означает, что
		\begin{align}
			\lim\limits_{n \to \infty} \Pr{A_n} &= \Pr{A_1} - \lim\limits_{n \to \infty}\Pr{A_1\setminus A_n} = \Pr{A_1} - \Pr{\bigcup\limits_{n=1}^{\infty}(A_1\setminus A_n)} \\
			&= \Pr{A_1} - \Pr{A_1 \setminus \left(\bigcap\limits_{n=1}^{\infty} A_n\right)} = \Pr{A_1} - \Pr{A_1} + \Pr{\bigcap\limits_{n=1}^{\infty} A_n} \\
			&= \Pr{\bigcap\limits_{n=1}^{\infty} A_n}
		\end{align}
		
		\item[{\([d \Rightarrow b]\)}] Достаточно взять последовательность \(A_n \downarrow \emptyset\). Тогда из непрерывности снизу будет следовать непрерывность в нуле.
		
		\item[{\([b \Rightarrow a]\)}] Покажем, что непрерывность в нуле даёт счётную аддитивность. Рассмотрим некоторую последовательность попарно непересекающихся событий \(\{A_n\}_{n = 1}^{\infty}\). Заметим, что
		\[
		\Pr{\bigsqcup\limits_{n=1}^{\infty}A_n} =
		\Pr{\bigsqcup\limits_{i=1}^{n}A_i} +
		\Pr{\bigsqcup\limits_{i=n + 1}^{\infty} A_i}.
		\]
		Осталось только заметить, что \(\bigsqcup\limits_{i= n + 1}^{\infty} A_i \downarrow \emptyset\) при \(n \to \infty\). Тогда
		\begin{align}
			\sum\limits_{i = 1}^{\infty}\Pr{A_i} &=
			\lim\limits_{n\to\infty} \sum\limits_{i = 1}^{n}\Pr{A_i} =
			\lim\limits_{n\to\infty} \Pr{\bigsqcup\limits_{i = 1}^{n}A_i} \\
			&= \lim\limits_{n\to\infty} \left(
			\Pr{\bigsqcup\limits_{i = 1}^{\infty}A_i} -
			\Pr{\bigsqcup\limits_{i = n+1}^{\infty}A_i}
			\right) \\
			&= \Pr{\bigsqcup\limits_{i = 1}^{\infty}A_i} -
			\lim\limits_{n\to\infty} \Pr{\bigsqcup\limits_{i = n+1}^{\infty}A_i} =
			\Pr{\bigsqcup\limits_{i = 1}^{\infty}A_i}.\qedhere
		\end{align}
	\end{itemize}
\end{proof}


\subsection{Вероятностные меры на $\left( \R, \mathcal{B}(\R) \right)$.}
Пусть $\Pr$ --- некоторая вероятностная мера.

\begin{definition}
	\emph{Функцией распределения} вероятностной меры $\Pr$ называют функцию $F: \R \to [0, 1]$ такую, что
	$F(x) = \Pr{(-\infty, x]}$.
\end{definition}

\begin{lemma}[Свойства функции распределения]
	\ 
	\begin{enumerate}
		\item
		$F(x)$ неубывающая;
		\item
		\(\lim\limits_{x\to +\infty} F(x)= 1;\)
		\item
		\(\lim\limits_{x\to -\infty} F(x)= 0;\)
		\item
		$F(x)$ непрерывна справа.
	\end{enumerate}
\end{lemma}
\begin{proof}
	\ 
	\begin{enumerate}
		\item Если $y > x$, то в силу аддитивности $\Pr$, $F(y) - F(x) = \Pr{(x, y]} \geq 0.$
		\item Пусть $x_n \uparrow +\infty$; тогда $(-\infty, x_n] \uparrow \R$. Значит,
		\(
		\lim\limits_{x_n\to +\infty} F(x_n) =
		\lim\limits_{x_n\to +\infty} \Pr{(-\infty, x_n]} =
		\Pr{\R} =
		1
		\)
		\item Аналогично.
		\item Пусть $x_n \downarrow x$; тогда, в силу непрерывности вероятностной меры, $F(x_n) \to
		F(x)$.
	\end{enumerate}
\end{proof}

\textbf{Примеры:}
\begin{itemize}
	\item
    Функция распределения константной случайной величины:
    \[
        F(x) =
        \begin{cases}
	        1, x\geq c;\\
	        0, x < c
	    \end{cases}
    \]

	\item
    Функция распределения случайной величины, равномерно распределённой на $[0, 1]$:
    \[
        F(x) =
        \begin{cases}
	        1, x > 1;\\
	        x, x\in [0, 1];\\
	        0, x < 0
	    \end{cases}
    \]

	Такой функции распределения соответствует вероятностная мера $\Pr$ такая, что \(\forall a < b \in
	[0, 1],\ \Pr{(a, b]} = b - a\). Эту меру также называют \emph{мерой Лебега} \footnote{Говоря простым
    языком, вероятность попадания в полуинтервал пропорциональна его длине.}.
\end{itemize}
