\section{Семинар от 07.10.2016}
Как обычно, начинаем с разбора домашнего задания.
\begin{problem}
    Пусть случайные величины \(\xi_1, \ldots, \xi_n, \eta_1, \ldots, \eta_m\) независимы в совокупности. Докажите, что для любых функций \(f : \R^n \to \R\) и \(g : \R^m \to \R\) случайные величины \(f(\xi_1, \ldots, \xi_n)\) и \(g(\eta_1, \ldots, \eta_n)\) тоже независимы.
\end{problem}
\begin{proof}
    Вообще, утверждение данной задачи почти что очевидно на интуитивном уровне. Но интуиция может подводить, так что докажем формально.
    
    Рассмотрим \(\Pr(f(\xi_1, \ldots, \xi_n) = a, g(\eta_1, \ldots, \eta_n) = b)\). Заметим, что это равно следующей сумме:
    
    \[\sum_{\substack{a_1, \ldots, a_n, b_1, \ldots, b_m: \\ f(a_1, a_2, \ldots, a_n) = a \\ g(b_1, b_2, \ldots, b_m) = b}} \Pr(\underbrace{\xi_1 = a_1, \ldots, \xi_n = a_n, \eta_1 = b_1, \ldots, \eta_m = b_m}_{\text{независимы по условию}}).\]
    
    Тогда \(\Pr(\xi_1 = a_1, \ldots, \xi_n = a_n, \eta_1 = b_1, \ldots, \eta_m = b_m) = \Pr(\xi_1 = a_1) \cdot \ldots \cdot \Pr(\xi_n = a_n) \cdot \Pr(\eta_1 = b_1) \cdot \ldots \cdot \Pr(\eta_m = b_m)\) и, распределяя слагаемые, получим, что эта сумма равна
    
    \[\left(\sum_{\substack{a_1, \ldots, a_n: \\ f(a_1, a_2, \ldots, a_n) = a}} \Pr(\xi_1 = a_1) \cdot \ldots \cdot \Pr(\xi_n = a_n) \right)\left(\sum_{\substack{b_1, \ldots, b_m: \\ g(b_1, b_2, \ldots, b_m) = b}} \Pr(\eta_1 = b_1) \cdot \ldots \cdot \Pr(\eta_m = b_m)\right).\]
    
    Но это есть ни что иное, как \(\Pr(f(\xi_1, \ldots, \xi_n) = a)\Pr(g(\eta_1, \ldots, \eta_n) = b)\). Отсюда получаем независимость \(f(\xi_1, \ldots, \xi_n)\) и \(g(\eta_1, \ldots, \eta_n)\), что и требовалось доказать.
\end{proof}

\begin{problem}
    Случайные величины \(X, Y, Z, W\) независимы в совокупности и одинаково распределены: каждая равновероятно принимает значения \(1\) и \(-1\). Являются ли независимыми в совокупности следующие наборы случайных величин:
    
    \begin{enumerate}
        \item \(XYZ\), \(XYW\), \(XW\);
        \item \(XY\), \(XYW\), \(XW\), \(XZW\);
        \item \(XYZ\), \(XYW\), \(XW\), \(XZW\)?
    \end{enumerate}

\end{problem}
\begin{proof}
    Для начала заметим следующее. Пусть у нас есть дву случайных независимые величины \(\xi\) и \(\eta\), равновероятно принимающие значения из \(\{-1, 1\}\). Тогда случайная величина \(\xi\eta\) тоже равновероятно принимает значения из \(\{-1, 1\}\). Покажем это:
    \[\Pr{\xi\eta = 1} = \Pr{\xi = 1, \eta = 1} + \Pr{\xi = -1, \eta = -1} = \frac{1}{2}.\]
    
    Это доказывает, что любое произведение \(X\), \(Y\), \(Z\), \(W\), в котором каждая случайная величина участвует не более одного раза, равновероятно принимает значение из \(\{-1, 1\}\). Теперь посмотрим на наборы переменных из условия.
    \begin{enumerate}
        \item В данном случае нам надо показать, что для любого набора \((a_1, a_2, a_3), a_i \in \{-1, 1\}\) выполнено
        \[\Pr{XYZ = a_1, XYW = a_2, XW = a_3} = \frac{1}{8} = \Pr{XYZ = a_1}\Pr{XYW = a_2}\Pr{XW = a_3}.\]
        
        Разложим вероятность слева по формуле полной вероятности для событий \(Z = 0\) и \(Z = 1\). В первом случае необходимо определить, какие значения \(X\), \(Y\) и \(W\) должны иметь для выполнения условия \(XY = a_1, XYW = a_2, XW = a_3, Z = 1\). Легко проверить, что из этого следует, что \(X = \frac{a_1a_3}{a_2}\), \(Y = \frac{a_2}{a_3}\), \(Z = 1\), \(W = \frac{a_2}{a_1}\). Тогда
        \[\Pr{XY = a_1, XYW = a_2, XW = a_3 \given Z = 1} = \frac{1}{16}.\]
        
        Во втором же случае она считается практически аналогично и равна
        \[\Pr{X = -\frac{a_1a_3}{a_2}, Y = \frac{a_2}{a_3}, Z = -1, W = -\frac{a_2}{a_1}} = \frac{1}{16}.\]
        
        Суммируя их, получаем, что \(\Pr{XYZ = a_1, XYW = a_2, XW = a_3} = \frac{1}{8}.\) Тогда случайные величины \(XYZ\), \(XYW\), \(XW\) независимы.
        
        \item Решим систему уравнений:
        \[\begin{cases}
        XY = a_1 \\ XYW = a_2 \\ XW = a_3 \\ XZW = a_4
        \end{cases}
        \implies
        \begin{cases}
        X = \frac{a_1 a_3}{a_2} \\ Y = \frac{a_2}{a_3} \\ Z = \frac{a_4}{a_3} \\ W = \frac{a_2}{a_1}
        \end{cases}
        \]
        
        Отсюда следует, что \(\Pr{XY = a_1, XYW = a_2, XW = a_3, XZW = a_4} = \frac{1}{16}\). Отсюда следует независимость.
        
        \item Опять же, решим систему уравнений:
        \[
        \begin{cases}
        XYZ = a_1 \\ XYW = a_2 \\ XW = a_3 \\ XZW = a_4
        \end{cases}
        \implies
        \begin{cases}
        X = \frac{a_1a_3^2}{a_2a_4} \\ Y = \frac{a_2}{a_3} \\ Z = \frac{a_4}{a_3} \\ W = \frac{a_2a_4}{a_1a_3}
        \end{cases}
        \]
        
        Тем самым снова получаем независимость.
    \end{enumerate}
\end{proof}

\begin{problem}
    Докажите, что если последовательность случайных величин \(\{\xi_n\}_{n = 1}^{\infty}\) сходится по вероятности к двум случайным величинам \(\xi\), \(\eta\), \(\xi_n \prto \xi\), \(\xi_n \prto \eta\), то \(\Pr(\xi = \eta) = 1\).
\end{problem}
\begin{proof}
    Суть у данной задачи следующая: сходимость по вероятности даёт настоящий предел, который единственен.
    
    Из определения сходимости по вероятности следует, что для любого \(\varepsilon > 0\)
    \[
    \begin{aligned}
    \Pr{|\xi_n - \xi| > \frac{\varepsilon}{2}} \to 0, \\
    \Pr{|\xi_n - \eta| > \frac{\varepsilon}{2}} \to 0. \\
    \end{aligned}
    \]
    
    Тогда рассмотрим \(\Pr{|\xi - \eta| > \varepsilon}\):
    \[\begin{aligned}
        \Pr{|\xi - \eta| > \varepsilon} &= \Pr{|\xi - \xi_n + \xi_n - \eta| > \varepsilon} \\
        &\leq \Pr{|\xi_n - \xi| + |\xi_n - \eta| > \varepsilon} \\
        &\leq \Pr{\left\{|\xi_n - \xi| > \frac{\varepsilon}{2}\right\} \cup \left\{|\xi_n - \eta| > \frac{\varepsilon}{2}\right\}} \\ 
        &\leq \Pr{|\xi_n - \xi| > \frac{\varepsilon}{2}} + \Pr{|\xi_n - \eta| > \frac{\varepsilon}{2}} \to 0.
    \end{aligned}\]
    Второе неравенство легко понять, если представить его графически. Изобразим первую четверть координатной плоскости с координатами \(x = |\xi_n - \xi|\) и \(y = |\xi_n - \eta|\). Тогда для выражения справа недопустимая область будет иметь вид прямоугольного треугольника с координатами вершин \((0, 0)\), \((0, 1)\) и \((1, 0)\). Для выражения справа недопустимая область~--- это квадрат с координатами \((0,0)\), \((0, 1/2)\), \((1/2, 1/2)\) и \((1/2, 0)\). Но данный квадрат полностью включается в треугольник.
\end{proof}

\begin{problem}
    Случайные величины \(\{\xi_n\}_{n = 1}^{\infty}\) независимы в совокупности. Обозначим \(S_n = \xi_1 + \xi_2 + \ldots + \xi_n\). Пусть случайная величина \(\xi_n\)
    
    \begin{enumerate}
        \item принимает три значения \(\{-n, 0, n\}\) с вероятностями \(\left(\frac{1}{2n^2}, 1 - \frac{1}{n^2}, \frac{1}{2n^2}\right)\);
        \item принимает три значения \(\{-2^n, 0, 2^n\}\) с вероятностями \(\left(2^{-n - 1}, 1 - 2^{-n}, 2^{-n - 1}\right)\).
    \end{enumerate}

    Выясните, в каком случае выполнен закон больших чисел: \(\frac{S_{n}}{n} \prto 0\) при \(n\to \infty\)?
\end{problem}
\begin{proof}
    Для начала заметим, что и в первом, и во втором случае матожидание \(\xi_n\) равно 0. Тогда \(\E{S_n} = 0\) и можно применить неравенство Чебышёва:
    \[\Pr{\left|\frac{S_n}{n}\right| \geq \alpha} \leq \frac{\D{S_n}}{\alpha n^2}.\]
    
    Теперь посчитаем \(\D{\xi_n}\):
    \begin{enumerate}
        \item \(\D{\xi_n} = \frac{1}{2n^2}n^2 + \frac{1}{2n^2}n^2 = 1\),
        \item \(\D{\xi_n} = \frac{1}{2^{n + 1}}2^{2n} + \frac{1}{2^{n + 1}}2^{2n} = 2^n\).
    \end{enumerate}
    
    Отсюда видно, что в первом случае закон больших чисел точно выполняется. Однако про второй мы ничего сказать не можем. Как быть?
    
    Зафиксируем некоторое \(\omega \in \Omega\) и рассмотрим последовательность \(\{\xi_n(\omega)\}_{n = 1}^{\infty}\). Утверждается следующее: в данной последовательности, начиная с некоторого номера \(N\), будут лишь нули. Докажем это. Зафиксируем некоторое натуральное число \(m\) и рассмотрим следующую вероятность:
    \[\Pr{\exists n \geq m : \xi_{n} \neq 0} = \Pr{\sum_{n = m}^{\infty} \{\xi_n \neq 0\}} \leq \sum_{n = m}^{\infty} \Pr{\xi_n \neq 0} = \sum_{n = m}^{\infty} \frac{1}{2^n} = \frac{1}{2^{m - 1}} \xrightarrow{m \to \infty} 0.\]
    
    Теперь посчитаем вероятность того, что последовательность \(\{\xi_n(\omega)\}_{n = 1}^{\infty}\) не финитна (то есть содержит бесконечное число ненулевых элементов). Она равна
    \[\Pr{\forall m \in \N\ \exists n \geq m : \xi_{n}(\omega) \neq 0} = \lim\limits_{m \to \infty} \Pr{\exists n \geq m : \xi_{n} \neq 0} = 0.\]
    
    Это и доказывает финитность \(\{\xi_n(\omega)\}_{n = 1}^{\infty}\). Тогда для любого \(\omega \in \Omega\) выполнено
    \[\lim\limits_{n \to \infty} \frac{\sum\limits_{i = 1}^{n} \xi_i(\omega)}{n} = 0 \implies \frac{S_n}{n} \prto 0.\]
    
    Это доказывает то, что и для второго случая тоже выполнен закон больших чисел.
\end{proof}

Теперь обсудим применения вероятностного метода к решению задач.

Рассмотрим полный граф на \(n\) вершинах. Покрасим его ребра в два цвета. Тогда имеет место следующая теорема:
\begin{theorem}[Рамсей]
    В достаточно большом полном графе с произвольной раскраской рёбер в белый и чёрный цвета найдётся черный полный подграф размера \(r\) и белый полный подграф размера \(s\) для наперёд заданных \(r\) и \(s\).
\end{theorem}

\begin{definition}
    Пусть \(m, n \in \N\). Число Рамсея \(\operatorname{R}(m, n)\)~--- это наименьшее из таких чисел \(x \in \N\), что при любой раскраске ребер полного графа на \(x\) вершинах в два цвета найдется клика на \(n\) вершинах с ребром цвета 1 или клика на \(m\) вершинах с ребром цвета 2.
\end{definition}

\begin{problem}[Спенсер, 1974]
    Докажите, что \[\operatorname{R}(k, k) \geq \frac{\sqrt{2}}{e}k2^{k/2}(1 + o(1)).\]
\end{problem}
\begin{proof}
     Положим, что у нас есть полный граф на \(n\) вершинах, раскрашенный в два цвета произвольным образом.
    
     Пусть \(S\)~--- это некоторый подграф на \(k\) вершинах. Тогда введём событие \(A_{S} = \{\)все рёбра в \(S\) одного цвета\(\}\). Легко понять, что \(\Pr{A_S} = 2^{1 - \binom{k}{2}}\). Заметим, что если для двух подграфов \(S\) и \(T\) размера \(k\) выполнено \(|S \cap T| \leq 1\), то события \(A_{S}\) и \(A_{T}\) независимы (так как эти подграфы не содержат общих рёбер). Тогда событие \(A_{S}\) зависит от \(d\) событий, где
     \[d = \left|\{T : |T| = k, |S \cap T| > 1\}\right| = \binom{k}{2}\binom{n - 2}{k - 2}.\]
     
     Эту оценку можно получить достаточно просто: достаточно выбрать какие-либо две вершины в \(S\) и сказать, что эти вершины принадлежат \(T\), после чего добавить \(k - 2\) каких-либо вершин в \(T\).
     
     Осталось лишь найти \(n\) такие, что выполняется требование локальной леммы:
     
     \[ep(d + 1) \leq 1 \iff e2^{1 - \binom{k}{2}}\left(\binom{k}{2}\binom{n - 2}{k - 2} + 1\right) \leq 1.\]
     
     Преобразуем это неравенство, пользуясь следствием из формулы Стирлинга: \(\binom{n}{k} \leq \left(\frac{en}{k}\right)^{k}\)
     
     \[\begin{aligned}
     \binom{k}{2}\binom{n - 2}{k - 2} &\leq \frac{2^{\binom{k}{2}}}{2e} - 1 \\
     \left(\frac{en}{k}\right)^{k - 2} &\leq \frac{2^{\binom{k}{2}}}{ek(k - 1)} - \frac{1}{k(k - 1)} \\
     \left(\frac{en}{k}\right)^{k - 2} &\leq 2^{\binom{k}{2}}\left(\frac{1}{ek(k - 1)} + o(1)\right) \\
     \left(\frac{en}{k}\right)^{k - 2} &\leq 2^{\binom{k}{2}}\left(1 + o(1)\right) \\
     n &\leq \frac{1}{e}k2^{\frac{k}{2} + \frac{k}{2(k - 2)}}(1 + o(1)) \\
     n &\leq \frac{\sqrt{2}}{e}k2^{\frac{k}{2}}(1 + o(1))
     \end{aligned}\]
     
     Отсюда получаем, что \(\operatorname{R}(k, k) \geq \frac{\sqrt{2}}{e}k2^{k/2}(1 + o(1))\).
     
     % TODO: дописать доказательство нормальным образом
\end{proof}