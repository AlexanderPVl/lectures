\section{Семинар от 29.11.2016}
Разбор домашки ждать не будет.
\begin{problem}
	Случайные величины \(\xi_1\), \(\xi_2\) независимы и имеют равномерное распределение на отрезке \([0, 1]\). Найдите плотности случайных величин \(\xi_1 + \xi_2\), \(\xi_1 - \xi_2\) и \(\xi_1\xi_2\).
\end{problem}
\begin{proof}
	Перед тем, как пользоваться формулами свёртки, найдём распределение \(\eta = -\xi_2\):
	\[F_{\eta}(x) = \Pr{\eta \leq x} = \Pr{\xi \geq -x} = 1 - \Pr{\xi \leq -x} = 1 - F_{\xi}(-x).\]
	\[p_{\eta} = p_{\xi}(-x) = I\left\{x \in [-1, 0]\right\}.\]
	
	Напомню формулу свёртки:
	\[p_{\xi + \eta} = \int\limits_{-\infty}^{+\infty} p_{\xi}(x - y)p_{\eta}(y)\,\mathrm{d}y.\]
	
	Теперь можем считать:
	\begin{enumerate}
		\item Для начала рассмотрим сумму. Заметим, что \(0 \leq \xi_1 + \xi_2 \leq 2\), поэтому при \(x \not\in [0, 2]\) \(p_{\xi_1 + \xi_2}(x) = 0\). Иначе же
		\begin{align}
			p_{\xi_1 + \xi_2}(x) &= \int\limits_{-\infty}^{+\infty} I\left\{(x - y) \in [0, 1]\right\}I\{y \in [0, 1]\}\,\mathrm{d}y \\
			&= \int\limits_{-\infty}^{+\infty} I\left\{y \in [x - 1, x]\right\}I\{y \in [0, 1]\}\,\mathrm{d}y.
		\end{align}
		Осталось рассмотреть два случая. Если \(x \in [0, 1]\), то \(p_{\xi_1 + \xi_2}(x) = x\). Если же \(x \in [1, 2]\), то \(p_{\xi_1 + \xi_2}(x) = 1 - x + 1 = 2 - x\).
		
		Запишем ответ:
		\[p_{\xi_1 + \xi_2}(x) = \begin{cases}
		x,& x \in [0, 1]; \\
		2 - x,& x \in [1, 2]; \\
		0,& \text{иначе}.
		\end{cases}\]
		
		\item 
	\end{enumerate}
\end{proof}