\section{Семинар от 21.10.2016}
Так как на прошлом семинаре была контрольная работа, то начнём с её разбора.
\begin{problem}
    В колоде содержится 52 карты четырёх мастей. Каждому из четырёх игроков раздают по шесть карт. Найдите вероятность того, что хотя бы у одного игрока будут карты одной масти.
\end{problem}
\begin{proof}
    Введём событие \(A_i = \{\)у \(i\)-го игрока все карты одной масти\(\}\). Тогда вероятность из условия равна \(\Pr{A_1 \cup A_2 \cup A_3 \cup A_4}\). Пользуясь формулой включений-исклю\-чений, получаем, что
    \[\begin{aligned}
    \Pr{A_1 \cup A_2 \cup A_3 \cup A_4} &= \Pr{A_1} + \dotsb + \Pr{A_4} - \Pr{A_1 \cap A_2} - \dotsb - \Pr{A_3 \cap A_4} \\ &+ \Pr{A_1 \cap A_2 \cap A_3} + \dotsb + \Pr{A_2 \cap A_3 \cap A_4} - \Pr{A_1 \cap A_2 \cap A_3 \cap A_4}.
    \end{aligned}\]
    
    Осталось посчитать эти вероятности:
    
    \[\Pr{A_i} = \frac{4 \cdot \binom{13}{6}}{\binom{52}{6}},\qquad \Pr{A_i \cap A_j} = \frac{4 \cdot \binom{13}{6}\binom{7}{6}}{\binom{52}{6}\binom{46}{6}} + \frac{4 \cdot 3 \cdot \binom{13}{6}^2}{\binom{52}{6}\binom{46}{6}},\]
    
    \[\Pr{A_i \cap A_j \cap A_k} = \frac{4 \cdot 3 \cdot \binom{3}{2}\binom{13}{6}^2\binom{7}{6}}{\binom{52}{6}\binom{46}{6}\binom{40}{6}} + \frac{4 \cdot 3 \cdot 2 \cdot \binom{13}{6}^3}{\binom{52}{6}\binom{46}{6}\binom{40}{6}},\]
    
    \[\Pr{A_1 \cap A_2 \cap A_3 \cap A_4} = \frac{4!\cdot \binom{13}{6}^4 + 4 \cdot 3 \cdot \binom{4}{2}\binom{13}{6}^2\binom{7}{6}^2 + 4 \cdot 3 \cdot 2 \cdot \binom{4}{2}\binom{13}{6}^3\binom{7}{6}}{\binom{52}{6}\binom{46}{6}\binom{40}{6}\binom{34}{6}}.\qedhere\]
\end{proof}

\begin{problem}
    Два ящика содержат красные и чёрные шары. Ящик \(A\) содержит 2 красных и 1 чёрный шар. Ящик \(B\)~--- 101 красный и 100 черных шаров. Вам предлагается сыграть в следующую игру. В тайне от вас равновероятно выбирается один из ящиков. Ваша задача~--- угадать, какой ящик был выбран. Для этого из ящика последовательно вынимается два шара. После вытаскивания первого шара и определения его цвета вы решаете, вернуть ли в ящик этот шар перед вторым вытаскиванием или нет. Какой должна быть ваша стратегия, чтобы максимизировать вероятность выигрыша?
\end{problem}
\begin{proof}
    Рассмотим вероятности вытащить определённый набор в зависимости от того, какой ящик выбран:
    
    \begin{center}
        \begin{tabular}{|c|c|c|c|c|}
            \hline \diaghead{Шарыящик}{Шары}{Ящик} & I(возврат) & II(возврат) & I(без возврата) & II(без возврата) \\
            \hline&&&&\\[-10pt]
            КК & \(\dfrac{4}{9}\) & \(\dfrac{101^2}{201^2}\) & \(\dfrac{1}{3}\) & \(\dfrac{101 \cdot 100}{201 \cdot 200}\) \\[10pt]
            \hline&&&&\\[-10pt]
            КЧ/ЧК & \(\dfrac{2}{9}\) & \(\dfrac{101 \cdot 100}{201^2}\) & \(\dfrac{1}{3}\) & \(\dfrac{101 \cdot 100}{201 \cdot 200}\) \\[10pt]
            \hline&&&&\\[-10pt]
            ЧЧ & \(\dfrac{1}{9}\) & \(\dfrac{100^2}{201^2}\) & \(0\) & \(\dfrac{100 \cdot 99}{201 \cdot 200}\) \\[10pt]
            \hline
        \end{tabular}
    \end{center}

    Теперь можно составить таблицу вероятностей того, какой ящик выбран, в зависимости от того, какой набор был вытащен:
    
    \begin{center}
        \begin{tabular}{|c|c|c|c|c|}
            \hline \diaghead{Шарыящик}{Шары}{Ящик} & I(возврат) & II(возврат) & I(без возврата) & II(без возврата) \\
            \hline&&&&\\[-10pt]
            КК & \(\dfrac{17956}{28157} \approx 0.638\) & \(\dfrac{10201}{28157}\approx 0.362\) & \(\dfrac{13400}{23500} \approx 0.57\) & \(\dfrac{10100}{23500} \approx 0.43\) \\[10pt]
            \hline&&&&\\[-10pt]
            КЧ/ЧК & \(\dfrac{8978}{19078} \approx 0.471\) & \(\dfrac{10100}{19078} \approx 0.529\) & \(\dfrac{13400}{23500} \approx 0.57\) & \(\dfrac{10100}{23500} \approx 0.43\) \\[10pt]
            \hline&&&&\\[-10pt]
            ЧЧ & \(\dfrac{4489}{14489} \approx 0.31\) & \(\dfrac{10000}{14489} \approx 0.69\) & \(0\) & \(1\) \\[10pt]
            \hline
        \end{tabular}
    \end{center}

    Допустим, что на первом вытаскивании нам попался красный шар. Что выгоднее: оставить его или же вернуть назад? Если мы оставим шар, то максимальный шанс выигрыша равен \(0.57\). Если же мы вернём его назад, то минимальный шанс выигрыша будет равен \(0.638\). Тогда выгоднее вернуть красный шар. Если же мы вынимаем чёрный, то его лучше оставить. При этом указывать на первый ящик стоит только тогда, когда выпало два красных шара или чёрный, а затем красный.
\end{proof}

\begin{problem}
    Случайные величины \(\{\xi_n\}_{n = 1}^{\infty}\) независимы в совокупности и принимают значения \(3\) и \(-2\) с вероятностью \(p\) и \(1 - p\) соответственно. Обозначим \(S_n = \xi_1 + \dots + \xi_n\). Найдите условную вероятность
    \[\Pr{S_n = x \given S_k = y, S_m = z}\text{ при } m < n < k, x, y, z \in \Z.\]
\end{problem}
\begin{proof}
    По определению условной вероятности
    \[\Pr{S_n = x \given S_k = y, S_m = z} = \frac{\Pr{S_k = y, S_n = x, S_m = z}}{\Pr{S_k = y, S_m = z}}.\]
    
    Преобразуем числитель и знаменатель:
    \[\begin{aligned}
    \Pr{S_k = y, S_n = x, S_m = z} &= \Pr(\underbrace{S_k - S_n = y - x, S_n - S_m = x - z, S_m = z}_{\text{независимы}}) \\ &= \Pr{S_k - S_n = y - x}\Pr{S_n - S_m = x - z}\Pr{S_m = z}, \\
    \Pr{S_k = y, S_m = z} &= \Pr{S_k - S_m = y - z}\Pr{S_m = z}.
    \end{aligned}\]
    
    Отсюда получаем, что
    \[\Pr{S_n = x \given S_k = y, S_m = z} = \frac{\Pr{S_k - S_n = y - x}\Pr{S_n - S_m = x - z}}{\Pr{S_k - S_m = y - z}}.\]
    
    Осталось посчитать какую-либо из вероятностей. Остальные считаются абсолютно аналогично. Допустим, посчитаем \(\Pr{S_k - S_n = y - x}\). Для этого необходимо, чтобы в сумме было \(l\) случайных величин, равных \(3\), и \(k - n - l\) случайных величин, равных \(-2\). Тогда
    \[y - x = 3l - 2n + 2k + 2l \implies l = \frac{y - x + 2n - 2k}{5}\]
    
    Отсюда получаем необходимые условия на то, что данная вероятность не равна 0: \(y - x + 2n - 2k\) должно быть больше 0 и делиться на 5. Тогда 
    \[\Pr{S_k - S_n = y - x} = \binom{k - n}{l}p^{l}(1 - p)^{k - n - l}.\]
    
    Посчитав две остальных вероятности и сократив, мы получим ответ. Что самое интересное, он не зависит от \(p\).
\end{proof}

\begin{problem}
    Каждую целочисленную точку числовой оси независимо от других назвоём белой с вероятностью \(p\) или чёрной с вероятностью \(1 - p\). Пусть \(S\)~--- множество всех таких целочисленных точек \(x\), что расстояние от \(x\) до ближайшей черной точки (включая \(x\), если \(x\)~--- чёрная точка) не меньше расстояния от \(x\) до начала координат. Найдите математическое ожидание и дисперсию числа элементов в \(S\).
\end{problem}
\begin{proof}
    Введём событие \(A_i = \{i \in S\}\). Тогда легко понять, что
    \[\Pr{A_i} = \begin{cases}
    1& i = 0, \\
    p^{2|i| - 1}& i \neq 0.
    \end{cases}\]
    
    Отсюда получаем, что
    
    \[\E{S} = 2\sum_{i = 1}^{\infty} p^{2i - 1} + 1 = \frac{2p}{1 - p^2} + 1.\]
    
    Теперь посчитаем дисперсию. Выразим её через сумму ковариаций:
    \[\D{S} = \sum_{i, j \in \Z} \cov(I_{A_i}, I_{A_j}) = \sum_{i \in \Z} \cov(I_{A_i}, I_{A_i}) + 2\sum_{\substack{i, j \in \Z \\ i > j}} \cov(I_{A_i}, I_{A_j}).\]
    
    Так как \(\cov(I_{A_i}, I_{A_j}) = \Pr{A_i \cap A_j} - \Pr{A_i}\Pr{A_j}\) и \(\cov(I_{A_i}, I_{A_j}) = 0\) при \(ij \leq 0\), то дисперсию можно записать в виде
    \[\D{S} = 2\sum_{i = 1}^{\infty}\left(p^{2i - 1} - p^{4i - 2}\right) + 1 + 2\sum_{i = 1}^{\infty}\sum_{j = i + 1}^{\infty}\left(p^{2j - 1} - p^{2(i + j) - 2}\right).\qedhere\]
\end{proof}

Теперь вернёмся к традиции разбирать домашнее задание на семинаре.

\begin{problem}
    Задача без условия на ЛЛЛ
\end{problem}
\begin{proof}
    содержимое...
\end{proof}

\begin{problem}
    Что-то там про депутатов
\end{problem}
\begin{proof}
    содержимое...
\end{proof}

Теперь обсудим задачи на тему общего понятия вероятности события.

\begin{problem}
    Палочку разламывают в двух произвольных местах. Найдите вероятность того, что из полученных кусков можно сложить треугольник.
\end{problem}
\begin{proof}[Решение]
    Без ограничения общности положим длину палочки, равную 1. 
    
    Пусть палочку разломали в точках \(x\) и \(y\), \(x < y\). При каких условиях на \(x\) и \(y\) можно сложить треугольник? Очевидно, если для всех трёх сторон выполнено неравенство треугольника: длина стороны не превосходит суммы длин двух других сторон. Так как стороны равны \(x\), \(y - x\) и \(1 - y\), то
    \[\begin{cases}
    x < y - x + 1 - y \\
    y - x < x + 1 - y \\
    1 - y < x + y - x
    \end{cases}
    \implies
    \begin{cases}
    x < \frac{1}{2} \\
    y < \frac{1}{2} + x \\
    y > \frac{1}{2}
    \end{cases}
    \]
    Изобразим решение этих неравенств и все возможные значения \(x\) и \(y\) (подходящие значения \((x, y)\) выделены серым):
    \begin{center}
        \begin{tikzpicture}
            \fill [pattern = north west lines] (0, 0) -- (2, 0) -- (2, 2) -- (0, 2) -- cycle;
            \fill [gray] (1, 1) -- (1, 2) -- (0, 1) -- cycle;
            \fill [gray] (1, 1) -- (1, 0) -- (2, 1) -- cycle;
            \draw [thick, ->] (0, -0.5) -- (0, 2.5) node [anchor = east] {\(y\)};
            \draw [thick, ->] (-0.5, 0) -- (2.5, 0) node [anchor = north] {\(x\)};
            \draw (0, 0) node [anchor = north east] {\(0\)};
            \draw (0, 2) node [anchor = east] {\(1\)};
            \draw (2, 0) node [anchor = north] {\(1\)};
            \draw (0, 0) -- (2, 0) -- (2, 2) -- (0, 2) -- cycle;
            \draw (1, 1) -- (1, 0) -- (2, 1) -- cycle;
            \draw (1, 1) -- (1, 2) -- (0, 1) -- cycle;
        \end{tikzpicture}
    \end{center}

    Отсюда получаем, что вероятность этого события равна \(\frac{1}{4}\).
\end{proof}

Немного усложним задачу.
\begin{problem}
    Пусть палочку длины 1 ломают следующим образом: сначала её ломают в произвольном месте, затем более длинный кусок ломают в произвольном месте. Найдите вероятность того, что из полученных кусков можно сложить треугольник.
\end{problem}
\begin{proof}[Решение]
    На первый взгляд может показаться, что эта задача ничем не отличается от предыдущей. Но нет. В данной задаче отличается множество допустимых пар \((x, y)\). Изобразим это графически, заштриховав допустимые значения:
    \begin{center}
        \begin{tikzpicture}
        \fill [pattern = north west lines] (0, 0) -- (2, 0) -- (2, 2) -- (0, 2) -- cycle;
        \fill [white] (1, 1) -- (1, 2) -- (2, 2) -- cycle;
        \fill [white] (1, 1) -- (1, 0) -- (0, 0) -- cycle;
        \draw [thick, ->] (0, -0.5) -- (0, 2.5) node [anchor = east] {\(y\)};
        \draw [thick, ->] (-0.5, 0) -- (2.5, 0) node [anchor = north] {\(x\)};
        \draw (0, 0) node [anchor = north east] {\(0\)};
        \draw (0, 2) node [anchor = east] {\(1\)};
        \draw (2, 0) node [anchor = north] {\(1\)};
        \draw (0, 0) -- (2, 0) -- (2, 2) -- (0, 2) -- cycle;
        \draw (1, 1) -- (1, 2) -- (2, 2) -- cycle;
        \draw (1, 1) -- (1, 0) -- (0, 0) -- cycle;
        \end{tikzpicture}
    \end{center}

    При этом подходящие пары \((x, y)\) те же, что и раньше. Отсюда получаем, что вероятность этого события равна \(\frac{1}{3}\).
\end{proof}