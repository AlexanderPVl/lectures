\section{Лекция от 29.11.2016}
\subsection{Многомерный случай}
\epigraph{Многомерный случай - это одномерный случай.}{Д.А. Шабанов}
Ранее мы рассматривали только одну случайную величину за раз. Но порой возникают ситуации, когда их нужно рассматривать по несколько одновременно. Введём необходимую теорию для этого.

\begin{definition}
	Пусть \(\Pr\)~--- это некоторая вероятностная мера на \(\R^{n}\) (\(n > 1\)). Тогда назовём \emph{функцией распределения} \(F: \R^{n} \to [0, 1]\) такую, что для любых \(x_1, x_2, \dots, x_n \in \R\) верно следующее:
	\[
	F(x_1, x_2, \dots, x_n) = \Pr{(-\infty, x_1] \times (-\infty, x_2] \times \dots \times (-\infty, x_n]}.
	\]
\end{definition}
\begin{remark}
	Для удобства это определение можно записать следующим образом (в данном случае \(\vec{x} = (x_1, x_2, \dots, x_n)\)):
	\[
	F(\vec{x}) = \Pr{(-\infty, \vec{x}]}.
	\]
	
	Вообще, запись \((-\infty, \vec{x}]\) не несёт особого смысла, и её стоит понимать, как сокращение для \((-\infty, x_1] \times (-\infty, x_2] \times \dots \times (-\infty, x_n]\).
\end{remark}

Как и в одномерном случае, фукнция распределения обладает несколькими свойствами. Непрервыность справа и предельные значения легко обобщить, а как быть с неубыванием? Для этого это свойство несколько изменяют. В итоге получается следующая лемма.

\begin{lemma}[Основные свойства функции распределения]
	Функция распределения в многомерном случае обладает следующими свойствами:
	\begin{enumerate}
		\item Пусть есть некоторая точка \(\vec{x}\) и последовательность точек \(\{\vec{x}^{(m)}\}_{m = 1}^{\infty}\), про которую известно, что \(\vec{x}^{(m)} \downarrow \vec{x}\) и \(x_{i}^{(m + 1)} \leq x_{i}^{(m)}\) для всех \(i \in \{1, 2, \dots, n\}\). Тогда
		\[
		\lim\limits_{m \to \infty} F(\vec{x}^{(m)}) = F(\vec{x}).
		\]
		
		Данное свойство выступает аналогом непрерывности справа в одномерном случае.
		
		\item 
		\[
		\lim\limits_{\substack{x_1 \to +\infty \\ \dots \\ x_n \to +\infty}} F(x_1, x_2, \dots, x_n) = 1.
		\]
		В данном случае этот предел множно понимать, как повторный.
		
		\item Для любого \(i \in \{1, 2, \dots, n\}\) выполнено
		\[
		\lim\limits_{x_i \to -\infty} F(x_1, x_2, \dots, x_n) = 0.
		\]
		
		\item Введём оператор \(\Delta_{a, b}^{i} F(x_1, \dots, x_n)\), равный
		\[
		F(x_1, \dots, x_{i - 1}, b, x_{i + 1}, \dots, x_n) - F(x_1, \dots, x_{i - 1}, a, x_{i + 1}, \dots, x_n).
		\]
		Тогда для любого набора чисел \(a_1, \dots, a_n, b_1, \dots, b_n\) такого, что \(a_i \leq b_i\) для любого \(i \in \{1, 2, \dots, n\}\), выполнено
		\[
		\Delta_{a_1, b_1}^{1} \Delta_{a_2, b_2}^{2} \dots \Delta_{a_n, b_n}^{n} F(x_1, \dots, x_n) \geq 0.
		\]
		
		Данное свойство обобщает неубывание на многомерное пространство.
	\end{enumerate}
\end{lemma}
\begin{proof}
	В принципе, доказательство первых трёх пунктов идейно ничем не отличается от одномерного случая. Но всё равно докажем их.
	\begin{enumerate}
		\item Если \(\{\vec{x}^{(m)}\}_{m = 1}^{\infty}\) монотонно сходится к \(\vec{x}\) сверху, то (все сходимости монотонные):
		\[
		\left\{\begin{matrix}
		\left(-\infty, x_1^{(m)}\right] \downarrow \left(-\infty, x_1\right] \\
		\dots \\
		\left(-\infty, x_n^{(m)}\right] \downarrow \left(-\infty, x_n\right]
		\end{matrix}\right\}
		\implies
		\left(-\infty, \vec{x}^{(m)}\right] \downarrow \left(-\infty, \vec{x}\right].
		\]
		Следовательно, пользуясь непрерывностью меры, получаем, что	
		\[
		\lim\limits_{m \to \infty} F(\vec{x}^{(m)}) = \lim\limits_{m \to \infty} \Pr{\left(-\infty, \vec{x}^{(m)}\right]} = \Pr{\left(-\infty, \vec{x}\right]} = F(\vec{x}).
		\]
		
		\item Для начала заметим следующее (все сходимости монотонные):
		\[
		\left\{\begin{matrix}
		\left(-\infty, x_1\right] \xrightarrow[x_1 \to +\infty]{} \left(-\infty, +\infty\right) \\
		\dots \\
		\left(-\infty, x_n\right] \xrightarrow[x_n \to +\infty]{} \left(-\infty, +\infty\right)
		\end{matrix}\right\}
		\implies
		\left(-\infty, \vec{x}\right] \to \R^{n}.
		\]
		
		Тогда по непрерывности меры получаем, что
		\[
		\lim\limits_{\substack{x_1 \to +\infty \\ \dots \\ x_n \to +\infty}} F(x_1, x_2, \dots, x_n) = \Pr{\R^{n}} = 1.
		\]
		
		\item Если \(x_{i} \to -\infty\), то \(\left(-\infty, x_i\right] \downarrow \emptyset\). Тогда 
		\[
		\left(-\infty, \vec{x}\right] \text{ сходится к } \left(-\infty, x_1\right] \times \dots \times \emptyset \times \dots \times \left(-\infty, x_n\right] = \emptyset.
		\]
		Тогда по непрерывности меры получаем, что
		\[
		\lim\limits_{x_i \to -\infty} F(x_1, x_2, \dots, x_n) = \Pr{\emptyset} = 0.
		\]
		
		\item Следующее свойство достаточно муторно доказывать в общем случае, так что разберём его для двумерного пространства~--- идея доказательства та же. Распишем это, пользуясь определением оператора \(\Delta_{a, b}^{i}\):
		\begin{align}
			\Delta_{a_1, b_1}^{1} \Delta_{a_2, b_2}^{2} F(x_1, x_2) &= \Delta_{a_1, b_1}^{1} \left(F(x_1, b_2) - F(x_1, a_2)\right) \\
			&= F(b_1, b_2) - F(a_1, b_2) - F(b_1, a_2) + F(a_1, a_2).
		\end{align}
		
		Теперь воспользуемся определением функции распределения\footnote{На это выражение можно посмотреть и графически, но это роскошь двумерного случая.}:
		\begin{multline*}
			F(b_1, b_2) - F(a_1, b_2) - F(b_1, a_2) + F(a_1, a_2) = \Pr{(-\infty, b_1] \times (-\infty, b_2]} - \\ - \Pr{(-\infty, a_1] \times (-\infty, b_2]} - \Pr{(-\infty, b_1] \times (-\infty, a_2]} + \Pr{(-\infty, a_1] \times (-\infty, a_2]} = \\
			= \Pr{(a_1, b_1] \times (-\infty, b_2]} - \Pr{(a_1, b_1] \times (-\infty, a_2]} = \Pr{(a_1, b_1] \times (a_2, b_2]} \geq 0.
		\end{multline*}
		
		Можно заметить, что эта сумма есть ни что иное, как формула включений-исключе\-ний. Тогда в общем случае
		\[
		\Delta_{a_1, b_1}^{1} \Delta_{a_2, b_2}^{2} \dots \Delta_{a_n, b_n}^{n} F(x_1, \dots, x_n) = \Pr{(a_1, b_1] \times \dots \times (a_n, b_n]} \geq 0.\qedhere
		\]
	\end{enumerate}
\end{proof}

Последнее свойство на первый взгляд кажется странным. Возникает вопрос: а нельзя ли его заменить на неубывание по всем координатам? Увы, но нет.
\begin{example}
	Рассмотрим следующую функцию:
	\[
	F(x, y) = \begin{cases}
	1,& x + y \geq 0 \\
	0,& \text{иначе}
	\end{cases}
	\]
	
	Изобразим её следующим образом: в заштрихованной области значение функции равно 1, в незаштрихованной~--- 0.
	\begin{center}
		\begin{tikzpicture}
			\tkzInit[xmax=2,ymax=2,xmin=-2,ymin=-2];
			\tkzGrid;
			\tkzAxeXY;
			\tkzDefPoints{-2/2/A, 2/-2/B, 2/2/C};
			\tkzDrawPolygon[draw=none,pattern=north west lines](A, B, C);
			\tkzDrawSegment(A, B);
		\end{tikzpicture}
	\end{center}

	Данная функция удовлетворяет свойствам 1-3 и не убывает по каждой координате. Проверим, выполняется ли свойство 4. Рассмотрим следующее значение:
	\[
	\Delta_{-1, 1}^{1} \Delta_{-1, 1}^{2} F(x_1, x_2) = F(1, 1) - F(-1, 1) - F(1, -1) + F(-1, -1) = -1 < 0.
	\]
	
	Как видим, оно не выполняется.
\end{example}

В одномерном случае между функцией распределения и вероятностной мерой есть биекция. Следующая теорема утверждает, что это верно и в многомерном случае:
\begin{theorem}
	Пусть \(F(x): \R^{n} \to [0, 1]\)~--- некоторая функция, удовлетворяющая основным свойствам функции распределения. Тогда существует единственная вероятностная мера \(\Pr\) на \(\R^{n}\) такая, что \(F\)~--- это её функция распределения. Другими словами, для любого \(\vec{x} \in \R^{n}\)
	\[
	\Pr{\left(-\infty, \vec{x}\right]} = F(\vec{x}).
	\]
\end{theorem}
Теорема хорошая, доказывать мы её, конечно, не будем.

Рассмотрим несколько примеров многомерных распределений.

Самый простое, что может быть~--- это произведение функций распределения для одномерных случаев. Другими словами, функция распределения определяется следующим образом: пусть есть \(n\) одномерных функций распределения \(F_1, F_2, \dots, F_n\). Тогда
\[
F(x_1, x_2, \dots, x_n) = \prod\limits_{i = 1}^{n} F_i(x_i).
\]
Проверим, что это действительно будет функцией распределения. Легко проверить, что свойства \(1-3\) выполнены. Теперь проверим последнее свойство. Заметим, что
\[
\Delta_{a_1, b_1}^{1} \Delta_{a_2, b_2}^{2} \dots \Delta_{a_n, b_n}^{n} F(x_1, \dots, x_n) = \prod\limits_{i = 1}^{n} \Delta_{a_i, b_i}^{1}F_i(x_i) \geq 0.
\]

Теперь рассмотрим несколько другой случай. Пусть есть неотрицательная функция \(p(x_1, x_2, \dots, x_n)\) такая, что
\[
\int\limits_{\R^{n}} p(t_1, t_2, \dots, t_n)\,\mathrm{d}t_1\dots\mathrm{d}t_n = 1.
\]

Тогда введём следующую функцию:
\[
F(x_1, x_2, \dots, x_n) = \int\limits_{-\infty}^{x_1} \dots \int\limits_{-\infty}^{x_n} p(t_1, t_2, \dots, t_n)\,\mathrm{d}t_1\dots\mathrm{d}t_n
\]

Покажем, что это функция распределения. Первые три свойства легко проверяются. Теперь проверим четвёртое:
\[
\Delta_{a_1, b_1}^{1} \Delta_{a_2, b_2}^{2} \dots \Delta_{a_n, b_n}^{n} F(x_1, \dots, x_n) = \int\limits_{a_1}^{b_1} \dots \int\limits_{a_n}^{b_n} p(t_1, t_2, \dots, t_n)\,\mathrm{d}t_1\dots\mathrm{d}t_n \geq 0.
\]

В этом случае \(p(x_1, x_2, \dots, x_n)\) принято называть \emph{плотностью} функции распределения \(F\).

Допустим, что есть некоторая функция распределения \(F(x_1, x_2, \dots, x_n)\). В каком случае она имеет плотность? Ответ даёт теорема Радона-Никодима\footnote{Эта теорема гласит о том, что если функция абсолютно непрерывна, то она представима в нужном нам виде. За доказательством обращайтесь к учебнику по функциональному анализу.}: функция должна быть абсолютно непрерывной. Что это значит?
\begin{definition}
	Вероятностная мера \(\Pr\) называется \emph{абсолютно непрерывной}, если для любого множества \(B \in \mathcal{B}(\R^n)\) с лебеговой мерой 0 выполнено \(\Pr{B} = 0\).
\end{definition}

Если функция \(F\) (а значит, и вероятностная мера \(\Pr\)) абсолютно непрерывна, то плотность можно получить следующим образом:
\[
p(x_1, x_2, \dots, x_n) = \frac{\partial^n}{\partial x_1 \partial x_2 \dots \partial x_n} F(x_1, \dots, x_n).
\]
\subsection{Случайные величины в многомерном случае}
Для одномерного случая были введены функции распределения и распределения случайных величин. Обобщим это.
\begin{definition}
	Пусть \(\xi_1, \xi_2, \dots, \xi_n\)~--- некоторые случайные величины. Тогда их \emph{совместной функцией распределения} называется следующая функция
	\[
	F_{\xi_1, \dots, \xi_n}(x_1, \dots, x_n) = \Pr{\xi_1 \leq x_1, \dots, \xi_n \leq x_n}.
	\]
\end{definition}
\begin{definition}
	\emph{Совместным распределением} случайных величин \(\xi_1, \xi_2, \dots, \xi_n\) называется вероятностная мера \(\Pr_{\xi_1, \dots, \xi_n}\) на \(\R^{n}\), заданная следующим правилом: для любого множества \(B \in \mathcal{B}(\R^{n})\)
	\[
	\Pr_{\xi_1, \dots, \xi_n}(B) = \Pr{(\xi_1, \dots, \xi_n) \in B}.
	\]
\end{definition}
Заметим следующее:
\begin{enumerate}
	\item \(F_{\xi_1, \dots, \xi_n}\)~--- это настоящая функция распределения.
	\item \(\Pr_{\xi_1, \dots, \xi_n}\)~--- это настоящая вероятностная мера.
	\item \(F_{\xi_1, \dots, \xi_n}\)~--- это функция распределения для \(\Pr_{\xi_1, \dots, \xi_n}\).
\end{enumerate}

В каких случаях совместная функция распределения хорошо считается? Например, в случае независимых случайных величин.
\begin{theorem}
	Если случайные величины \(\xi_1, \xi_2, \dots, \xi_n\) независимы (в совокупности) и функция распределения \(\xi_i\) равна \(F_{\xi_i}\) для всех \(i \in \{1, 2, \dots, n\}\), то
	\[
	F_{\xi_1, \dots, \xi_n}(x_1, \dots, x_n) = \prod\limits_{i = 1}^{n} F_{\xi_i}(x_i).
	\]
\end{theorem}
\begin{proof}
	Распишем совместную функцию распределения по определению:
	\begin{align}
		F_{\xi_1, \dots, \xi_n}(x_1, \dots, x_n) &= \Pr{\xi_1 \leq x_1, \dots, \xi_n \leq x_n} = \prod\limits_{i = 1}^{n} \Pr{\xi_i \leq x_i} = \prod\limits_{i = 1}^{n} F_{\xi_i}(x_i). \qedhere
	\end{align}
\end{proof}
Теперь, по аналогии со вторым примером функции распределения, введём плотность совместного распределения.
\begin{definition}
	Пусть есть случайные величины \(\xi_1, \dots, \xi_n\) с функцией совместного распределения \(F_{\xi_1, \dots, \xi_n}\). Если у \(F_{\xi_1, \dots, \xi_n}\) есть плотность \(p_{\xi_1, \dots, \xi_n}\), то \(p_{\xi_1, \dots, \xi_n}\) называют \emph{совместной плотностью распределения}. В данном случае
	\[
	F_{\xi_1, \dots, \xi_n}(x_1, \dots, x_n) = \int\limits_{-\infty}^{x_1} \dots \int\limits_{-\infty}^{x_n} p_{\xi_1, \dots, \xi_n}(t_1, t_2, \dots, t_n)\,\mathrm{d}t_1\dots\mathrm{d}t_n.
	\]
\end{definition}

Теперь докажем один весьма полезный факт.
\begin{theorem}
	Если у случайных величин \(\xi_1, \xi_2, \dots, \xi_n\) есть совместная плотность, то в каждой из случайных величин есть плотность.
\end{theorem}
\begin{proof}
	Вспомним определение плотности для одномерной величины: если \(p_{\xi}(x)\)~--- плотность, то для любого \(x \in \R\) выполнено
	\[
	\Pr{\xi \leq x} = \int\limits_{-\infty}^{x} p_{\xi}(t)\,\mathrm{d}t.
	\]
	
	Попробуем получить нечто подобное, используя совместную плотность. По определениям совместной плотности и совместной функции распределения:
	\[
	\Pr{\xi_1 \leq x_1, \dots, \xi_n \leq x_n} = \int\limits_{-\infty}^{x_1} \dots \int\limits_{-\infty}^{x_n} p_{\xi_1, \dots, \xi_n}(t_1, t_2, \dots, t_n)\,\mathrm{d}t_1\dots\mathrm{d}t_n.
	\]
	
	Допустим, что мы хотим получить плотность \(\xi_i\). Тогда устремим все \(x_k\), \(k \in \{1, 2, \dots, n\} \setminus \{i\}\) к бесконечности. Тогда все условия, кроме \(i\)-го будут гарантированно выполнены. Изменяя порядок интегрирования (так как плотность абсолютно непрерывна, то это легально), получаем, что
	\[
	\Pr{\xi_i \leq x_i} = \int\limits_{-\infty}^{x_i} \left(\int\limits_{-\infty}^{+\infty} \dots \int\limits_{-\infty}^{+\infty} p_{\xi_1, \dots, \xi_n}(t_1, t_2, \dots, t_n)\,\mathrm{d}t_1\dots\mathrm{d}t_n\right)\,\mathrm{d}t_i.
	\]
	
	Сравнивая это с определением плотности в одномерном случае, получаем, что
	\[
	p_{\xi_i}(x_i) = \int\limits_{-\infty}^{+\infty} \dots \int\limits_{-\infty}^{+\infty} p_{\xi_1, \dots, \xi_n}(t_1, t_2, \dots, t_n)\,\mathrm{d}t_1\dots\mathrm{d}t_{i - 1}\,\mathrm{d}t_{i + 1}\dots\mathrm{d}t_n.\qedhere
	\]
\end{proof}

А что можно сказать про плотности независимых случайных величин? Ответ на этот вопрос даёт следующая теорема.
\begin{theorem}
	Пусть есть случайные величины \(\xi_1, \xi_2, \dots, \xi_n\) с совместной плотностью распределения \(p_{\xi_1, \dots, \xi_n}\), и для каждой из них существует плотность \(p_{\xi_i}\). Тогда эти случайные величины независимы в совокупности тогда и только тогда, когда 
	\[
	p_{\xi_1, \dots, \xi_n}(x_1, \dots, x_n) = \prod\limits_{i = 1}^{n} p_{\xi_i}(x_i).
	\]
\end{theorem}
\begin{proof}\ 
	\begin{enumerate}
		\item[{\([\Rightarrow]\)}] Пусть случайные величины \(\xi_1, \xi_2, \dots, \xi_n\) независимы. Тогда
		\[
		F_{\xi_1, \dots, \xi_n}(x_1, \dots, x_n) = \prod\limits_{i = 1}^{n} F_{\xi_i}(x_i) = \prod\limits_{i = 1}^{n} \left(\int\limits_{-\infty}^{x_i} p_{\xi}(t_i)\,\mathrm{d}t_i\right).
		\]
		
		Заметим, что произведение интегралов можно представить в следующем виде:
		\[
		\prod\limits_{i = 1}^{n} \left(\int\limits_{-\infty}^{x_i} p_{\xi}(t_i)\,\mathrm{d}t_i\right) = \int\limits_{-\infty}^{x_1} \dots \int\limits_{-\infty}^{x_n} \left(\prod\limits_{i = 1}^{n} p_{\xi_i}(t_i)\right)\,\mathrm{d}t_1\dots\mathrm{d}t_n.
		\]
		
		Однако
		\[
		F_{\xi_1, \dots, \xi_n}(x_1, \dots, x_n) = \int\limits_{-\infty}^{x_1} \dots \int\limits_{-\infty}^{x_n} p_{\xi_1, \dots, \xi_n}(t_1, t_2, \dots, t_n)\,\mathrm{d}t_1\dots\mathrm{d}t_n.
		\]
		
		Сравнивая формулы, получаем, что
		\[
		p_{\xi_1, \dots, \xi_n}(x_1, \dots, x_n) = \prod\limits_{i = 1}^{n} p_{\xi_i}(x_i).
		\]
		
		\item[{\([\Leftarrow]\)}] Пусть известно, что
		\[
		p_{\xi_1, \dots, \xi_n}(x_1, \dots, x_n) = \prod\limits_{i = 1}^{n} p_{\xi_i}(x_i).
		\]
		
		Тогда
		\[
		F_{\xi_1, \dots, \xi_n}(x_1, \dots, x_n) = \int\limits_{-\infty}^{x_1} \dots \int\limits_{-\infty}^{x_n} \left(\prod\limits_{i = 1}^{n} p_{\xi_i}(t_i)\right)\,\mathrm{d}t_1\dots\mathrm{d}t_n.
		\]
		
		Разобъём его в произведение интегралов:
		\[
		\int\limits_{-\infty}^{x_1} \dots \int\limits_{-\infty}^{x_n} \left(\prod\limits_{i = 1}^{n} p_{\xi_i}(t_i)\right)\,\mathrm{d}t_1\dots\mathrm{d}t_n = \prod\limits_{i = 1}^{n} \left(\int\limits_{-\infty}^{x_i} p_{\xi}(t_i)\,\mathrm{d}t_i\right).
		\]
		
		Отсюда получаем, что
		\[
		F_{\xi_1, \dots, \xi_n}(x_1, \dots, x_n) = \prod\limits_{i = 1}^{n} \left(\int\limits_{-\infty}^{x_i} p_{\xi}(t_i)\,\mathrm{d}t_i\right) = \prod\limits_{i = 1}^{n} F_{\xi_i}(x_i).
		\]
		
		А это означает независимость \(\xi_1, \xi_2, \dots, \xi_n\).
	\end{enumerate}
\end{proof}

\subsection{Математическое ожидание в многомерном случае}
Простой вопрос: а зачем мы ввели всё это? Ответ тоже прост~--- для подсчёта математического ожидания от функции многих случайных величин. Поставим вопрос формально.
\begin{problem}
	Пусть есть \(n\) случайных величин \(\xi_1, \xi_2, \dots, \xi_n\) с совместной плотностью распределения \(p_{\xi_1, \dots, \xi_n}\) и некоторая борелевская функция \(f(x_1, \dots, x_n)\). Чему равно
	\[
	\E{f(\xi_1, \xi_2, \dots, \xi_n)}?
	\]
\end{problem}
Ответ на этот вопрос даёт следующая теорема, которую мы сформулируем без доказательства:
\begin{theorem}
	Пусть \(\xi_1, \xi_2, \dots, \xi_n\)~--- случайные величины с совместной плотностью распределения \(p_{\xi_1, \dots, \xi_n}\), а \(f(x_1, x_2, \dots, x_n)\)~--- некоторая борелевская функция. Тогда
	\[
	\E{f(\xi_1, \xi_2, \dots, \xi_n)} = \int\limits_{\R^n} f(x_1, x_2, \dots, x_n)p_{\xi_1, \dots, \xi_n}(x_1, x_2, \dots, x_n)\,\mathrm{d}x_1\dots\mathrm{d}x_n.
	\]
\end{theorem}

В данном случае мы пишем теорему без доказательства по той причине, потому что нам не хватает знаний по функциональному анализу и теории меры. Однако мы можем доказать этот результат в некоторых частных случаях. Попробуем доказать формулу для матожидания произведения двух случайных величин.
\begin{theorem}
	Пусть \(\xi\) и \(\eta\)~--- некоторые (не обязательно независимые) неотрицательные случайные величины с совместной плотностью распределения \(p_{\xi, \eta}\). Тогда
	\[
	\E{\xi\eta} = \int\limits_{\R^2} xyp_{\xi,\eta}(x, y)\,\mathrm{d}x\,\mathrm{d}y.
	\]
\end{theorem}
\begin{proof}
	Введём последовательность простых случайных величин \(\{\delta_n\}_{n = 1}^{\infty}\), устроенную следующим образом:
	\[
	\delta_n = \sum_{i = 1}^{n2^{n}}\sum_{j = 1}^{n2^{n}} \frac{i - 1}{2^n}\frac{j - 1}{2^n}\mathop{\mathrm{I}}\left\{\frac{i - 1}{2^n} < \xi \leq \frac{i}{2^n}, \frac{j - 1}{2^n} < \eta \leq \frac{j}{2^n}\right\}.
	\]
	
	Заметим, что \(\delta_n \uparrow \xi\eta\) и \(\delta_{n + 1} \geq \delta_{n}\). Тогда по определению математического ожидания
	\[
	\E{\xi\eta} = \lim\limits_{n \to \infty} \E{\delta_n}.
	\]
	
	Теперь посчитаем \(\E{\delta_n}\):
	\begin{align}
		\E{\delta_n} &= \sum_{k = 1}^{n2^{n}}\sum_{j = 1}^{n2^{n}} \frac{k - 1}{2^n}\frac{j - 1}{2^n}\Pr{\frac{k - 1}{2^n} < \xi \leq \frac{k}{2^n}, \frac{j - 1}{2^n} < \eta \leq \frac{j}{2^n}} \\
		&= \sum_{k = 1}^{n2^{n}}\sum_{j = 1}^{n2^{n}} \frac{k - 1}{2^n}\frac{j - 1}{2^n}\left(\int\limits_{\frac{k - 1}{2^n}}^{\frac{k}{2^n}}\int\limits_{\frac{j - 1}{2^n}}^{\frac{j}{2^n}} p_{\xi,\eta}(x, y)\,\mathrm{d}x\,\mathrm{d}y\right).
	\end{align}
	Теперь заметим, что
	\begin{align}
		\int\limits_{0}^{n}\int\limits_{0}^{n} p_{\xi,\eta}(x, y)\,\mathrm{d}x\,\mathrm{d}y - \mathcal{O}\left(\frac{n}{2^n}\right) \leq \E{\delta_n} \leq \int\limits_{0}^{n}\int\limits_{0}^{n} p_{\xi,\eta}(x, y)\,\mathrm{d}x\,\mathrm{d}y.
	\end{align}
	
	Отсюда получаем, что
	\[
	\E{\xi\eta} = \int\limits_{0}^{+\infty}\int\limits_{0}^{+\infty} p_{\xi,\eta}(x, y)\,\mathrm{d}x\,\mathrm{d}y.
	\]
\end{proof}