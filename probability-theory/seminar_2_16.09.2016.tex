\section{Семинар от 16.09.2016}
Начнём с разбора домашнего задания.
\begin{problem}
    \(n\) шаров раскладывают по \(N\) ящикам. Найдите вероятность того, что для каждого \(i = 1, 2, \ldots, N\) в \(i\)-м ящике лежит \(n_i\) шаров, где \(n_1 + n_2 + \ldots + n_N = n\), если
    \begin{enumerate}
        \item шары различимы,
        \item шары неразличимы.
    \end{enumerate}
\end{problem}
\begin{proof}[Решение]
    Начнём со случая различимых шаров. В первый ящик необходимо выбрать \(n_1\) шаров из \(n\), что можно сделать \(\binom{n}{n_1}\) способами. Для второго ящика надо выбрать \(n_2\) шаров из \(n - n_1\), что даёт \(\binom{n - n_1}{n_2}\). Рассуждая аналогично, получаем, что всего есть \(\binom{n}{n_1}\binom{n - n_1}{n_2} \ldots \binom{n - n_1 - \ldots - n_{N - 1}}{n_{N}} = \frac{n!}{n_{1}!(n - 1)!}\frac{(n - n_1)!}{n_2!(n - n_1 - n_2)!}\ldots\frac{(n - n_1 - \ldots - n_{N - 1})!}{n_N!(n - n_1 - \ldots - n_{N - 1} - n_N)!} = \frac{n!}{n_{1}!n_{2}!\ldots n_{N}!}\) успешных исходов. Всего же исходов \(N^{n}\). Тогда искомая вероятность равна \[\Pr = \frac{n!}{N^{n}n_{1}!n_{2}!\ldots n_{N}!}.\]
    
    Теперь рассмотрим случай, когда шары неразличимы. Заметим, что тогда есть лишь один подходящий случай. Всего же случаев \(\binom{n + N - 1}{N - 1}\). Тогда вероятность равна \[\Pr = \frac{1}{\binom{n + N - 1}{N - 1}}.\qedhere\]
\end{proof}
\begin{remark}
    Заметим, что число успешных исходов в случае различимых шаров, равное \(\frac{n!}{n_{1}!n_{2}!\ldots n_{N}!}\), принято называть \emph{мультиномиальным коэффициентом}. Его можно получить, рассматривая полином \((x_1 + x_2 + \ldots + x_N)^{n}\).
\end{remark}

\begin{problem}\ 
    \begin{enumerate}
        \item[1)] Случайно бросаются три \(N\)-гранных кубика, на гранях которых написаны числа от 1 до \(N\). Найдите вероятность события \(A_i = \{\)сумма чисел, выпавших на кубиках, равна \(i\}, i \leq N + 2\).
        \item[2)*] Случайно бросаются три \(N\)-гранных кубика, на гранях которых написаны числа от 1 до \(N\). Найдите вероятность события \(A_i = \{\)сумма чисел, выпавших на кубиках, равна \(i\}, i = 2,\ldots,2N\).
        \item[3)**] Случайно бросаются \(k\) различных \(N\)-гранных кубиков, на гранях которых написаны от 1 до \(N\). Найдите вероятность события \(A_i = \{\)сумма чисел, выпавших на кубиках, равна \(i\}, i = k,\ldots,kN\).
    \end{enumerate}
\end{problem}
\begin{proof}[Решение пункта (1)]
    Пусть \(i = k_1 + k_2 + k_3\), и \(1 \leq k_1, k_2, k_3 \leq N\). Как посчитать число подходящих наборов? Воспользуемся методом точек и перегородок. Пусть есть \(i\) точек и нужно расставить 2 перегородки по \(i - 1\) допустимой позиции. Тогда есть \(\binom{i - 1}{2}\) допустимых набора. Отсюда получаем, что искомая вероятность равна \[\Pr[i] = \frac{(i - 1)(i - 2)}{2N^3}.\qedhere\]
\end{proof}
\begin{remark}
    Пункты (2) и (3) на данный момент слишком сложны. Их адекватное решение будет рассказано ближе к концу курса.
\end{remark}

\begin{problem}
    Пусть выбирается произвольная перестановка из \(S_n\). Какова вероятность того, что 1 и 2 будут лежать в одном цикле?
\end{problem}
\begin{proof}[Решение]
    Воспользуемся тем фактом, что каждая перестановка однозначно представима в виде композиции циклов. Пусть цикл, содержащий 1 и 2, состоит из \(k + 1\) элемента (\(1 \leq k \leq n - 1\)). Позицию для 2 можно выбрать \(k\) способами. Далее будем заполнять цикл. Есть \((n - 2)(n - 3)\ldots(n - k)\) вариантов его заполнения. Остальное же мы можем заполнять, как хотим. Следовательно, итого есть \(k(n - 2)(n - 3)\ldots(n - k)(n - k - 1)! = k(n - 2)!\) допустимых перестановок с нужным циклом размера \(k\). Тогда, суммируя по \(k\) от \(1\) до \(n - 1\), получаем \(n - 2)!(1 + 2 + \ldots + (n - 1)) = \frac{1}{2}n!\). Но всего перестановок \(n!\). Тогда вероятность равна \(1/2\). 
\end{proof}

\begin{problem}
    Пусть в группе 25 студентов. Считаем, что дни рождения равновероятны и случайны. Найдите вероятность того, что найдётся ровно одна пара студентов такая, что
    \begin{itemize}
        \item дни рождения у них совпадают
        \item у всех других студентов дни рождения не совпадают с днём рождения данной пары студентов
    \end{itemize}
\end{problem}
\begin{proof}
    Для начала посчитаем вероятность дополнения к событию. В данном случае дополнением является событие ``нет такой пары студентов, что их дни рождения совпадают и у остальных они другие''.
    
    Рассмотрим событие \(A_i = \{\)дни рождения в день \(i\) совпадают только у двух человек\(\}\). Его вероятность равна \(\Pr(A_{i}) = \frac{\binom{25}{2}\cdot 364^{23}}{365^{25}}\). Теперь посчитаем вероятность объединения \(k\) событий \(A_{i_1}, A_{i_2}, \ldots, A_{i_k}\) равна \[\Pr(A_{i_1} \cap A_{i_2} \cap \ldots \cap A_{i_k}) = \frac{\binom{25}{2}\binom{23}{2} \ldots \binom{25 - 2(k - 1)}{2}(365 - k)^{25 - 2k}}{365^{25}}.\]
    Теперь посчитаем вероятность дополнения. Она равна \(\Pr\left(\bigcap\limits_{i = 1}^{365} \overline{A_i}\right)\). Её можно посчитать с помощью формулы включений-исключений. Теперь введём функцию \(\alpha(x, y)\), где \(x\)~--- количество студентов, а \(y\)~--- количество дней. Данная функция равна вероятности того, что ``среди \(x\) студентов нет такой пары студентов, что их дни рождения совпадают, и среди других студентов они другие и находятся среди \(y\) выбранных дней''. Она считается аналогично.
    
    Теперь посчитаем вероятность из условия. Для этого выберем двух человек из 25, выберем им день рождения. После чего посчитаем вероятность дополнения для 23 студентов и 364 дней и умножим на \(364^{23}\). Тогда ответ равен 
    \[\Pr = \frac{\binom{25}{2} \cdot 365 \cdot \alpha(23, 364) \cdot 364^{23}}{365^{25}}.\qedhere\]
\end{proof}

Теперь рассмотрим несколько классических задач на условную вероятность.

\begin{problem}[Парадокс Монти-Холла]
    Вы участвуете в игре, в которой надо выбрать одну дверь из трёх. За одной из них автомобиль, а за другими~--- козы. Вы выбрали первую дверь. Ведущий открыл третью дверь, за которой стоит коза. Ведущий предлагает изменить выбор с первой двери на вторую. Стоит ли это делать?
\end{problem}
\begin{proof}[Решение]
    Рассмотрим два решения~--- элементарное и через теорему Байеса. Начнём со второго.
    
    Пусть \(C_{i} = \{\)машина стоит за \(i\)-й дверью\(\}\) Очевидно, что \(\Pr(C_i) = 1/3\). Теперь введём событие \(H = \{\)ведущий открывает третью дверь\(\}\). Так как ведущий не желает открывать дверь с автомобилем, то условные вероятности будут равны
    \[\begin{array}{l}
    \Pr(H \mid C_1) = 1/2 \\
    \Pr(H \mid C_2) = 1 \\
    \Pr(H \mid C_3) = 0
    \end{array}\]
    Теперь посчитаем вероятность \(\Pr(C_2 \mid H)\). По теореме Байеса она равна
    \[\Pr(C_2 \mid H) = \frac{\Pr(H \mid C_2)\Pr(H)}{\Pr(H \mid C_1)\Pr(H) + \Pr(H \mid C_2)\Pr(H) + \Pr(H \mid C_3)\Pr(H)} = \frac{1}{1/2 + 1 + 0} = \frac{2}{3}\]
    Если же не менять дверь, то вероятность не изменится и будет равна \(1/3\). Поэтому выгоднее изменить выбор двери.
    
    Теперь рассмотрим элементарное решение. Так как вероятность того, что за дверью будет машина, равна \(1/3\), то на вторую и третью дверь вместе приходится \(2/3\). Так как после открытия третьей двери оказалось, что за ней стояла коза, то вероятность ``переходит'' второй двери. Тогда вероятность того, что за первой дверью будет машина, равна \(1/3\), а за второй~--- \(2/3\). Выбор очевиден.
\end{proof}

\begin{problem}[Задача о поручике Ржевском]
    Поручик Ржевский пришёл в казино и решил поиграть на деньги. Сначала у него есть 8 рублей, и он хочет выйти из казино с 256 рублями. Ему предлагают две тактики:
    \begin{itemize}
        \item Каждый раз идти ва-банк.
        \item Каждый раз играть на 1 рубль.
    \end{itemize}
    Какую тактику выбрать поручику, если вероятность выигрыша составляет: (a) \(1/4\), (b) \(3/4\)?
\end{problem}
\begin{proof}[Решение]
    В случае первой тактики всё просто~--- он не имеет права проиграть. Поэтому вероятность того, что он уйдёт с желаемой суммой, равна \(p^5\).
    
    Со второй тактикой дела обстоят интереснее. Введём событие \(p_{l} = \{\)поручик получил желаемое, изначально имея \(l\) рублей\(\}\). По формуле полной вероятности получим рекурсивную формулу. Вместе с начальными условиями получаем систему
    \[\begin{array}{l}
    p_{l} = (1 - p)p_{l - 1} + pp_{l + 1} \\
    p_{0} = 0 \\
    p_{256} = 1
    \end{array}\]
    Выпишем характеристическое уравнение: \(p\lambda^2 - \lambda + (1 - p) = 0\). Корни этого уравнения имеют вид \[\frac{1 \pm \sqrt{1 - 4p + 4p^2}}{2p} = \frac{1 \pm (1 - 2p)}{2p} = \left[\begin{array}{l}
    1 \\ \frac{1 - p}{p}
    \end{array}\right.\]
    Тогда получаем, что \(p_{l} = a_1 + a_2\left(\frac{1 - p}{p}\right)^{l}\). Теперь определим значение констант. Так как \(p_0 = 0\), то \(0 = a_1 + a_2\) и \(a_1 = -a_2 = -a\). Теперь рассмотрим \(p_{256}\):
    \[1 = -a + a\left(\frac{1 - p}{p}\right)^{256} \implies a = \frac{1}{\left(\frac{1 - p}{p}\right)^{256} - 1}\]
    Отсюда получаем, что Отсюда получаем, что \(p_{l} = \dfrac{\left(\frac{1 - p}{p}\right)^{l} - 1}{\left(\frac{1 - p}{p}\right)^{256} - 1}\) и вероятность успеха второй стратегии равна \(p_8\).
    
    Теперь осталось посчитать. Сначала посчитаем для \(p = 1/4\):
    \[\Pr_{\text{I}} = \left(\frac{1}{4}\right)^5 = \frac{1}{1024}\]
    \[\Pr_{\text{II}} = \frac{\left(\frac{1 - 1/4}{1/4}\right)^{8} - 1}{\left(\frac{1 - 1/4}{1/4}\right)^{256} - 1} = \frac{3^8 - 1}{3^{256} - 1} \approx \frac{1}{3^{248}}\]
    В таком случае шанс выйти из казино по своей воле выше, если каждый раз играть ва-банк.
    
    Теперь же посчитаем для \(p = 3/4\):
    \[\Pr_{\text{I}} = \left(\frac{3}{4}\right)^5 = \frac{243}{1024}\]
    \[\Pr_{\text{II}} = \frac{\left(\frac{1 - 3/4}{3/4}\right)^{8} - 1}{\left(\frac{1 - 3/4}{3/4}\right)^{256} - 1} = \frac{1 - (1/3)^{8}}{1 - (1/3)^{256}} \approx \frac{6560}{6561}\]
    В таком же случае гораздо безопаснее каждый раз играть на один рубль.
\end{proof}

\begin{problem}[Задача о контрольной работе]
    Пусть студенты A, B, C пишут контрольную. Студент A решает любую задачу с вероятностью \(3/4\), студент B~--- с вероятностью \(1/2\), а студент C~--- с вероятностью \(1/4\). В контрольной работе 4 задачи. Преподаватель получает анонимную работу, в которой решено 3 задачи. Кому эта работа скорее всего принадлежит?
\end{problem}
\begin{proof}[Решение]
    Пусть \(D = \{\)автор решил 3 задачи из 4\(\}\). Теперь введём ещё три события: \(D_A = \{\)автор~--- студент A\(\}\), \(D_B = \{\)автор~--- студент B\(\}\), \(D_C = \{\)автор~--- студент C\(\}\). Очевидно, что \(\Pr(D_A) = \Pr(D_B) = \Pr(D_C) = \frac{1}{3}\). Найдём условную вероятность события \(D\) для разных условий:
    \[\begin{array}{l}
    \Pr(D \mid D_A) = \left(\frac{3}{4}\right)^{3} \cdot \binom{4}{1} \cdot \frac{1}{4}  = \frac{27}{64} \\
    \Pr(D \mid D_B) = \left(\frac{1}{2}\right)^{3} \cdot \binom{4}{1} \cdot \frac{1}{2}  = \frac{1}{4} = \frac{16}{64} \\
    \Pr(D \mid D_C) = \left(\frac{1}{4}\right)^{3} \cdot \binom{4}{1} \cdot \frac{3}{4}  = \frac{3}{64}
    \end{array}\]
    Теперь воспользуемся теоремой Байеса:
    \[\begin{aligned}
    \Pr(D_A \mid D) &= \frac{\Pr(D \mid D_A)\Pr(D_A)}{\Pr(D \mid D_A)\Pr(D_A) + \Pr(D \mid D_B)\Pr(D_B) + \Pr(D \mid D_C)\Pr(D_C)} \\
    &= \frac{\frac{27}{64}}{\frac{27}{64} + \frac{16}{64} + \frac{3}{64}} = \frac{27}{46}
    \end{aligned}\]
    Аналогично получаем, что \(\Pr(D_B \mid D) = \frac{16}{46}\) и \(\Pr(D_C \mid D) = \frac{3}{46}\). Следовательно, эту работу, скорее всего, сдал студент \(A\).
\end{proof}

\begin{minipage}{0.5\textwidth}
    \begin{flushleft}
        \begin{problem}
            Пусть пять приборов соединены в схему. Каждый из них пропускает ток с вероятностью \(p\). Какова вероятность того, что схема пропускает ток? Какова вероятность того, что есть ток, но при этом E сломан?
        \end{problem}
    \end{flushleft}
\end{minipage}
\begin{minipage}{0.4\textwidth}
    \begin{flushright}
        \begin{circuitikz}[european resistors]
            \draw (0, 0) to[short, *-] (1,0);
            \draw (1, 0) to[short] (1, 1);
            \draw (1, 0) to[short] (1, -1);
            \draw (1, 1) to[R, l=A] (3, 1);
            \draw (1, -1) to[R, l=B] (3, -1);
            \draw (3, 1) to[R, l=E] (3, -1);
            \draw (3, 1) to[R, l=D] (5, 1);
            \draw (3, -1) to[R, l=C] (5, -1);
            \draw (5, 1) to[short] (5, 0);
            \draw (5, -1) to[short] (5, 0);
            \draw (5, 0) to[short, -*] (6, 0);
        \end{circuitikz}
    \end{flushright}
\end{minipage}
\begin{proof}[Решение]
    Для начал посмотрим, по каким путям может пройти ток:
    \begin{itemize}
        \item Ток может пойти по \(AD\)~--- тогда вероятность того, что ток будет, равна \(p^2\).
        \item Если \(D\) сломан, то ток может пойти по \(AEC\). Вероятность такого случая равна \(p^3(1 - p)\).
        \item Если \(A\) сломан, то ток может пойти по \(BC\). Вероятность этого равна \(p^2(1 - p)\).
        \item Если же сломаны и \(A\), и \(C\), то ток пойдёт по \(BED\). Вероятность такого равна \(p^3(1 - p)^2\).
    \end{itemize}
    Тогда итоговая вероятность равна сумме: \[\Pr(\text{в цепи есть ток}) = p^2 + p^3(1 - p) + p^2(1 - p) + p^3(1 - p)^2.\]
    
    Теперь ответим на второй вопрос. Воспользуемся определением условной вероятности:
    \[\Pr(\text{E не проводит ток} \mid \text{в цепи есть ток}) = \frac{\Pr(\text{E не проводит ток и в цепи есть ток})}{\Pr(\text{в цепи есть ток})}\]
    Вероятность сверху посчитать несложно~--- достаточно рассмотреть допустимые пути. Тогда ответ равен \[\Pr(\text{E не проводит ток} \mid \text{в цепи есть ток}) = \frac{p^2(1 - p) + p^2(1 - p)^2}{p^2 + p^3(1 - p) + p^2(1 - p) + p^3(1 - p)^2}.\qedhere\]
\end{proof}