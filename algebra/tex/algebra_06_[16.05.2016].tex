\documentclass[a4paper,10pt]{amsart}

\usepackage[T2A]{fontenc}
\usepackage[utf8x]{inputenc}
\usepackage{amssymb}
\usepackage[russian]{babel}
\usepackage{geometry}
\usepackage{hyperref}
\usepackage{enumitem}

\geometry{a4paper,top=2cm,bottom=2cm,left=2cm,right=2cm}

\setlength{\parindent}{0pt}
\setlength{\parskip}{\medskipamount}

\newcommand{\Ker}{\mathop{\mathrm{Ker}}}
\renewcommand{\Im}{\mathop{\mathrm{Im}}}
\DeclareMathOperator{\Tor}{\mathrm{Tor}}
%\newcommand{\Tor}{\mathop{\mathrm{Tor}}}

%\def\Ker{{\rm Ker}}%
%\def\Im{{\rm Im}}%
\def\Mat{{\rm Mat}}%
\def\cont{{\rm cont}}%
%\def\Tor{{\rm Tor}}%
\def\Char{{\rm Char}}%
\def\signum{{\rm sig}}%
\def\Sym{{\rm Sym}}%
\def\St{{\rm St}}%
\def\Aut{{\rm Aut}}%
\def\Chi{{\mathbb X}}%
\def\Tau{{\rm T}}%
\def\Rho{{\rm R}}%
\def\rk{{\rm rk}}%
\def\ggT{{\rm ggT}}%
\def\kgV{{\rm kgV}}%
\def\Div{{\rm Div}}%
\def\div{{\rm div}}%
\def\quot{/\!\!/}%
\def\mal{\! \cdot \!}%
\def\Of{{\mathcal{O}}}
%
\def\subgrpneq{\le}%
\def\subgrp{\le}%
\def\ideal#1{\le_{#1}}%
\def\submod#1{\le_{#1}}%
%
\def\Bild{{\rm Bild}}%
\def\Kern{{\rm Kern}}%
\def\bangle#1{{\langle #1 \rangle}}%
\def\rq#1{\widehat{#1}}%
\def\t#1{\widetilde{#1}}%
\def\b#1{\overline{#1}}%
%
\def\abs#1{{\vert #1 \vert}}%
\def\norm#1#2{{\Vert #1 \Vert}_{#2}}%
\def\PS#1#2{{\sum_{\nu=0}^{\infty} #1_{\nu} #2^{\nu}}}%
%
\def\C{{\rm C}}%
\def\O{{\rm O}}%
\def\HH{{\mathbb H}}%
\def\LL{{\mathbb L}}%
\def\FF{{\mathbb F}}%
\def\CC{{\mathbb C}}%
\def\KK{{\mathbb K}}%
\def\TT{{\mathbb T}}%
\def\ZZ{{\mathbb Z}}%
\def\RR{{\mathbb R}}%
\def\SS{{\mathbb S}}%
\def\NN{{\mathbb N}}%
\def\QQ{{\mathbb Q}}%
\def\PP{{\mathbb P}}%
\def\AA{{\mathbb A}}%
%
\def\eins{{\mathbf 1}}%
%
\def\AG{{\rm AG}}%
\def\Aut{{\rm Aut}}%
\def\Hol{{\rm Hol}}%
\def\GL{{\rm GL}}%
\def\SL{{\rm SL}}%
\def\SO{{\rm SO}}%
\def\Sp{{\rm Sp}}%
\def\gl{\mathfrak{gl}}%
\def\rg{{\rm rg}}%
\def\sl{\mathfrak{sl}}%
\def\HDiv{{\rm HDiv}}%
\def\CDiv{{\rm CDiv}}%
\def\Res{{\rm Res}}%
\def\Pst{{\rm Pst}}%
\def\Nst{{\rm Nst}}%
\def\rad{{\rm rad}}%
\def\GL{{\rm GL}}%
\def\Tr{{\rm Tr}}%
\def\Pic{{\rm Pic}}%
\def\Hom{{\rm Hom}}%
\def\hom{{\rm hom}}%
\def\Mor{{\rm Mor}}%
\def\codim{{\rm codim}}%
\def\Supp{{\rm Supp}}%
\def\Spec{{\rm Spec}}%
\def\Proj{{\rm Proj}}%
\def\Maps{{\rm Maps}}%
\def\cone{{\rm cone}}%
\def\ord{{\rm ord}}%
\def\pr{{\rm pr}}%
\def\id{{\rm id}}%
\def\mult{{\rm mult}}%
\def\inv{{\rm inv}}%
\def\neut{{\rm neut}}%
%
\def\AAA{\mathcal{A}}
\def\BBB{\mathcal{B}}
\def\CCC{\mathcal{C}}
\def\EEE{\mathcal{E}}
\def\FFF{\mathcal{F}}

\def\CF{{\rm CF}}
\def\GCD{{\rm GCD}}
\def\Mat{{\rm Mat}}
\def\End{{\rm End}}
\def\cont{{\rm cont}}
\def\Kegel{{\rm Kegel}}
\def\Char{{\rm Char}}
\def\Der{{\rm Der}}
\def\signum{{\rm sg}}
\def\grad{{\rm grad}}
\def\Spur{{\rm Spur}}
\def\Sym{{\rm Sym}}
\def\Alt{{\rm Alt}}
\def\Abb{{\rm Abb}}
\def\Chi{{\mathbb X}}
\def\Tau{{\rm T}}
\def\Rho{{\rm R}}
\def\ad{{\rm ad}}
\def\Frob{{\rm Frob}}
\def\Rang{{\rm Rang}}
\def\SpRang{{\rm SpRang}}
\def\ZRang{{\rm ZRang}}
\def\ggT{{\rm ggT}}
\def\kgV{{\rm kgV}}
\def\Div{{\rm Div}}
\def\div{{\rm div}}
\def\quot{/\!\!/}
\def\mal{\! \cdot \!}
\def\add{{\rm add}}
\def\mult{{\rm mult}}
\def\smult{{\rm smult}}

\def\subgrpneq{\le}
\def\subgrp{\le}
\def\ideal#1{\unlhd_{#1}}
\def\submod#1{\le_{#1}}

\def\Bild{{\rm Bild}}
\def\Kern{{\rm Kern}}
\def\Kon{{\rm Kon}}
\def\bangle#1{{\langle #1 \rangle}}
\def\rq#1{\widehat{#1}}
\def\t#1{\widetilde{#1}}
\def\b#1{\overline{#1}}

\def\abs#1{{\vert #1 \vert}}
\def\norm#1#2{{\Vert #1 \Vert}_{#2}}
\def\PS#1#2{{\sum_{\nu=0}^{\infty} #1_{\nu} #2^{\nu}}}


\def\eins{{\mathbf 1}}

\def\ElM{{\rm ElM}}
\def\ZOp{{\rm ZOp}}
\def\SpOp{{\rm SpOp}}
\def\Gal{{\rm Gal}}
\def\Def{{\rm Def}}
\def\Fix{{\rm Fix}}
\def\ord{{\rm ord}}
\def\Aut{{\rm Aut}}
\def\Hol{{\rm Hol}}
\def\GL{{\rm GL}}
\def\SL{{\rm SL}}
\def\SO{{\rm SO}}
\def\Sp{{\rm Sp}}
\def\Spann{{\rm Spann}}
\def\Lin{{\rm Lin}}
\def\gl{\mathfrak{gl}}
\def\rg{{\rm rg}}
\def\sl{\mathfrak{sl}}
\def\so{\mathfrak{so}}
\def\sp{\mathfrak{sp}}
\def\gg{\mathfrak{g}}
\def\HDiv{{\rm HDiv}}
\def\CDiv{{\rm CDiv}}
\def\Res{{\rm Res}}
\def\Pst{{\rm Pst}}
\def\Nst{{\rm Nst}}
\def\WDiv{{\rm WDiv}}
\def\GL{{\rm GL}}
\def\Tr{{\rm Tr}}
\def\Pic{{\rm Pic}}
\def\Hom{{\rm Hom}}
\def\hom{{\rm hom}}
\def\Mor{{\rm Mor}}
\def\codim{{\rm codim}}
\def\Supp{{\rm Supp}}
\def\Spec{{\rm Spec}}
\def\Proj{{\rm Proj}}
\def\Maps{{\rm Maps}}
\def\cone{{\rm cone}}
\def\ord{{\rm ord}}
\def\pr{{\rm pr}}
\def\id{{\rm id}}
\def\mult{{\rm mult}}
\def\inv{{\rm inv}}
\def\neut{{\rm neut}}
\def\trdeg{{\rm trdeg}}
\def\sing{{\rm sing}}
\def\reg{{\rm reg}}


%%%%%%%%%%%%%%%%%%%%%%%%%%%

\newtheorem{theorem}{Теорема}
\newtheorem{proposition}{Предложение}
\newtheorem{lemma}{Лемма}
\newtheorem{corollary}{Следствие}
\theoremstyle{definition}
\newtheorem{definition}{Определение}
\newtheorem{problem}{Задача}
%
\theoremstyle{remark}
\newtheorem{exercise}{Упражнение}
\newtheorem{remark}{Замечание}
\newtheorem{example}{Пример}

\renewcommand{\theenumi}{\textup{(\alph{enumi})}}
\renewcommand{\labelenumi}{\theenumi}
\newcounter{property}
\renewcommand{\theproperty}{\textup{(\arabic{property})}}
\newcommand{\property}{\refstepcounter{property}\item}
\newcounter{prooperty}
\renewcommand{\theprooperty}{\textup{(\arabic{prooperty})}}
\newcommand{\prooperty}{\refstepcounter{prooperty}\item}

\makeatletter
\def\keywords#1{{\def\@thefnmark{\relax}\@footnotetext{#1}}}
\let\subjclass\keywords
\makeatother
%
\begin{document}
%
\sloppy
%\thispagestyle{empty}
%
\centerline{\large \bf Лекции курса \guillemotleft
Алгебра\guillemotright{}, лектор Р.\,С.~Авдеев}

\smallskip

\centerline{\large ФКН НИУ ВШЭ, 1-й курс ОП ПМИ, 4-й модуль,
2015/2016 учебный год}


\bigskip

\section*{Лекция 6}

\medskip

{\it Три действия группы на себе. Теорема Кэли. Классы сопряжённости.
Кольца. Делители нуля, обратимые элементы, нильпотенты. Поля и алгебры.
Идеалы. }
% и факторкольца. Теорема о
%гомоморфизме. Центр алгебры матриц над полем. Простота алгебры
%матриц над полем.}

\medskip

Пусть $G$~--- произвольная группа. Рассмотрим три действия $G$ на
самой себе, т.\,е. положим $X=G$:

1) действие {\it умножениями слева (левыми сдвигами)}: $(g,h)\mapsto gh$;

2) действие {\it умножениями справа (правыми сдвигами)}: $(g,h)\mapsto hg^{-1}$;

3) действие {\it сопряжениями}: $(g,h)\mapsto ghg^{-1}$.

\begin{remark}
Для действий левыми и правыми сдвигами есть ровно одна орбита (сама $G$) и
стабилизатор любой точки тривиален, то есть $\St(x) = \{0\}$.
\end{remark}

\begin{definition}
Орбитой действия сопряжениями называются \textit{классами сопряженности}
\end{definition}

\begin{example}
В любой группе $G$ есть класс сопряженности $\{e\}$. \\Также, если $G$ коммутативна, то $\{x\}$ является классом сопряженности для всех $x$ из $G$.
\end{example}

\smallskip

{\bf Теорема Кэли.} Всякая конечная группа $G$ порядка $n$ изоморфна
подгруппе симметрической группы~$S_n$.

\begin{proof}
Рассмотрим действие группы $G$ на себе левыми сдвигами. Как мы
знаем, это действие свободно, поэтому соответствующий гомоморфизм $a
\colon G \to S(G) \simeq\nobreak S_n$ инъективен, т.\,е. $\Ker a =
\lbrace e \rbrace$. Учитывая, что $G / \lbrace e \rbrace \cong G$,
по теореме о гомоморфизме получаем $G \cong \Im a$.
\end{proof}

\medskip

Теперь приступим к изучению колец.
\begin{definition}
{\it Кольцом} называется множество $R$ с двумя бинарными операциями
\guillemotleft $+$\guillemotright{}~(сложение) и \guillemotleft
$\times$\guillemotright{}~(умножение), обладающими следующими
свойствами:

1) $(R,+)$ является абелевой группой (называемой {\it аддитивной
группой} кольца $R$);

2) выполнены {\it левая и правая дистрибутивности}, т.е.
$$
a(b+c)=ab+ac \quad \text{и} \quad (b+c)a=ba+ca \quad \text{для всех}
\ a,b,c\in R.
$$

В этом курсе мы рассматриваем только ассоциативные кольца с
единицей, поэтому дополнительно считаем, что выполнены ещё два
свойства:

3) $a(bc)=(ab)c$ для всех $a,b,c\in R$ (\textit{ассоциативность
умножения});

4) существует такой элемент $1\in R$ (называемый \textit{единицей}),
что
\begin{equation} \label{eq1}
a1 = 1a = a \ \text{для всякого} \ a \in R.
\end{equation}
\end{definition}

\begin{remark}
В произвольном кольце $R$ выполнены равенства
\begin{equation} \label{eq2}
a0 = 0a = 0 \ \text{для всякого} \ a \in R.
\end{equation}
В самом деле, имеем $a0 = a(0 + 0) = a0 + a0$, откуда $0 = a0$.
Аналогично устанавливается равенство $0a = 0$.
\end{remark}

\begin{remark}
Если кольцо $R$ содержит более одного элемента, то $0\ne 1$. Это
следует из соотношений~(\ref{eq1}) и~(\ref{eq2}).
\end{remark}

\textbf{Примеры колец:}
\begin{enumerate}[label=\textup{(\arabic*)},ref=\textup{\arabic*}]
\item \label{ex_num}
числовые кольца $\ZZ$, $\QQ$, $\RR$, $\CC$;

\item
кольцо $\ZZ_n$ вычетов по модулю~$n$;

\item \label{ex_mat}
кольцо $\Mat(n\times n, \RR)$ матриц с коэффициентами из~$\RR$;

\item \label{ex_pol}
кольцо $\RR[x]$ многочленов от переменной $x$ с коэффициентами
из~$\RR$;

\item
кольцо $\RR[[x]]$ \textit{формальных степенных рядов} от переменной
$x$ с коэффициентами из~$\RR$:
$$
\RR[[x]] := \lbrace \sum \limits_{i = 0}^\infty a_i x^i \mid a_i \in
\RR \rbrace;
$$

\item \label{ex_func}
кольцо $\FFF(M, \RR)$ всех функций из множества $M$ во
множество~$\RR$ с операциями поточечного сложения и умножения:
$$
(f_1 + f_2)(m) := f_1(m) + f_2(m); \quad (f_1f_2)(m) := f_1(m)
f_2(m) \quad \text{для всех} \quad f_1,f_2 \in \FFF(M, \RR), m \in
M.
$$
\end{enumerate}

\begin{remark}
В примерах (\ref{ex_mat})--(\ref{ex_func}) вместо $\RR$ можно брать
любое кольцо, в частности $\ZZ$, $\QQ$, $\CC$, $\ZZ_n$.
\end{remark}

\begin{remark}
Обобщая пример~(\ref{ex_pol}), можно рассматривать кольцо $\RR[x_1,
\ldots, x_n]$ многочленов от нескольких переменных $x_1, \ldots,
x_n$ с коэффициентами из~$\RR$.
\end{remark}

\begin{definition}
Кольцо $R$ называется {\it коммутативным}, если $ab=ba$ для всех
$a,b\in R$.
\end{definition}

Все перечисленные в примерах (\ref{ex_num})--(\ref{ex_func}) кольца,
кроме $\Mat(n\times n, \RR)$ при $n \geqslant 2$, коммутативны.

Пусть $R$~--- кольцо.

\begin{definition}
Элемент $a\in R$ называется {\it обратимым}, если найдётся такой
$b\in R$, что $ab=ba=1$. Такой элемент $b$ обозначается классическим образом как $a^{-1}$.
\end{definition}

\begin{remark}
Все обратимые элементы кольца $R$ образуют группу относительно
операции умножения.
\end{remark}

\begin{definition}
Элемент $a\in R$ называется \textit{левым} (соответственно
\textit{правым}) \textit{делителем нуля}, если $a \ne 0$ и найдётся
такой $b \in R$, $b\ne 0$, что $ab=0$ (соответственно $ba = 0$).
\end{definition}

\begin{remark}
В~случае коммутативных колец понятия левого и правого делителей нуля
совпадают, поэтому говорят просто о делителях нуля.
\end{remark}

\begin{remark}
Все делители нуля в $R$ необратимы: если $ab = 0$, $a \ne 0$, $b \ne
0$ и существует $a^{-1}$, то получаем $a^{-1}ab = a^{-1}0$, откуда
$b = 0$~--- противоречие.
\end{remark}

\begin{definition}
Элемент $a\in R$ называется {\it нильпотентом}, если $a \ne 0$ и
найдётся такое $m \in \NN$, что $a^m=0$.
\end{definition}

\begin{remark}
Всякий нильпотент в $R$ является делителем нуля: если $a \ne 0$,
$a^m = 0$ и число $m$ наименьшее с таким свойством, то $m \geqslant
2$ и $a^{m-1} \ne 0$, откуда $aa^{m-1} = a^{m-1}a = 0$.
\end{remark}

%\begin{definition}
%Элемент $a\in R$ называется {\it идемпотентом}, если $a^2=a$.
%\end{definition}

\begin{definition}
{\it Полем} называется коммутативное ассоциативное кольцо $K$ с
единицей, в котором всякий ненулевой элемент обратим.
\end{definition}

\begin{remark}
Тривиальное кольцо $\lbrace 0 \rbrace$ полем не считается, поэтому
$0 \ne 1$ в любом поле.
\end{remark}

\textbf{Примеры полей:} $\QQ$, $\RR$, $\CC$, $\ZZ_2$.

\begin{proposition}
Кольцо вычетов $\ZZ_n$ является полем тогда и только тогда, когда
$n$~--- простое число.
\end{proposition}

\begin{proof}
Если число $n$ составное, то $n = m k$, где $1 < m, k < n$. Тогда
$\overline{m} \overline{k} = \overline{n} = \overline{0}$.
Следовательно, $\overline k$ и $\overline m$~--- делители нуля в
$\ZZ_n$, ввиду чего не все ненулевые элементы там обратимы.

Если $n = p$~--- простое число, то возьмём произвольный ненулевой
вычет $\overline{a} \in \ZZ_p$ и покажем, что он обратим. Рассмотрим
вычеты
\begin{equation} \label{eq3}
\overline{1} \overline{a}, \overline{2} \overline{a}, \ldots,
\overline{(p-1)} \overline{a}.
\end{equation}
Если $\overline{r} \overline{a} = \overline{s} \overline{a}$ при $1
\leqslant r,s \leqslant p-1$, то число $(r - s)a$ делится на~$p$. В
силу взаимной простоты чисел $a$ и $p$ получаем, что число $r - s$
делится на~$p$. Тогда из условия $|r-s| \leqslant p - 2$ следует,
что $r = s$. Это рассуждение показывает, что все вычеты~(\ref{eq3})
попарно различны. Поскольку все они отличны от нуля, среди них
должна найтись единица: существует такое $b \in \lbrace 1, \ldots,
p-1 \rbrace$, что $\overline{b} \overline{a}=\overline{1}$. Это и
означает, что вычет $\overline{a}$ обратим.
\end{proof}

\begin{definition}
{\it Алгеброй} над полем $K$ (или кратко \textit{$K$-алгеброй})
называется множество $A$ с операциями сложения, умножения и
умножения на элементы поля $K$, обладающими следующими свойствами:

1) относительно сложения и умножения $A$ есть кольцо;

2) относительно сложения и умножения на элементы из $K$ множество
$A$ есть векторное пространство;

3 $(\lambda a)b=a(\lambda b)=\lambda(ab)$ для любых $\lambda\in K$ и
$a,b\in A$.

{\it Размерностью} алгебры $A$ называется её размерность как
векторного пространства над~$K$. (Обозначение: $\dim_K A$.)
\end{definition}

\textbf{Примеры.}

1) Алгебра матриц $\Mat(n\times n, K)$ над
произвольным полем~$K$. Её размерность равна $n^2$.

2) Алгебра $K[x]$ многочленов от переменной $x$ над произвольным
полем~$K$. Её размерность равна~$\infty$.

3) $K, F$ --- поля, $K \subset F$, $F$ --- алгебра над $K$. \\
Если это $\RR \subset \CC$, то $\dim_\RR\CC = 2$.\\
Если это $\QQ \subset \RR$, то $\dim_\QQ\RR = \infty$.

\begin{definition}
\textit{Подкольцом} кольца $R$ называется всякое подмножество $R'
\subseteq R$, замкнутое относительно операций сложения и умножения
(т.\,е. $a + b \in R'$ и $ab \in R'$ для всех $a,b \in R'$) и
являющееся кольцом относительно этих операций. \textit{Подполем}
называется всякое подкольцо, являющееся полем.
\end{definition}

Например, $\ZZ$ является подкольцом в~$\QQ$, а~скалярные матрицы
образуют подполе в кольце $\Mat(n \times n, \RR)$.

\begin{remark}
Если $K$~--- подполе поля~$F$, то $F$ является алгеброй над~$K$.
Так, поле $\CC$ является бесконечномерной алгеброй над~$\QQ$, тогда
как над $\RR$ имеет размерность~$2$.
\end{remark}

\begin{definition}
\textit{Подалгеброй} алгебры $A$ (над полем~$K$) называется всякое
подмножество $A' \subseteq A$, замкнутое относительно всех трёх
имеющихся в $A$ операций (сложения, умножения и умножения на
элементы из~$K$) и являющееся алгеброй (над~$K$) относительно этих
операций.
\end{definition}

Легко видеть, что подмножество $A' \subseteq A$ является алгеброй
тогда и только тогда, когда оно является одновременно подкольцом и
векторным подпространством в~$A$.

Гомоморфизмы колец, алгебр определяются естественным образом как
отображения, сохраняющие все операции.

\begin{exercise}
Сформулируйте точные определения гомоморфизма колец и гомоморфизма
алгебр.
\end{exercise}

\begin{definition}
\textit{Изоморфизмом} колец, алгебр называется всякий гомоморфизм,
являющийся биекцией.
\end{definition}

В теории групп нормальные подгруппы обладают тем свойством, что по
ним можно \guillemotleft факторизовать\guillemotright{}. В~этом
смысле аналогами нормальных подгрупп в теории колец служат идеалы.

\begin{definition}
Подмножество $I$ кольца $R$ называется (\textit{двусторонним}) {\it
идеалом}, если оно является подгруппой по сложению и $ra\in I$,
$ar\in I$ для любых $a\in I$, $r\in R$.
\end{definition}

\begin{remark}
В~некоммутативных кольцах рассматривают также левые и правые идеалы.
\end{remark}

В каждом кольце $R$ есть {\it несобственные} идеалы $I=0$ и $I=R$.
Все остальные идеалы называются {\it собственными}.

\begin{exercise}
Пусть $R$~--- коммутативное кольцо. С~каждым элементом $a \in R$
связан идеал $(a) := \{ ra \mid r \in R \}$.
\end{exercise}

\begin{definition}
Идеал $I$ называется {\it главным}, если существует такой элемент
$a\in R$, что $I=(a)$. (В~этой ситуации говорят, что $I$ порождён
элементом~$a$.)
\end{definition}

\textbf{Пример.} В~кольце $\ZZ$ подмножество $k \ZZ$ ($k \in \ZZ$)
является главным идеалом, порождённым элементом~$k$. Более того, все
идеалы в $\ZZ$ являются главными.

\begin{remark}
Главный идеал $(a)$ является несобственным тогда и только тогда,
когда $a=0$ или $a$ обратим.
\end{remark}

Более общо, с каждым подмножеством $S \subseteq R$ связан идеал
$$
(S) := \{ r_1 a_1 + \ldots + r_k a_k \mid a_i \in S, r_i \in R,
k\in\NN\}.
$$
(Проверьте, что это действительно идеал!) Это наименьший по
включению идеал в~$R$, содержащий подмножество~$S$. В~этой ситуации
говорят, что идеал $I=(S)$ порождён подмножеством~$S$.

%
%Вернёмся к случаю произвольного кольца $R$. Поскольку любой идеал
%$I$ является подгруппой абелевой группы $(R,+)$, мы можем
%рассмотреть факторгруппу $R/I$. Введём на ней умножение по формуле
%$$
%(a+I)(b+I) := ab + I.
%$$
%Покажем, что это определение корректно. Пусть элементы $a',b' \in R$
%таковы, что $a' + I = a + I$ и $b' + I = b + I$. Проверим, что $a'b'
%+ I = ab + I$. Заметим, что $a' = a + x$ и $b' = b + y$ для
%некоторых $x, y \in I$. Тогда
%$$
%a'b' + I = (a + x)(b + y) + I = ab + ay + xb + xy + I = ab + I,
%$$
%поскольку $ay, xb, xy \in I$ в силу определения идеала.
%
%\begin{exercise}
%Проверьте, что множество $R/I$ является кольцом относительно
%имеющейся там операции сложения и только что введённой операции
%умножения.
%\end{exercise}
%
%\begin{definition}
%Кольцо $R/I$ называется {\it факторкольцом} кольца $R$ по
%идеалу~$I$.
%\end{definition}
%
%\textbf{Пример.} $\ZZ / n \ZZ = \ZZ_n$.
%
%Пусть $\varphi\colon R\to R'$~--- гомоморфизм колец. Тогда
%определены его ядро $\Ker \varphi = \lbrace r \in R \mid \varphi(r)
%= 0 \rbrace$ и образ $\Im \varphi = \lbrace \varphi(r) \mid r \in R
%\rbrace \subseteq R'$.
%
%\begin{lemma}
%Ядро $\Ker \varphi$ является идеалом в~$R$.
%\end{lemma}
%
%\begin{proof}
%Так как $\varphi$~--- гомоморфизм абелевых групп, то $\Ker \varphi$
%является подгруппой в $R$ по сложению. Покажем теперь, что $ra \in
%\Ker \varphi$ и $ar \in \Ker \varphi$ для произвольных элементов $a
%\in \Ker \varphi$ и $r \in R$. Имеем $\varphi(ra) = \varphi(r)
%\varphi(a) = \varphi(r) 0 = 0$, откуда $ra \in \Ker \varphi$.
%Аналогично получаем $ar \in \Ker \varphi$.
%\end{proof}
%
%\begin{exercise}
%Проверьте, $\Im \varphi$~--- подкольцо в~$R'$.
%\end{exercise}
%
%\smallskip
%
%{\bf Теорема о гомоморфизме для колец.}\ Пусть $\varphi\colon R\to
%R'$~-- гомоморфизм колец. Тогда имеет место изоморфизм
%$$
%R/\Ker\,\varphi\cong\Im\varphi.
%$$
%
%\smallskip
%
%\begin{proof}
%Положим для краткости $I = \Ker \varphi$ и рассмотрим отображение
%$$
%\pi \colon R/I \to \Im \varphi, \quad a+I \mapsto \varphi(a).
%$$
%Из доказательства теоремы о гомоморфизме для групп следует, что
%отображение $\pi$ корректно определено и является изоморфизмом
%абелевых групп (по сложению). Покажем, что $\pi$~--- изоморфизм
%колец. Для этого остаётся проверить, что $\pi$ сохраняет операцию
%умножения:
%$$
%\pi((a+I)(b+I)) = \pi(ab+I) = \varphi(ab) = \varphi(a) \varphi(b) =
%\pi(a+I) \pi(b+I).
%$$
%\end{proof}
%
%\begin{example}
%Пусть $R = \FFF(M, \RR)$. Зафиксируем произвольную точку $m_0 \in M$
%и рассмотрим гомоморфизм $\varphi \colon R \to \RR$, $f \mapsto
%f(m_0)$. Ясно, что гомоморфизм $\varphi$ сюръективен. Его ядром
%является идеал $I$ всех функций, обращающихся в нуль в точке $m_0$.
%По теореме о гомоморфизме получаем $R / I \cong \RR$.
%\end{example}
%
%\begin{definition}
%Кольцо $R$ называется {\it простым}, если в нём нет собственных
%(двусторонних) идеалов.
%\end{definition}
%
%\textbf{Пример.} Всякое поле является простым кольцом.
%
%\begin{definition}
%\textit{Центром} алгебры $A$ над полем $K$ называется её
%подмножество
%$$
%Z(A) = \{ a \in A \mid ab = ba \ \text{для всех} \ b \in A \}.
%$$
%\end{definition}
%
%\begin{theorem}
%Пусть $K$~--- поле, $n$~--- натуральное число и $A = \Mat(n \times
%n, K)$~--- алгебра квадратных матриц порядка~$n$ над полем~$K$.
%
%\textup{(1)} $Z(A) = \lbrace \lambda E \mid \lambda \in K \rbrace$,
%где $E$~--- единичная матрица \textup(в частности, $Z(A)$~---
%одномерное подпространство в~$A$\textup);
%
%\textup{(2)} алгебра $A$ проста \textup(как кольцо\textup).
%\end{theorem}
%
%\begin{proof}
%Для каждой пары индексов $i,j \in \lbrace 1, \ldots, n \rbrace$
%обозначим через $E_{ij}$ соответствующую \textit{матричную
%единицу}~--- такую матрицу, в которой на $(i,j)$-месте стоит
%единица, а на всех остальных местах~--- нули. Непосредственная
%проверка показывает, что
%$$
%E_{ij}E_{kl} =
%\begin{cases}
%E_{il}, & \ \text{если} \ j = k;\\
%0, & \ \text{если} \ j \ne k.
%\end{cases}
%$$
%Заметим, что матричные единицы образуют базис в~$A$ и всякая матрица
%$X = (x_{kl})$ представима в виде $X = \sum \limits_{k,l = 1}^n
%x_{kl} E_{kl}$.
%
%(1) Пусть матрица $X = \sum \limits_{k,l = 1}^n x_{kl} E_{kl}$ лежит
%в $Z(A)$. Тогда $X$ коммутирует со всеми матричными единицами.
%Выясним, что означает условие $XE_{ij} = E_{ij}X$. Имеем
%$$
%XE_{ij} = (\sum \limits_{k,l = 1}^n x_{kl} E_{kl})E_{ij} = \sum
%\limits_{k = 1}^n x_{ki}E_{kj}; \qquad E_{ij}X = E_{ij}(\sum
%\limits_{k,l = 1}^n x_{kl} E_{kl}) = \sum \limits_{l = 1}^n
%x_{jl}E_{il}.
%$$
%Сравнивая правые части двух равенств, получаем $x_{ii} = x_{jj}$,
%$x_{ki}=0$ при $k \ne i$ и $x_{jl}=0$ при $j \ne l$. Поскольку эти
%равенства имеют место при любых значениях $i,j$, мы получаем, что
%матрица $X$ скалярна, т.\,е. $X = \lambda E$ для некоторого $\lambda
%\in K$. С~другой стороны, ясно, что всякая скалярная матрица лежит в
%$Z(A)$.
%
%(2) Пусть $I$~--- двусторонний идеал алгебры~$A$. Если $I \ne
%\lbrace 0 \rbrace$, то $I$ содержит ненулевую матрицу~$X$. Покажем,
%что тогда $I = A$. Пусть индексы $k,l$ таковы, что $x_{kl} \ne 0$.
%Тогда
%$$
%E_{ik} X E_{lj} = E_{ik}(\sum_{p,q = 1}^n x_{pq} E_{pq}) E_{lj} =
%E_{ik} \sum \limits_{p = 1}^n x_{pl}E_{pj} = x_{kl} E_{ij} \in I.
%$$
%Домножая $x_{kl}E_{ij}$ на скалярную матрицу $(x_{kl})^{-1}E$, мы
%получаем, что $E_{ij} \in I$. Из произвольности выбора $i,j$
%следует, что все матричные единицы лежат в~$I$. Отсюда $I = A$, что
%и требовалось.
%\end{proof}

\bigskip

\begin{thebibliography}{99}
\bibitem{Vi}
Э.\,Б.~Винберг. Курс алгебры. М.: Факториал Пресс, 2002 (глава~1,
\S\,3,4,6,8,9 и глава~9, \S\,2)
\bibitem{Ko1}
А.\,И.~Кострикин. Введение в алгебру. Основы алгебры. М.: Наука.
Физматлит, 1994 (глава~4, \S\,3)
\bibitem{Ko3}
А.\,И.~Кострикин. Введение в алгебру. Основные структуры алгебры.
М.: Наука. Физматлит, 2000 (глава~4, \S\,1,4)
\bibitem{SZ}
Сборник задач по алгебре под редакцией А.\,И.~Кострикина. Новое
издание. М.: МЦНМО, 2009 (глава 14, \S\,63--64)
\end{thebibliography}

\end{document}
