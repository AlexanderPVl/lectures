\documentclass[a4paper,10pt]{amsart}

\usepackage[T2A]{fontenc}
\usepackage[utf8x]{inputenc}
\usepackage{amssymb}
\usepackage[russian]{babel}
\usepackage{geometry}
\usepackage{hyperref}
\usepackage{enumitem}

\geometry{a4paper,top=2cm,bottom=2cm,left=2cm,right=2cm}

\setlength{\parindent}{0pt}
\setlength{\parskip}{\medskipamount}

\newcommand{\Ker}{\mathop{\mathrm{Ker}}}
\renewcommand{\Im}{\mathop{\mathrm{Im}}}
\DeclareMathOperator{\Tor}{\mathrm{Tor}}
%\newcommand{\Tor}{\mathop{\mathrm{Tor}}}

%\def\Ker{{\rm Ker}}%
%\def\Im{{\rm Im}}%
\def\Mat{{\rm Mat}}%
\def\cont{{\rm cont}}%
%\def\Tor{{\rm Tor}}%
\def\Char{{\rm Char}}%
\def\signum{{\rm sig}}%
\def\Sym{{\rm Sym}}%
\def\St{{\rm St}}%
\def\Aut{{\rm Aut}}%
\def\Chi{{\mathbb X}}%
\def\Tau{{\rm T}}%
\def\Rho{{\rm R}}%
\def\rk{{\rm rk}}%
\def\ggT{{\rm ggT}}%
\def\kgV{{\rm kgV}}%
\def\Div{{\rm Div}}%
\def\div{{\rm div}}%
\def\quot{/\!\!/}%
\def\mal{\! \cdot \!}%
\def\Of{{\mathcal{O}}}
%
\def\subgrpneq{\le}%
\def\subgrp{\le}%
\def\ideal#1{\le_{#1}}%
\def\submod#1{\le_{#1}}%
%
\def\Bild{{\rm Bild}}%
\def\Kern{{\rm Kern}}%
\def\bangle#1{{\langle #1 \rangle}}%
\def\rq#1{\widehat{#1}}%
\def\t#1{\widetilde{#1}}%
\def\b#1{\overline{#1}}%
%
\def\abs#1{{\vert #1 \vert}}%
\def\norm#1#2{{\Vert #1 \Vert}_{#2}}%
\def\PS#1#2{{\sum_{\nu=0}^{\infty} #1_{\nu} #2^{\nu}}}%
%
\def\C{{\rm C}}%
\def\O{{\rm O}}%
\def\HH{{\mathbb H}}%
\def\LL{{\mathbb L}}%
\def\FF{{\mathbb F}}%
\def\CC{{\mathbb C}}%
\def\KK{{\mathbb K}}%
\def\TT{{\mathbb T}}%
\def\ZZ{{\mathbb Z}}%
\def\RR{{\mathbb R}}%
\def\SS{{\mathbb S}}%
\def\NN{{\mathbb N}}%
\def\QQ{{\mathbb Q}}%
\def\PP{{\mathbb P}}%
\def\AA{{\mathbb A}}%
%
\def\eins{{\mathbf 1}}%
%
\def\AG{{\rm AG}}%
\def\Aut{{\rm Aut}}%
\def\Hol{{\rm Hol}}%
\def\GL{{\rm GL}}%
\def\SL{{\rm SL}}%
\def\SO{{\rm SO}}%
\def\Sp{{\rm Sp}}%
\def\gl{\mathfrak{gl}}%
\def\rg{{\rm rg}}%
\def\sl{\mathfrak{sl}}%
\def\HDiv{{\rm HDiv}}%
\def\CDiv{{\rm CDiv}}%
\def\Res{{\rm Res}}%
\def\Pst{{\rm Pst}}%
\def\Nst{{\rm Nst}}%
\def\rad{{\rm rad}}%
\def\GL{{\rm GL}}%
\def\Tr{{\rm Tr}}%
\def\Pic{{\rm Pic}}%
\def\Hom{{\rm Hom}}%
\def\hom{{\rm hom}}%
\def\Mor{{\rm Mor}}%
\def\codim{{\rm codim}}%
\def\Supp{{\rm Supp}}%
\def\Spec{{\rm Spec}}%
\def\Proj{{\rm Proj}}%
\def\Maps{{\rm Maps}}%
\def\cone{{\rm cone}}%
\def\ord{{\rm ord}}%
\def\pr{{\rm pr}}%
\def\id{{\rm id}}%
\def\mult{{\rm mult}}%
\def\inv{{\rm inv}}%
\def\neut{{\rm neut}}%
%
\def\AAA{\mathcal{A}}
\def\BBB{\mathcal{B}}
\def\CCC{\mathcal{C}}
\def\EEE{\mathcal{E}}
\def\FFF{\mathcal{F}}

\def\CF{{\rm CF}}
\def\GCD{{\rm GCD}}
\def\Mat{{\rm Mat}}
\def\End{{\rm End}}
\def\cont{{\rm cont}}
\def\Kegel{{\rm Kegel}}
\def\Char{{\rm Char}}
\def\Der{{\rm Der}}
\def\signum{{\rm sg}}
\def\grad{{\rm grad}}
\def\Spur{{\rm Spur}}
\def\Sym{{\rm Sym}}
\def\Alt{{\rm Alt}}
\def\Abb{{\rm Abb}}
\def\Chi{{\mathbb X}}
\def\Tau{{\rm T}}
\def\Rho{{\rm R}}
\def\ad{{\rm ad}}
\def\Frob{{\rm Frob}}
\def\Rang{{\rm Rang}}
\def\SpRang{{\rm SpRang}}
\def\ZRang{{\rm ZRang}}
\def\ggT{{\rm ggT}}
\def\kgV{{\rm kgV}}
\def\Div{{\rm Div}}
\def\div{{\rm div}}
\def\quot{/\!\!/}
\def\mal{\! \cdot \!}
\def\add{{\rm add}}
\def\mult{{\rm mult}}
\def\smult{{\rm smult}}

\def\subgrpneq{\le}
\def\subgrp{\le}
\def\ideal#1{\unlhd_{#1}}
\def\submod#1{\le_{#1}}

\def\Bild{{\rm Bild}}
\def\Kern{{\rm Kern}}
\def\Kon{{\rm Kon}}
\def\bangle#1{{\langle #1 \rangle}}
\def\rq#1{\widehat{#1}}
\def\t#1{\widetilde{#1}}
\def\b#1{\overline{#1}}

\def\abs#1{{\vert #1 \vert}}
\def\norm#1#2{{\Vert #1 \Vert}_{#2}}
\def\PS#1#2{{\sum_{\nu=0}^{\infty} #1_{\nu} #2^{\nu}}}


\def\eins{{\mathbf 1}}

\def\ElM{{\rm ElM}}
\def\ZOp{{\rm ZOp}}
\def\SpOp{{\rm SpOp}}
\def\Gal{{\rm Gal}}
\def\Def{{\rm Def}}
\def\Fix{{\rm Fix}}
\def\ord{{\rm ord}}
\def\Aut{{\rm Aut}}
\def\Hol{{\rm Hol}}
\def\GL{{\rm GL}}
\def\SL{{\rm SL}}
\def\SO{{\rm SO}}
\def\Sp{{\rm Sp}}
\def\Spann{{\rm Spann}}
\def\Lin{{\rm Lin}}
\def\gl{\mathfrak{gl}}
\def\rg{{\rm rg}}
\def\sl{\mathfrak{sl}}
\def\so{\mathfrak{so}}
\def\sp{\mathfrak{sp}}
\def\gg{\mathfrak{g}}
\def\HDiv{{\rm HDiv}}
\def\CDiv{{\rm CDiv}}
\def\Res{{\rm Res}}
\def\Pst{{\rm Pst}}
\def\Nst{{\rm Nst}}
\def\WDiv{{\rm WDiv}}
\def\GL{{\rm GL}}
\def\Tr{{\rm Tr}}
\def\Pic{{\rm Pic}}
\def\Hom{{\rm Hom}}
\def\hom{{\rm hom}}
\def\Mor{{\rm Mor}}
\def\codim{{\rm codim}}
\def\Supp{{\rm Supp}}
\def\Spec{{\rm Spec}}
\def\Proj{{\rm Proj}}
\def\Maps{{\rm Maps}}
\def\cone{{\rm cone}}
\def\ord{{\rm ord}}
\def\pr{{\rm pr}}
\def\id{{\rm id}}
\def\mult{{\rm mult}}
\def\inv{{\rm inv}}
\def\neut{{\rm neut}}
\def\trdeg{{\rm trdeg}}
\def\sing{{\rm sing}}
\def\reg{{\rm reg}}


%%%%%%%%%%%%%%%%%%%%%%%%%%%

\newtheorem{theorem}{Теорема}
\newtheorem{proposition}{Предложение}
\newtheorem{lemma}{Лемма}
\newtheorem{corollary}{Следствие}
\theoremstyle{definition}
\newtheorem{definition}{Определение}
\newtheorem{problem}{Задача}
%
\theoremstyle{remark}
\newtheorem{exercise}{Упражнение}
\newtheorem{remark}{Замечание}
\newtheorem{example}{Пример}

\renewcommand{\theenumi}{\textup{(\alph{enumi})}}
\renewcommand{\labelenumi}{\theenumi}
\newcounter{property}
\renewcommand{\theproperty}{\textup{(\arabic{property})}}
\newcommand{\property}{\refstepcounter{property}\item}
\newcounter{prooperty}
\renewcommand{\theprooperty}{\textup{(\arabic{prooperty})}}
\newcommand{\prooperty}{\refstepcounter{prooperty}\item}

\makeatletter
\def\keywords#1{{\def\@thefnmark{\relax}\@footnotetext{#1}}}
\let\subjclass\keywords
\makeatother
%
\begin{document}
%
\sloppy
%\thispagestyle{empty}
%
\centerline{\large \bf Лекции курса \guillemotleft
Алгебра\guillemotright{}, лектор Р.\,С.~Авдеев}

\smallskip

\centerline{\large ФКН НИУ ВШЭ, 1-й курс ОП ПМИ, 4-й модуль,
2015/2016 учебный год}


\bigskip

\section*{Лекция~8}

\medskip

{\it Элементарные симметрические многочлены. Основная теорема о
симметрических многочленах. Лексикографический порядок. Теорема
Виета. Дискриминант многочлена.}

\medskip

Вернемся ненадолго к теме прошлой лекции. Рассмотрим кольцо $R = K[x_1, \ldots, x_n]$, где 	$K$ --- поле. 
На семинарах разбиралось, что оно не является кольцом главных идеалов и, соответственно, евклидовым кольцом. Однако несмотря на это:

\textbf{Теорема.} Кольцо $R$ факториально.

Впрочем, доказывать эту теорему мы не будем.

Вернемся теперь к теме текущей лекции. Пусть $K$~--- произвольное поле.

\begin{definition}
Многочлен $f(x_1,\ldots,x_n)\in K[x_1,\ldots,x_n]$ называется {\it
симметрическим}, если
$f(x_{\tau(1)},\ldots,x_{\tau(n)})=f(x_1,\ldots,x_n)$ для всякой
перестановки $\tau \in S_n$.
\end{definition}

\textbf{Примеры:}

1) Многочлен $x_1x_2 + x_2x_3$ не является симметрическим, а вот многочлен $x_1x_2 + x_2x_3 + x_1x_3$ --- является.

2) {\it Степенные суммы} $s_k(x_1, \ldots, x_n) = x_1^k + x_2^k +
\ldots + x_n^k$ являются симметрическими многочленами.

3) {\it Элементарные симметрические многочлены}
$$
\sigma_1(x_1, \ldots, x_n) = x_1 + x_2 + \ldots + x_n;
$$
$$
\sigma_2(x_1, \ldots, x_n) = \sum \limits_{1 \leqslant i < j
\leqslant n} x_i x_j;
$$
$$
..................................................
$$
$$
\sigma_k(x_1, \ldots, x_n) = \sum \limits_{1 \leqslant i_1 < i_2 <
\ldots < i_k \leqslant n} x_{i_1} x_{i_2} \ldots x_{i_k};
$$
$$
..................................................
$$
$$
\sigma_n(x_1, \ldots, x_n) = x_1 x_2 \ldots x_n
$$
являются симметрическими.

5) Определитель Вандермонда
$$
V(x_1, \ldots, x_n) =
\begin{vmatrix}
1 & x_1 & x_1^2 & \ldots & x_1^{n-1} \\
1 & x_2 & x_2^2 & \ldots & x_2^{n-1} \\
\ldots & \ldots & \ldots & \ldots & \ldots \\
1 & x_n & x_n^2 & \ldots & x_n^{n-1}
\end{vmatrix} =
\prod \limits_{1 \leqslant i < j \leqslant n} (x_j - x_i)
$$
симметрическим многочленом не является (при перестановке индексов
умножается на её знак), а вот его квадрат уже является.

Основная цель этой лекции~--- понять, как устроены все
симметрические многочлены.

Легко видеть, что все симметрические многочлены образуют подкольцо
(и даже подалгебру) в $K[x_1, \ldots, x_n]$. В~частности, если
$F(y_1, \ldots, y_k)$~--- произвольный многочлен и $f_1(x_1, \ldots,
x_n)$, $\ldots$, $f_k(x_1, \ldots, x_n)$~--- симметрические
многочлены, то многочлен
$$
F(f_1(x_1, \ldots, x_n), \ldots, f_k(x_1, \ldots, x_n)) \in K[x_1,
\ldots, x_n]
$$
также является симметрическим. Мы покажем, что всякий симметрический
многочлен однозначно выражается через элементарные симметрические
многочлены.

\medskip

{\bf Основная теорема о симметрических многочленах.}\ Для всякого
симметрического многочлена $f(x_1, \ldots, x_n)$ существует и
единственен такой многочлен $F(y_1, \ldots, y_n)$, что
$$
f(x_1, \ldots, x_n) = F(\sigma_1(x_1, \ldots, x_n), \ldots,
\sigma_n(x_1, \ldots, x_n)).
$$

\textbf{Пример.} $s_2(x_1, \ldots, x_n) = x_1^2 + \ldots + x_n^2 =
(x_1 + \ldots + x_n)^2 - 2\sum \limits_{1 \leqslant i < j \leqslant
n} x_i x_j = \sigma_1^2 - 2\sigma_2$, откуда $F(y_1, \ldots, y_n) =
y_1^2 - 2y_2$.


Доказательство этой теоремы потребует некоторой подготовки. Начнём с
того, что определим старший член многочлена от многих переменных.

Пусть $M_n$~--- множество всех одночленов от переменных $x_1,
\ldots, x_n$. Определим на $M_n$ {\it лексикографический порядок}
следующим образом:
$$
ax_1^{i_1}x_2^{i_2}\ldots x_n^{i_n} \prec bx_1^{j_1}x_2^{j_2}\ldots
x_n^{j_n} \quad \Leftrightarrow \quad \exists k: \: i_1=j_1,\ldots,
i_{k-1}=j_{k-1}, i_k<j_k.
$$
Например, $x_1^2x_3^9 \prec x_1^2x_2$.

\ \\
\textbf{Свойства:}

1) Лексикографический порядок обладает свойством
транзитивности: если $u,v,w \in M_n$, $u \prec v$ и $v \prec w$, то
$u \prec w$ (докажите это).

2) Если $u,v,w \in M_n$ и $u \prec v$, то $uw \prec vw$.

Свойство транзитивности лексикографического порядка позволяет
корректно определить следующее понятие.

\begin{definition}
{\it Старшим членом} ненулевого многочлена $f(x_1,\ldots,x_n)$
называется наибольший в лексикографическом порядке встречающий в нём
одночлен. Обозначение: $L(f)$.
\end{definition}

\textbf{Примеры:}

1) $L(s_k) = x_1^k$;

2) $L(\sigma_k) = x_1x_2 \ldots x_k$.


{\bf Лемма о старшем члене.}\, Пусть $f(x_1, \ldots, x_n), g(x_1,
\ldots, x_n) \in K[x_1, \ldots, x_n]$~--- произвольные ненулевые
многочлены. Тогда $L(f g ) = L(f) L(g)$.

\begin{proof}
Пусть $u$~--- какой-то одночлен многочлена~$f$ и $v$~--- какой-то
одночлен многочлена~$g$. По определению старшего члена имеем
\begin{equation} \label{inequalities}
u \preccurlyeq L(f), \quad v \preccurlyeq L(g).
\end{equation}
Тогда $uv \preccurlyeq uL(g) \preccurlyeq L(f)L(g)$, т.\,е. $uv
\preccurlyeq L(f) L(g)$. Более того, легко видеть, что $uv \prec
L(f) L(g)$ тогда и только тогда, когда хотя бы одно из
\guillemotleft неравенств\guillemotright{} (\ref{inequalities})
является строгим. Отсюда следует, что после раскрытия скобок в
произведении $fg$ одночлен $L(f)L(g)$ будет старше всех остальных
возникающих одночленов. Ясно, что после приведения подобных членов
этот одночлен сохранится и будет по-прежнему старше всех остальных
одночленов, поэтому $L(f)L(g) = L(fg)$.
\end{proof}

\begin{lemma} \label{lemma_1}
Если $ax_1^{k_1}x_2^{k_2}\ldots x_n^{k_n}$~--- старший член
некоторого симметрического многочлена~$f(x_1, \ldots, x_n)$, то $k_1
\geqslant k_2 \geqslant \ldots \geqslant k_n$.
\end{lemma}

\begin{proof}
От противного. Пусть $k_i < k_{i+1}$ для некоторого~$i \in \lbrace
1, \ldots, n-1 \rbrace$. Тогда, будучи симметрическим, многочлен $f$
содержит одночлен $ax_1^{k_1} \ldots
x_{i-1}^{k_{i-1}}x_i^{k_{i+1}}x_{i+1}^{k_i}x_{i+2}^{k_{i+2}} \ldots
x_n^{k_n}$, который старше $L(f)$. Противоречие.
\end{proof}

\begin{lemma} \label{lemma_2}
Пусть $k_1, \ldots, k_n$~--- целые неотрицательные числа. Если $k_1
\geqslant k_2 \geqslant \ldots \geqslant k_n$, то существуют и
единственны такие целые неотрицательные числа $l_1, l_2, \ldots,
l_n$, что
$$
x_1^{k_1}x_2^{k_2}\ldots x_n^{k_n}=
L(\sigma_1(x_1,\ldots,x_n)^{l_1}\sigma_2(x_1,\ldots,x_n)^{l_2}\ldots\sigma_n(x_1,\ldots,x_n)^{l_n}).
$$
\end{lemma}

\begin{proof}
С учётом леммы о старшем члене требуемое условие означает, что
искомые числа $l_1, \ldots, l_n$ удовлетворяют системе
$$
\begin{cases}
l_1 + l_2 + \ldots + l_n = k_1; \\
\phantom{l_1 + {}} l_2 + \ldots + l_n = k_2;\\
\phantom{l_1 + l_2 + .} \ldots\ldots\ldots\ldots\\
\phantom{l_1 + l_s + \ldots + {}} l_n = k_n,
\end{cases}
$$
из которой они легко находятся:
\begin{align*}
l_i &= k_i - k_{i+1} \quad \text{при} \ 1 \leqslant i \leqslant n-1; \\
l_n &= k_n.
\end{align*}
\end{proof}

\begin{proof}[Доказательство основной теоремы о симметрических многочленах]
Пусть $f(x_1, \ldots, x_n)$~--- произвольный симметрический
многочлен.

Сначала докажем существование искомого многочлена~$F(y_1, \ldots,
y_n)$. Если $f(x_1, \ldots, x_n)$~--- нулевой многочлен, то можно
взять $F(y_1,\ldots, y_n) = 0$. Далее считаем, что $f(x_1, \ldots,
x_n) \ne 0$. Пусть $L(f) = ax_1^{k_1} \ldots x_n^{k_n}$, $a \ne 0$.
Тогда $k_1 \geqslant k_2 \geqslant \ldots \geqslant k_n$ в силу
леммы~\ref{lemma_1}. По лемме~\ref{lemma_2} найдётся одночлен от
элементарных симметрических многочленов
$a\sigma_1^{l_1}\ldots\sigma_n^{l_n}$, старший член которого
совпадает с~$L(f)$. Положим $f_1 := f -
a\sigma_1^{l_1}\ldots\sigma_n^{l_n}$. Если $f_1 = 0$, то $f = a
\sigma_1^{l_1} \ldots \sigma_n^{l_n}$ и искомым многочленом будет
$F(y_1, \ldots, y_n) = ay_1^{l_1} \ldots y_n^{l_n}$. Если же $f_1
\ne 0$, то $L(f_1) \prec L(f)$. Повторим ту же процедуру: вычтя из
$f_1$ подходящий одночлен от $\sigma_1, \ldots, \sigma_n$, мы
получим новый многочлен~$f_2$ со следующим свойством: либо $f_2 = 0$
(и тогда мы получаем выражение $f$ через элементерные симметрические
многочлены), либо $L(f_2) \prec L(f_1)$. Многократно повторяя эту
процедуру, мы получим последовательность многочленов $f, f_1, f_2,
\ldots$ со свойством $L(f) \succ L(f_1) \succ L(f_2) \succ \ldots$.
Покажем, что процесс закончится, т.\,е. найдётся такое~$m$, что $f_m
= 0$ (и тогда мы получим представление $f$ в виде многочлена от
$\sigma_1, \ldots, \sigma_n$). Для этого заметим, что переменная
$x_1$ входит в старший член каждого из многочленов $f_1, f_2,
\ldots$ в степени, не превышающей $k_1$. Но в силу
леммы~\ref{lemma_1} одночленов с таким условием имеется лишь
конечное число, поэтому процесс не может продолжаться бесконечно.

Теперь докажем единственность многочлена $F(y_1, \ldots, y_n)$.
Предположим, что
$$
f(x_1, \ldots, x_n) = F(\sigma_1(x_1, \ldots, x_n), \ldots,
\sigma_n(x_1, \ldots, x_n) = G(\sigma_1(x_1, \ldots, x_n), \ldots,
\sigma_n(x_1, \ldots, x_n))
$$
для двух различных многочленов $F(y_1, \ldots, y_n), G(y_1, \ldots,
y_n) \in K[y_1, \ldots, y_n]$. Тогда многочлен $$H(y_1, \ldots, y_n)
:= F(y_1, \ldots, y_n) - G(y_1, \ldots, y_n)$$ является ненулевым,
но $H(\sigma_1(x_1, \ldots, x_n), \ldots, \sigma_n(x_1, \ldots,
x_n)) = 0$. Покажем, что такое невозможно. Пусть $H_1, \ldots,
H_s$~--- все ненулевые одночлены в~$H$. Обозначим через $w_i$
старший член многочлена $$H_i(\sigma_1(x_1, \ldots, x_n), \ldots,
\sigma_n(x_1, \ldots, x_n)) \in K[x_1, \ldots, x_n].$$ В~силу
леммы~\ref{lemma_2} среди одночленов $w_1, \ldots, w_s$ нет
пропорциональных. Выберем из них старший в лексикографическом
порядке. Он не может сократиться ни с одним членов в выражении
$$
H_1(\sigma_1(x_1, \ldots, x_n), \ldots, \sigma_n(x_1, \ldots, x_n))
+ \ldots + H_s(\sigma_1(x_1, \ldots, x_n), \ldots, \sigma_n(x_1,
\ldots, x_n)),
$$
поэтому $H(\sigma_1(x_1, \ldots, x_n), \ldots, \sigma_n(x_1, \ldots,
x_n)) \ne 0$, и мы пришли к противоречию.
\end{proof}

На практике многочлен $F(y_1, \ldots, y_n)$ можно искать, повторяя
описанный в доказательстве алгоритм, однако он может потребовать
много вычислений. Более эффективным для нахождения многочлена
$F(y_1, \ldots, y_n)$ является метод неопределённых коэффициентов,
который планируется разобрать на семинарах.

{\bf Теорема Виета}. Пусть $\alpha_1,\ldots,\alpha_n$~--- корни
многочлена $x^n + a_{n-1}x^{n-1} + \dots + a_1x + a_0$. Тогда
$$
\sigma_k(\alpha_1,\ldots,\alpha_n)=(-1)^k a_{n-k}, \quad k = 1,
\ldots, n.
$$

\begin{proof}
Достаточно приравнять коэффициенты при $x^{n-k}$ в левой и правой
частях равенства
$$
x^n+a_{n-1}x^{n-1}+\dots+a_1x+a_0=(x-\alpha_1)(x-\alpha_2)\ldots
(x-\alpha_n).
$$
\end{proof}

Из теоремы Виета и основной теоремы о симметрических многочленах
следует, что мы можем выразить значение любого симметрического
многочлена от корней данного многочлена через коэффициенты, не
находя самих корней.

\begin{definition}
{\it Дискриминантом} многочлена $h(x)=a_nx^n+\ldots+a_1x+a_0$ с
корнями $\alpha_1, \ldots, \alpha_n$ называется выражение
$$
D(h) = a_n^{2n-2} \prod \limits_{1 \leqslant i < j \leqslant n}
(\alpha_i - \alpha_j)^2.
$$
\end{definition}

\begin{remark}
Дискриминант $D(h)$ является симметрическим многочленом от
$\alpha_1, \ldots, \alpha_n$, а значит, в соответствии с
вышесказанным он является многочленом от коэффициентов $a_n,
a_{n-1}, \ldots, a_0$.
\end{remark}

\begin{remark}
Непосредствено из определения следует, что $D(h) = 0$ тогда и только
тогда, когда многочлен $h$ имеет кратный корень.
\end{remark}
%
%\begin{example}
%Пусть $h(x)=ax^2+bx+c$. Тогда
%$$
%D(h)=a^2(\alpha_2-\alpha_1)^2=a^2((\alpha_1+\alpha_2)^2-4\alpha_1\alpha_2)=
%a^2((-b/a)^2-4c/a)=b^2-4ac.
%$$
%\end{example}
%
%\bigskip
%
%\textbf{Понятие о базисе Грёбнера\footnote{Это необязательный
%материал, в программу экзамена он не войдёт.}.} Рассмотрим в кольце
%$K[x_1, \ldots, x_n]$ идеал~$I$, порождённый многочленами $f_1,
%\ldots, f_k$. Как выяснить алгоритмически, принадлежит ли данный
%многочлен $f \in K[x_1, \ldots, x_n]$ идеалу~$I$? Другими словами,
%представим ли многочлен $f$ в виде $f_1h_1+\ldots+f_kh_k$ для
%некоторых многочленов $h_1, \ldots, h_k \in K[x_1, \ldots, x_n]$?
%При $k=1$ или $n=1$ ответить на этот вопрос легко, в общем случае
%сложнее.
%
%{\it Базисом Грёбнера} идеала $I$ в кольце $K[x_1,\ldots,x_n]$
%называется такой набор многочленов $g_1, \ldots, g_m \in I$, что для
%всякого $g \in I$ старший член $g$ делится на старший член одного из
%$g_i$. Оказывается, базис Грёбнера данного идеала всегда существует
%и его можно эффективно построить, исходя из набора порождающих $f_1,
%\ldots, f_k$ (алгоритм Бухбергера и его модификации). Имея такой
%базис, мы можем проводить редукции, т.\,е. вычитать из данного
%многочлена $f$ один из элементов базиса Грёбнера, умноженный на
%некоторый многочлен так, чтобы старший член сократился. Осуществляя
%редукции, мы за конечное число шагов выясним, лежит ли $f$ в идеале.
%
%Базисы Грёбнера позволяют алгоритмически решать и многие другие
%задачи, связанные с системами полиномиальных уравнений.

\bigskip

\begin{thebibliography}{99}
\bibitem{Vi}
Э.\,Б.~Винберг. Курс алгебры. М.: Факториал Пресс, 2002 (глава~3,
\S\,8)
\bibitem{Ko1}
А.\,И.~Кострикин. Введение в алгебру. Основы алгебры. М.: Наука.
Физматлит, 1994 (глава~6, \S\,2)
\bibitem{SZ}
Сборник задач по алгебре под редакцией А.\,И.~Кострикина. Новое
издание. М.: МЦНМО, 2009 (глава~6, \S\S\,31,32)
\end{thebibliography}



\end{document}
