\documentclass[../main.tex]{subfiles}

\begin{document}
	\section{Лекция 8. Двоичные деревья поиска.}
	
	Пусть мы хотим реализовать структуру данных, моделирующую множество. Для этого нам надо уметь быстро \textit{вставлять, искать} и \textit{удалять} элементы из множества. В этом нам поможет такая структура, как \textit{двоичное дерево поиска}.
	
	\subsection{Двоичное дерево поиска}
	
	Будем называть двоичное дерево $T$ \textit{деревом поиска}, если для него выполняются следующие условия:
	\begin{enumerate}
		\item У любого узла существует не более двух детей
		\item Если узел $y$ лежит в левом поддереве с корнем $x$, то $y.key \leqslant x.key$; если в правом, то $y.key \geqslant x.key$.
	\end{enumerate}

	\begin{remark}
		Строгость неравенства зависит от поставленной задачи. Например, при реализации multiset'а неравенство нестрогое, а при реализиации set'а -- строгое.
	\end{remark}
	
	Приведем пример двоичного дерева поиска:
	
	\begin{center}
		\begin{tikzpicture}
		\begin{scope}[every node/.style={circle,thick,draw}]
		\node (top) at (10, 10) {$9$};
		
		\node (2-1) at (6, 	8) 	{$7$};
		\node (2-2) at (14, 8) 	{$20$};
		
		\node (3-1) at (5,  6) 	{$2$};
		\node (3-2) at (7,  6) 	{$8$};
		\node (3-3) at (12, 6)  {$15$};
		
		\node (4-2) at (6,  4)	{$4$};
		\node (4-3) at (11, 4)	{$12$};
		\node (4-4) at (13, 4) 	{$17$};
		
		\node (5-1) at (5,  3)  {$3$};
		\node (5-2) at (7,  3)  {$5$};
		\end{scope}
		
		\begin{scope}[>={Stealth[black]}, every edge/.style={draw=black}]
		
		\path [->] (top) edge (2-1);
		\path [->] (top) edge (2-2);
		
		
		\path [->] (2-1) edge (3-1);
		\path [->] (2-1) edge (3-2);
		\path [->] (2-2) edge (3-3);
		
		\path [->] (3-1) edge (4-2);
		\path [->] (3-3) edge (4-3);
		\path [->] (3-3) edge (4-4);
		
		
		\path [->] (4-2) edge (5-1);
		\path [->] (4-2) edge (5-2);
		
		\end{scope}
		\end{tikzpicture}
	\end{center}

	\textit{Весь код в этой лекции будет написан в Python-подобном синтаксисе}	
	
	Для начала опишем поиск в такой структуре:
	\begin{lstlisting}[language=python]
def tree_search(x, k):
    if x:
        if k < x.key:
            return tree_search(x.left, k)
        elif k > x.key:
            return tree_search(x.right, k)
        else:
            return x
    return None
	\end{lstlisting}
	
	Словами: на каждом шаге сравниваем с текущим и идем в нужную ветку.
	
	\begin{time}
		В худшем случае время работы данного алгоритма займет $O(height(T))$, где $T$ -- дерево.
	\end{time}
	
	Затем нужно научиться вставлять ноды в дерево. Основная идея: вставляем в корень, проверяем выполнение инварианта, останавливаемся, если все хорошо, и идем дальше иначе. 
	
	\begin{lstlisting}[language=python]
def tree_insert(t, k): # t -- tree
    z = Node(k)
    x = t.root
    while x:
	    z.parent = x
        if z.key > x.key:
            x = x.right
        else:
            x = x.left

    if z.parent:
        if z.key > z.parent.key:
            z.parent.right = z
        else:
            z.parent.left = z
    else:
        t.root = z
	\end{lstlisting}
	
	
	
	
	
	
	
	
	
	
	
	
	
	
	
	
	
	
	
	
\end{document}