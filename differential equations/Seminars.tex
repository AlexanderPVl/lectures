\documentclass[a4paper,12pt]{article}
 
%% Начало шапки
 
%% Настройка поддержки русского языка
\usepackage{cmap}                   % Поиск по кириллице
\usepackage{mathtext}               % Кириллица в формулах
\usepackage[T1,T2A]{fontenc}        % Кодировки шрифтов
\usepackage[utf8]{inputenc}         % Кодировка текста
\usepackage[english,russian]{babel} % Подключение поддержки языков
 
%% Настройка размеров полей
\usepackage[top=0.7in, bottom=0.75in, left=0.625in, right=0.625in]{geometry}
 
%% Математические пакеты
\usepackage{mathtools}              % Тот же amsmath, только с некоторыми поправками
\usepackage{amssymb}                % Математические символы
\usepackage{amsthm}                 % Оформление теорем
\usepackage{amstext}                % Текстовые вставки в формулы
\usepackage{amsfonts}               % Математические шрифты
\usepackage{icomma}                 % "Умная" запятая: $0,2$ --- число, $0, 2$ --- перечисление
\usepackage{enumitem}               % Для выравнивания itemize (\begin{itemize}[align=left])
\usepackage{array}                  % Таблицы и матрицы
\usepackage{multirow}
 
%% Алгоритмические пакеты и их настройки
\usepackage{algorithm}              % Шапка алгоритма
\usepackage{algorithmicx}           % Написание алгоритмов
\usepackage[noend]{algpseudocode}   % Написание псевдокода; убраны end
\usepackage{listings}               % Для кода на каком-либо языке программиования
\renewcommand{\algorithmicrequire}{\textbf{Ввод:}}              % Ввод
\renewcommand{\algorithmicensure}{\textbf{Вывод:}}              % Вывод
\floatname{algorithm}{Алгоритм}                                 % Название алгоритма
\renewcommand{\algorithmiccomment}[1]{\hspace*{\fill}\{// #1\}} % Комментарии
\newcommand{\algname}[1]{\textsc{#1}}                           % Вызов функции в алгоритме
\usepackage{physics}
 
%% Шрифты
\usepackage{euscript}               % Шрифт Евклид
\usepackage{mathrsfs}               % \mathscr{}
 
%% Графика
\usepackage[pdftex]{graphicx}       % Вставка картинок
\graphicspath{{images/}}            % Стандартный путь к картинкам
\usepackage{tikz}  
\usetikzlibrary{patterns}                 % Рисование всего
\usepackage{pgfplots}               % Графики
 
%% Прочие пакеты
\usepackage{indentfirst}                    % Красная строка в начале текста
\usepackage{epigraph}                       % Эпиграфы
\usepackage{fancybox,fancyhdr}              % Колонтитулы
\usepackage[colorlinks=true, urlcolor=blue]{hyperref}   % Ссылки
\usepackage{titlesec}                       % Изменение формата заголовков
\usepackage[normalem]{ulem}                 % Для зачёркиваний
\usepackage[makeroom]{cancel}               % И снова зачёркивание (на этот раз косое)

 
%% Прочее
\mathtoolsset{showonlyrefs=true}        % Показывать номера только у тех формул,
                                        % на которые есть \eqref{} в тексте.
\renewcommand{\headrulewidth}{1.8pt}    % Изменяем размер верхнего отступа колонтитула
\renewcommand{\footrulewidth}{0.0pt}    % Изменяем размер нижнего отступа колонтитула

%Прочее
\usepackage{forest} % Деревья

\usetikzlibrary{arrows,calc}
\usetikzlibrary{quotes,angles}

\usetikzlibrary{positioning,intersections}

\usetikzlibrary{through}

\NewDocumentCommand{\bissectrice}{%
	O{}     % drawing options
	mmm     % bissector of mmm
	m       % intersection point between base and bissector
	O{1}O{1}% extended drawing of the bissector
}{%
\path[name path=Bis#2#3#4] let
\p1 = ($(#2) - (#3)$),
\p2 = ($(#4) - (#3)$),
\n1 = {veclen(\x1,\y1)/2} ,
\n2 = {veclen(\x2,\y2)/2} ,
\n3 = {max(\n1,\n2)},
\p1 = ($(#3)!\n3!(#2)$),
\p2 = ($(#3)!\n3!(#4)$),
\p3 = ($(\p1) + (\p2) - (#3)$)
in
(#3) -- (\p3) ;

\path[name path = foo] (#2)--(#4) ;

\path[name intersections={of=foo and Bis#2#3#4, by=#5}] ;

\path[#1] ($(#3)!#6!(#5)$) -- ($(#5)!#7!(#3)$) ;
}


\tikzset{
	right angle quadrant/.code={
		\pgfmathsetmacro\quadranta{{1,1,-1,-1}[#1-1]}     % Arrays for selecting quadrant
		\pgfmathsetmacro\quadrantb{{1,-1,-1,1}[#1-1]}},
	right angle quadrant=1, % Make sure it is set, even if not called explicitly
	right angle length/.code={\def\rightanglelength{#1}},   % Length of symbol
	right angle length=2ex, % Make sure it is set...
	right angle symbol/.style n args={3}{
		insert path={
			let \p0 = ($(#1)!(#3)!(#2)$),     % Intersection
			\p1 = ($(\p0)!\quadranta*\rightanglelength!(#3)$), % Point on base line
			\p2 = ($(\p0)!\quadrantb*\rightanglelength!(#2)$), % Point on perpendicular line
			\p3 = ($(\p1)+(\p2)-(\p0)$) in  % Corner point of symbol
			(\p1) -- (\p3) -- (\p2)
		}
	}
}

%% Определения
\newtheorem{definition}{Определение}

\newtheorem*{task}{Задача}
\newtheorem*{task1}{Задача №1}
\newtheorem*{task2}{Задача №2}
\newtheorem*{task3}{Задача №3}
\newtheorem*{task4}{Задача №4}
\newtheorem*{task5}{Задача №5}

\newtheorem{fulllemma}{Лемма}
\newtheorem*{sl1}{Следствие 1}
\newtheorem*{sl2}{Следствие 2}
\newtheorem*{scheme}{Утверждение 1}
\newtheorem*{theorem}{Теорема}
\newtheorem*{proposal}{Предложение}
\newtheorem*{notice}{Замечание}
\newtheorem{statement}{Утверждение}
\newtheorem*{consequence}{Следствие}
\newtheorem*{lemma}{Лемма}

\newcommand{\note}{\underline{Замечание:} }
\newcommand{\fact}{\underline{\textbf{Факт}:} }
\newcommand{\example}{\underline{Пример:} }
\newcommand{\sign}{\underline{Обозначения:} }
\newcommand{\statements}{\underline{Утверждения:} }

\renewcommand{\Re}{\mathrm{Re\:}}
\renewcommand{\Im}{\mathrm{Im\:}}
\newcommand{\Arg}{\mathrm{Arg\:}}
\renewcommand{\arg}{\mathrm{arg\:}}
\newcommand{\Mat}{\mathrm{Mat}}
\newcommand{\id}{\mathrm{id}}
\newcommand{\isom}{\xrightarrow{\sim}} 
\newcommand{\leftisom}{\xleftarrow{\sim}}
\newcommand{\Hom}{\mathrm{Hom}}
\newcommand{\Ker}{\mathrm{Ker}\:}
\newcommand{\rk}{\mathrm{rk}\:}
\newcommand{\diag}{\mathrm{diag}}
\newcommand{\ort}{\mathrm{ort}}
\newcommand{\pr}{\mathrm{pr}}
\newcommand{\vol}{\mathrm{vol\:}}

\newcommand{\Z}{\mathbb{Z}}
\newcommand{\N}{\mathbb{N}}
\newcommand{\Q}{\mathbb{Q}}
\newcommand{\R}{\mathbb{R}}
\renewcommand{\C}{\mathbb{C}}
\renewcommand{\L}{\mathscr{L}}
\renewcommand{\P}{\mathcal{P}}
\newcommand{\p}{\mathsf{p}}
\newcommand{\E}{\mathsf{E}}
\newcommand{\D}{\mathsf{D}}

\renewcommand{\G}{\mathsf{G}}
\renewcommand{\d}{\mathsf{d}}
\newcommand{\du}{\dot{u}}
\newcommand{\dx}{\dot{x}}
\newcommand{\ddx}{\ddot{x}}

\newcommand{\cov}{\mathsf{cov}}

\renewcommand{\l}{\mathcal{L}}
\renewcommand{\O}{\mathcal{O}}
\newcommand{\F}{\mathsf{F}}
\newcommand{\ds}{\displaystyle}
\renewcommand{\S}{\mathsf{S}}
 
%% Информация об авторах
\title{\Huge{Дифференциальные Уравнения \\ Семинарские занятия}}
\author{Вадим Гринберг \\ по семинарам Войнова А. С.}
\date{}

\begin{document}
\maketitle
\tableofcontents
\newpage

\section{Семинар 1, 10 января}

\subsection{Общие факты}
Пускай у нас имеется функция $x(t)$ (вообще говоря, вектор-функция $x = (x_1,\ \ldots,\ x_d)$) от переменной $t \in \R$, действующая из интервала $(a,\ b)$ (по умолчанию считаем всей числовой прямой), такая, что для переменной $t$, функции $x(t)$ и $n$ её первых производных выполнено уравнение:
\[\F\big(t,\ x(t),\ \dot{x}(t),\ \ldots,\ x^{(n)}(t)\big) = 0\] --- это и есть дифференциальное уравнение $n$-го порядка. $\F$ в данном случае, грубо говоря, <<функция от $n + 1$ переменной>>, которая неявно задаёт $x(t)$ (за точным определением --- на лекцию). 

\textbf{Решить диффур} означает найти такую функцию $x(t)$, что выполняется вышеуказанное равенство.

Тупой пример: $\dx(t) = x(t)$. Функция совпадает со своей производной. Решением, очевидно, будет $x(t) = \lambda \cdot e^t,\ \lambda \in \R$.

Любой диффур можно привести к удобоваримому виду: \[\dx(t) = f(t,\ x)\] где $f$ --- некая хорошая функция (доказательство на лекции). С такими диффурами мы в основном и будем иметь дело.

Разберёмся, а как вообще можно решать диффуры. Пускай у нас имеется диффур $\dx = f(t,\ x)$, который мы хотим решить. Попробуем приблизить график нашей кривой $x(t)$ некоей ломаной линией. Возьмём какую-то начальную точку $(t_0,\ x_0)$, и будем смотреть на направление движения, то бишь на направление вектора $(\d t,\ \d x)$. Будем делать маленькие шаги вдоль этого направления. Тогда каждый раз, находясь в точке $(t,\ x)$, мы будем переходить в точку $(t + \d t,\ x + \d x)$. 

После многих таких шагов мы получим ломаную линию, приближающую график нашей кривой $x(t)$. Эта ломаная называется \textbf{Ломаной Эйлера}.

\begin{center}
	\begin{tikzpicture}
	
	%\draw (0, -2) node[draw,circle,fill=white,minimum size=6pt,inner sep=0pt] (0) [label=below:$0$] {$A$};
	%\draw (0, 2) node[draw,circle,fill=white,minimum size=6pt,inner sep=0pt] (pi) [label=above:$\pi$] {};
	%\draw (1.4, 1.4) node[draw,circle,fill=white,minimum size=6pt,inner sep=0pt] (B) [label=45:$b$] {$B$};
	%\draw (-2, 0) node[draw,circle,fill=white,minimum size=6pt,inner sep=0pt] (C) [label=180:$c$] {$C$};
	
	
	\draw (5, 0) node[draw,circle,fill=black,minimum size=6pt,inner sep=2pt] (C) {};
	
	\draw[step=1cm,gray,very thin] (4,-2) grid (8,2);
	\draw[very thick,->] (4,-2) -- (8,-2) node[anchor=south] {$t$};
	\draw[very thick,->] (4,-2) -- (4,2) node[anchor=south] {$x$};
	
	\draw[dashed, very thick, ->] (5, 0) -- (7, 1);
	\node [above] at (7, 1) {$(\d t,\ \d x)$};
	%\draw[very thick] (4, 0) -- (8, 0);
	%\draw[very thick] (6, -2) -- (8, 0);
	%\draw[very thick] (6, -2) -- (6, 2);
	
	%\fill [pattern = crosshatch] (4, 0) -- (6, 2) -- (6, 0) -- cycle;
	%\fill [pattern = crosshatch] (6, -2) -- (8, 0) -- (6, 0) -- cycle;
	
	\node [left] at (4, 0) {$x_0$};
	\node [below] at (5, -2) {$t_0$};
	\node [left] at (4, -2) {$0$};
	
	\draw[very thick, ->] (12, -1) -- (13, -1);
	\draw[very thick, ->] (13, -1) -- (14, 0);
	\draw[very thick, ->] (14, 0) -- (15, 4);
	
	\draw[step=1cm,gray,very thin] (12,-2) grid (16,4);
	\draw[very thick,->] (12,-2) -- (16,-2) node[anchor=north] {$t$};
	\draw[very thick,->] (12,-2) -- (12, 4) node[anchor=east] {$x$};
	
	\node [left] at (12, -1) {$1$};
	\node [below] at (13, -2) {$1$};
	\node [left] at (12, 0) {$2$};
	\node [left] at (12, 1) {$3$};
	\node [left] at (12, 2) {$4$};
	\node [left] at (12, 3) {$5$};
	\node [below] at (14, -2) {$2$};
	\node [below] at (15, -2) {$3$};
	\node [left] at (12, -2) {$0$};
	
	
	\end{tikzpicture}
\end{center}

Для удобства можно делать шаг $\d t$ всегда равным 1, поделив вектор направления на $\d t$. Тогда соответственно шаг $\d x$ станет $\dfrac{\d x}{\d t} = \dx = f(t, x)$, и вектор направления в точке $(t,\ x)$ будет иметь вид $\big(1,\ f(t,\ x)\big)$.

Пример: $\dx = tx$. Построим Ломаную Эйлера, стартуя из точки $(t_0,\ x_0) = (0,\ 1)$:
\begin{enumerate}
	\item $t = 0,\ x = 1 \Rightarrow \dx = 0 \cdot 1 = 0 \Rightarrow \big(1,\ f(t,\ x)\big) = (1,\ 0) \Rightarrow (t + \d t,\ x + \d x) = (1,\ 1)$
	\item $t = 1,\ x = 1 \Rightarrow \dx = 1 \cdot 1 = 1 \Rightarrow \big(1,\ f(t,\ x)\big) = (1,\ 1) \Rightarrow (t + \d t,\ x + \d x) = (2,\ 2)$
	\item $t = 2,\ x = 2 \Rightarrow \dx = 2 \cdot 2 = 4 \Rightarrow \big(1,\ f(t,\ x)\big) = (1,\ 4) \Rightarrow (t + \d t,\ x + \d x) = (3,\ 6)$
	\item \dotfill
\end{enumerate}

\begin{definition}
	Пусть у нас есть диффур $\dx = f(t,\ x)$.
	
	\textbf{Изоклина} -- геометрическое место точек плоскости, в которых одно и то же направление движения, то есть, угол наклона вектора $(\d t,\ \d x)$ один и тот же для любой точки $(t,\ x)$ изоклины. Стоит отметить, что это определение работает только для диффуров 1 порядка.
	
	\textbf{Изолиния поля} -- подмножество точек изоклины (являющееся линией), в которых вектор $(\d t,\ \d x)$ один и тот же для любой точки  $(t,\ x)$ изолинии. То есть, вектор $(\d t, \d x) \sim \big(1, f(t,\ x)\big) = const$, откуда тут же следует $f(t,\ x) = const$. Для каждой изолинии константа своя.
\end{definition}

Для примера выше изоклиной будет являться множество $\left\{xt = k \iff x = \dfrac{k}{t},\ k \in \R\right\}$ --- гиперболы.

\subsection{Диффуры с разделяющимися переменными}

Это суть дифференциальные уравнения вида:
\[\dx = \dfrac{\d x}{\d t} = \dfrac{a(t)}{b(x)}\]
Перемножим крест-накрест и получим:
\begin{gather*}
	b(x)\d x = a(t)\d t \\
	\ds\int \text{ --- теперь интегрируем каждую часть независимо от другой} \ds\int\\
	B(x) = A(t) + C \text{ --- это и будет решением диффура}
\end{gather*}

\textbf{Пример №1}
\begin{gather*}
	\dx = tx \\
	\dx= tx = \dfrac{\d x}{\d t} \Longrightarrow \dfrac{\d x}{x} = t \cdot \d t \Longrightarrow \ds \int\dfrac{\d x}{x}=\int t \cdot \d t \\
	\ln |x| = \dfrac{t^2}{2} + C \Longrightarrow |x| = e^{\frac{t^2}{2}} \cdot \underbrace{e^C}_{\text{какая-то константа}} \Longrightarrow |x| = \lambda \cdot e^{\frac{t^2}{2}},\ \lambda > 0 \Longrightarrow x = \lambda \cdot e^{\frac{t^2}{2}},\ \lambda \in \R
\end{gather*}
В последних двух действиях мы взяли экспоненту от обеих частей и избавились от модуля.

\textbf{Пример №2}

Найдите кривую $x(t)$, такую, что для любой $t_0 \in \R$ отрезки, соединяющую точку касания $\big(t_0,\ x(t_0)\big)$ с точками пересечения касательной в данной точке с осями координат, будут равны.

\begin{center}
	\begin{tikzpicture}
	
	
	
	\draw (5.5, -0.5) node[draw,circle,fill=black,minimum size=5pt,inner sep=1pt][label=45:$M$] (M) {};
	\draw (4, -2) node[draw,circle,fill=black,minimum size=0pt,inner sep=0pt] (o) {};
	\draw (7, -2) node[draw,circle,fill=black,minimum size=0pt,inner sep=0pt] (p) [label=-90:$P$] {};
	
	\draw[step=1cm,gray,very thin] (4,-2) grid (8,2);
	\draw[very thick,->] (4,-2) -- (8,-2) node[anchor=south] (t) {$t$};
	\draw[very thick,->] (4,-2) -- (4,2) node[anchor=south](x) {$x$};
	
	\draw[very thick] (7, -2) -- (4, 1);
	\draw[very thick] (4, -2) -- (5.5, -0.5);
	
	\draw[very thick, dashed] (5.5, -2) -- (M);
	\draw[very thick, dashed] (4, -0.5) -- (M);
	
	\node [left] at (4, -0.5) {$x_0$};
	\node [below] at (5.5, -2) (t0) {$t_0$};
	\node [left] at (4, -2) (O) {$O$};
	
	\pic[draw=red, <->, angle eccentricity=1, angle radius=1.1cm]	{angle=M--p--o};
	
	\draw[very thick, dashed, color=violet] (M) to [out=120,in=-80] (4.5, 2);
	\draw[very thick, dashed, color=violet] (M) to [out=-30,in=180] (8, -1.5);
	
	\end{tikzpicture}
\end{center}

Пусть мы касаемся нашей кривой $x(t)$ в точке $(t_0,\ x_0)$ -- обозначим её $M$. Можно заметить, что тогда $OM$ -- медиана. Отсюда следует, что координаты точек пересечения с осями абсцисс и ординат равны соответственно $(2t_0,\ 0)$ и $(0,\ 2x_0)$. Тогда тангенс угла наклона касательной $\tan \angle MPO = -\dfrac{2x_0}{2t_0} = -\dfrac{x_0}{t_0} = \dx(t_0)$, так как тангенс угла наклона касательной к функции $x(t)$ в точке $t_0$ есть не что иное, как производная $x(t)$ -- $\dx(t)$ -- в данной точке. Таким образом, мы получили диффур:
\[\dx = -\dfrac{x}{t}\] Решим его, тем самым найдя $x(t)$.
\begin{gather*}
	\dx = -\dfrac{x}{t} = \dfrac{\d x}{\d t} \Longrightarrow -\dfrac{\d x}{x} = \dfrac{\d t}{t} \Longrightarrow \ds\int = \ds\int \\
	 -\ln |x| = \ln |t| + C \Longrightarrow \dfrac{1}{|x|} = |t| \cdot \lambda,\ \lambda > 0 \Longrightarrow x = \dfrac{\lambda}{t},\ \lambda \in \R
\end{gather*}

\textbf{Пример №3}

\begin{gather*}
	xt + (t + 1) \cdot \dx = 0\\
	xt + (t + 1) \cdot \dx = 0 \Longrightarrow xt + (t + 1) \cdot \dfrac{\d x}{\d t} = 0 \Longrightarrow \dfrac{\d x}{\d t}  = -\dfrac{xt}{t + 1} \Longrightarrow -\dfrac{\d x}{x} = \dfrac{t \cdot \d t}{t + 1} \Longrightarrow \ds\int = \int
\end{gather*}
Возьмём правый интеграл.
\begin{gather*}
	\ds\int \dfrac{t\cdot \d t}{t + 1} = \ds\int 1 - \dfrac{1}{t + 1}\ \d t = t - \ln |t + 1|
\end{gather*}
Тогда:
\begin{gather*}
	 -\ln|x| = t - \ln |t + 1| + C \Longrightarrow \dfrac{1}{|x|} = \lambda \cdot e^{\frac{t}{t + 1}},\ \lambda > 0 \Longrightarrow x = \lambda \cdot e^{-\frac{t}{t + 1}},\ \lambda \in \R
\end{gather*}

\subsection{$n$-параметрическое семейство кривых}

Это система дифференциальных уравнений вида:
\[\left\{\begin{gathered}
	\F\big(t, x(t), c_1,\ \ldots,\ c_n\big) = 0\hfill\\
	\F'\big(t, x(t), c_1,\ \ldots,\ c_n\big) = 0\hfill\\
	\dotfill\\
	\F^{(n)}\big(t, x(t), c_1,\ \ldots,\ c_n\big) = 0\hfill
\end{gathered}\right.\] -- всего $n + 1$ уравнение, константы $c_1,\ \ldots,\ c_n$ неизвестны. Необходимо, как и раньше, найти подходящую $x(t)$.

Метод решения таков: сначала мы выражаем константы $c_1,\ \ldots,\ c_n$ через $t,\ x(t),\ \dx(t), \ldots,\ x^{(n)}(t)$, и потом подставляем всё в одно уравнение, тем самым получая диффур вида:
\[\G\big(t,\ x(t),\ \dx(t),\ \ldots,\ x^{(n)}(t)\big) = 0\] который мы умеем решать.

\textbf{Пример №4}

Необходимо найти диффур, задающий множество окружностей, касающихся оси абсцисс.

Чего делать, сходу и не вдуплишь, да?) Однако, выход есть --- если видим слово "окружность", нужно тут же писать её уравнение. 

Пускай у нас есть окружность радиуса $R$, касающаяся оси абсцисс в точке $t_0$. Тогда выполнено тождество:
\[(x - R)^2 + (t - t_0)^2 = R^2\]
В данном случае $R$ и $t_0$ и есть наши неизвестные константы. Составим систему уравнений из производных:
\[\left\{\begin{gathered}
(x - R)^2 + (t - t_0)^2 - R^2 = 0\hfill\\
(2x\cdot \dx - 2R \cdot \dx) + 2t - 2t_0 = 0\hfill\\
2(\dx)^2 + 2x\cdot \ddx - 2R \cdot \ddx + 2 = 0\hfill
\end{gathered}\right.\]

Осталось выразить $R$ через $\dx$ и $\ddx$ из последнего уравнения, подставить во второе и выразить $t_0$, после чего загнать всё в первое уравнение и получить нужный диффур. 

\subsection{Замена переменных}

Разберём на примере. Пускай у нас есть диффур \[\dot{x} = x - \sqrt{x}\] Решать его в таком виде не очень приятно. Поэтому сделаем замену переменных (название -- сущая формальность, так как вообще говоря мы заменяем одну функцию на другую, а не переменную): \[y(t) = \sqrt{x(t)}\] Тогда диффур примет вид:
\[2\dot{y}\cdot y = y^2 - y \Longrightarrow 2\dot{y} = y - 1 \Longrightarrow \dfrac{2\d y}{y - 1} = \d t\] --- получили простое уравнение с разделяющими переменными.

Рассмотрим ещё несколько примеров замен.

\subsubsection{Линейная замена}

Пускай у нас есть диффур вида:
\[\dx = f(at + bx)\]
Можно сделать замену $u = at + bx$, получив уравнение $\dx = f(u)$. Решим этот диффур относительно переменной $u$, получив функцию $x(u)$, после чего, сделав обратную замену, выразить искомую $x(t)$.
\begin{gather*}
	u = at + bx\\
	\d u = a\cdot \d t + b \cdot \d x \Longrightarrow \d t = \dfrac{\d u - b\cdot \d x}{a}\\
	\dx = \dfrac{\d x}{\d t} = \dfrac{a \cdot \d x}{\d u - b \cdot \d x} = f(u)\\
	a \cdot \d x = f(u)\d u - b\cdot f(u)\d x \Longrightarrow \big(a + b\cdot f(u)\big)\d x = f(u)\d u\\
	\d x = \dfrac{f(u)}{a + b \cdot f(u)} \d u
\end{gather*}
После этих махинаций всё легко решается как уравнение с разделяющими переменными.

\textbf{Пример №5}
\[\dx \cos (x - t)\]
Ну тут совсем толсто: $u = x - t$. В данном случае $a = -1,\ b = 1$. По формуле выше:
\[\d x = \dfrac{\cos u}{\cos u - 1}\d u\]
Теперь интегрируем, получаем $x(u)$ и делаем обратную замену.

\subsubsection{Общий вид}

Пускай у нас есть диффур вида:
\[\dx = f\big(\varphi(t,\ x)\big)\]
Можно сделать замену $u = \varphi(t,\ x)$, получив новое уравнение. Решаем его и делаем обратную замену, получая $x(t)$.

\textbf{Пример №6}

\[\dx \cdot t = 2x^2 \cdot t^3 - x\]

Здесь можно сделать замену $u = xt$, откуда $\d u = \dx \cdot t + x \cdot 1$. Подставим:
\begin{gather*}
	\dx \cdot t = 2x^2 \cdot t^3 - x \iff \dx \cdot t + x = 2x^2 \cdot t^3 \Longrightarrow \du = 2u^2 \cdot t \\
	\dfrac{\d u}{\d t} = 2u^2 \cdot t \Longrightarrow \dfrac{\d u}{u^2} = 2t \cdot \d t \Longrightarrow \ds\int = \int\\
	-\dfrac{1}{u} = t^2 + C \Longrightarrow u = -\dfrac{1}{t^2 + C}
\end{gather*}
Делаем обратную замену и выражаем $x(t)$:
\begin{gather*}
	u = xt \Longrightarrow xt = -\dfrac{1}{t^2 + C} \Longrightarrow x = -\dfrac{1}{t^3 + Ct}
\end{gather*}

\subsection{Домашнее задание №1}

\begin{task1}
	Найти все кривые $x(t)$, такие, что длина отрезка, соединяющего точку касания и точку пересечения касательной в данной точке с одной из осей, была постоянной.
\end{task1}
\begin{proof}[Подсказки к решению]
	В зависимости от того, какую ось вы выберете, получится либо $x(t)$, либо $t(x)$, оба варианта правильные.
	
	Изобразите ситуацию на графике. Затем вспомните, что $\dot{x} = \dfrac{\d x}{\d t} = \tan \alpha$, где $\alpha$ -- угол наклона касательной. Также вам понадобится Теорема Пифагора.
\end{proof}

\begin{task2}
	Придумать диффур 1 порядка, не обладающий решением на всей прямой. То бишь, не для всех $t$ решение $\dot{x} = f(t,\ x)$ должно существовать. 
\end{task2}
\begin{proof}[Подсказки к решению]
	Подумайте о не всюду определённых функциях.
\end{proof}

\begin{task3}
	Решите диффур: 	\[(t^2 - 1) \cdot \dot{x} + 2tx^2 = 0, \text{начальное условие: }x(0) = 1\]
\end{task3}
%\begin{proof}[Подсказки к решению]
%	
%\end{proof}

\begin{task4}
	Изоклинами найти приближённое решение: 	\[\dot{x} = \dfrac{x}{t + x}\]
	Также изобразите изоклины на графике и покажите все различные (с точностью до топологии и асимптотики) решения.
\end{task4}
%\begin{proof}[Подсказки к решению]
%	
%\end{proof}

\begin{task5}
	Придумайте (вообще говоря, найдите) диффур 1 порядка, задающий множество прямых, являющихся касательными к единичной окружности с центром в нуле.
\end{task5}
\begin{proof}[Подсказки к решению]
	Пускай вы касаетесь в точке $(t_0, x_0)$. Однако, координата $x_0$ зависит от $t_0$. Используйте уравнение окружности, чтобы ликвидировать эту зависимость. Ну а дальше придётся малость подумать и чутка посчитать.
\end{proof}


\newpage
\section{Common Tasks}

\begin{enumerate}
	\item Найти такую кривую $x(t)$, что для любой $t_0 \in \R$ касательная к $x(t)$ в точке $(t_0,\ x(t_0))$ пересекает ось абсцисс в точке $\dfrac{t_0}{2}$.
	\item Найти диффур 1 порядка, задающий на плоскости параболы, проходящие через точку $(0,\ 1)$ и касающиеся прямой $x = t$.
\end{enumerate}

\end{document}