\documentclass[a4paper,12pt]{article}
 
%% Начало шапки
 
%% Настройка поддержки русского языка
\usepackage{cmap}                   % Поиск по кириллице
\usepackage{mathtext}               % Кириллица в формулах
\usepackage[T1,T2A]{fontenc}        % Кодировки шрифтов
\usepackage[utf8]{inputenc}         % Кодировка текста
\usepackage[english,russian]{babel} % Подключение поддержки языков
 
%% Настройка размеров полей
\usepackage[top=0.7in, bottom=0.75in, left=0.625in, right=0.625in]{geometry}
 
%% Математические пакеты
\usepackage{mathtools}              % Тот же amsmath, только с некоторыми поправками
\usepackage{amssymb}                % Математические символы
\usepackage{amsthm}                 % Оформление теорем
\usepackage{amstext}                % Текстовые вставки в формулы
\usepackage{amsfonts}               % Математические шрифты
\usepackage{icomma}                 % "Умная" запятая: $0,2$ --- число, $0, 2$ --- перечисление
\usepackage{enumitem}               % Для выравнивания itemize (\begin{itemize}[align=left])
\usepackage{array}                  % Таблицы и матрицы
\usepackage{multirow}
 
%% Алгоритмические пакеты и их настройки
\usepackage{algorithm}              % Шапка алгоритма
\usepackage{algorithmicx}           % Написание алгоритмов
\usepackage[noend]{algpseudocode}   % Написание псевдокода; убраны end
\usepackage{listings}               % Для кода на каком-либо языке программиования
\renewcommand{\algorithmicrequire}{\textbf{Ввод:}}              % Ввод
\renewcommand{\algorithmicensure}{\textbf{Вывод:}}              % Вывод
\floatname{algorithm}{Алгоритм}                                 % Название алгоритма
\renewcommand{\algorithmiccomment}[1]{\hspace*{\fill}\{// #1\}} % Комментарии
\newcommand{\algname}[1]{\textsc{#1}}                           % Вызов функции в алгоритме
\usepackage{physics}
 
%% Шрифты
\usepackage{euscript}               % Шрифт Евклид
\usepackage{mathrsfs}               % \mathscr{}
 
%% Графика
\usepackage[pdftex]{graphicx}       % Вставка картинок
\graphicspath{{images/}}            % Стандартный путь к картинкам
\usepackage{tikz}  
\usetikzlibrary{patterns}                 % Рисование всего
\usepackage{pgfplots}               % Графики
 
%% Прочие пакеты
\usepackage{indentfirst}                    % Красная строка в начале текста
\usepackage{epigraph}                       % Эпиграфы
\usepackage{fancybox,fancyhdr}              % Колонтитулы
\usepackage[colorlinks=true, urlcolor=blue]{hyperref}   % Ссылки
\usepackage{titlesec}                       % Изменение формата заголовков
\usepackage[normalem]{ulem}                 % Для зачёркиваний
\usepackage[makeroom]{cancel}               % И снова зачёркивание (на этот раз косое)

 
%% Прочее
\mathtoolsset{showonlyrefs=true}        % Показывать номера только у тех формул,
                                        % на которые есть \eqref{} в тексте.
\renewcommand{\headrulewidth}{1.8pt}    % Изменяем размер верхнего отступа колонтитула
\renewcommand{\footrulewidth}{0.0pt}    % Изменяем размер нижнего отступа колонтитула

%Прочее
\usepackage{forest} % Деревья

\usetikzlibrary{arrows,calc}
\usetikzlibrary{quotes,angles}

\usetikzlibrary{positioning,intersections}

\usetikzlibrary{through}

\NewDocumentCommand{\bissectrice}{%
	O{}     % drawing options
	mmm     % bissector of mmm
	m       % intersection point between base and bissector
	O{1}O{1}% extended drawing of the bissector
}{%
\path[name path=Bis#2#3#4] let
\p1 = ($(#2) - (#3)$),
\p2 = ($(#4) - (#3)$),
\n1 = {veclen(\x1,\y1)/2} ,
\n2 = {veclen(\x2,\y2)/2} ,
\n3 = {max(\n1,\n2)},
\p1 = ($(#3)!\n3!(#2)$),
\p2 = ($(#3)!\n3!(#4)$),
\p3 = ($(\p1) + (\p2) - (#3)$)
in
(#3) -- (\p3) ;

\path[name path = foo] (#2)--(#4) ;

\path[name intersections={of=foo and Bis#2#3#4, by=#5}] ;

\path[#1] ($(#3)!#6!(#5)$) -- ($(#5)!#7!(#3)$) ;
}


\tikzset{
	right angle quadrant/.code={
		\pgfmathsetmacro\quadranta{{1,1,-1,-1}[#1-1]}     % Arrays for selecting quadrant
		\pgfmathsetmacro\quadrantb{{1,-1,-1,1}[#1-1]}},
	right angle quadrant=1, % Make sure it is set, even if not called explicitly
	right angle length/.code={\def\rightanglelength{#1}},   % Length of symbol
	right angle length=2ex, % Make sure it is set...
	right angle symbol/.style n args={3}{
		insert path={
			let \p0 = ($(#1)!(#3)!(#2)$),     % Intersection
			\p1 = ($(\p0)!\quadranta*\rightanglelength!(#3)$), % Point on base line
			\p2 = ($(\p0)!\quadrantb*\rightanglelength!(#2)$), % Point on perpendicular line
			\p3 = ($(\p1)+(\p2)-(\p0)$) in  % Corner point of symbol
			(\p1) -- (\p3) -- (\p2)
		}
	}
}

%% Определения
\newtheorem{definition}{Определение}

\newtheorem*{task}{Задача}
\newtheorem*{task1}{Задача №1}
\newtheorem*{task2}{Задача №2}
\newtheorem*{task3}{Задача №3}
\newtheorem*{task4}{Задача №4}
\newtheorem*{task5}{Задача №5}

\newtheorem{fulllemma}{Лемма}
\newtheorem*{sl1}{Следствие 1}
\newtheorem*{sl2}{Следствие 2}
\newtheorem*{scheme}{Утверждение 1}
\newtheorem*{theorem}{Теорема}
\newtheorem*{proposal}{Предложение}
\newtheorem*{notice}{Замечание}
\newtheorem{statement}{Утверждение}
\newtheorem*{consequence}{Следствие}
\newtheorem*{lemma}{Лемма}

\newcommand{\note}{\underline{Замечание:} }
\newcommand{\fact}{\underline{\textbf{Факт}:} }
\newcommand{\example}{\underline{Пример:} }
\newcommand{\sign}{\underline{Обозначения:} }
\newcommand{\statements}{\underline{Утверждения:} }

\renewcommand{\Re}{\mathrm{Re\:}}
\renewcommand{\Im}{\mathrm{Im\:}}
\newcommand{\Arg}{\mathrm{Arg\:}}
\renewcommand{\arg}{\mathrm{arg\:}}
\newcommand{\Mat}{\mathrm{Mat}}
\newcommand{\id}{\mathrm{id}}
\newcommand{\isom}{\xrightarrow{\sim}} 
\newcommand{\leftisom}{\xleftarrow{\sim}}
\newcommand{\Hom}{\mathrm{Hom}}
\newcommand{\Ker}{\mathrm{Ker}\:}
\newcommand{\rk}{\mathrm{rk}\:}
\newcommand{\diag}{\mathrm{diag}}
\newcommand{\ort}{\mathrm{ort}}
\newcommand{\pr}{\mathrm{pr}}
\newcommand{\vol}{\mathrm{vol\:}}

\newcommand{\Z}{\mathbb{Z}}
\newcommand{\N}{\mathbb{N}}
\newcommand{\Q}{\mathbb{Q}}
\newcommand{\R}{\mathbb{R}}
\renewcommand{\C}{\mathbb{C}}
\renewcommand{\L}{\mathscr{L}}
\renewcommand{\P}{\mathcal{P}}
\newcommand{\p}{\mathsf{p}}
\newcommand{\E}{\mathsf{E}}
\newcommand{\D}{\mathsf{D}}

\renewcommand{\G}{\mathsf{G}}
\renewcommand*\d{\mathop{}\!\mathrm{d}}

\newcommand{\dy}{\dot{y}}
\newcommand{\ddy}{\ddot{y}}

\newcommand{\dt}{\dot{t}}
\newcommand{\ddt}{\ddot{t}}

\newcommand{\dw}{\dot{w}}
\newcommand{\ddw}{\ddot{w}}

\newcommand{\du}{\dot{u}}
\newcommand{\ddu}{\ddot{u}}

\renewcommand{\dv}{\dot{v}}
\newcommand{\ddv}{\ddot{v}}

\newcommand{\dx}{\dot{x}}
\newcommand{\ddx}{\ddot{x}}

\newcommand{\dz}{\dot{z}}
\newcommand{\ddz}{\ddot{z}}

\newcommand{\wx}{\widetilde{x}}
\newcommand{\wt}{\widetilde{t}}

\newcommand{\cov}{\mathsf{cov}}

\renewcommand{\l}{\mathcal{L}}
\renewcommand{\O}{\mathcal{O}}
\newcommand{\F}{\mathsf{F}}
\newcommand{\ds}{\displaystyle}
\renewcommand{\S}{\mathsf{S}}
 
%% Информация об авторах
\title{\Huge{Дифференциальные Уравнения \\ Семинарские занятия}}
\author{Вадим Гринберг \\ по семинарам Войнова А. С.}
\date{}

\begin{document}
\maketitle
\tableofcontents
\newpage

\section{Семинар 1, 10 января}

\subsection{Общие факты}
Пускай у нас имеется функция $x(t)$ (вообще говоря, вектор-функция $x = (x_1,\ \ldots,\ x_d)$) от переменной $t \in \R$, действующая из интервала $(a,\ b)$ (по умолчанию считаем всей числовой прямой), такая, что для переменной $t$, функции $x(t)$ и $n$ её первых производных выполнено уравнение:
\[\F\big(t,\ x(t),\ \dot{x}(t),\ \ldots,\ x^{(n)}(t)\big) = 0\] --- это и есть дифференциальное уравнение $n$-го порядка. $\F$ в данном случае, грубо говоря, <<функция от $n + 1$ переменной>>, которая неявно задаёт $x(t)$ (за точным определением --- на лекцию). 

\textbf{Решить диффур} означает найти такую функцию $x(t)$, что выполняется вышеуказанное равенство.

Тупой пример: $\dx(t) = x(t)$. Функция совпадает со своей производной. Решением, очевидно, будет $x(t) = \lambda \cdot e^t,\ \lambda \in \R$.

Любой диффур можно привести к удобоваримому виду: \[\dx(t) = f(t,\ x)\] где $f$ --- некая хорошая функция (доказательство на лекции). С такими диффурами мы в основном и будем иметь дело.

Разберёмся, а как вообще можно решать диффуры. Пускай у нас имеется диффур $\dx = f(t,\ x)$, который мы хотим решить. Попробуем приблизить график нашей кривой $x(t)$ некоей ломаной линией. Возьмём какую-то начальную точку $(t_0,\ x_0)$, и будем смотреть на направление движения, то бишь на направление вектора $(\d t,\ \d x)$. Будем делать маленькие шаги вдоль этого направления. Тогда каждый раз, находясь в точке $(t,\ x)$, мы будем переходить в точку $(t + \d t,\ x + \d x)$. 

После многих таких шагов мы получим ломаную линию, приближающую график нашей кривой $x(t)$. Эта ломаная называется \textbf{Ломаной Эйлера}.

\begin{center}
	\begin{tikzpicture}	
	
	\draw (5, 0) node[draw,circle,fill=black,minimum size=6pt,inner sep=2pt] (C) {};
	
	\draw[step=1cm,gray,very thin] (4,-2) grid (8,2);
	\draw[very thick,->] (4,-2) -- (8,-2) node[anchor=south] {$t$};
	\draw[very thick,->] (4,-2) -- (4,2) node[anchor=south] {$x$};
	
	\draw[dashed, very thick, ->] (5, 0) -- (7, 1);
	\node [above] at (7, 1) {$(\d t,\ \d x)$};
	
	\node [left] at (4, 0) {$x_0$};
	\node [below] at (5, -2) {$t_0$};
	\node [left] at (4, -2) {$0$};
	
	\draw[very thick, ->] (12, -1) -- (13, -1);
	\draw[very thick, ->] (13, -1) -- (14, 0);
	\draw[very thick, ->] (14, 0) -- (15, 4);
	
	\draw[step=1cm,gray,very thin] (12,-2) grid (16,4);
	\draw[very thick,->] (12,-2) -- (16,-2) node[anchor=north] {$t$};
	\draw[very thick,->] (12,-2) -- (12, 4) node[anchor=east] {$x$};
	
	\node [left] at (12, -1) {$1$};
	\node [below] at (13, -2) {$1$};
	\node [left] at (12, 0) {$2$};
	\node [left] at (12, 1) {$3$};
	\node [left] at (12, 2) {$4$};
	\node [left] at (12, 3) {$5$};
	\node [below] at (14, -2) {$2$};
	\node [below] at (15, -2) {$3$};
	\node [left] at (12, -2) {$0$};
	
	
	\end{tikzpicture}
\end{center}

Для удобства можно делать шаг $\d t$ всегда равным 1, поделив вектор направления на $\d t$. Тогда соответственно шаг $\d x$ станет $\dfrac{\d x}{\d t} = \dx = f(t, x)$, и вектор направления в точке $(t,\ x)$ будет иметь вид $\big(1,\ f(t,\ x)\big)$.

Пример: $\dx = tx$. Построим Ломаную Эйлера, стартуя из точки $(t_0,\ x_0) = (0,\ 1)$:
\begin{enumerate}
	\item $t = 0,\ x = 1 \Rightarrow \dx = 0 \cdot 1 = 0 \Rightarrow \big(1,\ f(t,\ x)\big) = (1,\ 0) \Rightarrow (t + \d t,\ x + \d x) = (1,\ 1)$
	\item $t = 1,\ x = 1 \Rightarrow \dx = 1 \cdot 1 = 1 \Rightarrow \big(1,\ f(t,\ x)\big) = (1,\ 1) \Rightarrow (t + \d t,\ x + \d x) = (2,\ 2)$
	\item $t = 2,\ x = 2 \Rightarrow \dx = 2 \cdot 2 = 4 \Rightarrow \big(1,\ f(t,\ x)\big) = (1,\ 4) \Rightarrow (t + \d t,\ x + \d x) = (3,\ 6)$
	\item \dotfill
\end{enumerate}

\subsection{Изоклины}

\begin{definition}
	Пусть у нас есть диффур $\dx = f(t,\ x)$.
	
	\textbf{Интегральная кривая} --- график функции $x(t)$ -- решения диффура. Тогда $\dx$ --- это угловой коэффициент интегральной кривой в точке $\big(t,\ x(t)\big)$, то бишь тангенс угла наклона касательной к $x(t)$ в данной точке.
	
	\textbf{Изоклина} -- геометрическое место точек плоскости, в которых одно и то же направление движения (направление касательных), то есть, угол наклона вектора $(\d t,\ \d x)$ один и тот же для любой точки $(t,\ x)$ изоклины. Иными словами, $\dx = const$.
	
	\textbf{Изолиния поля} -- подмножество точек изоклины (являющееся линией), в которых вектор $(\d t,\ \d x)$ один и тот же для любой точки  $(t,\ x)$ изолинии. То есть, вектор $(\d t, \d x) \sim \big(1, f(t,\ x)\big) = const$. Для каждой изолинии константа своя.
\end{definition}

Семейство изоклин определяется уравнением
\[\dx = k = f(t,\ x)\] где $k$ -- параметр. Придавая параметру $k$ близкие значения, получаем достаточно густую сеть изоклин, с помощью которых можно приближенно построить интегральные кривые дифференциального yравнения.

Для примера выше изоклинами будут являться множества $\left\{xt = k \iff x = \dfrac{k}{t},\ k \in \R\right\}$ --- гиперболы.

Научимся находить приближённые решения диффура, строя интегральную кривую при помощи изоклин. Стоит отметить сразу же, что \textbf{нулевая изоклина} $f(t,\ x) = 0$ даёт уравнение линий, на которых могут находиться точки максимума и минимума интегральных кривых.

Для большей точности построения интегральных кривых хорошо находить ГМТ точек перегиба, исследуя вторую производную $\ddx$ при помощи уравнения:
\[\ddx = \dfrac{\d f}{\d t} + \dfrac{\d f}{\d x}\cdot \dx = \dfrac{\d f}{\d t} + f(t,\ x) \cdot \dfrac{\d f}{\d x} = 0\]
Линия, определяемая данным уравнением, и есть возможное ГМТ точек перегиба.

\textbf{Пример №1}

Изоклинами найти приближённое решение диффура
\[\dx = 2t - x\]

Для получения изоклин положим $\dx = const = k$, откуда:
\[2t - x = k \iff x = 2t - k\] --- параллельные прямые.

Пусть $k = 0$, тогда получим изоклину $x = 2t$ --- эта прямая делит плоскость на две части, в каждой из которых производная $\dx$ имеет один и тот же знак --- интегральные кривые, пересекая $x = 2t$, из области убывания $x(t)$ переходят в область возрастания. Отсюда получаем, что на данной прямой лежат точки минимума.

Возьмём ещё две изоклины: $x = 2t + 1,\ k = -1$ и $x = 2t - 1,\ k = 1$. Изобразим их на графике. Касательные, проведённые к интегральным кривым в точках пересечения с изоклинами $k = -1,\ k = 0$ и $k = 1$ образуют с осью абсцисс углы в $135,\ 0$ и $45$ градусов соответственно. На графике направление показано чёрточками.

\begin{center}
	\begin{tikzpicture}
	
	
	
	\draw[very thick,->] (-8,0) -- (8,0) node[anchor=north] (t) {$t$};
	\draw[very thick,->] (0,-8) -- (0,8) node[anchor=east](x) {$x$};
	
	\draw[very thick] (4, 8) -- (-2, -4) node[anchor=east]  {$x = 2t$};
	\draw[very thick] (3, 8) -- (-2, -2) node[anchor=east]  {$x = 2t + 1$};
	\draw[very thick] (5, 8) -- (-2, -6) node[anchor=east]  {$x = 2t - 1$};
	\draw[very thick] (6.5, 9) -- (-2, -8) node[anchor=east]  {$x = 2t - 2$};
	
	\draw[very thick] (-1, -8) -- (7, 8) node[anchor=west]  {$x = 2t - 4$};
	
	\foreach \x in {-7,...,16}
		\draw [very thick] (6pt + 0.25*\x cm, 0.5*\x cm) -- (-6pt + 0.25*\x cm, 0.5*\x cm);
		
	\foreach \x in {-7,...,12}
		\draw [very thick] (5pt + 0.25*\x cm, 0.5*\x cm + 60 -6pt) -- (-4pt + 0.25*\x cm, 0.5*\x cm + 60 +6pt);
		
	\foreach \x in {-7,...,20}
		\draw [very thick] (4pt + 0.25*\x cm, 0.5*\x cm - 60 +3pt) -- (-7pt + 0.25*\x cm, 0.5*\x cm - 60 -3pt);
		
	\foreach \x in {-1,...,27}
	\draw [very thick] (-1pt + 0.25*\x cm, 0.5*\x cm - 180-4pt) -- (1pt+ 0.25*\x cm, 0.5*\x cm - 180 +20pt);
	
	\node [below right] at (0, 0) (O) {$0$};
	
	\draw[very thick, color=red] (-1, 6) to [out=-90,in=120] (0, 2.75);
	\draw[very thick, color=red] (0, 2.75) to [out=-60,in=170] (1, 2);
	\draw[very thick, color=red] (1, 2) to [out=0,in=-120] (5, 6);
	
	\draw[very thick, color=red] (-1, 6) to [out=-90,in=120] (0, 2.75);
	\draw[very thick, color=red] (0, 2.75) to [out=-60,in=170] (1, 2);
	\draw[very thick, color=red] (1, 2) to [out=0,in=-120] (5, 6);
	
	
	\draw[very thick, color=red] (0, 8) to [out=-90,in=120] (1, 4.75);
	\draw[very thick, color=red] (1, 4.75) to [out=-60,in=170] (2, 4);
	\draw[very thick, color=red] (2, 4) to [out=0,in=-120] (6, 8);
	
	\draw[very thick, color=red] (0, 8) to [out=-90,in=120] (1, 4.75);
	\draw[very thick, color=red] (1, 4.75) to [out=-60,in=170] (2, 4);
	\draw[very thick, color=red] (2, 4) to [out=0,in=-120] (6, 8);
	
	\draw[very thick, color=red] (-2, 4) to [out=-90,in=120] (-1, 0.75);
	\draw[very thick, color=red] (-1, 0.75) to [out=-60,in=170] (0, 0);
	\draw[very thick, color=red] (0, 0) to [out=0,in=-120] (4, 4);
	
	\draw[very thick, color=red] (-2, 4) to [out=-90,in=120] (-1, 0.75);
	\draw[very thick, color=red] (-1, 0.75) to [out=-60,in=170] (0, 0);
	\draw[very thick, color=red] (0, 0) to [out=0,in=-120] (4, 4);
	
	\draw[very thick, color=red] (-3, 2) to [out=-90,in=120] (-2, -1.25);
	\draw[very thick, color=red] (-2, -1.25) to [out=-60,in=170] (-1, -2);
	\draw[very thick, color=red] (-1, -2) to [out=0,in=-120] (3, 2);
	
	\draw[very thick, color=red] (-3, 2) to [out=-90,in=120] (-2, -1.25);
	\draw[very thick, color=red] (-2, -1.25) to [out=-60,in=170] (-1, -2);
	\draw[very thick, color=red] (-1, -2) to [out=0,in=-120] (3, 2);
	
	\draw[very thick, color=blue] (2, -2) to [out=75,in=-120] (5, 6);
	\draw[very thick, color=blue] (1.5, -8) to [out=87,in=-100] (2, -2);
	
	\draw[very thick, color=blue] (2, -2) to [out=75,in=-120] (5, 6);
	\draw[very thick, color=blue] (1.5, -8) to [out=87,in=-100] (2, -2);
	
	\draw[very thick, color=blue] (2, -2) to [out=75,in=-120] (5, 6);
	\draw[very thick, color=blue] (1.5, -8) to [out=87,in=-100] (2, -2);
	
	\draw[very thick, color=blue] (3, 0) to [out=75,in=-120] (6, 8);
	\draw[very thick, color=blue] (2.5, -6) to [out=87,in=-100] (3, 0);
	
	\draw[very thick, color=blue] (3, 0) to [out=75,in=-120] (6, 8);
	\draw[very thick, color=blue] (2.5, -6) to [out=87,in=-100] (3, 0);
	
	\draw[very thick, color=blue] (1.5, -3) to [out=75,in=-120] (4.5, 5);
	\draw[very thick, color=blue] (1, -9) to [out=87,in=-100] (1.5, -3);
	
	\draw[very thick, color=blue] (1.5, -3) to [out=75,in=-120] (4.5, 5);
	\draw[very thick, color=blue] (1, -9) to [out=87,in=-100] (1.5, -3);
	
	\draw[very thick, color=blue] (3.5, 1) to [out=75,in=-120] (6.5, 9);
	\draw[very thick, color=blue] (3, -5) to [out=87,in=-100] (3.5, 1);
	
	\draw[very thick, color=blue] (3.5, 1) to [out=75,in=-120] (6.5, 9);
	\draw[very thick, color=blue] (3, -5) to [out=87,in=-100] (3.5, 1);
	
	\end{tikzpicture}
\end{center}

Вторая производная: $\ddx = 2 - \dx = 2 - 2t + x$.

Рассмотрим прямую $x = 2t - 2$, на которой $\ddx = 0$. Это изоклина при $k = 2$. Заметим, что в таком случае угол наклона касательной равен углу наклона самой изоклины. Значит, ни одна интегральная кривая не будет пересекать эту изоклину, но при этом они будут к ней стремиться на бесконечности.

Прямая $x = 2t - 2$ делит плоскость на две части, в одной из которых (над прямой) $\ddx > 0$, а значит, интегральные кривые выпуклы вниз, а в другой $\ddx < 0$, и интегральные кривые выпуклы вверх. Кроме того, поскольку точки минимума расположены над этой прямой, то интегральные кривые, проходящие ниже изоклины $x = 2t - 2$ не имеют точек экстремума. 

Рассмотрим также изоклину $x = 2t - 4$, $ k = 4$. В данном случае угол наклона касательной будет равен 75 градусов. При этом интегральные кривые будут также стремиться к $x = 2t - 2$, но являясь выпуклыми вверх. Тем самым мы получили другое семейство решений диффура.

На графике выше изображены интегральные кривые, приближающие $x(t)$, полученные в соответствии с проведённым исследованием. Как видим, в точках пересечения с изоклинами кривые параллельны направлению касательных в точках пересечения.


\subsection{Диффуры с разделяющимися переменными}

Это суть дифференциальные уравнения вида:
\[\dx = \dfrac{\d x}{\d t} = \dfrac{a(t)}{b(x)}\]

В данном случае стоит быть осторожным и проверять вырожденные случаи ($b(x) = 0$, $a(t) = 0$, чтобы нечаянно не убить некоторые решения).

Проверив особые случаи, перемножим крест-накрест и получим:
\begin{gather*}
	b(x)\d x = a(t)\d t \\
	\ds\int \text{ теперь интегрируем каждую часть независимо от другой } \ds\int\\
	B(x) = A(t) + C \text{ --- это и будет решением диффура}
\end{gather*}

\textbf{Пример №2}
\begin{gather*}
	\dx = tx \\
	\dx= tx = \dfrac{\d x}{\d t} \Longrightarrow \dfrac{\d x}{x} = t \cdot \d t \Longrightarrow \ds \int\dfrac{\d x}{x}=\int t \cdot \d t \\
	\ln |x| = \dfrac{t^2}{2} + C \Longrightarrow |x| = e^{\frac{t^2}{2}} \cdot \underbrace{e^C}_{\text{какая-то константа}} \Longrightarrow |x| = \lambda \cdot e^{\frac{t^2}{2}},\ \lambda > 0 \Longrightarrow x = \lambda \cdot e^{\frac{t^2}{2}},\ \lambda \in \R
\end{gather*}
В последних двух действиях мы взяли экспоненту от обеих частей и избавились от модуля.

\textbf{Пример №3}

Найдите кривую $x(t)$, такую, что для любой $t_0 \in \R$ отрезки, соединяющую точку касания $\big(t_0,\ x(t_0)\big)$ с точками пересечения касательной в данной точке с осями координат, будут равны.

\begin{center}
	\begin{tikzpicture}
	
	
	
	\draw (5.5, -0.5) node[draw,circle,fill=black,minimum size=5pt,inner sep=1pt][label=45:$M$] (M) {};
	\draw (4, -2) node[draw,circle,fill=black,minimum size=0pt,inner sep=0pt] (o) {};
	\draw (7, -2) node[draw,circle,fill=black,minimum size=0pt,inner sep=0pt] (p) [label=-90:$P$] {};
	
	\draw[step=1cm,gray,very thin] (4,-2) grid (8,2);
	\draw[very thick,->] (4,-2) -- (8,-2) node[anchor=south] (t) {$t$};
	\draw[very thick,->] (4,-2) -- (4,2) node[anchor=south](x) {$x$};
	
	\draw[very thick] (7, -2) -- (4, 1);
	\draw[very thick] (4, -2) -- (5.5, -0.5);
	
	\draw[very thick, dashed] (5.5, -2) -- (M);
	\draw[very thick, dashed] (4, -0.5) -- (M);
	
	\node [left] at (4, -0.5) {$x_0$};
	\node [below] at (5.5, -2) (t0) {$t_0$};
	\node [left] at (4, -2) (O) {$O$};
	
	\pic[draw=red, <->, angle eccentricity=1, angle radius=1.1cm, very thick]	{angle=M--p--o};
	
	\draw[very thick, dashed, color=violet] (M) to [out=120,in=-80] (4.5, 2);
	\draw[very thick, dashed, color=violet] (M) to [out=-30,in=180] (8, -1.5);
	
	\end{tikzpicture}
\end{center}

Пусть мы касаемся нашей кривой $x(t)$ в точке $(t_0,\ x_0)$ -- обозначим её $M$. Можно заметить, что тогда $OM$ -- медиана. Отсюда следует, что координаты точек пересечения с осями абсцисс и ординат равны соответственно $(2t_0,\ 0)$ и $(0,\ 2x_0)$. Тогда тангенс угла наклона касательной $\tan \angle MPO = -\dfrac{2x_0}{2t_0} = -\dfrac{x_0}{t_0} = \dx(t_0)$, так как тангенс угла наклона касательной к функции $x(t)$ в точке $t_0$ есть не что иное, как производная $x(t)$ -- $\dx(t)$ -- в данной точке. Таким образом, мы получили диффур:
\[\dx = -\dfrac{x}{t}\] Решим его, тем самым найдя $x(t)$.
\begin{gather*}
	\dx = -\dfrac{x}{t} = \dfrac{\d x}{\d t} \Longrightarrow -\dfrac{\d x}{x} = \dfrac{\d t}{t} \Longrightarrow \ds\int = \ds\int \\
	 -\ln |x| = \ln |t| + C \Longrightarrow \dfrac{1}{|x|} = |t| \cdot \lambda,\ \lambda > 0 \Longrightarrow x = \dfrac{\lambda}{t},\ \lambda \in \R
\end{gather*}

\textbf{Пример №4}

\begin{gather*}
	xt + (t + 1) \cdot \dx = 0\\
	xt + (t + 1) \cdot \dx = 0 \Longrightarrow xt + (t + 1) \cdot \dfrac{\d x}{\d t} = 0 \Longrightarrow \dfrac{\d x}{\d t}  = -\dfrac{xt}{t + 1} \Longrightarrow -\dfrac{\d x}{x} = \dfrac{t \cdot \d t}{t + 1} \Longrightarrow \ds\int = \int
\end{gather*}
Возьмём правый интеграл.
\begin{gather*}
	\ds\int \dfrac{t\cdot \d t}{t + 1} = \ds\int 1 - \dfrac{1}{t + 1}\ \d t = t - \ln |t + 1|
\end{gather*}
Тогда:
\begin{gather*}
	 -\ln|x| = t - \ln |t + 1| + C \Longrightarrow \dfrac{1}{|x|} = \lambda \cdot \dfrac{e^{t}}{t + 1},\ \lambda > 0 \Longrightarrow x = \lambda \cdot e^{-t} \cdot (t + 1),\ \lambda \in \R
\end{gather*}

\subsection{$n$-параметрическое семейство кривых}

Это система дифференциальных уравнений вида:
\[\left\{\begin{gathered}
	\F\big(t, x(t), c_1,\ \ldots,\ c_n\big) = 0\hfill\\
	\F'\big(t, x(t), c_1,\ \ldots,\ c_n\big) = 0\hfill\\
	\dotfill\\
	\F^{(n)}\big(t, x(t), c_1,\ \ldots,\ c_n\big) = 0\hfill
\end{gathered}\right.\] -- всего $n + 1$ уравнение, константы $c_1,\ \ldots,\ c_n$ неизвестны. Необходимо, как и раньше, найти подходящую $x(t)$.

Метод решения таков: сначала мы выражаем константы $c_1,\ \ldots,\ c_n$ через $t,\ x(t),\ \dx(t), \ldots,\ x^{(n)}(t)$, и потом подставляем всё в одно уравнение, тем самым получая диффур вида:
\[\G\big(t,\ x(t),\ \dx(t),\ \ldots,\ x^{(n)}(t)\big) = 0\] который мы умеем решать.

\textbf{Пример №5}

Необходимо найти диффур, задающий множество окружностей, касающихся оси абсцисс.

Чего делать, сходу и не вдуплишь, да?) Однако, выход есть --- если видим слово "окружность", нужно тут же писать её уравнение. 

Пускай у нас есть окружность радиуса $R$, касающаяся оси абсцисс в точке $t_0$. Тогда выполнено тождество:
\[(x - R)^2 + (t - t_0)^2 = R^2\]
В данном случае $R$ и $t_0$ и есть наши неизвестные константы. Составим систему уравнений из производных:
\[\left\{\begin{gathered}
(x - R)^2 + (t - t_0)^2 - R^2 = 0\hfill\\
(2x\cdot \dx - 2R \cdot \dx) + 2t - 2t_0 = 0\hfill\\
2(\dx)^2 + 2x\cdot \ddx - 2R \cdot \ddx + 2 = 0\hfill
\end{gathered}\right.\]

Осталось выразить $R$ через $\dx$ и $\ddx$ из последнего уравнения, подставить во второе и выразить $t_0$, после чего загнать всё в первое уравнение и получить нужный диффур. 

\subsection{Замена переменных}

Разберём на примере. Пускай у нас есть диффур \[\dot{x} = x - \sqrt{x}\] Решать его в таком виде не очень приятно. Поэтому сделаем замену переменных (название -- сущая формальность, так как вообще говоря мы заменяем одну функцию на другую, а не переменную): \[y(t) = \sqrt{x(t)}\] Тогда диффур примет вид:
\[2\dot{y}\cdot y = y^2 - y \Longrightarrow 2\dot{y} = y - 1 \Longrightarrow \dfrac{2\d y}{y - 1} = \d t\] --- получили простое уравнение с разделяющими переменными.

Рассмотрим ещё несколько примеров замен.

\subsubsection{Линейная замена}

Пускай у нас есть диффур вида:
\[\dx = f(at + bx)\]
Можно сделать замену $u = at + bx$, получив уравнение $\dx = f(u)$. Решим этот диффур относительно переменной $u$, получив функцию $x(u)$, после чего, сделав обратную замену, выразить искомую $x(t)$.
\begin{gather*}
	u = at + bx                                                                                        \\
	\d u = a\cdot \d t + b \cdot \d x \Longrightarrow \d t = \dfrac{\d u - b\cdot \d x}{a}             \\
	\dx = \dfrac{\d x}{\d t} = \dfrac{a \cdot \d x}{\d u - b \cdot \d x} = f(u)                        \\
	a \cdot \d x = f(u)\d u - b\cdot f(u)\d x \Longrightarrow \big(a + b\cdot f(u)\big)\d x = f(u)\d u \\
	\d x = \dfrac{f(u)}{a + b \cdot f(u)} \d u
\end{gather*}
После этих махинаций всё легко решается как уравнение с разделяющими переменными.

\textbf{Пример №6}
\[\dx = \cos (x - t)\]
Ну тут совсем толсто: $u = x - t$. В данном случае $a = -1,\ b = 1$. По формуле выше:
\[\d x = \dfrac{\cos u}{\cos u - 1}\d u\]
Теперь интегрируем, получаем $x(u)$ и делаем обратную замену.

\subsubsection{Общий вид}

Пускай у нас есть диффур:
\[\dx = f(t,\ x)\]
Можно сделать замену $u = \varphi(t,\ x)$, получив новое уравнение (весьма удачно, если получится диффур вида $\du = f(t,\ u)$, но такое бывает далеко не всегда). Решаем его и делаем обратную замену, получая $x(t)$.

\textbf{Пример №7}

\[\dx \cdot t = 2x^2 \cdot t^3 - x\]

Здесь можно сделать замену $u = xt$, откуда $\du = \dx \cdot t + x \cdot 1$. Подставим:
\begin{gather*}
	\dx \cdot t = 2x^2 \cdot t^3 - x \iff \dx \cdot t + x = 2x^2 \cdot t^3 \Longrightarrow \du = 2u^2 \cdot t \\
	\dfrac{\d u}{\d t} = 2u^2 \cdot t \Longrightarrow \dfrac{\d u}{u^2} = 2t \cdot \d t \Longrightarrow \ds\int = \int\\
	-\dfrac{1}{u} = t^2 + C \Longrightarrow u = -\dfrac{1}{t^2 + C}
\end{gather*}
Делаем обратную замену и выражаем $x(t)$:
\begin{gather*}
	u = xt \Longrightarrow xt = -\dfrac{1}{t^2 + C} \Longrightarrow x = -\dfrac{1}{t^3 + Ct}
\end{gather*}

\subsection{Домашнее задание №1}

\begin{task1}
	Найти все кривые $x(t)$, такие, что длина отрезка, соединяющего точку касания и точку пересечения касательной в данной точке с одной из осей, была постоянной.
\end{task1}
\begin{proof}[Решение]
	\begin{center}
		\begin{tikzpicture}
		
		
		\draw (9.5, 1) node[draw,circle,fill=black,minimum size=5pt,inner sep=1pt][label=135:$M$] (M) {};
		\draw (4, -2) node[draw,circle,fill=black,minimum size=0pt,inner sep=0pt] (o) {};
		\draw (11, -2) node[draw,circle,fill=black,minimum size=0pt,inner sep=0pt] (t) {};
		\draw (4, -1) node[draw,circle,fill=black,minimum size=0pt,inner sep=0pt] [label=135:$P$] (c) {};
		\draw (5.87, 0.92) node[draw,circle,fill=black,minimum size=0pt,inner sep=0pt] (p) {};
		\draw (8, -1) node[draw,circle,fill=black,minimum size=0pt,inner sep=0pt] (t0) {};
		
		\draw[step=1cm,gray,very thin] (4,-2) grid (12,3);
		\draw[very thick,->] (4,-2) -- (12,-2) node[anchor=north] {$t$};
		\draw[very thick,->] (4,-2) -- (4,3) node[anchor=east](x) {$x$};
		
		\draw[very thick] (12, 2) -- (4, -1);
		
		
		\draw[very thick, dashed] (9.5, -2) -- (M);
		\draw[very thick, dashed] (4, 1) -- (M);
		\draw[very thick, dashed] (c) -- (12, -1);
				
		\node [left] at (4, 1) {$x_0$};
		\node [below] at (9.5, -2) {$t_0$};
		\node [left] at (4, -2) (O) {$O$};
		
		\node [above right] at (9.5, -1) {$Q$};
		
		\pic[draw=red, <->, angle eccentricity=2, angle radius=4cm, very thick]	{angle=t0--c--M};
		
		\draw[very thick, dashed, color=violet] (M) to [out=40,in=-90] (11.5, 3);
		\draw[very thick, dashed, color=violet] (M) to [out=190,in=0] (4, -0.35);
		
		\end{tikzpicture}
	\end{center}
	
	Пусть длина отрезка $MP = l$. Рассмотрим треугольник $MPQ$ -- на рисунке выше. Мы знаем, что $PQ = t_0$. Известно, что $\tan \angle MPQ = \dx = \dfrac{MQ}{PQ}$, откуда получаем, что 
	\[MQ = PQ\tan \angle MPQ = t_0\dx\]. По условию, равенство можно продлить на всю числовую прямую, получая:
	\[PQ = t \Longrightarrow MQ = t\dx\]
	Теперь применим Теорему Пифагора, чтобы получить дифференциальное уравнение на искомую кривую:
	\begin{gather*}
		MP^2 = PQ^2 + MQ^2\\
		l^2 = t^2 + t^2(\dx)^2 \Longrightarrow l^2 - t^2 = t^2(\dx)^2\\
		(\dx)^2 = \dfrac{l^2 - t^2}{t^2} \Longrightarrow \dx = \ds \pm \dfrac{\sqrt{l^2 - t^2}}{t}
	\end{gather*}
	Получили диффур (вообще говоря, два диффура). Будем рассматривать случай со знаком $+$, так как знак $-$ приведёт нас к почти аналогичному результату (это обговорится далее).
	
	Итак, имеем диффур с разделяющимися переменными:
	\begin{gather*}
		\dfrac{\d x}{\d t} = \dfrac{\sqrt{l^2 - t^2}}{t} \Longrightarrow \d x = \dfrac{\sqrt{l^2 - t^2}}{t}\d t \Longrightarrow \ds\int \d x = \ds \int \dfrac{\sqrt{l^2 - t^2}}{t} \d t
	\end{gather*}
	Возьмём правый интеграл, сделав замену переменных.
	\begin{gather*}
		\ds \int \dfrac{\sqrt{l^2 - t^2}}{t} \d t = \Big\{t = l \cdot \sin \beta,\ \d t = l\cdot \cos \beta \d \beta\Big\} = \ds\int \dfrac{\sqrt{l^2(1 - \sin^2 \beta)}}{l \cdot \sin \beta} \cdot l\cdot \cos \beta \d \beta =\\
		= \ds\int\dfrac{l \cdot \cos \beta \cdot \cos \beta}{\sin \beta}\d \beta = l \cdot \ds\int \dfrac{1 - \sin^2 \beta}{\sin \beta}\d \beta = l \left(\ds\int \dfrac{\d \beta}{\sin\beta} - \ds\int \sin\beta \d \beta\right) = \\
		=\Big\{\ds\int\dfrac{\d \beta}{\sin\beta} = \ds\int\dfrac{\d \beta}{2\sin(\frac{\beta}{2})\cos(\frac{\beta}{2})} = \ds\int\dfrac{\cos(\frac{\beta}{2})\d\beta}{2\sin(\frac{\beta}{2})\cos[2](\frac{\beta}{2})} = \ds\int\dfrac{\d \tan(\frac{\beta}{2})}{\tan(\frac{\beta}{2})} = \ln\left|\tan(\frac{\beta}{2})\right| + C\Big\} = \\
		l\cdot \ln\left|\tan(\frac{\beta}{2})\right| + l \cdot \cos(\beta) + C
	\end{gather*}
	Теперь сделаем обратную замену:
	\begin{gather*}
		t = l\cdot \sin\beta \Longrightarrow \sin\beta = \dfrac{t}{l} \Longrightarrow \cos\beta = \ds\pm\sqrt{1 - \dfrac{t^2}{l^2}}\\
		\ln\left|\tan(\frac{\beta}{2})\right| = \ln\left|\dfrac{\sin(\frac{\beta}{2})}{\cos(\frac{\beta}{2})}\right| = \ln\left|\dfrac{2\cdot \sin(\frac{\beta}{2})\cdot\cos(\frac{\beta}{2})}{2\cdot \cos[2](\frac{\beta}{2})}\right| = \ln\left|\dfrac{l\cdot \sin\beta}{\left(1 + \cos\beta\right)\cdot l}\right| = \ln\left|\dfrac{t}{l \pm \sqrt{l^2 - t^2}}\right|
	\end{gather*}
	Теперь запишем полный результат интегрирования обеих частей, помня, что $\ds\int \d x = x + C$:
	\[x = \ds\pm \sqrt{l^2 - t^2} + l\cdot \ln\left|\dfrac{t}{l \pm \sqrt{l^2 - t^2}}\right| + C\]
	У нас тут есть модуль, что не очень хорошо. Кроме того, мы не рассмотрели случай с минусом. Убьём двух зайцев одним ударом, немного преобразовав ответ:
	\begin{gather*}
		\ds\pm \sqrt{l^2 - t^2} + l\cdot \ln\left|\dfrac{t}{l \pm \sqrt{l^2 - t^2}}\right| = \ds\pm \sqrt{l^2 - t^2} + \dfrac{l}{2}\cdot \ln\left|\dfrac{t}{l \pm \sqrt{l^2 - t^2}}\right|^2 =\\
		= \ds\pm \sqrt{l^2 - t^2} + \dfrac{l}{2}\cdot \ln\left(\dfrac{t^2}{(l \pm \sqrt{l^2 - t^2})\cdot(l \pm \sqrt{l^2 - t^2})}\right) = \\
		= \ds\pm \sqrt{l^2 - t^2} + \dfrac{l}{2}\cdot \ln\left(\dfrac{t^2\cdot (l \mp \sqrt{l^2 - t^2})}{(l \pm \sqrt{l^2 - t^2})\cdot(l \pm \sqrt{l^2 - t^2})\cdot (l \mp \sqrt{l^2 - t^2})}\right) = \\
		= \ds\pm \sqrt{l^2 - t^2} + \dfrac{l}{2}\cdot \ln\left(\dfrac{t^2\cdot (l \mp \sqrt{l^2 - t^2})}{(l \pm \sqrt{l^2 - t^2})\cdot\big(l^2 - (l^2 - t^2)\big)}\right) = \\
		= \ds\pm \sqrt{l^2 - t^2} + \dfrac{l}{2}\cdot \ln\left(\dfrac{l \mp \sqrt{l^2 - t^2}}{(l \pm \sqrt{l^2 - t^2})}\right)
	\end{gather*}
	Заметим, что если бы мы взяли случай с минусом, то тогда перед слагаемым с логарифмом стоял бы знак минус. Тогда, домножив на $-\dfrac{1}{2}$, мы бы возводили подлогарифменное выражение не в 2 степень, а в -2, соответственно абсолютно аналогичными преобразованиями получив под логарифмом такую же, но перевёрнутую дробь. Однако, и в числителе, и в знаменателе, у нас возникает по $\pm$ или $\mp$ -- следовательно, рассмотрев случай с плюсом, мы уже получили все возможные варианты ответа.
	
	Итоговый полный ответ:
	\[x = \ds\pm \sqrt{l^2 - t^2} + \dfrac{l}{2}\cdot \ln\left(\dfrac{l \mp \sqrt{l^2 - t^2}}{(l \pm \sqrt{l^2 - t^2})}\right) + C\]
	
	Осталось лишь указать, что $t \in [-l,\ l]$, и $x(\pm l) = C$. 
\end{proof}

\begin{task2}
	Придумать диффур 1 порядка, не обладающий решением на всей прямой. То бишь, не для всех $t$ решение $\dot{x} = f(t,\ x)$ должно существовать. 
\end{task2}
\begin{proof}[Решение]
	$f(t, x)$  должна быть всюду определённой функцией. Поэтому достаточно взять такую $x = x(t)$ в качестве решения диффура, чтобы она не была всюду определена, при выполнении условия выше.
	
	Пример: $\dx = -x^2$. Проверим, что подходит:
	\begin{gather*}
		\dx = -x^2 \Longrightarrow \dfrac{\d x}{x^2} = -\d t \Longrightarrow \ds\int = \int\\
		\dfrac{1}{x} = t \Longrightarrow x = \dfrac{1}{t}
	\end{gather*}
\end{proof}

\begin{task3}
	Решите диффур: 	\[(t^2 - 1) \cdot \dot{x} + 2tx^2 = 0, \text{начальное условие: }x(0) = 1\]
\end{task3}
\begin{proof}[Решение]
	\begin{gather*}
		(t^2 - 1)\cdot \dx + 2tx^2 = 0 \Longrightarrow (t^2 - 1)\cdot \d x + 2tx^2\d t = 0\\
		\dfrac{\d x}{x^2} = \dfrac{2t \cdot \d t}{1 - t^2}\Longrightarrow \ds\int = \int\\
		\dfrac{1}{x} = \ln|1-t^2| + C
	\end{gather*}
	Найдём константу: \[x(0) = 1 \Longrightarrow 1 = \ln 1 + C \Longrightarrow C = 1\]
	Итоговая кривая:
	\[\dfrac{1}{x} = \ln|1 - t^2| + 1 \Longrightarrow x = \dfrac{1}{\ln|1 - t^2| + 1}\]
\end{proof}

\begin{task4}
	Изоклинами найти приближённое решение: 	\[\dot{x} = \dfrac{x}{t + x}\]
	Также изобразите изоклины на графике и покажите все различные (с точностью до топологии и асимптотики) решения (то есть, как рассмотрено выше в примере).
\end{task4}
\begin{proof}[Решение]
	Уравнения изоклин:
	\[\dx = \dfrac{x}{t + x} = k \Longrightarrow x = t \cdot \dfrac{k}{1 - k}\] Получим прямые изменения характера роста и выпуклости, исследовав первые две производные:
	\begin{gather*}
		\dx = 0 = \dfrac{x}{t + x} \Longrightarrow x = 0\\
		\ddx = \dfrac{\dx(t + x) - x(1 + \dx)}{(t + x)^2} = \dfrac{\dx t - x}{(t + x)^2} = 0 \Longrightarrow \dfrac{\frac{x}{t + x} \cdot t - x}{(t + x)^2} = 0 \Longrightarrow \\
		\Longrightarrow \dfrac{-x^2}{(t + x)^3} = 0 \Longrightarrow x = 0\\
		\ddx > 0 \Longrightarrow t + x < 0 \Longrightarrow x < -t\\
		\ddx < 0 \Longrightarrow t + x > 0 \Longrightarrow x > -t
	\end{gather*}
	Таким образом, характер роста функции меняется в нуле, а выпуклость изменяется, проходя через прямые $x = 0,\ x = -t$. На основании этого и нескольких изоклин можно построить приблизительный график кривой $x(t)$.
\end{proof}

\begin{task5}
	Придумайте (вообще говоря, найдите) диффур 1 порядка, задающий множество прямых, являющихся касательными к единичной окружности с центром в нуле.
\end{task5}
\begin{proof}[Решение]
	Уравнение касательной к окружности в точке $(t_0,\ x_0)$ --- $x \cdot x_0 + t \cdot t_0 = 1$. Таким образом, у нас есть два уравнения:
	\[\left\{\begin{gathered}
		x_0^2 + t_0^2 = 1\hfill\\
		x \cdot x_0 + t \cdot t_0 = 1\hfill
	\end{gathered}\right.\]
	Будем выражать из них $t_0$ и $x_0$, чтобы остались только переменные $t$ и $x$. Для этого продифференцируем второе уравнение и попреобразуем:
	\begin{gather*}
		x \cdot x_0 + t \cdot t_0 = 1 \Longrightarrow \dx x_0 + t_0 = 0\\
		\dx x_0 = -t_0 \text{ --- здесь возведём в квадрат, запоминая знак минус --- } (\dx)^2x_0^2 = t_0^2
	\end{gather*}
	Из уравнения окружности мы  получаем выражение на квадрат $t_0$:
	\[x_0^2 + t_0^2 = 1 \Longrightarrow t_0^2 = 1 - x_0^2\] --- теперь подставим $t_0^2$ в уравнение выше:
	\begin{gather*}
		(\dx)^2x_0^2 = 1 - x_0^2 \Longrightarrow x_0^2\big(1 + (\dx)^2\big) = 1\\
		x_0^2 = \dfrac{1}{1 + (\dx)^2} \Longrightarrow t_0^2 = 1 - x_0^2 = \dfrac{(\dx)^2}{1 + (\dx)^2}\\
		x_0 = \ds\pm\dfrac{1}{\sqrt{1 + (\dx)^2}},\ \ \ t_0 = \ds\pm\dfrac{\dx}{\sqrt{1 + (\dx)^2}}
	\end{gather*}
	Осталось подставить выраженные константы в уравнение касательной и получить некоторыми преобразованиями искомый диффур:
	\begin{gather*}
		\ds\pm\dfrac{x}{\sqrt{1 + (\dx)^2}} \ds\pm\dfrac{\dx t}{\sqrt{1 + (\dx)^2}} = 1\\
		\ds\pm x \ds\pm \dx t = \sqrt{1 + (\dx)^2}\\
		x^2 \ds\pm 2\dx x t + (\dx)^2t^2 = 1 + (\dx)^2\\
		(\dx)^2(t^2 - 1) \ds\pm (2xt)\dx + (x^2 - 1) = 0 \text{ --- квадратное уравнение }\\
		\dx = \dfrac{\pm 2xt \pm \sqrt{4x^2t^2 - 4(t^2 - 1)(x^2 - 1)}}{2(t^2 - 1)}\\
		\dx = \dfrac{\pm xt \pm \sqrt{x^2 + t^2 - 1}}{t^2 - 1}
	\end{gather*}
\end{proof}


\newpage
\section{Семинар 2, 17 января}

\subsection{Специальные замены. Однородные уравнения.}

\begin{definition}
	Функцию $g(t,\ x)$ назовём \textbf{однородной}, если
	\[\forall\ \lambda \in \R:\ g(\lambda t,\ \lambda x) = \lambda \cdot g(t,\ x)\]
	
	Функции $M(t,\ x)$ и $N(t,\ x)$ -- \textbf{одинаково однородны}, если 
	\[\forall\ \lambda \in \R:\ M(\lambda t,\ \lambda x) = \lambda \cdot M(t,\ x) \iff N(\lambda t,\ \lambda x) = \lambda \cdot N(t,\ x)\]
\end{definition}
\begin{definition}
	\textbf{Однородное} дифференциальное уравнение -- это диффур вида
	\[M(t,\ x) \d t + N(t,\ x)\d x = 0\] где функции $M$ и $N$ -- одинаково однородны.
\end{definition}

Решать такие диффуры можно путём сведения к уравнению с разделяющимися переменными при помощи различных замен. О них, а также о том, как сводить иные уравнения к однородным, мы и поговорим.

\subsection{Однородные уравнения: $y = \dfrac{x}{t}$}

Пускай у нас имеется диффур вида:

\[\dx = f\left(\dfrac{x}{t}\right)\]

Оно уже однородное. Мы хотим привести его к уравнению с разделяющими переменными. Следующая замена позволит нам это сделать: $y = \dfrac{x}{t}$:
\[y = \dfrac{x}{t} \Longrightarrow x = y\cdot t \Longrightarrow \d x = y\d t + t \d y\] -- подставляем в однородное уравнение и решаем, находя $y(t)$. После этого обратная замена.
\ \\

\textbf{Пример №1}

\[t \d x = (x + t) \d t\]

Данное уравнение уже является однородным. Преобразуем его и сделаем вышеуказанную замену (предварительно рассмотрев вырожденные случаи):
\begin{gather*}
	t \d x = (x + t) \d t \iff \d x = \dfrac{x}{t}\d t + \d t,\ y = \dfrac{x}{t}\\
	(y\d t + t \d y) = (y + 1)\d t\\
	t\d y = \d t \Longrightarrow \d y = \dfrac{\d t}{t} \Longrightarrow \ds\int=\int
\end{gather*}
Осталось проинтегрировать, получить $y(t)$ и подставить $y = \dfrac{x}{t}$.
\ \\

\textbf{Пример №2}

\[x^2 + \dx t^2 = tx\dx\]

Помня, что $\dx = \dfrac{\d x}{\d t}$, преобразуем диффур и сделаем ту же замену.
\begin{gather*}
	x^2 + \dx t^2 = tx\dx \iff x^2 \d t = (tx - t^2)\d x,\ y = \dfrac{x}{t}\\
	y^2t^2\d t = t^2(y - 1)(y\d t + t\d y)\\
	y^2\d t = y^2\d t + yt\d y - y\d t - t\d y\\
	y\d t = t(y - 1)\d y \iff 
	\dfrac{\d t}{t} = \left(1 - \dfrac{1}{y}\right)\d y\\
	\ln |t| = y - \ln |y| + C
\end{gather*}
Теперь удобно делать обратную замену:
\[\ln |yt| = y + C \iff \ln |x| = \dfrac{x}{t} + C\] и преобразовать получившееся выражение до вида $x = x(t)$.
\ \\

\textbf{Пример №3}

\[t\dx = x - t \cdot \exp(\dfrac{x}{t})\] --- здесь руки сами просят поделить на $t$ ($t \neq 0$, так как иначе уравнение не определено):
\begin{gather*}
	\dx = \dfrac{x}{t} - \exp(\dfrac{x}{t}),\ y = \dfrac{x}{t}\\
	\dfrac{y\d t + t\d y}{\d t} = y - e^y \Longrightarrow \dfrac{t\d y}{\d t} = - e^y\\
	\dfrac{\d y}{e^y} = -\dfrac{\d t}{t} \Longrightarrow \ds\int = \int\\
	e^{-y} = -\ln |t| + C
\end{gather*} --- далее обратная замена.

\subsection{Однородные уравнения: дробно-линейный вид}

Пускай мы имеем диффур вида:
\[\dx = f\left(\dfrac{a_1t + b_1x + c_1}{a_2t + b_2x + c_2}\right),\ a_1,\ a_2,\ b_1,\ b_2,\ c_1,\ c_2 \in \R\]

Ясно, что числитель и знаменатель суть уравнения прямых на координатной плоскости. Мы можем преобразовать уравнения такого типа к только что рассмотренным $\dx = f\left(\dfrac{x}{t}\right)$, если перенесём систему координат в точку пересечения данных прямых. 

Теперь подробнее о методе. Имеем систему уравнений:
\[\left\{\begin{gathered}
	a_1t + b_1x + c_1 = 0\hfill\\
	a_2t + b_2x + c_2 = 0\hfill
\end{gathered}\right.\] --- решением данной системы будет точка пересечения двух прямых $(t^*,\ x^*)$. Теперь перенесём систему координат в данную точку, произведя замену:
\[\begin{gathered}
	\wt = t - t^*\hfill\\
	\wx = x - x^*\hfill
\end{gathered}\] --- теперь, произведя простые преобразования, получаем однородный диффур $\dot{(\wx)} = \left(\dfrac{\wx}{\wt}\right)$. 	Решив его, осталось провести обратную замену, прибавив соответствующие константы.

Здесь стоит обговорить случай, когда выражения в числителе и знаменателе задают параллельные прямые, то есть:
\[\dx = f\left(\dfrac{at + bx + c_1}{at + bx + c_2}\right)\] --- отличаются только на свободный член, ибо $c_2 = c_1 + c,\ c \in \R$. В данном случае уравнение тривиально сводится к уравнению вида $\dx = \widetilde{f}(at + bx)$, которое мы умеем решать, линейной заменой сводя к диффуру с разделяющимися переменными.
\ \\

\textbf{Пример №4}

\[(x + 2)\d t = (2t + x - 4)\d x\]
Перезапишем в дробно-линейном виде:
\[\dx = \dfrac{x + 2}{2t + x - 4}\]
Решим систему:
\[\left\{\begin{gathered}
x + 2 = 0\hfill\\
2t + x - 4 = 0\hfill
\end{gathered}\right. \Longrightarrow \left\{\begin{gathered}
	x = -2\hfill\\
	t = 3\hfill
\end{gathered}\right.\]
Делаем замену, получая однородное уравнение:
\[\left\{\begin{gathered}
\wx = x + 2\hfill\\
\wt = t - 3\hfill
\end{gathered}\right. \Longrightarrow \dfrac{\d x}{\d t} = \dfrac{\d \wx}{\d\wt} = \dfrac{\wx}{2\wt + \wx}\] --- осталось решить его как уравнение с разделяющимися переменными.
\ \\

\textbf{Пример №5}

\[\dx = \dfrac{5t - x - 3}{3t + 2x - 7}\]
Решим систему:
\[\left\{\begin{gathered}
5t - x - 3 = 0\hfill\\
3t + 2x - 7 = 0\hfill
\end{gathered}\right. \Longrightarrow \left\{\begin{gathered}
x = 2\hfill\\
t = 1\hfill
\end{gathered}\right.\]
Делаем замену, получая однородное уравнение:
\[\left\{\begin{gathered}
\wx = x - 2\hfill\\
\wt = t - 1\hfill
\end{gathered}\right. \Longrightarrow \dfrac{\d \wx}{\d\wt} = \dfrac{5\wt - \wx}{3\wt + 2\wx}\]
Поделим числитель и знаменатель правой части уравнения на $\wt$	и положим $y = \dfrac{\wx}{\wt}$. Получаем уравнение с разделяющимися переменными:
\[\wt \cdot \dfrac{\d y}{\d \wt} = \dfrac{5 - 4y - 2y^2}{3 + 2y}\]

\subsection{Однородные уравнения: $x = y^m$}

Пускай мы имеем уравнение
\[M(t,\ x) \d t + N(t,\ x)\d x = 0\] где функции $M$ и $N$ -- НЕ одинаково однородные. Однако, мы были бы рады привести его к таковому. В этом нам поможет замена $x = y^m$, где $m \in \Q$. 

Идея в том, что у однородного уравнения степени каждого из слагаемых уравнения должны совпадать --- тогда при домножении на константу мы сможем её спокойно вынести за функции. Поэтому сначала мы найдём степень $m$, после чего будем решать обычное однородное уравнение при помощи известных методов. Разберём на примерах.
\ \\

\textbf{Пример №6}

\[2t^4\cdot x\dx + x^4 = 4t^6\] --- перепишем с дифференциалами:
\[2t^4 \cdot x\d x + x^4\d t= 4t^6\d t\]
Делаем замену:
\[x = y^m \Longrightarrow \d x = m \cdot y^{m - 1}\d y\] --- получаем уравнение:
\[2t^4 \cdot y^m \cdot m \cdot y^{m - 1}\d y + y^{4m}\d t = 4t^6\d t\]

Приравняем степени:
\[3 + 2m = 4m = 6 \Longrightarrow m = \dfrac{3}{2}\] --- теперь подставляем и решаем однородный диффур.
\ \\

\textbf{Пример №7}

\[\dx = x^2 - \dfrac{2}{t^2}\]
Перепишем с дифференциалами и сделаем замену $x = y^m$:
\begin{gather*}
	t^2\d x = x^2t^2\d t - 2\d t\\
	x = y^m,\ \d x =  m \cdot y^{m - 1}\d y\\
	m\cdot t^2 \cdot y^{m - 1}\d y = y^{2m}\cdot t^2\d t - 2\d t
\end{gather*}
Ищем степень:
\[2 + m - 1 = 2m + 2 = 0 \Longrightarrow m = -1\]
Тогда:
\begin{gather*}
	-t^2\cdot y^{-2}\d y = y^{-2} \cdot t^2\d t - 2\d t\\
	-t^2\d y = t^2\d t - 2y^2\d t
\end{gather*}
Получили однородное уравнение, которое решается классически: $z = \dfrac{y}{t}$:
\begin{gather*}
	z = \dfrac{y}{t},\ \ y = zt,\ \ \d y = t \d z + z\d t\\
	-t^2\big(t\d z + z\d t\big) = t^2\d t - 2z^2t^2\d t\\
	-t\d z - z \d t =  \d t - 2z^2\d t\\
	t\d z = (2z^2 - z - 1)\d t
\end{gather*}

Получили уравнение с разделяющимися переменными:
\[\dfrac{\d t}{t} = \dfrac{\d z}{2z^2 - z - 1} \Longrightarrow \ds\int = \int\] --- находим решение $z = z(t)$, после чего производим череду обратных замен, находя $x = x(t)$.

\subsection{Домашнее задание №2}


\begin{task1}
	Решить диффур:
	\[\dx = 2 \cdot \left(\dfrac{x + 2}{x + t - 1}\right)^2\]
\end{task1}
\begin{proof}[Решение]

\end{proof}

\begin{task2}
	Решить диффур:
	\[\dx = \dfrac{x + 2}{t + 1} + \tan(\dfrac{x - 2t}{t + 1})\]	 
\end{task2}
\begin{proof}[Решение]
	
\end{proof}

\begin{task3}
	Решить диффур:
	\[2x + (xt^2 + 1)\cdot t\dx = 0\]
\end{task3}
\begin{proof}[Решение]
	
\end{proof}

\begin{task4}
	Найти все такие $\alpha,\ \beta,\ a,\ b \in \R$, такие, что дифференциальное уравнение
	\[\dx = at^\alpha + bx^\beta\] \textbf{сводится} к однородному.
\end{task4}
\begin{proof}[Решение]

\end{proof}

\begin{task5}
	Найдите все кривые $x(t)$, такие, что расстояние от начала координат до касательной к $x(t)$ в любой точке $\big(t_0,\ x(t_0)\big)$ совпадает с абсциссой данной точки (равно $t_0$):
	\begin{center}
		\begin{tikzpicture}
		
		
		\draw (8, -0.3) node[draw,circle,fill=black,minimum size=5pt,inner sep=1pt][label=45:$M$] (M) {};
		\draw (4, -2) node[draw,circle,fill=black,minimum size=0pt,inner sep=0pt] (o) {};
		\draw (11, -2) node[draw,circle,fill=black,minimum size=0pt,inner sep=0pt] (t) {};
		\draw (8, -1) node[draw,circle,fill=black,minimum size=0pt,inner sep=0pt] (c) {};
		\draw (5.87, 0.92) node[draw,circle,fill=black,minimum size=0pt,inner sep=0pt] (p) {};
		\draw (8, -2) node[draw,circle,fill=black,minimum size=0pt,inner sep=0pt] (t0) {};
		
		\draw[step=1cm,gray,very thin] (4,-2) grid (12,3);
		\draw[very thick,->] (4,-2) -- (12,-2) node[anchor=north] {$t$};
		\draw[very thick,->] (4,-2) -- (4,3) node[anchor=east](x) {$x$};
		
		\draw[very thick] (11, -2) -- (4, 2);
		\draw[very thick] (4, -2) -- (5.87, 0.92);
		
		\draw[very thick, dashed] (8, -2) -- (M);
		\draw[very thick, dashed] (4, -0.3) -- (M);
		
		\draw [right angle symbol={o}{p}{M}];
		
		\draw [right angle symbol={o}{t}{M}];
		
		\node [left] at (4, -0.3) {$x_0$};
		\node [below] at (8, -2) {$t_0$};
		\node [left] at (4, -2) (O) {$O$};
		
		\node [above left] at (5.25, 0) {$t_0$};
		
		\pic[draw=red, <->, angle eccentricity=2, angle radius=1.75cm, very thick]	{angle=M--t--o};
		
		\draw[very thick, dashed, color=violet] (M) to [out=140,in=-80] (5, 3);
		\draw[very thick, dashed, color=violet] (M) to [out=-25,in=180] (12, -1.3);
		
		\end{tikzpicture}
	\end{center}
\end{task5}
\begin{proof}[Решение]
	
\end{proof}

\newpage
\section{Семинар 3, 24 января}

\subsection{Линейные уравнения. Базовый случай}

Линейные диффуры первого порядка имеют следующий общий вид: \[\dx + a(t)x = b(t)\]

Опишем 2 метода решения таких уравнений в общем виде.

\subsubsection{Метод замены}

Имеем диффур: $\dx + a(t)x = b(t)$. Делаем следующую замену: $x = uv$, где $u = u(t),\ v = v(t)$ -- функции от $t$. Естественным образом изменяется и дифференциал:
\[\left\{\begin{gathered}
x = uv\hfill\\
\dx = \du v + u\dv\hfill
\end{gathered}\right.\]
Тогда уравнение примет вид:
\[\du v + u\dv + a(t)uv = b(t)\]
Сгруппируем слагаемые в левой части по $u$:
\[\du v + u\big(\dv + a(t)v\big) = b(t)\]
Теперь приравниваем выражение в скобках к нулю, получая систему:
\[\left\{\begin{gathered}
\dv + a(t)v = 0\hfill\\
\du v = b(t)\hfill
\end{gathered}\right.\]
Решим первое уравнение системы -- это уравнение с разделяющимися переменными:
\begin{gather*}
\dfrac{\d v}{\d t} + a(t)v = 0\\
\dfrac{\d v}{v} = -a(t)\d t \Longrightarrow \ds\int = \int\\
\ln|v| = -\ds\int a(t) \d t \Longrightarrow v = \exp(-\ds\int a(t)\d t)
\end{gather*}
Прошу заметить, что в данном случае константа $C = 0$ -- это важно. Теперь подставляем данное выражение во второе уравнение.
\begin{gather*}
\du\exp(-\ds\int a(t)\d t) = b(t)\\
\d u = b(t) \cdot \exp(\ds\int a(t)\d t)\d t \Longrightarrow \ds\int = \int\\
u = \ds\int b(t) \cdot \exp(\ds\int a(t) \d t)\d t
\end{gather*}
Осталось подставить полученные $u$ и $v$, получив $x = x(t) = u \cdot v$:
\begin{gather*}
x = u \cdot v = \left[\ds\int b(t) \cdot \exp(\ds\int a(t) \d t)\d t\right] \cdot \exp(-\ds\int a(t) \d t)
\end{gather*}
\ \\

\textbf{Пример №1}

\[\dx +2tx = t\cdot \exp(-t^2)\]
Будем действовать по алгоритму.

\begin{gather*}
	\left\{\begin{gathered}
	x = uv\hfill\\
	\dx = \du v + u\dv\hfill
	\end{gathered}\right.\\
	\du v + u\dv + 2tuv = t\cdot \exp(-t^2) \iff \du v + u\big(\dv + 2tv\big) = t\cdot \exp(-t^2)\\
	\left\{\begin{gathered}
	\dv + 2tv = 0\hfill\\
	\du v = t\cdot \exp(-t^2)\hfill
	\end{gathered}\right. \Longrightarrow
	\left\{\begin{gathered}
	v = \exp(-t^2)\hfill\\
	\du \exp(-t^2) = t\cdot \exp(-t^2)\hfill
	\end{gathered}\right. \\
	\du = t \Longrightarrow u = \dfrac{t^2}{2} + C\\
	x = uv = \left(\dfrac{t^2}{2} + C\right)\cdot \exp(-t^2)
\end{gather*}

\subsubsection{Метод вариации произвольной постоянной}

Имеем диффур: $\dx + a(t)x = b(t)$. Если $b(t) := 0$, то у нас имеется однородный диффур $\dx + a(t)x = 0$. Ясно, что каждое решение линейного диффура является решением однородного диффура с нулевой правой частью, смещённого на $b(t)$ -- по аналогии с СЛУ из Линала, мы сначала решаем однородную СЛУ, получая общее решение, а затем смещаем на вектор значений (столбец правых частей из расширенной СЛУ). Таким образом, получаем метод:
\begin{gather*}
\dx + a(t)x = 0\\
\d x + a(t)x\d t = 0\\
\dfrac{\d x}{x} = -a(t)\d t \Longrightarrow \ds\int = \int \\
\ln|x| = -\ds\int a(t)\d t + C \Longrightarrow x = u \cdot \exp(-\ds\int a(t) \d t),\ u \in \R
\end{gather*}
Теперь сделаем вариацию постоянной (тем самым получая все возможные <<сдвиги>> решения однородного уравнения). Полагаем теперь, что $u$ -- не константа, а тоже некоторая функция от $t$: $u = u(t)$. 
Тогда можно сделать замену $x = x(t,\ u)$ и подставить в исходное уравнение:
\begin{gather*}
\dot{\left[u \cdot \exp(-\ds\int a(t)\d t)\right]} + a(t) \cdot u \cdot \exp(-\ds\int a(t) \d t) = b(t)\\
\du\exp(-\ds\int a(t) \d t) - u \cdot a(t) \cdot \exp(-\ds\int a(t)\d t) + a(t) \cdot u \cdot \exp(-\ds\int a(t)\d t) = b(t)
\end{gather*}
Заметим, что последние два слагаемых в левой части равны по модулю и противоположны по знаку, следовательно, их можно попросту сократить, получая диффур:
\begin{gather*}
\du\exp(-\ds\int a(t)\d t) = b(t)\\
\d u = b(t) \cdot \exp(\ds\int a(t) \d t) \d t \Longrightarrow \ds\int = \int\\
u = \ds\int b(t) \cdot \exp(\ds\int a(t) \d t) \d t
\end{gather*}
Осталось подставить полученное решение $u$ в выражение для $x$, чтобы получить общее решение диффура:
\begin{gather*}
x = u \cdot \exp(-\ds\int a(t)\d t) = \left[\ds\int b(t) \cdot \exp(\ds\int a(t) \d t)\d t\right] \cdot \exp(-\ds\int a(t) \d t)
\end{gather*}

Внимательный читатель может заметить, что оба метода являются по факту одним и тем же. Однако, в зависимости от ситуации, пользоваться можно любым подходом.
\ \\

\textbf{Пример №2}

\[t\dx - 2x = 2t^4\]

Сначала решим однородное.
\begin{gather*}
	t\dx = 2x\\
	t\d x = 2x \d t \Longrightarrow \dfrac{\d x}{x} = 2\dfrac{\d t}{t} \Longrightarrow \ds\int=\int\\
	\ln|x| = 2\ln|t| \Longrightarrow x = u\cdot t^2,\ u \in \R
\end{gather*}
Теперь сделаем вариацию постоянной $u = u(t)$:
\begin{gather*}
	\left\{\begin{gathered}
	x = ut^2\hfill\\
	\d x = \du t^2 + 2ut\hfill
	\end{gathered}\right.\\
	t\left(\du t^2 + 2ut\right) - 2ut^2 = 2t^4\\
	\du t^3 +2ut^2 - 2ut^2 = 2t^4\\
	\du t^3 = 2t^4 \Longrightarrow \du = 2t \Longrightarrow u = t^2 + C\\
	x = ut^2 = (t^2 + C)t^2 = t^4 + Ct^2
\end{gather*}
Получили общее решение.
\ \\

\textbf{Пример №3}

\[x = t(\dx - t\cdot\cos t)\]
Приведём к <<классическому>> виду:
\[\dx t - x = t^2\cdot \cos t\]
Однородное:
\begin{gather*}
\dx t = x\\
\d x t \cdot t = x\d t\\
\dfrac{\d x}{x} = \dfrac{\d t}{t} \Longrightarrow \ds\int = \int\\
\ln|x| = \ln|t| + C \Longrightarrow x = ut,\ u \in \R
\end{gather*}
Сделаем вариацию постоянной $u = u(t)$:\
\begin{gather*}
\left\{\begin{gathered}
x = ut\hfill\\
\d x = \du t + u\hfill
\end{gathered}\right.\\
(\du t + u)t - ut = t^2\cos t \Longrightarrow \du t^2 = t^2\cos t\\
\du = \cos t \Longrightarrow u = \sin t + C
\end{gather*}
Подставляем в выражение для $x$, чтобы получить итоговый ответ:
\[x = ut = (\sin t + C)t = t\sin t + Ct\]

\subsubsection{Функция от $x$, сведение к линейному}

Пускай у нас есть уравнение вида: \[a'(x)\dx + b(t)\cdot a(x) = c(t)\] где $a = a(x)$ -- функция от \underline{$x$}. Его можно свести к линейному ввиду свойств дифференциала:
\[a'(x)\dx = \dfrac{\d a}{\d x} \cdot \dfrac{\d x}{\d t} = \dfrac{\d a}{\d t}\]
Подставляем в исходное уравнение, получая диффур, линейный по \underline{$a$}:
\[\dfrac{\d a}{\d t} + b(t) \cdot a = c(t)\]

Его уже можно решить приведёнными выше методами, получив решение $a = f(t)$, после чего выразить $x$ через $a(x) = f(t)$.

\subsection{<<Обратное>> решение: $t = t(x)$}

Ранее мы рассматривали уравнения, линейные относительно переменной $x$ и её производной. То есть мы считали, что $t$ является независимой переменной, а $x$ является зависимой переменной. Однако, всегда стоит иметь в виду, что возможен противоположный подход. То есть можно считать переменную $x$ независимой переменной, а $t$ – зависимой переменной. На практике часто встречаются задачи, в которых уравнение линейно относительно переменной $t$ и её производной, а не $x$. В общем виде такое уравнение можно записать так:

\[\big(a(x)t + b(x)\big)\dx = c(x)\]
Преобразуем его:
\begin{gather*}
	\big(a(x)t + b(x)\big)\dfrac{\d x}{\d t} = c(x)\\
	a(x)t + b(x) = c(x)\dfrac{\d t}{\d x}\\
	c(x)\dfrac{\d t}{\d x} - a(x)t = b(x)\\
	\dt - \dfrac{a(x)}{c(x)}\cdot t = \dfrac{b(x)}{c(x)}
\end{gather*}
Теперь решаем диффур любым из рассмотренных выше методов, но для функции $t = t(x)$.
\ \\

\textbf{Пример №4}

\[(2t + x^3)\dx = x\]

Преобразуем его к диффуру от $t$, подставив нужные выражения в формулу выше.
\[\dt - \dfrac{2}{x} \cdot t = x^2\]
Теперь решим методом вариации постоянной:
\begin{gather*}
\dt = \dfrac{2}{x} \cdot t\\
\dfrac{\d t}{t} = 2\dfrac{\d x}{x} \Longrightarrow \ds\int = \int\\
\ln|t| = 2\ln|x| + C \Longrightarrow t = ux^2,\ u \in \R
\end{gather*}
Теперь положим $u = u(x)$:
\begin{gather*}
\left\{\begin{gathered}
t = ux^2\hfill\\
\d t = \du x^2 + 2ux\hfill
\end{gathered}\right.\\
\du x^2 + 2ux - 2ux = x^2 \Longrightarrow \du x^2 = x^2\\
\du = 1 \Longrightarrow u = x + C
\end{gather*}
Получаем общее решение $t = t(x)$:
\[t = ux^2 = (x + C)x^2 = x^3 + Cx^2\]

\subsection{Уравнение Бернулли}

Это дифференциальное уравнение имеет следующий вид:
\[\dx + a(t)x = b(t)\cdot x^m\]

Характерный признак -- степень $m$ в правой части. Стоит отметить, что при $m = 0,\ 1$ это обычное линейное уравнение, которое мы умеем решать. Более того, степень $m$ может быть какой угодно -- положительной ли, отрицательной или вообще дробью.

Как и линейное неоднородное уравнение первого порядка, уравнение Бернулли может приходить на новогодний утренник в разных костюмах:
\begin{itemize}
	\item Волком:
		\[a(t)\dx +  b(t)x = c(t) \cdot x^m\]
	\item Зайчиком:
		\[\dx + x = c(t) \cdot x^m\]
	\item Или белочкой:
		\[\dx + a(t)x = x^m\]
\end{itemize}
Важно, чтобы всегда присутствовала Ёлочка -- $x^m$, которая иногда может маскироваться под корень. Вокруг неё и будем водить хороводы.

Стоит также обратить внимание, что у данных уравнений при $m > 0$ всегда есть очевидное частное решение $x = 0$. Понятно, что когда просят найти частное решение диффура, на этот факт можно забить, но при нахождении общего решения терять данный случай нельзя.

Теперь обговорим метод решения. Пусть мы имеем диффур в <<классическом>> виде:
\[\dx + a(t)x = b(t)\cdot x^m\]
Избавимся от $x$ в правой части уравнения, поделив всё на его степень:
\[\dfrac{\dx}{x^m} + \dfrac{a(t)}{x^{m - 1}} = b(t)\]
Теперь делаем хитрую (нет) замену:
\[\left\{\begin{gathered}
w = \dfrac{1}{x^{m - 1}}\hfill\\
\dw = \dfrac{(1 - m)\dx}{x^m} \Longrightarrow \dfrac{\dx}{x^m} = \dfrac{\dw}{1 - m}\hfill
\end{gathered}\right.\]
Тогда диффур примет вид:
\[\dfrac{\dw}{1 - m} + a(t)w = b(t) \Longrightarrow \dw + (1 - m)\cdot a(t)w = (1 - m)\cdot b(t)\] --- получили обычный линейный диффур, который только что научились решать.
\ \\

\textbf{Пример №5}

\[\dx = x^4\cdot \cos t + x \cdot \tan t\]
Запомним, что $x = 0$ -- решение, далее полагаем $x \neq 0$. Перепишем в <<классическом>> виде:
\[\dx - x \cdot \tan t = x^4 \cdot \cos t\]
Поделим на $x^4$:
\[\dfrac{\dx}{x^4} - \dfrac{\tan t}{x^3} = \cos t\]
Замена:
\begin{gather*}
\left\{\begin{gathered}
w = \dfrac{1}{x^3}\hfill\\
\dw = \dfrac{(1 - 4)\dx}{x^4} \Longrightarrow \dfrac{\dx}{x^4} = -\dfrac{\dw}{3}\hfill
\end{gathered}\right.\\
-\dfrac{\dw}{3} - w \cdot \tan t = \cos t \iff \dw + 3w \cdot \tan t = -3\cos t
\end{gather*}
Теперь решаем такой линейный диффур.
\begin{gather*}
\dw = -3w\tan t \Longrightarrow \dfrac{\d w}{w} = -3 \tan t \d t \Longrightarrow \ds\int = \int \\
\ln|w| = 3\ln|\cos t| + C \Longrightarrow w = u \cdot \cos^3t,\ u \in \R\\
\end{gather*}
Делаем вариацию постоянной $u = u(t)$:
\begin{gather*}
\left\{\begin{gathered}
w = u \cdot \cos^3t\hfill\\
\dw = \du\cos^3t - 3u \cdot \sin t \cdot \cos^2 t\hfill
\end{gathered}\right.\\
\du\cos^3t - 3u \cdot \sin t \cdot \cos^2 t + 3u \cdot \cos^3t \cdot \tan t = -3\cos t \iff\\
\iff  \du\cos^3t - 3u \cdot \sin t \cdot \cos^2 t + 3u \cdot \cos^3t \cdot \dfrac{\sin t}{\cos t}= -3\cos t\\
\du \cos^3 t = -3\cos t \iff \d u = -\dfrac{3\d t}{\cos^2 t} \Longrightarrow \ds\int = \int \\
u = -3 \tan t + C
\end{gather*}
Подставляем в выражение для $w$ и делаем обратную замену.
\begin{gather*}
w = u \cdot \cos^3t = \left(-3 \tan t + C\right)\cdot \cos^3t\\
w = \dfrac{1}{x^3} \Longrightarrow x^3 = \dfrac{1}{\left(-3\tan t + C\right)\cdot \cos^3 t}
\end{gather*}

\textbf{Пример №6}

Решить задачу Коши:
\[\dx - \dfrac{2x}{t} = 2t\cdot \sqrt{x},\ \ \ \ \text{ начальное условие: } x(1) = 1\]
Так как нужно решить задачу Коши, то $x \neq 0$. Решаем стандартным для Бернулли способом:
\begin{gather*}
\dfrac{\dx}{\sqrt{x}} - \dfrac{2\sqrt{x}}{t} = 2t\\
\left\{\begin{gathered}
w = \sqrt{x}\hfill\\
\dw = \dfrac{\dx}{2\sqrt{x}} \Longrightarrow \dfrac{\dx}{\sqrt{x}} = 2\dw\hfill
\end{gathered}\right.\\
2\dw - \dfrac{2w}{t} = 2t \iff \dw - \dfrac{w}{t} = t
\end{gather*}
Решим заменой $w = uv$:
\begin{gather*}
	\left\{\begin{gathered}
	w = uv\hfill\\
	\dw = \du v + u\dv\hfill
	\end{gathered}\right.\\
	\du v + u\dv - \dfrac{uv}{t} = t \iff \du v + u\left(\dv - \dfrac{v}{t}\right) = t\\
	\left\{\begin{gathered}
	\dv - \dfrac{v}{t} = 0\hfill\\
	\du v = t\hfill
	\end{gathered}\right.\\
	\dv = \dfrac{v}{t} \iff \dfrac{\d v}{v} = \dfrac{\d t}{t} \Longrightarrow \ds\int = \int\\
	\ln|v| = \ln|t| \iff v = t\\
	\du t = t \iff \du = 1 \Longrightarrow u = t + C
\end{gather*}
Получаем общее решение диффура от $w$
\[w = (t + C)\cdot t\]
Делаем обратную замену $w = \sqrt{x} \iff x = w^2$:
\[x = w^2 = \big[(t + C)\cdot t\big]^2 = (t + C)^2 \cdot t^2\]
Теперь найдём частное решение:
\[x(1) = 1 \Longrightarrow 1 = (1 + C)^2\] --- внезапно, уравнение имеет 2 корня: $C = 0$ и $C = -2$, откуда получается 2 частных решения:
\[x = t^4,\ \ \ \ \ x = (t - 2)^2 \cdot t^2\]
Каждое из них удовлетворяет начальному условию. Это объясняется тем, что задача Коши имеет единственное решение только при выполнении определённых условий (функция $f$ в диффуре $\dx = f(t, x)$ должна быть Липшицевой). В данном случае условие единственности нарушено, и в точке $(1,\ 1)$ \textbf{пересекаются} графики кривых $x = t^4$ и $x = (t - 2)^2 \cdot t^2$.

\subsection{Уравнение Риккати}

Это дифференциальное уравнение вида:
\[\dx + a(t)x^2 + b(t)x + c(t) = 0\]
Сразу отметим, что если $a,\ b,\ c$ -- константы, то это обычное уравнение с разделяющимися переменными, решение которого можно записать в виде функции $t = t(x)$:
\[t = C - \ds\int\dfrac{\d x}{ax^2 + bx + c}\]

Методы решения данного уравнения весьма и весьма интересны.

\subsubsection{Одно частное решение}

Пускай у нас есть одно \textbf{частное} решение $x_1 = x_1(t)$ этого диффура. Тогда мы можем представить общее решение -- функцию $x = x(t)$ -- как сумму функций:
\[x = x_1 + z\] где $z = z(t)$ -- новая неизвестная функция. Подставим выражение для $x$ в исходное уравнение:
\begin{gather*}
	\left\{\begin{gathered}
	x = x_1 + z\hfill\\
	\dx = \dx_1 + \dz\hfill
	\end{gathered}\right.\\
	\dx_1 + \dz + a(t)\big(x_1^2  + 2x_1z + z^2\big) + b(t)(x_1 + z) + c(t) = 0\\
\end{gather*}
Поскольку $x_1$ -- решение диффура, то выполнено:
\[\dx_1 + a(t)x_1^2 + b(t)x_1 + c(t) = 0\]
Таким образом, раскрыв скобки, мы можем убрать обнулившуюся часть, получая диффур:
\begin{gather*}
\dz + a(t)\big(2x_1z + z^2\big) + b(t)z = 0\\
\dz + a(t)z^2 + \big(2a(t)x_1 + b(t)\big) \cdot z = 0\\
\dz + \big(2a(t)x_1 + b(t)\big) \cdot z = -a(t)z^2
\end{gather*}
--- а это уже знакомое нам уравнение Бернулли. Помним про корень $z = 0$, стреляем в уравнение стандартной заменой:
\begin{gather*}
\dz + \big(2a(t)x_1 + b(t)\big) \cdot z = -a(t)z^2\\
\dfrac{\dz}{z^2} + \dfrac{2a(t)x_1 + b(t)}{z} = -a(t)\\
\left\{\begin{gathered}
w = \dfrac{1}{z}\hfill\\
\dw = -\dfrac{\dx}{x^2} \Longrightarrow \dfrac{\dx}{x^2} = -\dw\hfill
\end{gathered}\right.\\
-\dw + \big(2a(t)x_1 + b(t)\big) \cdot w = -a(t) \iff \dw - \big(2a(t)x_1 + b(t)\big) \cdot w = a(t)
\end{gather*}
и добиваем каким-нибудь методом решения линейных диффуров.

Стоит отметить, что часто бывает более выгодна немного иная замена $x$:
\[x = x_1 + \dfrac{1}{z}\] --- это сразу же приводит наше уравнение к линейному виду:
\[\dz - \big(2a(t)x_1 + b(t)\big) \cdot z = a(t)\] Однако, в данном случае нужно дополнительно помнить про нулевое решение (так как вот таким способом мы найдём все \underline{ненулевые} решения).
\ \\

\textbf{Пример №7}

\[\dx + x^2 = t^2 - 2t\]
Частное решение находится тривиально -- достаточно прибавить 1 к обеим частям уравнения, чтобы его увидеть:

\begin{gather*}
\dx_1 + x_1^2 + 1 = t^2 - 2t + 1 \iff (1 - t)^2 = x_1^2 + \dx_1 + 1 \Longrightarrow x_1 = 1 - t
\end{gather*}
Делаем представление через сумму (тут удобнее инвертировать $z$, помня про нулевое решение):
\[\left\{\begin{gathered}
x = 1 - t + \dfrac{1}{z}\hfill\\
\dx = -1 -\dfrac{\dz}{z^2}\hfill
\end{gathered}\right.\]
Тогда мы сразу получаем линейный диффур:
\begin{gather*}
\dz - \big(2 - 2t + 0\big) \cdot z = 1\\
\dz + 2(t - 1)z = 1\\
\end{gather*}
Решаем однородное:
\begin{gather*}
\dz = 2(1 - t)z \iff \dfrac{\d z}{z} = 2(1 - t)\d t \Longrightarrow \ds\int = \int\\
\ln|z| = -2\dfrac{(1 - t)^2}{2} + C \iff z = u \cdot \exp(-(1 - t)^2),\ u \in \R
\end{gather*}
Вариация постоянной:
\begin{gather*}
\left\{\begin{gathered}
z = u \cdot \exp(-(1 - t)^2)\hfill\\
\dz = \du\exp(-(1 - t)^2) + 2u \cdot (1 - t) \cdot \exp(-(1 - t)^2) \hfill
\end{gathered}\right.\\
\du\exp(-(1 - t)^2) + 2u \cdot (1 - t) \cdot \exp(-(1 - t)^2) + 2(t - 1) \cdot u \cdot \exp(-(1 - t)^2) = 1\\
\du\exp(-(1 - t)^2) = 1 \iff \du = \exp((1 - t)^2) \iff u = \ds\int \exp((1 - t)^2)
\end{gather*}
Осталось подставить выражение для $z$ и получить общее решение:
\begin{gather*}
z = u \cdot \exp(-(1 - t)^2) = \exp(-(1 - t)^2) \cdot \ds\int \exp((1 - t)^2)\\
x = 1 - t + \dfrac{\exp((1 - t)^2)}{\ds\int \exp((1 - t)^2)}
\end{gather*}

\textbf{Пример №8}

\[3\dx + x^2 + \dfrac{2}{t^2} = 0\]

Здесь частное решение находится также весьма просто:
\[-3\dx_1 = x_1^2 + \dfrac{2}{t^2} \Longrightarrow x_1 = \dfrac{1}{t}\]
В сумму (опять инвертируем, помня про нулевое решение):
\[\left\{\begin{gathered}
x = \dfrac{1}{t} + \dfrac{1}{z}\hfill\\
\dx = -\dfrac{1}{t^2} - \dfrac{\dz}{z^2}\hfill
\end{gathered}\right.\]
Теперь подставим всё в диффур от $z$, заменив нужные функции от $t$:
\begin{gather*}
3\dx + x^2 + \dfrac{2}{t^2} = 0 \iff \dx + \dfrac{1}{3}x^2 + 0 \cdot x + \dfrac{2}{t^2} = 0\\
\dz - \dfrac{2}{3t}\cdot z = \dfrac{1}{3}
\end{gather*}
Решаем как однородное:
\begin{gather*}
\dz = \dfrac{2}{3t}z \iff \dfrac{\d z}{z} = \dfrac{2}{3} \cdot \dfrac{\d t}{t} \Longrightarrow\ds\int = \int\\
\ln|z| = \dfrac{2}{3} \cdot \ln|t| + C \iff z = u \cdot t^{\frac{2}{3}},\ u \in \R
\end{gather*}
Варьируем:
\begin{gather*}
\left\{\begin{gathered}
z = u \cdot t^{\frac{2}{3}}\hfill\\
\dz = \du t^{\frac{2}{3}} + \dfrac{2}{3} \cdot t^{-\frac{1}{3}} \hfill
\end{gathered}\right.\\
\du t^{\frac{2}{3}} + \dfrac{2}{3} \cdot u\cdot t^{-\frac{1}{3}} - \dfrac{2}{3t}\cdot u \cdot t^{\frac{2}{3}} = \dfrac{1}{3}\\
\du t^{\frac{2}{3}} = \dfrac{1}{3} \iff \du = \dfrac{1}{3} \cdot t^{-\frac{2}{3}} \iff u = t^{\frac{1}{3}} + C
\end{gather*}
Теперь можем получить общее решение:
\begin{gather*}
z = u \cdot t^{\frac{2}{3}} = \left(t^{\frac{1}{3}} + C\right) \cdot t^{\frac{2}{3}} = t + Ct^{\frac{2}{3}}\\
x = \dfrac{1}{t} + \dfrac{1}{z} = \dfrac{1}{t} + \dfrac{1}{t + Ct^{\frac{2}{3}}}
\end{gather*}

\subsubsection{Два частных решения}

Пусть теперь известны два частных решения уравнения Риккати: $x_1 = x_1(t)$ и $x_2 = x_2(t)$.

Мы знаем, что для первого решения выполнено тождество:
\[\dx_1 + a(t)x_1^2 + b(t)x_1 + c(t) = 0\]
Проделаем то же представление:
\begin{gather*}
\left\{\begin{gathered}
x = x_1 + z\hfill\\
\dx = \dx_1 + \dz\hfill
\end{gathered}\right.\\
\dx_1 + \dz + a(t)\big(x_1^2  + 2x_1z + z^2\big) + b(t)(x_1 + z) + c(t) = 0\\
\dz + \big(2a(t)x_1 + b(t)\big) \cdot z = -a(t)z^2
\end{gather*}
Теперь же сделаем хитрость: поделим на $z$ и подставим вместо него $x - x_1$:
\begin{gather*}
\dfrac{\dz}{z} + 2a(t)x_1 + b(t) = -a(t)z\\
\dfrac{1}{x - x_1}\cdot\dfrac{\d (x - x_1)}{\d t} + 2a(t)x_1 + b(t) = -a(t)x + a(t)x_1\\
\dfrac{1}{x - x_1}\cdot\dfrac{\d (x - x_1)}{\d t} = -a(t)\cdot \big(x + x_1\big) - b(t)
\end{gather*}
А сейчас начинаем колдовать: заносим дробь в левой части уравнения под дифференциал:
\[\dfrac{1}{x - x_1}\d (x - x_1) = \d \ln\big(x - x_1\big)\]\newpage
Тогда:
\[
\dfrac{\d \ln\big(x - x_1\big)}{\d t} = -a(t)\cdot \big(x + x_1\big) - b(t)
\]
Для второго частного решения также выполнено тождество:
\[\dx_2 + a(t)x_2^2 + b(t)x_2 + c(t) = 0\]
Откуда мы аналогичным (матемагическим) способом получаем:
\[\dfrac{\d \ln\big(x - x_2\big)}{\d t} = -a(t)\cdot \big(x + x_2\big) - b(t)\]
А теперь совсем сакрально поступим: вычтем из первого уравнения второе:
\[\left\{\begin{gathered}
\dfrac{\d \ln\big(x - x_1\big)}{\d t} = -a(t)\cdot \big(x + x_1\big) - b(t)\hfill\\
\dfrac{\d \ln\big(x - x_2\big)}{\d t} = -a(t)\cdot \big(x + x_2\big) - b(t) \hfill
\end{gathered}\right. \Longrightarrow \dfrac{\d \ln\frac{x - x_1}{x - x_2}}{\d t} = a(t)\cdot (x_2 - x_1)\] --- но ведь это диффур с разделяющимися переменными! Из него получаем уравнение, задающее $x = x(t)$ неявно через частные решения:
\begin{gather*}
\dfrac{\d \ln\frac{x - x_1}{x - x_2}}{\d t} = a(t)\cdot (x_2 - x_1) \\
\d \ln\dfrac{x - x_1}{x - x_2} = a(t)\cdot (x_2 - x_1)\d t \Longrightarrow \ds\int = \int\\
\ln\dfrac{x - x_1}{x - x_2} + C = \ds\int a(t)\cdot (x_2 - x_1)\d t\\
\dfrac{x - x_1}{x - x_2} = \lambda \cdot \exp(\ds\int a(t)\cdot (x_2 - x_1)\d t),\ \lambda \in \R
\end{gather*}
Конечно, не всегда бывает возможно угадать сразу два частных решения, однако если вам так повезло, то можно захотеть выпендриться и задать ответ формулой выше.
\ \\

\textbf{Пример № 9}

\[\dx = \dfrac{k^2}{t^2} - x^2,\ k \in \R\]

Частные решения:
\[\left\{\begin{gathered}
x_1 = \dfrac{1}{t} + \dfrac{k}{t^2}\hfill\\
x_2 = \dfrac{1}{t} - \dfrac{k}{t^2}\hfill
\end{gathered}\right.\]
Тогда можем сразу получить общее решение:
\begin{gather*}
\dfrac{x - \frac{1}{t} - \frac{k}{t^2}}{x - \frac{1}{t} + \frac{k}{t^2}} = \lambda \cdot \exp(\ds\int -\dfrac{2k}{t^2}\d t)\\
\dfrac{xt^2 - x - k}{xt^2 - x + k} = \lambda \cdot \exp(\dfrac{2k}{x})
\end{gather*}
Кому интересно, может попробовать решить этот диффур по-старинке. Авось что покрасивше выйдет. Или нет.

\subsection{Домашнее задание №3}

\begin{task1}
Решить диффур:
\[(tx + e^t)\d t - t\d x = 0\]
\end{task1}
%\begin{proof}[Решение]
%
%\end{proof}

\begin{task2}
Решить диффур:
\[t\dx = x - t\cdot \exp(\dfrac{x}{t})\]
\end{task2}
%\begin{proof}[Решение]
%%	
%\end{proof}

\begin{task3}
Решить диффур:
\[t\dx - 2t^2\cdot \sqrt{x} = 4x\]
\end{task3}
%\begin{proof}[Решение]
%	
%\end{proof}

\begin{task4}
Решить диффур:
\[t^2\dx + tx + t^2x^2 = 4\]
\end{task4}
%\begin{proof}[Решение]
%
%\end{proof}

\begin{task5}
Пусть $x_1,\ x_2$ -- независимые  частные решения линейного диффура 1 порядка:
\[\dx + a(t)x = b(t)\]
Найти общее решение (выразить через $x_1$ и $x_2$).
\end{task5}
%\begin{proof}[Решение]
%	
%\end{proof}

\newpage
\section{Common Tasks}

\begin{enumerate}
	\item Найти все гладкие функции $x(t)$ такие, что для любой точки $t_0 \in \R$ касательная к $x(t)$ в точке $\big(t_0,\ x(t_0)\big)$ пересекает ось абсцисс в точке $\dfrac{t_0}{2}$.
	\item Найти дифференциальное уравнение первого порядка, задающее на плоскости семейство парабол
	$x = at^2 + bt + c$, проходящих через 
	точку $(0,1)$ и касающихся прямой $x = t$. 
	\item Решить диффур: \[\dfrac{t}{x}\dx = \ln x - \ln t + 1\]
	\item Решить задачу Коши:
	\[	\dx = \dfrac{1}{2x}\cdot \exp(\frac{x^2}{t}) + \dfrac{x}{2t},\ \ 	\text{начальное условие: }x(1) = 1\]
	\item Решить диффур: \[(t + 1)(x\dx - 1) = x^2\]
	\item Решить задачу Коши: \[\dx - 2tx + x^2 = 5 - t^2,\ \ \text{ начальное условие: }x(0) = 0\]
\end{enumerate}

\end{document}