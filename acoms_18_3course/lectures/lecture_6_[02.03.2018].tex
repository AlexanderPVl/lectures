\section{Лекция 6}
На этой лекции разбирались задачи из первого домашнего задания. Здесь мы оставим только формулировки и ответы.
\begin{problem}
    Пусть $\vec{X} = (X_{1}, \ldots, X_{n})$ "--- выборка из равномерного распределения на отрезке $[\,0, \theta]$. Проверьте на несмещенность, состоятельность, сильную состоятельность и асимптотическую нормальность следующие оценки параметра $\theta$: $2\overline{\vec{X}}$, $\overline{\vec{X}} + X_{(n)}/2$, $(n + 1)X_{(1)}$, $X_{(1)} + X_{(n)}$, $\frac{n + 1}{n}X_{(n)}$.
\end{problem}
\begin{answer}
    Скомпонуем ответ в таблицу:

    \begin{center}
        \begin{tabular}{|c|c|c|c|}
            \hline
            Оценка & Несмещённость & Состоятельность & Асимптотическая нормальность \\
            \hline
            $2\overline{\vec{X}}$ & Есть & Сильная & Есть с $\sigma^{2}(\theta) = \theta^{2}/3$ \\
            \hline
            $\overline{\vec{X}} + X_{(n)}/2$ & Нет & Сильная & Есть с $\sigma^{2}(\theta) = \theta^{2}/12$ \\
            \hline
            $(n + 1)X_{(n)}$ & Есть & Нет & Нет \\
            \hline
            $X_{(1)} + X_{(n)}$ & Есть & Сильная & Нет \\
            \hline
            $\frac{n + 1}{n}X_{(n)}$ & Есть & Сильная & Нет \\
            \hline
        \end{tabular}
    \end{center}
\end{answer}

\begin{problem}
    Пусть $\vec{X} = (X_{1}, \ldots, X_{n})$ "--- выборка из гамма-распределения с параметрами $(\alpha, \lambda)$. Предложите асимптотически нормальную оценку $\alpha > 0$ и вычислите ее асимптотическую дисперсию, если 
    \begin{enumerate}[label=(\alph*)]
        \item $\lambda$ известно;
        \item $\lambda$ тоже неизвестно.
    \end{enumerate}
\end{problem}
\begin{answer}
    Оценка $\alpha$ при известном $\lambda$ имеет вид $\hat{\alpha}_{n}(\vec{X}) = \lambda\overline{\vec{X}}$. Для неё
    \[
        \sqrt{n}(\hat{\alpha}_{n}(\vec{X}) - \alpha) 
        \xrightarrow{d_{\alpha}} \mathcal{N}(0, \alpha).
    \]
    Оценка $\alpha$ при неизвестном $\lambda$ имеет вид 
    \[
        \hat{\alpha}_{n}(\vec{X}) = \frac{\overline{\vec{X}}^{2}}{\overline{\vec{X}^{2}} - \overline{\vec{X}}^{2}}.
    \]
    Для неё
    \[
        \sqrt{n}(\hat{\alpha}_{n}(\vec{X}) - \alpha) 
        \xrightarrow{d_{\alpha}} \mathcal{N}(0, 2\alpha(\alpha + 1)).
    \]
\end{answer}

\begin{problem}
    Пусть $\vec{X} = (X_{1}, \ldots, X_{n})$ "--- выборка из равномерного распределения на отрезке $[0, \theta]$. Сравните следующие оценки параметра $\theta$ в равномерном подходе с квадратичной функцией потерь:
    \begin{enumerate}
        \item $\hat{\theta}_{1}(\vec{X}) = 2\overline{\vec{X}}$;
        \item $\hat{\theta}_{2}(\vec{X}) = X_{(1)} + X_{(n)}$;
        \item $\hat{\theta}_{3}(\vec{X}) = \frac{n + 1}{n}X_{(n)}$.
    \end{enumerate}
\end{problem}
\begin{answer}
    Отсортируем оценки от худшей к лучшей: $2\overline{\vec{X}}$, $X_{(1)} + X_{(n)}$, $(1 + n^{-1})X_{(n)}$.
\end{answer}


\begin{problem}
    Пусть $\theta^{*}_{1}(\vec{X})$ и $\theta^{*}_{2}(\vec{X})$ — две ``почти наилучшие'' оценки параметра $\theta$ в среднеквадратичном походе (т.е. каждая из них не хуже любой другой оценки), имеющие одинаковые математические ожидания. Докажите, что тогда для любого $\theta$ они совпадают почти наверное, т.е. $\theta^{*}_{1}(\vec{X}) = \theta^{*}_{2}(\vec{X})$ $\Pr_{\theta}$-п.н.
\end{problem}
\begin{hint}
    Рассмотрите оценку $\hat{\theta}(\vec{X}) = (\theta^{*}_{1}(\vec{X}) + \theta^{*}_{2}(\vec{X})) / 2$ и, пользуясь ей, докажите, что $\EE_{\theta}[(\theta^{*}_{1}(\vec{X}) - \theta^{*}_{2}(\vec{X}))^{2}] = 0$.
\end{hint}

\begin{problem}
    $\vec{X} = (X_{1}, \ldots, X_{n})$ "--- выборка из распределения $\mathcal{N}(\theta, 1)$, $\theta > 0$. Сравните в равномерном подходе относительно квадратичной функции потерь оценки $\overline{\vec{X}}$ и $\max(0, \overline{\vec{X}})$.
\end{problem}
\begin{answer}
    Оценка $\max(0, \overline{\vec{X}})$ лучше.
\end{answer}

\begin{problem}
    Пусть $\vec{X} = (X_{1}, \ldots, X_{n})$ "--- выборка из гамма-распределения с плотностью
    \[
        p_{\theta}(x) 
        = \frac{2^{\theta}}{\Gamma(\theta)}x^{\theta - 1}e^{-2x}[x \geq 0].
    \]
    где $\theta > 0$ "--- неизвестный параметр. Для каких функций $\tau(\theta)$ существует эффективная оценка? Найдите информацию Фишера $i(\theta)$ одного элемента выборки.
\end{problem}
\begin{answer}
    Эффективная оценка существует только для линейного преобразования дигамма-функции $\psi(\theta)$. Информация Фишера одного элемента равна $\psi'(\theta)$.
\end{answer}

\begin{problem}
    Пусть $\vec{X} = (X_{1}, \ldots, X_{n})$ "--- выборка из нормального распределения с параметрами $(\mu, \sigma^{2})$. Найдите эффективную оценку
    \begin{enumerate}[label=(\alph*)]
        \item параметра $\mu$, если $\sigma$ известно;
        \item параметра $\sigma^{2}$, если $\mu$ известно.
    \end{enumerate}
    Вычислите информацию Фишера одного наблюдения в обоих случаях. Найдите информационную матрицу в случае, когда оба параметра $\mu$ и $\sigma^{2}$ неизвестны.
\end{problem}
\begin{answer}
    Информация Фишера при известной $\sigma^{2}$ равна $i(\mu) = \sigma^{-2}$. Информация Фишера при известной $\mu$ равна $(2\sigma^{4})^{-1}$. Информационная матрица Фишера равна
    \[
        \matr{\mathcal{I}}(\mu, \sigma^{2})
        = \begin{pmatrix}
            n\sigma^{-2} & 0 \\
            0 & n(2\sigma^{4})^{-1}
        \end{pmatrix}
    \]
\end{answer}

\begin{problem}
    Пусть $\vec{X}$ "--- наблюдение из ``регулярного'' семейства $\set{\Pr_{\vec{\theta}} \mid \vec{\theta} \in \Theta}$, $\Theta \subseteq \mathbb{R}^{k}$, $k > 1$. Докажите, что если $\hat{\vec{\theta}}(\vec{X})$ "--- эффективная оценка $\vec{\theta}$, то она является оценкой максимального правдоподобия для $\vec{\theta}$.
\end{problem}
\begin{hint}
    Покажите, что есть лишь одна точка, которая может претендовать на точку экстремума. Далее возникает проблема: гессиан посчитать нельзя, так как $\mathcal{I}(\vec{\theta})$ зависит от $\vec{\theta}$. Поэтому предлагается воспользоваться достаточным условием глобального максимума первого порядка.
\end{hint}
\begin{theorem}[Достаточное условие глобального максимума первого порядка]
    Пусть $f \colon \mathbb{R}^{n} \mapsto \mathbb{R}$ "--- дифференцируемая функция. Тогда если для любого $\vec{x} \neq \vec{x}_{0}$ $\langle \nabla f(\vec{x}), \vec{x} - \vec{x}_{0} \rangle < 0$, то точка $\vec{x}_{0}$ будет точкой глобального максимума.
\end{theorem}
\begin{proof}
    Для начала покажем, что теорема о среднем выполнена и в многомерном случае, то есть для любых $\vec{x}$ и $\vec{x}_{0}$ существует точка $\vec{z}$, лежащая на отрезке между $\vec{x}$ и $\vec{x}_{0}$, для которой выполнено, что
    \[
        f(\vec{x}) - f(\vec{x}_{0}) = \langle \nabla f(\vec{z}), \vec{x} - \vec{x}_{0} \rangle.
    \]
    Для этого параметризуем отрезок между $\vec{x}$ и $\vec{x}_{0}$ и введём функцию $g \colon [0, 1] \mapsto \mathbb{R}$, действующую по правилу
    \[
        g(t) = f(t\vec{x} + (1 - t)\vec{x}_{0}).
    \]
    Тогда $g(0) = f(\vec{x}_{0})$, $g(1) = f(\vec{x})$ и $g$ дифференцируема. Следовательно, по теореме о среднем существует точка $t' \in [0, 1]$ такая, что
    \[
        g(1) - g(0) = g'(t') = \left.\pdv{t} f(t\vec{x} + (1 - t)\vec{x}_{0})\right|_{t = t'}
    \]
    Скажем, что $t'\vec{x} + (1 - t)\vec{x}_{0} = \vec{z}$. Тогда
    \[
        f(\vec{x}) - f(\vec{x}_{0}) = \langle \nabla f(\vec{z}), \vec{x} - \vec{x}_{0} \rangle.
    \]
    Однако $\vec{x} - \vec{x}_{0} = \lambda(\vec{z} - \vec{x_{0}})$. Тогда
    \[
        f(\vec{x}) - f(\vec{x}_{0}) = \frac{1}{\lambda}\langle \nabla f(\vec{z}), \vec{z} - \vec{x}_{0} \rangle < 0.
    \]
    Тогда мы получаем, что $\vec{x}_{0}$ есть глобальный максимум.
\end{proof}

\begin{problem}
    Пусть $\vec{X} = (X_{1}, \ldots, X_{n})$ "--- выборка из распределения с плотностью
    \[
        p_{\vec{\theta}}(x) = \frac{1}{\alpha}\exp\left\{-\frac{x - \beta}{\alpha}\right\}[x \geq \beta],
    \]
    где $\vec{\theta} = (\alpha, \beta)$, $\alpha > 0$ "--- двумерный параметр. Найдите для $\vec{\theta}$ оценку максимального правдоподобия. Докажите, что полученная для $\alpha$ оценка $\hat{\alpha}_{n}(\vec{X})$ является асимптотически нормальной, и найдите ее асимптотическую дисперсию.
\end{problem}
\begin{answer}
    Оценк максимального правдоподобия имеет вид $(\overline{\vec{X}} - X_{(1)}, X_{(1)})$. Далее,
    \[
        \sqrt{n}(\hat{\alpha}_{n}(\vec{X}) - \alpha) \xrightarrow{d_{\theta}} \mathcal{N}(0, \alpha^{2}).
    \]
\end{answer}