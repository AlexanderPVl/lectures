\section{Лекция 12 от 28.11.2016 \\ Степенные ряды}

\subsection{Основные определения и свойства}
\begin{Def}
	Степенной ряд --- это функциональный ряд вида $\sum\limits_{n = 0}^{\infty}c_n (x-x_0)^n$, где $\{c_n\}$ --- \textit{последовательность коэффициентов}, а $x_0 \in \mathbb{R}$ ---  \textit{центр ряда}.
\end{Def}

Отметим, что ряд начинается с $n = 0$. Это будет важно в дальнейшем, давая возможность представлять рядами функции, в нуле (точнее, в $x_0$) не равные нулю.

В процессе всех дальнейших рассуждений в рамках этой лекции будем полагать, что $x_0 = 0$. Это не умаляет общности, так как фактически это сдвиг по оси (иными словами, замена переменной).

\begin{Theorem} [Абеля I]
	Пусть ряд $\sum\limits_{n = 0}^{\infty}c_n x^n$ сходится в точке $x_1$. Тогда $\forall x:\ |x| < |x_1|$ этот ряд сходится абсолютно. Более того, $\forall x_2 \in (0, |x_1|)$ сходимость на $[-x_2, x_2]$ --- равномерная.
\end{Theorem}
\begin{proof}
	Так как ряд $\zseries c_nx_1^n$ сходится, то его члены стремятся к нулю, а значит, $\exists C\ \forall n \in \N:\ |c_nx_1^n| < C$. 
	
	Тогда $\forall x:\ |x| < |x_1|$ верно, что 
	$$
	|c_nx^n| \leq |c_nx_1^n| \cdot \left| \dfrac{x}{x_1} \right|^n \leq C \left| \dfrac{x}{x_1} \right|^n.
	$$
	Вместе с тем, несложно заметить, что ряд $\zseries C \left|\dfrac{x}{x_1}\right|^n$ сходится как геометрическая прогрессия с $q = |x/x_1| < 1$, а значит, по признаку сравнения сходится и ряд $\zseries |c_nx^n|$, то есть ряд $\zseries c_nx^n$ сходится абсолютно.
	
	Для доказательства равномерной сходимости воспользуемся признаком Вейерштрасса:
	$$
	\forall n \in \N\ \forall x \in [-x_2; x_2]:\ |c_nx^n| \leq C\left|\frac{x}{x_1} \right|^n \leq C\left|\frac{x_2}{x_1}\right|^n.
	$$
	
	Так как ряд $\sum\limits_{n=0}^{\infty} C\left|\frac{x_2}{x_1} \right|^n$ сходится, то ряд $\sum\limits_{n = 0}^{\infty}c_n x^n$ сходится равномерно на $[-x_2; x_2]$.
\end{proof}


\begin{Def}
	Радиусом сходимости $R$ степенного ряда $\zseries c_nx^n$ называется точная верхняя грань множества модулей точек, в которых ряд сходится.
\end{Def}

\begin{Def}
Интервал $(-R, R)$ называется интервалом сходимости степенного ряда.
\end{Def}

\begin{Consequence}
	Ряд  $\sum\limits_{n = 0}^{\infty}c_n x^n$ сходится абсолютно в каждой точке интервала сходимости, расходится в каждой точке $x \not\in [-R, R]$, и более того, $\forall r \in (0, R)$ сходимость на $[-r, r]$ равномерная.
\end{Consequence}
\begin{proof}
Для $x \not \in [-R, R]$ --- следует из определения радиуса сходимости.

Пусть теперь $x \in (-R, R)$. Так как $|x|$ меньше точной верхней грани множества модулей точек сходимости, то существует такая точка $x_1$, что $\zseries c_nx_1^n$ сходится, и при этом $|x_1| > |x|$. Аналогично для равномерной сходимости на $[-r, r]$  при $r \in (0, R)$.

Теперь осталось просто воспользоваться теоремой Абеля.
\end{proof}

\subsection{Нахождение радиуса сходимости}

Факт существования у рядов радиуса сходимости --- это прекрасно, но хотелось бы уметь его находить.


\begin{Theorem} [Формула Коши--Адамара]
	Пусть $\sum\limits_{n = 0}^{\infty}c_n x^n$  --- степенной ряд. Тогда радиус сходимости этого ряда $R= \frac{1}{\varlimsup\limits_{n \to \infty} \sqrt[n]{|c_n|}}$ (полагая при $\varlimsup\limits_{n \to \infty} \sqrt[n]{|c_n|} = \infty$, что $R = 0$ и при $\varlimsup\limits_{n \to \infty} \sqrt[n]{|c_n|} = 0$, что $R = \infty$ ).
\end{Theorem}
\begin{proof}
	Заметим, что если $b_n \to b$, $b \geq 0$, то $\varlimsup\limits_{n \to \infty} (a_nb_n) = b\varlimsup\limits_{n \to \infty} a_n$.
	
	Вспомним радикальный признак Коши: пусть $\varlimsup\limits_{n \to \infty} \sqrt[n]{|c_nx^n|} = A$, тогда если $A < 1$, то ряд $\zseries |c_nx^n|$ сходится, а если $A > 1$, то расходится. Но вместе с тем, $\sqrt[n]{|c_nx^n|} = |x|\sqrt[n]{|c_n|}$.
	
	Следовательно, если $|x| < 1 / \varlimsup\limits_{n \to \infty} \sqrt[n]{|c_n|} = R$, то ряд сходится, а если $|x| > R$ --- расходится.
\end{proof}

Зная эту формулу, можно легко придумать ряд с любым радиусом сходимости.

\begin{Statement}
Пусть существует предел $\lim \left|\dfrac{c_{n+1}}{c_n}  \right| = A$. Тогда радиус сходимости ряда $\zseries c_nx^n$ равен $\dfrac{1}{A}$.
\end{Statement}

\begin{proof}
Для $x \neq 0$ рассмотрим предел $\lim\limits_{n \to \infty} \dfrac{|c_{n+1}x^{n+1}|}{|c_nx^n|} = A|x|$. Тогда, по признаку Д'Аламбера, если $A|x| < 1$, то ряд сходится, а если $A|x| > 1$, то расходится.
\end{proof}

\subsection{Поведение в концах интервала сходимости}

В концах интервала сходимости может происходить разное.
Простые примеры:
\\$\sum\limits_{n = 0}^{\infty}x^n$ --- радиус сходимости равен 1, при $x= \pm1$ ряд расходится.
\\$\sum\limits_{n = 0}^{\infty}\frac{1}{n} x^n$ --- радиус сходимости равен 1, при $x= 1$ ряд расходится, при $x = -1$ ряд сходится условно.
\\$\sum\limits_{n = 0}^{\infty}\frac{1}{n^2} x^n$ --- радиус сходимости равен 1, при $x= \pm 1$ ряд сходится абсолютно.



\begin{Theorem} [Абеля II]
	Пусть ряд $\sum\limits_{n = 0}^{\infty}c_n x^n$ сходится в точке $x_1$. Тогда он равномерно сходится на отрезке с концами $0$ и $x_1$.
\end{Theorem}

\begin{proof}
Рассмотрим $x$ из отрезка с концами 0 и $x_1$. Представим исходный ряд в уже знакомом нам виде $\zseries c_n x^n = \zseries c_n x_1^n \left|\dfrac{x}{x_1} \right|^n$.

Посмотрим на члены этого ряда как на произведение. Ряд $\zseries c_nx_1^n$ сходится (и не зависит от $x$), а последовательность $\left\{\left|x/x_1\right|^n \right\}$ либо монотонно убывает к нулю, либо постоянна (когда $x=1x_1$), и ограничена единицей. Следовательно, по признаку Абеля ряд $\zseries c_n x_1^n \left|\dfrac{x}{x_1} \right|^n$ равномерно сходится, то есть ряд $\zseries c_nx^n$ равномерно сходится на $[0, x_1]$.
\end{proof}

\begin{Consequence}
Сумма степенного ряда непрерывна на всём множестве сходимости.
\end{Consequence}

\begin{proof}
Согласно первой теореме Абеля, ряд равномерно сходится на $\left [ -r,r \right ] \subset \left ( -R, R \right )$, однако про весь интервал это точно утверждать нельзя, так как на интервале $\left ( -R, R \right )$ ряд может сходиться и неравномерно. Пусть $x_{0}\in\left ( -R, R \right )$. Выберем такое $r$, что $|x_{0}|<r<R$. Так как $x_{0}$ --- внутренняя точка отрезка $\left [ -r, r \right ]$ и на $\left [ -r, r \right ]$ ряд  сходится равномерно, то, по теореме о непрерывности суммы равномерно сходящегося ряда непрерывных функций, сумма ряда является непрерывной функцией на $\left [ -r,r \right ]$, включая точку $x_{0}$.
Таким образом, сумма ряда непрерывна во всех точках интервала $\left ( -R, R \right )$.
 
Однако, множество сходимости может включать в себя точки $\pm R$. В этом случае нам поможет вторая теорема Абеля. Согласно ей, если $R$ входит в множество сходимости, то ряд сходится равномерно на отрезке $[0, R]$ (аналогично для $-R$ и $[-R, 0]$). А значит, по всё той же теореме о непрерывности суммы равномерно сходящегося ряда непрерывных функций, сумма ряда непрерывна и в точках $\pm R$, если множество сходимости их в себя включает.
\end{proof}

\begin{Consequence}
Степенной ряд сходится равномерно на каждом отрезке, лежащем в его множестве сходимости.
\end{Consequence}



