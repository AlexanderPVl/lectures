\pagestyle{fancy}
\section{Лекция 22 от 13.03.2017 \\  Признак Дини. Принцип локализации}
\subsection{Признак Дини}
\begin{Theorem}[Признак Дини]
Пусть $f$ --- $2\pi$--периодическая функция, $f \in \Ri[-\pi, \pi]$, $x_0$ --- некоторая точка. Если существуют $S \in \R$ и $\delta > 0$ такие, что несобственный интеграл
$$
\int\limits_{0}^{\delta}\dfrac{f(x_0 + t) + f(x_0 - t) - 2S}{t}\d t 
$$
абсолютно сходится, то тогда $S_N(x_0, f) \ito S$.
\end{Theorem}
\begin{proof}
Надо доказать, что при больших $N$ маленькой является разность $|S_N(x_0, f) -~S|$. Распишем свертку:
\begin{gather*}
|S_N(x_0, f) - S| = \left| \dfrac{1}{\pi} \int\limits_{-\pi}^{\pi}f(x_0 - t)D_N(t)\d t - \dfrac{1}{\pi} \int\limits_{-\pi}^{\pi}SD_N(t)\d t \right|.
\end{gather*}
Разделим первый интеграл на два, по отрезкам $[0, \pi]$ и $[-\pi, 0]$. Во втором сделаем замену $t \to -t$ и воспользуемся четностью ядра Дирихле, то есть из $\int_{-\pi}^{0}f(x_0 - t)D_N(t)\d t$ получим $\int_0^\pi f(x_0 + t)D_N(t)\d t$.
\begin{gather*}
|S_N(x_0, f) - S| = \dfrac{1}{\pi} \left| \int\limits_0^\pi f(x_0 - t)D_n(t)\d t + \int\limits_0^{\pi} f(x_0 + t)D_N(t)\d t - 2\int\limits_0^\pi SD_N(t) \d t \right| = \\ = \dfrac{1}{\pi} \left| \int\limits_0^\pi \left( f(x_0 - t) + f(x_0 + t) - 2S  \right) D_N(t) \d t \right| \leq  \\
\leq \dfrac{1}{\pi} \left( \left| \int\limits_0^\delta \left( f(x_0 - t) + f(x_0 + t) - 2S  \right) D_N(t) \d t \right| + \left| \int\limits_\delta^\pi \left( f(x_0 - t) + f(x_0 + t) - 2S  \right) D_N(t) \d t \right| \right).
\end{gather*}
На последнем шаге мы просто разделили один интеграл на два, а так же внесли модуль внутрь суммы.

Зафиксируем произвольное $\eps > 0$ и положим $\eps_1 = \eps / 3$. Тогда найдется такое $\delta_0 > 0$, что $\displaystyle \int\limits_0^{\delta_0} \left| \dfrac{f(x_0 + t) + f(x_0 - t) - 2S}{t}\d t \right| < \eps_1$ (так как нам дана абсолютная сходимость этого интеграла на $[0, \delta]$, а значит, его <<хвосты>>, как у рядов, стремятся к нулю). Заменим теперь в последнем полученном выражении $\delta$ на $\delta_0$:
$$
\dfrac{1}{\pi} \left( \left| \int\limits_0^{\delta_0} \left( f(x_0 - t) + f(x_0 + t) - 2S  \right) D_N(t) \d t \right| + \left| \int\limits_{\delta_0}^\pi \left( f(x_0 - t) + f(x_0 + t) - 2S  \right) D_N(t) \d t \right| \right).
$$
Напомним, мы хотим, чтобы оба этих интеграла стремились к нулю.

Рассмотрим первый:
\begin{gather*}
\dfrac{1}{\pi} \left| \int\limits_0^{\delta_0} ( f(x_0 - t) + f(x_0 + t) - 2S  ) D_N(t) \d t \right| \leq \dfrac{1}{\pi}  \int\limits_0^{\delta_0} \left| \dfrac{(f(x_0 - t) + f(x_0 + t) - 2S) \sin (N + 1/n)t \cdot t}{2 \sin t/2 \cdot t} \right|\d t \leq \\
\leq \dfrac{1}{\pi}  \int\limits_0^{\delta_0} \left| \dfrac{f(x_0 - t) + f(x_0 + t) - 2S}{t} \right|  \cdot \dfrac{1 \cdot t}{2 \frac{t}{2} \frac{2}{\pi}} \d t < \dfrac{\eps_1}{2} < \eps_1.
\end{gather*}
Здесь мы воспользовались ограничением синуса снизу: $\sin u \geq \dfrac{2u}{\pi}$.

Со вторым интегралом все не так просто. Нам понадобится \textit{фокус} --- свести все к коэффициентам Фурье, которые на бесконечности сходятся к нулю.

Напомним, что $I_{[a, b]}(t)$ --- индикатор события $t \in [a, b]$.
\begin{gather*}
\dfrac{1}{\pi} \int\limits_{\delta_0}^{\pi} (f(x_0 - t) + f(x_0 + t) - 2S)D_N(t)\d t =  \dfrac{1}{\pi} \int\limits_{\delta_0}^{\pi} \dfrac{(f(x_0 - t) + f(x_0 + t) - 2S) \sin (N + 1/n)t}{2 \sin t/2} \d t = \\ 
= \dfrac{1}{\pi} \int\limits_{-\pi}^{\pi} \dfrac{f(x_0 - t) + f(x_0 + t) - 2S}{2 \sin t/2} \cdot I_{[\delta_0, \pi]}(t) \cdot \underbrace{(\sin t/2 \cos Nt + \cos t/2 \sin Nt)}_{=\sin(N + 1/2)t}\d t =
\end{gather*}
\vspace{-0.5cm}
\begin{multline}
= \dfrac{1}{\pi} \int\limits_{-\pi}^{\pi} \underbrace{\dfrac{f(x_0 - t) + f(x_0 + t) - 2S}{2} \cdot I_{[\delta_0, \pi]}(t)}_{g(t) \in \Ri[-\pi, \pi]} \cos Nt \d t + \\ + \dfrac{1}{\pi} \int\limits_{-\pi}^{\pi} \underbrace{\dfrac{(f(x_0 - t) + f(x_0 + t) - 2S)\cos t/2}{2\sin t/2} \cdot I_{[\delta_0, \pi]}(t)}_{h(t) \in \Ri[-\pi, \pi]} \sin Nt \d t =
\end{multline}
$$
= a_N(g) + b_N(h) \ito 0.
$$
Отметим, что несмотря на то, что в определении функции $h(t)$ в знаменателе есть $\sin t/2$, он не будет создавать проблем, так как благодаря индикатору функция отделена от нуля.

Таким образом, введя функции $g(t)$ и $h(t)$, мы свели второй интеграл к сумме двух коэффициентов Фурье. А значит, $\exists N_0: \forall N > N_0\ |a_N(g) + b_N(h)| < \eps_1$.

Итого, собрав все воедино:
\begin{gather*}
|S_N(x_0, f) - S| \leq \eps_1 + \eps_1 < \eps.
\end{gather*}
А значит, $S_N(x_0, f) \ito S$.
\end{proof}
\begin{Comment}
Такой странный фокус в конце доказательства нужен из-за того, что ядро Дирихле очень плохо ведет себя около нуля.
\end{Comment}
\subsection{Следствия и продолжения признака Дини}
Хорошо, мы доказали теорему. Но что вообще имеет смысл брать в качестве такого $S$?
\begin{Comment}
Если $f$ дифференцируема в точке $x_0$, то условие из признака Дини выполняется для $S = f(x_0)$.
\end{Comment}
\begin{proof}
Подинтегральная функция из условия имеет вид 
$$
\dfrac{f(x_0 + t) - f(x_0)}{t} + \dfrac{f(x_0 - t) - f(x_0)}{t}.
$$
Первая часть при $t \to 0+$ стремится к $f'(x_0)$. Во второй части удобно сделать замену $\Delta = -t$ и тогда становится понятно, что при $t \to 0-$ выражение стремится к $-f'(x_0)$.

Итого, около нуля функция ограничена, следовательно, сходится.
\end{proof}

Фактически мы только что доказали, что во всех точках, где функция дифференцируема, ее ряд Фурье сходится к самой функции. Однако условие дифференцируемости можно ослабить.
\begin{Comment}
Если в точке $x_0$ существуют правая и левая производные, то условие из признака Дини выполняется для $S = f(x_0)$.
\end{Comment}
\begin{Comment}
Пусть $x_0$ --- точка разрыва I--го рода, и существуют пределы $\lim\limits_{t \to 0+}\dfrac{f(x_0 + t) - f(x_0 + 0)}{t}$ и $\lim\limits_{t \to 0+} \dfrac{f(x_0-t) - f(x_0 - 0)}{t}$ (фактически --- правая и левая производные). Тогда условие из признака Дини выполняется для $S = \dfrac{f(x_0 + 0) + f(x_0 - 0)}{2}$.
\end{Comment}
А вот только непрерывности не хватит. При большом желании можно придумать контрпример (но он от дьявола).

\subsection{Принцип локализации Римана}
Теперь сформулируем теорему, которая на самом деле является простым следствием признака Дини.

\begin{Theorem}[Принцип локализации Римана]
Пусть $f$ и $g$ --- интегрируемые по Риману на $[-\pi, \pi]$ $2\pi$--периодические функции, совпадающие в некоторой окрестности точки $x_0$. Тогда в каждой точке этой окрестности ряды Фурье $f$ и $g$ по тригонометрической системе равносходятся (то есть сходятся или расходятся одновременно, а если сходятся, то к одной величине).
\end{Theorem}
\begin{proof}
Ряды Фурье равносходятся, если $S_N(x, f) - S_N(x, g) = S_N(x, f - g) \to 0$. Но в рассматриваемой окрестности $f - g \equiv 0$, и тогда по признаку Дини $S_N(x, f - g) \ito 0$.
\end{proof}

На самом деле, это самый неожиданный результат по рядам Фурье, который мы получали в данном курсе. Действительно, при составлении ряда мы используем \textit{всю} функцию на $[-\pi, \pi]$, и если ее поменять где-то на концах отрезка, то коэффициенты Фурье поменяются и мы получим дугой ряд. Но не смотря на это, значения в совпадающих точках останутся теми же.