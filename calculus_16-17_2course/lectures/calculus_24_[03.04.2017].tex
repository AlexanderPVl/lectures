\pagestyle{fancy}
\section{Лекция 24 от 03.04.2017 \\ Простые множества. Измеримость по Жордану}

\subsection{Простые множества}

\begin{Def}
Брус $\Pi$ называется \textit{невырожденным}, если его мера не равна нулю.
\end{Def}

\begin{Statement}
Если $I_1, \ldots, I_n$ --- невырожденные интервалы, то $\Pi = I_1 \times \ldots \times I_n$ --- открытое множество (открытый брус).

Если $I_1, \ldots, I_n$ --- невырожденные отрезки, то $\Pi = I_1 \times \ldots \times I_n$ --- замкнутое множество (замкнутый брус).
\end{Statement}

\begin{Def}
Множество, представимое в виде конечного объединения брусов --- \textit{простое множество}.

Простое множество представимо в виде конечного объединения \textit{попарно непересекающихся} брусов.
\end{Def}

\begin{Statement}
Объединение, пересечение и разность простых множеств --- простое множество.
\end{Statement}

\begin{proof}\ \\
Объединение --- все ясно.

Пересечение: $\left( \bigcup\limits^J\Pi_j \right) \cap \left( \bigcup\limits^K\PPi_k \right) = \bigcup\limits^{J, K}\left(\underbrace{\Pi_j \cap \PPi_k}_{\text{брус}} \right)$.

Разность: $\left( \bigcup\limits^J\Pi_j \right) \setminus \left( \bigcup\limits^K\PPi_k \right) = \bigcup\limits^{J}\left( \Pi_j \setminus \bigcup\limits^K\PPi_k \right) = \bigcup\limits^J \overbrace{\left( \bigcap\limits^K\left(\underbrace{\Pi_j \setminus \PPi_k}_{\text{брус}} \right) \right)}^{\text{простое множество}}$.
\end{proof}

\begin{Def}
Пусть $P = \bigsqcup\limits^J\Pi_j$ --- простое множество. Тогда $\mu P = \sum\limits^J \mu\Pi_j$.
\end{Def}

\begin{Def}
Брусы \textit{не перекрываются}, если мера их пересечения равна нулю.
\end{Def}

\begin{Statement}
Если брус $\Pi$ является конечным объединением попарно не перекрывающихся брусьев $\pi_j$, то $\mu \Pi = \sum\limits^J \mu \pi_j$.
\end{Statement}
\begin{proof}[Идея доказательства]
Проблема исключительно в том, что <<нарезка>> бруса $\Pi$ может быть нерегулярной (не <<сеточкой>>). Но в таком случае можно просто продлить все разрезы и получить таки регулярную нарезку.

Рассмотрим регулярную нарезку в двумерном случае. Пусть ширину бруса мы разрезали на отрезки $I_1, \ldots, I_n$, а длину на $\I_1, \ldots, \I_m$. Тогда $\mu\Pi = (|I_1| + \ldots + |I_n|) \cdot (|\I_1| + \ldots + |\I_m|) \hm= \sum\limits^n\sum\limits^m|I_i||\I_k| = \sum\mu\pi_j$.
\end{proof}

\begin{Statement}
Определение меры простого множества корректно.
\end{Statement}
\begin{proof}
Пусть простое множество представимо как объедение двух разных наборов брусьев: $P = \bigsqcup\limits^J\Pi_j = \bigsqcup\limits^K\PPi_k$. Пусть также $\Pi_j \cap \PPi_k = \Delta_{jk}$. Тогда $\sum\limits^J \mu\Pi_j = \sum\limits^J\left( \underbrace{\sum\limits^K \mu \Delta_{jk}}_{\mu\Pi_j}  \right)$ и $\sum\limits^K\mu \PPi_k = \sum\limits^K\left( \underbrace{\sum\limits^J\mu \Delta_{jk}}_{\mu\PPi_k} \right)$. Но фактически это одна и та же сумма, следовательно, мера простого равна $\sum\limits^j\mu\Pi_j = \mu P  = \sum\limits_K\PPi_k$ и определена корректно.
\end{proof}
\begin{Statement}
Пусть $P_1$ и $P_2$ --- простые множества. Тогда
\begin{itemize}
\item $\mu(P_1 \sqcup P_2) = \mu P_1 + \mu P_2$;
\item $\mu(P_1 \cup P_2) = \mu P_1 + \mu P_2 - \mu(P_1 \cap P_2)$;
\item если $P_1 \subset P_2$, то $\mu P_1 \leq \mu P_2$.
\end{itemize}
\end{Statement}

\subsection{Измеримость по Жордану}

Вспомним наконец, что наша задача --- определить площадь/объем.

Пусть $A$ --- ограниченное подмножество $\R^n$.

\begin{Def}\ \\
\textit{Нижняя мера Жордана} множества $A$: $\mu_*A = \sup\limits_{\substack{P \text{ --- простые},\\ P \subset A}}. \mu P$.

\textit{Верхняя мера Жордана} множества $A$: $\mu^*A = \inf\limits_{\substack{P \text{ --- простые},\\ A \subset P}}. \mu P$.
\end{Def}

\begin{Def}
Множество $A$ называется \textit{измеримым по Жордану (по $J$)}, если $\mu_*A = \mu^*A$. В этом случае мера множества $A$ равна $\mu A = \mu^*A$ (иногда это обозначается как $\mu_J A$).
\end{Def}

\begin{Statement}
Для любого ограниченного множества $A$ нижняя мера Жордана не больше верхней меры Жордана: $\mu_*A \leq \mu^*A$.
\end{Statement}
\begin{proof}
Пусть $p$ и $P$ --- произвольные простые множества такие, что $p \subset A \subset P$. Очевидно, что $\mu p \leq \mu P$. Соответственно, $\mu_* A = \sup\limits_{p \subset A} \mu p \leq \mu P$. Аналогично, $\mu p \leq \inf\limits_{A \subset P} = \mu^* A$. Итого, $\mu_* A \leq \mu^* A$.
\end{proof}

\begin{Statement}
Если $A$ --- простое, то его мера по Жордану совпадает с его мерой в старом смысле.
\end{Statement}

\begin{Statement}[Критерий измеримости 1]
Ограниченное множество $A$ измеримо по Жордану тогда и только тогда, когда $\forall \eps > 0 \exists p, P$ --- простые множества такие, что $p \subset A \subset P$ и $\mu P - \mu p < \eps$.
\end{Statement}
\begin{proof}
Очевидно.
\end{proof}

\begin{Comment}
В этом утверждении множество $p$ можно считать открытым, а $P$ --- замкнутым множеством.
\end{Comment}

\begin{Statement}[Критерий измеримости 2]
Пусть $A$ --- ограниченное подмножество $\R^n$. Тогда следующие утверждения эквивалентны:
\begin{enumerate}
\item $A$ --- измеримо по Жордану;
\item $\mu^* \delta A = 0$, где $\delta A$ --- граница множества.
\end{enumerate}
\end{Statement}
\begin{proof}\ \\
$(1) \Rightarrow (2)$. Зафиксируем произвольное $\eps > 0$. Тогда существует такое открытое множество $p$ и замкнутое множество $P$ такие, что $p \subset A \subset P$ и $\mu P- \mu p < \eps$. Но $\delta A \subset P$ и $P \subset \mathrm{int} A$, где $\mathrm{int} A$ --- совокупность внутренних точек множества. Получается, что $P \setminus p \supset \delta A$. Но $P \setminus p$ --- простое множество меры меньше $\eps$, то есть имеющее меру ноль. Значит, $\mu^* \delta A = 0$.

$(2) \Rightarrow (1)$. В следующий раз.
\end{proof}

\begin{Comment}
На основе этого критерия строится классический пример неизмеримого по Жордану множества. Берем квадрат, и пусть в $A$ входят все его внутренние точки с рациональными координатами. Границей такого множества будет весь куб, и его мера очевидно не равна нулю, соответственно, множество $A$ не измеримо.
\end{Comment}
\begin{Comment}
Из сходимости по Жордану следует сходимость по Лебегу, но не наоборот. Это связано с тем, что в сходимости по Лебегу используется не более чем \textit{счетное} объединение, а по Жордану --- \textit{конечное}.
\end{Comment}