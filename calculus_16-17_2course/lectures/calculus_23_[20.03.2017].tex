\pagestyle{fancy}
\section{Лекция 23 от 20.03.2017 \\ Средние Чезаро. Многомерный интеграл Римана (начало). Брусья и их свойства}
	\subsection{Средние Чезаро. Ядро Фрейра}
	Начнем сразу с определений в тему лекции:
	\begin{Def}
		Средними Чезаро (иногда называемыми средними Фейера) порядка $N$ функции $f$ в точке $x$ будем называть:
		$$
			\sigma_N(x) = \cfrac{1}{N} \sum\limits_{n =1}^{N -1}S_n(x,f),
		$$
		 где $S_n(x,f)$ есть $n$-ая частичная сумма ряда Фурье по тригонометрической системе.
	\end{Def}
	Вспоминая материал предыдущего раза, можем переписать $S_n$ как
	$$
		S_n(x,f) = \cfrac{1}{\pi} f*D_n(x),
	$$
	 то есть как свёртку с ядром Дини.
	\begin{Theorem}
		Пусть $f$ --- непрерывная на $\R$ $2\pi$-периодическая функция. Тогда её средние Фейера сходятся к ней равномерно на $\R$. 
	\end{Theorem}
	Для доказательства введем понятие ядра Фейера.
	\begin{Def}
		Выражение
		$$
			K_n(t) = \cfrac{1}{N}\sum\limits_{n=0}^{N-1}D_n(t)
		$$
		называют ядром Фейера.
	\end{Def}
	Теперь перейдем к доказательству
	\begin{proof}
		Попытаемся переписать средние Фейера в удобной форме как свертку функций:
        \begin{multline}	
			\sigma_n(x,f) = \cfrac{1}{N}\sum\limits_{n=0}^{N-1}S_n(x,f) = \cfrac{1}{N}\sum\limits_{n=0}^{N-1}\cfrac{1}{\pi}\int\limits_{-\pi}^{\pi}f(x-t)D_N(t)dt = \\ 
            = \cfrac{1}{N\pi}\int\limits_{-\pi}^{\pi}f(x-t)\sum\limits_{n=0}^{N-1}D_n(t)dt =   \cfrac{1}{\pi}\int\limits_{-\pi}^{\pi}f(x-t)K_N(t)dt = \cfrac{1}{\pi}f*K_n(x)
        \end{multline}
		Нам будет вполне достаточно доказать, что $\left\{\frac{1}{\pi}K_n(t)\right\}_{n=1}^{\infty}$ --- \lmao. 
		$2\pi$-периодичность и непрерывность очевидны. Далее
		$$
			\int\limits_{-\pi}^{\pi}D_n(t) = \pi \Rightarrow \cfrac{1}{\pi} \int\limits_{-\pi}^{\pi}K_N(t) dt = 1.
		$$
		Далее докажем неотрицательность. Для всех $t \in (0; \pi]$ мы можем записать ядро Дирихле как
		$$
			D_n(t) = \cfrac{\sin((n + 1/2)t)}{2\sin(t/2)}.
		$$
		Пользуясь тригонометрическими формулами произведения синусов и понижения степени, получаем
		\begin{align}
			K_N(t) =& \cfrac{1}{N}\sum\limits_{n=0}^{N-1} D_n(t) = \cfrac{1}{N}\sum\limits_{n}^{N-1} \cfrac{\sin((n + 1/2)t)}{2\sin(t/2)}\cdot\cfrac{\sin(t/2)}{\sin(t/2)} = \\
			=& \cfrac{1}{N}\cdot \cfrac{1 - \cos(t) + \cos(t) - \cos(2t) + \cos(2t) - \ldots - \cos(Nt)}{2\cdot 2\sin^2(t/2)} = \cfrac{1}{N}\cdot \cfrac{1 - \cos(Nt)}{4\sin^2(t/2)} =\\
			=&\cfrac{1}{N}\cdot \cfrac{\sin^2\left(\frac{Nt}{2}\right)}{2\sin(t/2)}
	    \end{align}
	    Отсюда следует неотрицательность.
	    Далее для произвольного $\delta \in (0;\pi)$ грубо оценим интеграл
	    \begin{gather}
	    	\cfrac{1}{\pi}\left(\int\limits_{-\pi}^{-\delta}K_N(t)dt + \int\limits_{\delta}^{\pi}K_N(t)dt\right) = \cfrac{2}{\pi}\int\limits_{\delta}^{\pi}\cfrac{\sin^2(Nt)}{2N\sin^2(t/2)}dt \leqslant \cfrac{2}{\pi}\int\limits_{\delta}^{\pi} \cfrac{1}{2N\sin^2(t/2)}dt \leqslant \cfrac{2}{\pi}\cfrac{\pi}{2N \sin^2(\delta/2)}.
	    \end{gather}
	    И видно в последнем выражении, что оно стремится к 0 при $N \to \infty.$
	\end{proof}
	На этом мы завершим разговор о рядах Фурье.
	\subsection{Многомерный интеграл Римана (начало). \\Брусья и их свойства}
	Мы, было дело, уже вводили понятие определенного интеграла Римана по отрезку. Для этого мы для функции $f$ разбивали отрезок $[a; b]$ на $N$ подотрезков, выбирали по одной точке на отрезках и составляли интегральную сумму Римана
	$$
		\sigma(f, \tau, \xi) = \sum\limits_{n =1}^{N}f(\xi_n)|\Delta_n|.
	$$
	Хотелось бы поступать подобным образом и с пространством произвольной размерности, не только с одномерной прямой. Например в случае с двумерным пространством и интегралом по области, разбивать область на кусочки и брать не длину, а площадь каждого кусочка. 
	\par Стоп. С длиной отрезка все понятно --- это просто разность правого конца и левого. А что делать с площадью (а что такое площадь четырёх- и более многомерной области)? Попробуем ввести её!
	\begin{Def}
		Брусом в $\R^n$ будем называть декартово произведение $I_1\times I_2\times\ldots \times I_n$, где $I_i,\ i = 1\dots n$ --- ограниченные промежутки.
	\end{Def}
	Брус невырожден, если все $I_i$ невырождены.
	\begin{Def}
		Простое множество в $\R^n$ --- это объединение конечного числа брусов. 
	\end{Def}
	И сразу же теорема.
	\begin{Theorem}
		Простое множество можно представить в виде конечного числа попарно непересекающихся брусов.
	\end{Theorem} 
	\begin{proof}
		Пусть $\Pi = I_1 \times \ldots I_n$ и $\widetilde{\Pi} = \I_1 \times \ldots \I_n$ --- два бруса в $\R^n$. Их пересечение $\Pi \cap \widetilde{\Pi} = (I_1 \cap \I_1) \times \ldots (I_n\cap \I_n)$ --- тоже брус, поскольку пересечение промежутков --- промежуток.
		Докажем сначала, что разность двух брусов $\Pi \setminus \widetilde{\Pi}$ представима как конечное объединение попарно непересекающихся брусов в количестве не более $2n$ штук. Докажем это индукцией по $n$.
		\par Базовый случай $n=1$ оставим как упражнение (это и впрямь очевидно).
		\par Пусть для всех $n \leqslant k$ уже все доказано. Докажем для $n = k + 1$. 
		\par Вспомним, что для произвольных множеств $A, B, C, D$ верно
		$$
			(A\times B) \setminus (C \times D) = (A \setminus C) \times B \sqcup (A\cap C)\times (B \setminus D),
		$$
		где значок $\sqcup$ означает объединение непересекающихся множеств.
		Обозначим за $\Pi' = I_1 \times \ldots \times I_k$, $\widetilde{\Pi}' = \I_1 \times \ldots \times \I_k$. Тогда
		$$
			\Pi \setminus \widetilde{\Pi} = (\Pi' \times I_{k+1}) \setminus (\widetilde{\Pi}' \times \I_{k + 1}) = (\Pi' \setminus \widetilde{\Pi}')\times I_{k + 1} \sqcup (\Pi' \cap \widetilde{\Pi}')\times (I_{k + 1} \setminus \I_{k + 1})
		$$
		Применяя предположение индукции, получаем то, что нужно.
		\par Теперь пусть $\Pi_1, \ldots, \Pi_k$ --- брусы. Докажем, что их объединение можно представить как объединение попарно непересекающихся брусов. Сделаем это индукцией по $k$. Для $k = 1$ это верно. Докажем для $k = k + 1$.
		$$
			\Pi_1 \cup \Pi_2 \ldots \cup \Pi_{k + 1} = \Pi_1 \cup \ldots \cup \Pi_{k} \sqcup \left(\Pi_{k + 1} \setminus \bigcup\limits_{n =1}^k\Pi_k\right)
		$$
		Первое объединение представимо в искомом виде по предположению индукции. Вторая разность есть ни что иное как
		$$
			\left(\Pi_{k + 1} \setminus \bigcup\limits_{n =1}^k\Pi_k\right) = \bigcap\limits_{n=1}^{k}(\Pi_{k + 1} \setminus \Pi_n).
		$$
		Выражение в скобках, по доказанному выше, есть объединение попарно непересекающихся брусов, а пересечение брусов --- брус. Получаем то, что нужно.
	\end{proof}