\pagestyle{fancy}
\section{Лекция 21 от 06.03.2017 \\ Равномерная сходимость рядов Фурье. Ядро Дирихле.}
\subsection{Равномерная сходимость рядов Фурье}
Мы уже пользовались в задачах почленным интегрированием ряда Фурье. Сформулируем общее утверждение.
\begin{Statement}
	Если $f$ --- интегрируемая по Риману на отрезку $[-\pi;\pi]$  функция, и её ряд Фурье по тригонометрической системе есть
	$$
		\cfrac{a_0(f)}{2} + \sum\limits_{n=1}^{\infty}(a_n(f)\cos(nx) + b_n(f)\sin(nx)),
	$$
	 то на любом подотрезке $[a;b] \subset [-\pi; \pi]$ интеграл по этому отрезку равен сумме числового ряда, полученного почленным интегрированием из ряда Фурье.
\end{Statement}
\begin{proof}
    Достаточно применить равенство Парсеваля (теорема \ref{pars}, четвёртый пункт) и заметить, что $\int\limits_a^bf(x)\dx$ равен $(f, I\{a; b\})$, где $I\{a; b\}$ --- индикатор отрезка $[a; b]$.
\end{proof}
Изначально сходимость ряда Фурье в метрике $\mathcal{R}^2[-\pi; \pi]$ не гарантирует нам ни поточечной, ни равномерной сходимости ряда. Помогает следующая теорема
\begin{Theorem}
	Пусть $f(x)$ --- непрерывно дифференцируемая на $\R$ $2\pi$-периодическая функция. Тогда её ряд Фурье сходится к ней равномерно.
\end{Theorem}
\begin{proof}
	Заметим, что ряд $\sum\limits_{n=1}^{\infty} |a_n(f)| + |b_n(f)|$ сходится. Действительно,
	$$
		|a_n(f)| = \left|\cfrac{b_n(f')}{n}\right| \leqslant \cfrac{1}{2} |b_n^2(f')| + \cfrac{1}{n^2}.
	$$
	Но
	\begin{enumerate}
		\item $\sum\limits_{n=1}^{\infty} b_n^2(f')$ сходится по неравенству Бесселя.
		\item $\sum\limits_{n=1}^{\infty} \cfrac{1}{n^2}$ тоже сходится.
	\end{enumerate}
	Следовательно, сходится и $\sum\limits_{n=1}^{\infty}\left(b_n^2(f') + \cfrac{1}{n^2}\right)$, потому сходится и $\sum \limits_{n=1}^{\infty}|a_n(f)|$. Аналогично сходится и ряд $\sum \limits_{n=1}^{\infty}|b_n(f)|$. Таким образом, ряд 
	$$
		\cfrac{a_0(f)}{2} + \sum\limits_{n=1}^{\infty}(a_n(f)\cos(nx)+ b_n(f)\sin(nx)) \overset{\R}{\rightrightarrows} g(x)
	$$
	для некоторой функции $g$ по признаку Вейерштрасса. Поэтому мы можем утверждать, что
	$$
		\cfrac{a_0(f)}{2} + \sum\limits_{n=1}^{\infty}(a_n(f)\cos(nx)+ b_n(f)\sin(nx)) \overset{\mathcal{R}^2}{=} g.
	$$
	По теореме единственности для ортогональной системы получаем, что коэффициенты Фурье функций $f$ и $g$ по тригонометрической системе совпадают.
	\par Таким образом, $f-g$ ортогональна всем элементам тригонометрической системы. Но тригонометрическая система замкнута, а, следовательно, полна. Следовательно, $f-g \overset{\mathcal{R}^2}{=} 0$. А в силу непрерывности $f \equiv g$.
\end{proof}

\subsection{Ядро Дирихле}
Посмотрим, чему будет равна частичная сумма тригонометрического ряда Фурье функции $f$ в точке $x$.
\begin{gather}
	S_N(x,f) = \cfrac{a_0}{2} + \sum\limits_{n=1}^{N}(a_n(f) \cos(nx) + b_n(f) \sin(nx)) = \cfrac{1}{2\pi}\int\limits_{-\pi}^{\pi} f(t)dt +\\
	+\sum\limits_{n=1}^{N}\left(\left( \cfrac{1}{\pi}\int\limits_{-\pi}^{\pi}f(t)\cos(nt) dt\right)\cos(nx) +
	\left( \cfrac{1}{\pi}\int\limits_{-\pi}^{\pi}f(t)\sin(nt) dt\right)\sin(nx) \right) = \\
	= [\cos(nt)\cos(nx) + \sin(nt)\sin(nx) = \cos(n(x-t))] 
	=\cfrac{1}{\pi}\left( \int\limits_{-\pi}^{\pi}f(t) \cdot \left( \cfrac{1}{2} + \sum \limits_{n=1}^{N}\cos(n(x-t))\right)dt\right).
\end{gather}
Ведём обозначение:
$$
	D_N(u) = \cfrac{1}{2} + \sum\limits_{n=1}^{N}\cos (nu).
$$
\begin{Def}
	$D_N(u)$ называется $N$-м ядром Дирихле. 
\end{Def}
Тогда мы можем переписать нашу частичную сумму как свёртку с ядром Дирихле:
\[
	S_N(x, f) = \cfrac{1}{\pi} f*D_N(x).\tag{*}
    \label{21aster}
\]
Но определение ядра через сумму не очень удобно само по себе. Попробуем выписать её в более удобном виде. Пусть $\sin(u/2) \neq 0$. Тогда мы вправе умножить и разделить на него.
\begin{gather}
	D_N(u) = \cfrac{1}{2} + \sum\limits_{n=1}^{N}\cos(nu) = \cfrac{\sin(u/2)}{2\sin(u/2)} + \sum\limits_{n=1}^{\infty} \cfrac{\cos(nu)\sin(u/2)}{\sin(u/2)} =\\
	= \left[\cos(nu)\sin(u/2) = \cfrac{\sin((n + 1/2)u) - \sin((n-1/2)u)}{2}\right]=\\
	= \cfrac{\sin(u/2) - \sin(u/2) + \sin(3u/2) - \sin(3u/2) + \ldots + \sin((N + 1/2)u)}{2\sin(u/2)} = \cfrac{\sin((N + 1/2)u)}{2\sin(u/2)}.
\end{gather}
Из равенства \eqref{21aster} сразу следует, что $D_N(u)$ --- непрерывная четная $2\pi$-периодическая функция и $\int\limits_{-\pi}^{\pi}D_N(t) = \pi.$