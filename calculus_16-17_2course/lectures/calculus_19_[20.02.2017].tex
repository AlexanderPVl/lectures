\pagestyle{fancy}
\section{Лекция 19 от 20.02.2017 \\  Теорема Вейерштрасса. Замкнутость тригонометрической системы.}
\subsection{Теорема Вейерштрасса}
В прошлый раз мы сформулировали теорему Вейерштрасса:
\begin{Theorem}
	Для всякой непрерывной на всей числовой прямой $2\pi$-периодической функции для всякого $\eps>0$ существует такой тригонометрический многочлен $T(x)$, что в каждой точке $x \in \R$ $$
	 |f(x)-T(x)| < \eps.
	$$
\end{Theorem}
	Для доказательства мы ввели понятие свёртки двух функций
	$$
		f*g(x) = \int\limits_{-\pi}^{\pi} f(t)g(x-t)dt = \int\limits_{-\pi}^{\pi}f(x-t)g(t)dt.
	$$
	Теперь сформулируем и докажем три леммы.
\begin{Lemma}
	Пусть $\{K_n(t)\}_{n=1}^{\infty}$ --- \lmao, $f(x)$ --- непрерывная $2\pi$-периодическая функция, $g_n(x) = f(x)*K_n(x)$. Тогда 
	$$
		\forall \eps>0\ \exists N\in \N\; \forall n>N\; \forall x\in\R\; |f(x) - g_n(x)| < \eps.
	$$
	Или, что то же самое, $g_n(x)$ сходятся к $f(x)$ равномерно на $\R$.
	\begin{proof}
		Зафиксируем произвольное $\eps > 0$. Так как $f$ --- периодическая и непрерывная на $\R$, она равномерно непрерывна на $\R$. Найдём такое $C>0$, что $|f(x)| <C\; \forall x$. Для $\eps_1 = \eps/2$ найдём такое $\delta > 0$, что
		$$
			\forall x_1, x_2\in  \R\; |x_1-x_2| < \delta \Rightarrow |f(x_1) - f(x_2)| < \eps_1.
		$$
		Не ограничивая общности, считаем, что $\delta \in (0; \pi)$.
		Для $\eps_2 = \frac{\eps}{5C}$ найдём такое $N\in \N$, что
		$$
			\forall n>N\; \int\limits_{-\pi}^{-\delta}  K_n(t)dt+ \int\limits_{\delta}^{\pi} K_n(t)dt < \eps_2.
		$$
		Для произвольного $x\in \R$ и $n >N$ оценим
		\begin{align*}
			&|g_n(x) - f(x)|= \left| \int \limits_{-\pi}^{\pi}f(x-t)K_n(t)dt - \int \limits_{-\pi}^{\pi}f(x)K_n(t)dt\right| \leqslant \int \limits_{-\pi}^{\pi}|f(x-t) -f(x)|K_n(t) dt =\\&= \int \limits_{-\delta}^{\delta}|f(x-t) - f(x)|K_n(t)dt + \int \limits_{-\pi}^{-\delta}|f(x-t) - f(t)|K_n(t)dt + \int \limits_{\delta}^{\pi}|f(x-t) - f(x)|K_n(t)dt \leqslant \\ &\leqslant
			\eps_1 \int \limits_{-\delta}^{\delta}K_n(t)dt + 2C\eps_2 < \eps/2 + \eps/2 = \eps.
		\end{align*}
	\end{proof}
\end{Lemma}
\begin{Lemma}
	Пусть $f(x)$ --- $2\pi$-периодическая непрерывная функция, а $T(x)$ --- тригонометрический многочлен. Тогда $f*T(x)$ --- тоже тригонометрический многочлен.
\end{Lemma}
\begin{proof}
	\begin{align*}
	f*T(x) = \int \limits_{-\pi}^{\pi}f(x)T(x - t)dt = \int \limits_{-\pi}^{\pi}f(t)\frac{\alpha_0}{2}dt +\sum\limits_{n=1}^{N} \int \limits_{-\pi}^{\pi}f(t)(\alpha_n(\cos(nx)\cos(nt) + \sin(nt)\sin(nx)) +\\+ \beta_n(\sin(nx)\cos(nt) - \cos(nx)\sin(nt)))dt = \frac{\widetilde{\alpha}_0}{2} + \sum \limits_{n=1}^{N}\left(\widetilde{\alpha}_n\cos(nx) + \widetilde{\beta}_n\sin(nx)\right)
	\end{align*}
	Здесь мы вынесли за знак интеграла выражения, не зависящие от $t$, а сами интегралы посчитали и обозначили их как константы $\widetilde{\alpha}_n$ и $\widetilde{\beta}_n$.
\end{proof}
\begin{Lemma}
	Определим последовательность функций $\{K_n(t)\}_{n=1}^{\infty}$ следующим образом:
	$$
		K_n(t) = \cfrac{\left(\frac{1+\cos(t)}{2}\right)^n}{\int\limits_{-\pi}^{\pi} \left(\frac{1 + \cos(u)}{2}\right)^ndu}.
	$$
	Тогда эта последовательность является неотрицательной $2\pi$-периодической непрерывной аппроксимативной единицей, каждая функция которой --- тригонометрический многочлен. 
\end{Lemma}
\begin{proof}
    Почти все свойства этой последовательности как аппроксимативной единицы выглядят очевидно, нужно проверить только, что для всякого $\delta > 0$
	$$
		\int\limits_{-\pi}^{-\delta}K_n(t) dt + \int\limits_{\delta}^{\pi}K_n(t) dt \underset{n\to \infty}{\longrightarrow} 0.
	$$

Разберёмся отдельно с числителем и знаменателем.
Числитель --- чётная функция, поэтому оба слагаемых равны, и работать будем лишь с одним.
$$
	\int\limits_{\delta}^{\pi}\left(\cfrac{1 + \cos(t)}{2}\right)^n dt = \begin{bmatrix} q = \frac{1 + \cos(\delta)}{2}\\
	q \in (0;1)
	\end{bmatrix} \leqslant q^n\pi.
$$
Перейдём к знаменателю
\begin{align*}
	\int\limits_{-\pi}^{\pi} \left(\cfrac{1 + \cos u}{2}\right)^n du = 2\int\limits_{0}^{\pi} \left(\cfrac{1 + \cos u}{2}\right)^n du = 4 \int\limits_{0}^{\pi} \left(\cos\frac{u}{2}\right)^{2n} d\frac{u}{2} = \\ =
	4 \int\limits_{0}^{\frac{\pi}{2}} \left(\cos y\right)^{2n} dy = \begin{bmatrix}
		\cos y = z\\
		dy = \cfrac{-dz}{\sqrt{1 - z^2}}
		\end{bmatrix} = 4 \int\limits_{0}^{1} z^{2n} \cfrac{1}{\sqrt{1-z^2}}dz \geqslant 4\int\limits_{0}^{1} z^{2n}dz = \cfrac{4}{2n+1}.
\end{align*}
Итого получаем
$$
	\int\limits_{\delta}^{\pi}K_n(t) dt \leqslant \cfrac{q^n\pi}{4}(2n+1) \underset{n \to \infty}{\longrightarrow} 0.
$$
\end{proof}
Из этих трёх лемм получаем сразу же теорему Вейерштрасса. Из неё же сразу несколько следствий.
\begin{Consequence}
	Тригонометрическая система --- замкнутая ОГС в $\mathcal{R}^2[-\pi;\pi].$
\end{Consequence}
\begin{Consequence}
	Для всякой функции $f \in \mathcal{R}^2[-\pi;\pi]$ тригонометрический ряд Фурье сходится к $f$ в $\mathcal{R}^2[-\pi; \pi]$ (то есть $|| f - S_N||_2\underset{N \to \infty}{\longrightarrow} 0$).
\end{Consequence}
\subsection{Несколько слов о равенстве Парсеваля}
Вспомним формулировку равенства Парсеваля:
$$
	||x||^2 = \sum\limits_{n=1}^{\infty} \x_n^2 ||e_n||^2.
$$
Как же будет оно выглядеть в случае тригонометрической системы? 
$$\int\limits_{-\pi}^{\pi} f^2(x)dx =2\pi\left( \frac{a_0}{2}\right)^2 + \sum\limits_{n=1}^{\infty}(a_n^2\pi + b_n^2\pi)
$$
\begin{Consequence}
	$$
	\forall f \in \mathcal{R}^2[-\pi;\pi]\; \frac{1}{\pi} \int\limits_{-\pi}^{\pi} f^2(x)dx = \frac{a_0^2}{2} + \sum\limits_{n=1}^{\infty}(a_n^2 + b_n^2),
	$$ где $a_n, b_n$ --- коэффициенты разложения в ряд Фурье по тригонометрической системе.
\end{Consequence}
\begin{Examples}
	Выведем формулу для суммы ряда обратных квадратов. Пусть $f(x) = x$. Тогда $a_n = 0$, ибо функция нечётная. Вычислим $b_n$:
	\begin{gather*}
		\frac{1}{\pi} \int \limits_{-\pi}^{\pi} x \sin(nx)dx = \frac{-2}{\pi n} \int \limits_{0}^{\pi}x d\cos(nx) = \frac{-2}{\pi}\left(x \cos(nx)\right)\biggr|_0^{\pi} - \int \limits_{0}^{\pi}\cos(nx) dx = \frac{2\cdot(-1)^{n+1}}{n}.
	\end{gather*}
	Интеграл $x^2$ по отрезку $[-\pi;\pi]$ фукнции $x^2$ равен $\cfrac{x^3}{3}\biggr|_{-\pi}^{\pi} = \cfrac{2\pi^3}{3}$. Отсюда получаем по равенству Парсеваля:
    $$
		\cfrac{2\pi^2}{3} = \sum\limits_{n=1}^{\infty} \cfrac{4}{n^2}.
    $$
    Следовательно,
    $$
		\sum\limits_{n=1}^{\infty} \cfrac{1}{n^2} = \cfrac{\pi^2}{6}.
	$$
\end{Examples}
\subsection{От тригонометрических многочленов к алгебраическим}
Пусть функция $f(x)$ непрерывна на отрезке $[a;b]$. Можно ли её так же эффективно приблизить не тригонометрическим, а просто алгебраическим многочленом? Ответ на этот вопрос положителен, и даёт его ещё одна теорема, названная именем Вейерштрасса:
\begin{Theorem}[Вейерштрасс]
	Для всякой непрерывной на отрезке функции $f(x)$ для любого $\eps > 0$ существует такой многочлен $P(x)$, что $|P(x) - f(x)| < \eps$ во всех точках этого отрезка.
\end{Theorem}
Введём ещё пару определений:
\begin{Def}
	\textit{Свёрткой} функций $f$ и $g$ на $\R$ называется
	$$
		f*g(x) = \int\limits_{\infty}^{\infty} f(x-t)g(t) \d t.
	$$
\end{Def}
\begin{Def}
	Последовательность $\{K_n(t)\}_{n=1}^{\infty}$ называется непрерывной неотрицательной аппроксимативной единицей, если все $K_n$ непрерывны, неотрицательны и
	\begin{enumerate}
		\item $\int\limits_{-\infty}^{+\infty}K_n(t) dt = 1$.
		\item $\forall \delta > 0 \; \int\limits_{-\infty}^{-\delta}K_n(t)dt + \int\limits_{\delta}^{+\infty}K_n(t) dt \underset{n \to \infty}{\longrightarrow}0$.
	\end{enumerate}
\end{Def}
