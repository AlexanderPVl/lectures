\section{Лекция 18 от 13.02.2017 \\ Тригонометрические многочлены и что-то ещё}
Вспомним, что мы хотели доказать замкнутость тригонометрической системы в $\mathcal{R}^2[-\pi;\pi]$.
\begin{Def}
\textit{Тригонометрическим многочленом} называется выражение вида
$$
\frac{\alpha_0}{2} + \sum\limits_{n = 1}^N\left(\alpha_n \cos(nx) +\beta \sin(nx)\right).
$$
\end{Def}
\begin{Theorem}[Вейерштрасса]
    Для любой непрерывной на $\R$ $2\pi$-периодической функции \\$\forall \eps>0\  \exists T$ --- тригонометрический многочлен, такой что $\forall x \in \R \ |f(x)-T(x)|<\eps.$
\end{Theorem}
Доказывать теорему Вейерштрасса мы сейчас не будем, но зато выясним, что из неё, когда мы её докажем, будет сразу следовать замкнутость тригонометрической системы. Для этого докажем несколько красиво подводящих нас к этому лемм.

\begin{Lemma}
   $\forall f \in \mathcal{R}^2[-\pi;\pi] \forall \eps > 0$ существует функция $ h(x)$ --- кусочно-постоянная на $[-\pi, \pi]$, такая что $||f-g||_2 < \eps$.
\end{Lemma}
\begin{proof}
    Зафиксируем произвольное $\eps >0$. Найдём $C>0$, такое что $\forall x \in [-\pi, \pi]\ |f(x)|~<~C$. Положим $\eps_1 = \frac{\eps^2}{100C}$. Существует разбиение  $\tau$ отрезка $[-\pi, \pi]$ такое что $S^*(\tau, f)- s_*(\tau, f) = \sum\limits_{n=1}^N (M_n-m_n)|\Delta_n|< \eps_1.$\\
    Определим $h(x) = M_n$ при $x \in \Delta_n$, и $h_l(x) = m_n$ при $x \in \Delta_n$. Тогда 
     \begin{multline} ||f-h||_2 = \sqrt{\int\limits_{-\pi}^{\pi}(f(x)-h(x))^2dx} \leq \sqrt{2C\int\limits_{-\pi}^{\pi}(h(x)-f(x))dx} \leq \sqrt{2C\int\limits_{-\pi}^{\pi}(h(x)-h_l(x))dx}\\ \leq \sqrt{2C\sum\limits_{n=1}^N (M_n-m_n)|\Delta_n|} < \sqrt{2C\eps_1} < \eps. \end{multline}
\end{proof}

\begin{Lemma}
    $\forall h$ --- кусочно-постоянной функции на $[-\pi, \pi]$, $ \forall \eps > 0\  $ существует функция $ g(x)\in C[-\pi, \pi]$, такая что $||h-g||_2 < \eps$.
\end{Lemma}
\begin{proof}
    $\exists C: |g| < C.$
    
    Для всякого достаточно большого $m \in \N$ определим $g_m(x)$ следующим образом: на расстоянии больше или равном $\frac{1}{m}$ от точек разрыва функции $h$ наша $g_m(x) = h(x)$, а в  $\frac{1}{m}$-окрестности точек разрыва $g_m$ --- линейная функция, соединяющая значения $h$ слева и справа от разрыва.
    
    Обозначив количество точек разрыва за $N$ мы можем оценить $||h-g_m||_2$
   $$ ||h-g_m||_2 = \sqrt{\int\limits_{-\pi}^{\pi}(h(x)-g_m(x))^2dx}\leq\left(N\cdot\frac{2}{m}\cdot4C^2 \right) .$$
   А значит $||h-g_m||_2\rarr0$ при $m \rarr 0$.
\end{proof}


\begin{Lemma}
    $\forall g \in C[-\pi, \pi]\ \forall \eps > 0$\ существует $\tilde{g}(x)$ --- $2\pi$-периодическая, непрерывная на $\R$ на $[-\pi, \pi]$ функция, такая что $||f-g||_2 < \eps$.
\end{Lemma}
\begin{proof}
    Аналогично предыдущему доказательству, изменим функцию на линейную на $(\pi- \frac{1}{m}; \pi)$.
\end{proof}
Теперь наконец воспользуемся ранее сформулированной теоремой Вейерштрасса.
\begin{Lemma}
    $\forall \eps > 0\ \forall\tilde{g}(x)$ --- $2\pi$-периодической, непрерывной на $\R$ на $[-\pi, \pi]$ функции существует тригонометрический многочлен $T$, такой что $||\tilde{g}-T||_2 < \eps$.
\end{Lemma}
\begin{proof}
   Возьмём $\eps_1 = \frac{\eps}{\sqrt{100\pi}}$.
   Пусть $T$ --- многочлен из теоремы Вейерштрасса для $\tilde{g}$ и $\eps_1$.
   Тогда $$||\tilde{g}-T||_2=\sqrt{\int\limits_{-\pi}^{\pi}(\tilde{g}(x)-T)^2dx}\leq \sqrt{\int\limits_{-\pi}^{\pi}(\eps_1)^2dx} = \sqrt{2\pi(\eps_1)^2}=\eps_1\sqrt{2\pi} < \eps.$$
\end{proof}
\begin{Def}
    \textit{Сверткой} функций $f$ и $g$ на $\mathcal{R}^2[-\pi;\pi]$ называется
	$$
    f*g(x) = \int\limits_{-\pi}^{\pi} f(t)g(x-t)dt = \int\limits_{-\pi}^{\pi}f(x-t)g(t)dt.
    $$
\end{Def}

\begin{Def}
    Последовательность функций $\{K_n(t)\}_{n=1}^\infty$ называется  \textit{ $2\pi$-периодической непрерывной неотрицательной аппроксимативной единицей} если
    \begin{enumerate}
        \item все $K_n$ --- $2\pi$-периодические, непрерывные и неотрицательные функции на $\R$.
        \item $\forall n \in \N\  \int\limits_{-\pi}^{\pi} K_n(t)dt = 1.$        \item $\forall \delta \in (0; \pi)\  \int\limits_{-\pi}^{-\delta} K_n(t)dt + \int\limits_{\delta}^{\pi} K_n(t)dt\rarr 0$ при $n \rarr \infty.$
    \end{enumerate}
\end{Def}
