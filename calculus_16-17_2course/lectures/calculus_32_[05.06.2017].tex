\section{Лекция 32 от 05.06.2017 \\ Формула Грина. Поверхностные интегралы}
\subsection{Формула Грина}
\begin{Def}
Область $D$ \textit{элементарна относительно $Ox$}, если 
$$D = \{ (x, y) \mid x \in (a, b), y \in (\phi(x), \psi(x) \},$$
где $\phi, \psi \in C^{1}[a, b]$ и $\psi > \phi$ на $[a, b]$.
\end{Def}
\begin{Def}
Область $D$ \textit{элементарна относительно $Oy$}, если 
$$D = \{ (x, y) \mid y \in (c, d), x \in (\phi(y), \psi(y) \},$$
где $\phi, \psi \in C^{1}[c, d]$ и $\psi > \phi$ на $[c, d]$.
\end{Def}

\begin{Lemma}
Пусть $D$ --- элементарная относительно $Ox$ область, $\Gamma$ --- граница $D$, ориентированная положительно, $P \in C^1(\overline{D})$. Тогда $\oint\limits_\Gamma P\dx = \iint\limits_D - P'_y\dx \d y$.
\end{Lemma}
\begin{proof}
Просто поймем, как считаются обе части равенства, вспомнив, как устроена элементарная относительно $OX$ область.

Левая часть:
$$
\int\limits_a^b P(x,\phi(x))\dx - \int\limits_a^bP(x, \psi(x))\dx.
$$
Правая часть:
$$
\int\limits_a^b\dx\int\limits_{\phi(x)}^{\psi(x)} -P'y\d y = \int\limits_a^b-P(x, \phi(x)) - P(x, \phi(x))\dx.
$$
Как видим, они равны.
\end{proof}

\begin{Consequence}
Пусть область $D$ разбивается на конечное число подобластей элементарных относительно  $Ox$, $\Gamma$ --- граница $D$, ориентированная положительно, $P \in C^1(\overline{D})$. Тогда $\oint\limits_\Gamma P\dx = \iint\limits_D - P'_y\dx \d y$.
\end{Consequence}

Почти аналогично:
\begin{Consequence}
Пусть область $D$ разбивается на конечное число подобластей элементарных относительно  $Oy$, $\Gamma$ --- граница $D$, ориентированная положительно, $Q \in C^1(\overline{D})$. Тогда $\oint\limits_\Gamma Q\d y = \iint\limits_D Q'_x\dx \d y$.
\end{Consequence}

Знаки подынтегральной функции разные, так как мы фактически в разные стороны бежим от дальней границе к ближней, в данной области.

\begin{Theorem}[Формула Грина]
Пусть $D$ --- область на плоскости, ограниченная кривой $\Gamma$,  ориентированной положительно, и допускающая разбиение как на конечное число подобластей, элементарных относительно $Ox$, так и на конечное количество подобластей, элементарных относительно $Oy$ (такие разбиения могут быть различны). Пусть также $(P, Q) \in C'(\overline{D})$. Тогда
$$
\int\limits_{\Gamma} P\dx + Q\d  y = \int\limits_{\overline{D}} (Q'_x - P'_y)\dx\d y.
$$
\end{Theorem}
\begin{proof}
Прямо следует из доказанного выше.
\end{proof}

\begin{Consequence}
Если $(P, Q)$ --- потенциальное, то $P'_y = Q'_x$.
\end{Consequence}
\begin{proof}
Как мы знаем, для потенциального поля кратный интеграл зависит только от начальной и конечной точки, а также кратный интеграл по любой \textit{простой} (без самопересечений) замкнутой прямой равен нулю. Отсюда и следует, что $Q'_x - P'_y = 0$.
\end{proof}

Проблема в том, что обратное не верно. А ведь так хотелось бы... Что ж, давайте поймем, чего нам не хватает для равносильности.

\begin{Def}
Область односвязна, если любая простая замкнутая кривая в этой области стягиваема в точку.
\end{Def}
Если рассматривается область на плоскости, то это можно сформулировать более человеческим языком: в области нет дырок.

\begin{Theorem}
Пусть $D$ --- односвязная область на плоскости, $(P, Q) \in C^2(D)$. Тогда то, что поле $(P, Q)$ потенциально, равносильно тому, что $P'_y = Q'_x$.
\end{Theorem}
\begin{proof}
Осталось доказать в обратную сторону.

Пусть $\Gamma$ --- простая замкнутая кусочно-гладка ломанная в $D$, $\Omega$ --- область, ограничиваемая $\Gamma$. $\overline \Omega \in D$ в силу односвязности. Тогда: 
$$
\int\limits_{\Gamma} P\dx + Q\d y = \iint\overbrace{(Q'_x - P'_y)}_{=0}\dx \d y = 0.
$$
Итого, поле потенциально.
\end{proof}

\subsection{Гладкие поверхности, их площадь}
Казалось бы, площадь поверхности интуитивно хочется определить через, например, приближение фигуры многоугольниками, то есть примерно так же, как мы вводили длину кривой, оценивая ее вписанными ломанными. Проблема в том, что такое определение все-таки несостоятельно. В качестве обоснования приведем классический пример.

\begin{Examples}[Сапог Шварца]
Будем извращаться над цилиндром. Его площадь мы знаем, так как можем развернуть и проверить: $2\pi R H$.

Теперь давайте разрежем цилиндр на $m$ одинаковых слоев плоскостями, перпендикулярными образующим. Разобьем самую верхнюю окружность на $n$ равных дуг. Следующую после верхней окружность разобьем тоже на $n$ дуг, но так, чтобы точки
разреза попадали ровно на середины дуг верхней окружности. Соединим эти точки разреза так, чтобы нарезать боковую стенку цилиндра на треугольники. В общем, разметим так всю сторону.

Попытаемся теперь посчитать суммарную площадь таких треугольников:
\begin{gather*}
S = \lim\limits_{n, m \to \infty} m \cdot 2n \cdot S_\vartriangle = \lim 2mn R \sin \dfrac{\pi}{n}\sqrt{R^2(1 - \cos \pi/n + (H/m)^2)} =\\= \lim 2\pi R \sqrt{H^2 + R^2m^2 \pi^4/(4n^4) + o(1/n^4)} = [m = An^2] = 2\pi R\sqrt{H^2 + A^2R^2\pi^/4}.
\end{gather*}
А ведь $A$ может быть любым... Получается, площадь такой фигуры можно сделать сколь угодно большим!

В общем, плохое приближение площади поверхности цилиндра. Очень плохое.
\end{Examples}

Как же тогда определить площадь поверхности?

\begin{Def}
Пусть в системе координат $(u, v)$ у нас есть область $D$, ограниченная простой замкнутой кусочно-гладкой кривой $\Gamma$. Инъективное отображение $\phi$ отображает $D$ в область $\Sigma$ в системе координат $(x, y, z)$: $\overline\Sigma = \phi(\overline D)$. Тогда поверхность $\Sigma$ называется \textit{гладкой}, если ее матрица Якоби имеет ранг два:
$$
\rk J = \rk \begin{pmatrix}
\phi'_u \\ \phi'_v
\end{pmatrix} = \rk \begin{pmatrix}
x'_u & y'_u & z'_u \\
x'_v & y'_v & z'_v
\end{pmatrix} = 2.
$$
\end{Def}

Введем обозначение $N(u, v) = \phi'_u \times \phi'_v$, где $x \times y$ --- векторное произведение. 
\begin{Def}
$||N(u, v)||$ --- локальный коэффициент изменения площади.
\end{Def}

\begin{Def}
Площадь гладкой поверхности $\Sigma$ определяется как
$$
\iint\limits_D||N(u, v)||\d u \d v.
$$
\end{Def}

\subsection{Поверхностные интегралы I рода}

Теперь пусть на $\Sigma$ задано скалярное поле $f(x, y, z)$.

\begin{Def}
\textit{Поверхностным интегралом I-го рода} $f$ по $\Sigma$ называют интеграл
$$
\iint\limits_{\Sigma}f \d S = \iint\limits_D f(x(u, v), y(u, v), z(u, v)) ||N(u, v)|| \d u \d v.
$$
\end{Def}

\begin{Comment}
Очевидно, что тогда $\iint\limits_{\Sigma}1 \d S$ просто равно площади $\Sigma$. 
\end{Comment}

Пусть у нас есть еще отображение $C$ между координатами $(u, v)$ и $(\tilde u, \tilde v')$, которое переводит область $D$ в область $\widetilde{D}$ и обратно. Тогда:
\begin{gather*}
\phi'_u = \frac{\partial \tilde u}{\partial u} \phi'_{\tilde u} + \frac{\partial \tilde v}{\partial u} \phi'_{\tilde v}, \\
\phi'_v = \frac{\partial \tilde u}{\partial v} \phi'_{\tilde u} + \frac{\partial \tilde v}{\partial v} \phi'_{\tilde v}, \\
\phi'_u \times \phi'_v = \left( \frac{\partial \tilde u}{\partial u} \cdot \frac{\partial \tilde v}{\partial v} - \frac{\partial \tilde v}{\partial u} \cdot \frac{\partial \tilde u}{\partial v} \right) = \begin{vmatrix}
\frac{\partial \tilde u}{\partial u} & \frac{\partial \tilde u}{\partial v} \\
\frac{\partial \tilde v}{\partial u} & \frac{\partial \tilde v}{\partial v}
\end{vmatrix}
\end{gather*} 

\subsection{Поверхностные интегралы II рода}

\begin{Def}
\textit{Ориентация} простого гладкого куска поверхности $\Sigma$ это
задание на нем непрерывного поля единичных нормалей $\overline n$.
\end{Def}

Пусть на $\Sigma$ заданы ориентация $\overline n$ и векторное поле $F(P, Q, R)$. 

\begin{Def}
\textit{Поверхностным интегралом II-го рода} называют интеграл
$$
\iint\limits_{\Sigma} \overline F \d \overline S = \iint\limits_{\Sigma} P\d y \d z + Q \dx \d z+ R \dx \d y = \iint\limits_{\Sigma} (\overline F, \overline n) \d S.
$$
\end{Def}

\begin{Def}
Смежные куски гладких поверхностей называются \textit{согласованными}, если ориентация на их границе противоположна.
\end{Def}

\subsection{Формулы для поверхностных интегралов}

\begin{Def}
Область $V$ в $\R^3$ \textit{элементарна относительно $Oz$}, если 
$$V = \{ (x, y, z) \mid (x, y) \in D, z \in (\phi(y), \psi(y) \},$$
 где $D$ --- область в $\R^2$, ограниченная простым кусочно-гладким замкнутым контуром,  где $\phi, \psi \in C^{1}(D)$, $\psi > \phi$ на $D$.
\end{Def}

\begin{Theorem}[Формула Гаусса--Остроградского]
Пусть $V$ --- область в $\R^3$, ограниченная кусочно-гладкой поверхностью $\Sigma$, ориентированной полем \underline{внешних} нормалей, $\overline F = (P, Q, R) \in C^1(\overline V)$ и $V$ допускает разбиение на конечное число подобластей, элементарных относительно $Ox$, $Oy$ и $Oz$ (это могут быть разные разбиения). Тогда
$$
\iint\limits_{\Sigma}\overline F \d \overline S = \iiint\limits_V \underbrace{(P'_x + Q'_y + R'_z)}_{\mathrm{div} \overline F}\dx \d y \d z.
$$
\end{Theorem}
\begin{proof}
Перенесем в $\R^3$ формулу Грина, получив следующее:
$$
\iint\limits_{\Sigma} R\dx\d y = \iint\limits_{\Sigma} (0, 0, R)\d \overline S = \iint\limits_V R'_z \d x \d y \d z. 
$$
Доказывается она в целом аналогично, так что опустим это. Аналогичные формулы можно привести для $R$ и $Q$. Тогда доказательство формулы Гаусса--Остроградского становится тривиальным, ведь $(P, Q, R) = (P, 0, 0) + (0, Q, 0) + (0, 0, R)$.
\end{proof}

\begin{Theorem}[Формула Стокса]
Пусть $\overline{\Sigma}\subset {\mathbb R}^3_{x,y,z}$ --- гладкий кусок поверхности с границей
$\Gamma$, заданный отображением $(x(u,v),y(u,v),z(u,v))$ замыкания
области $D\subset{\mathbb R}^2_{u,v}$ c границей $\gamma$ (где $\gamma$ --- простой замкнутый
кусочно--гладкий контур), причем это отображение
дважды непрерывно дифференцируемо на $\overline{D}$. Пусть также
$F(x,y,z)=(P(x,y,z), Q(x,y,z),R(x,y,z))$ --- векторное поле, непрерывно дифференцируемое
на $\overline{\Sigma}$, поверхность $\Sigma$ и ее граница $\Gamma$ ориентированы согласованно.
Тогда
$$
\int\limits_{\Gamma} P\,dx + Q\,dy + R\,dz = \int\limits_{\Sigma}\!\!\!\!\int {\rm rot} F\, \overline{dS},
$$
где
$$
{\rm rot} F = (R'_y - Q'_z, P'_z - R'_x, Q'_x - P'_y) = {\rm det}
\left(
\begin{array}{ccc}
i & j & k \\
\frac{\partial}{\partial x} & \frac{\partial}{\partial y} & \frac{\partial}{\partial z} \\
P & Q & R \\
\end{array}
\right).
$$
\end{Theorem}
\begin{proof}
Для сокращения выкладок докажем теорему для поля вида
$(P,0,0)$ --- аналогичным образом она доказывается для полей вида $(0,Q,0)$ и
$(0,0,R)$, а сложение этих трех результатов и дает требуемое утверждение.
Заметим, что для векторного поля $F=(P,0,0)$ его ротор равен $(0, P'_z, -P'_y)$.

Пусть $e_u (u,v) = (x'_u (u,v), y'_u (u,v), z'_u (u,v))$,
$e_v = (x'_v (u,v), y'_v (u,v), z'_v (u,v))$,
$$
N(u,v) = [e_u(u,v) \times e_v (u,v)] =
(y'_u z'_v - y'_v z'_u, x'_v z'_u - x'_u z'_v, x'_u y'_v- x'_v y'_u).
$$
Пусть также $(u(t), v(t))$ ($t\in[t_0, t_1]$) --- положительная параметризация кривой $\gamma$.

Предположим, что поверхность $\Sigma$ ориентирована полем единичных нормалей $\frac{N}{\|N\|}$,
тогда
$$
(x(u(t),v(t)), y(u(t),v(t)), z(u(t),v(t)))
$$
--- параметризация $\Gamma$,
ориентированная согласованно с $\Sigma$. В противном случае --- когда поверхность
$\Sigma$ ориентирована полем единичных нормалей $\left(-\frac{N}{\|N\|}\right)$ --- согласованная
ориентация $\Gamma$ также противоположна, то есть оставшийся
случай сводится к рассматриваемому просто сменой знака в левой и правой частях
доказываемого равенства.

Заметим, что левая часть доказываемого равенства равна
$$
\int\limits_{t_0}^{t_1} P(x(u(t),v(t)), y(u(t),v(t)), z(u(t),v(t)))\,\, d x(u(t),v(t)) =
$$
$$
= \int\limits_{t_0}^{t_1} P\bigl(x(u(t),v(t)), y(u(t),v(t)), z(u(t),v(t)\bigr)
\bigl(x'_u (u(t),v(t)) u'(t) + x'_v (u(t),v(t)) v'(t)\bigr)\,dt =
$$
$$
= \int\limits_{\gamma} P(x(u,v), y(u,v), z(u,v)) x'_u (u,v)\,du + 
P(x(u,v), y(u,v), z(u,v)) x'_v (u,v)\,dv = 
$$
$$
= \int\limits_{\gamma} \Bigl(P(x(u,v), y(u,v), z(u,v)) \,x'_u(u,v), \,\,\,P(x(u,v), y(u,v), z(u,v)) \,x'_v (u,v)\Bigr)
\,\overline{ds}.
$$
Правая часть доказываемого равенства равна
$$
\int\limits_{D}\!\!\!\!\int \bigl({\rm rot}\, F (x(u,v), y(u,v), z(u,v)) , N(u,v) \bigr)\, du\,dv =
$$
$$
= \int\limits_{D}\!\!\!\!\int \Bigl(P'_z \bigl(x(u,v), y(u,v), z(u,v)\bigr) \bigl(x'_v (u,v) z'_u (u,v) - x'_u (u,v) z'_v (u,v) \bigr) -
$$
$$
{} - P'_y \bigl(x(u,v), y(u,v), z(u,v)\bigr) \bigl(x'_u (u,v) y'_v (u,v) - x'_v(u,v) y'_u (u,v) \bigr) \Bigr)\,du\,dv =
$$
$$
= \int\limits_{D}\!\!\!\!\int
\Bigl(P'_z \,\,(x'_v z'_u - x'_u z'_v ) \, - P'_y \,\,(x'_u y'_v  - x'_v y'_u)\Bigr) \,du\,dv
$$
(в последнем интеграле для упрощения восприятия просто удалено явное указание на то, что все функции
являются функциями переменных $u$ и $v$).

Для того, чтобы показать равенство левой и правой частей, остается
применить формулу Грина:
$$
\Bigl(P\bigl(x(u,v), y(u,v), z(u,v)\bigr)\, x'_v (u,v)\Bigr)'_u - \Bigl(P\bigl(x(u,v), y(u,v), z(u,v)\bigr)\, x'_u (u,v)\Bigr)'_v = 
$$
$$
= \Bigl(\bigl(P'_x x'_u + P'_y y'_u + P'_z z'_u\bigr) \, x'_v + P\, x''_{vu} \Bigr)-
\Bigl(\bigl(P'_x x'_v + P'_y y'_v + P'_z z'_v\bigr) \,x'_u + P\, x''_{uv}\Bigr).
$$
Так как отображение $x (u,v)$ по условию дважды непрерывно дифференцируемо,
смешанные частные производные $x''_{vu}$ и $x''_{uv}$ равны. Соответственно,
слагаемые $P x''_{vu}$ и $-P x''_{uv}$ в сумме дают ноль, также взаимно уничтожаются
слагаемые $P'_x \,x'_u \,x'_v$, и остается выражение
$$
\Bigl(\bigl(P'_y y'_u + P'_z z'_u\bigr) \, x'_v - \bigl(P'_y y'_v + P'_z z'_v\bigr) \,x'_u \Bigr),
$$
в точности равное выражению под двойны интегралом, к которому была сведена правая часть доказываемой формулы.
\end{proof}







