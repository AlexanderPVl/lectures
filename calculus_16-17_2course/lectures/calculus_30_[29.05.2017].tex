\section{Лекция 30 от 29.05.2017 \\ Замена переменной, интегрирование индикатора}

\subsection{Замена переменной в кратном интеграле}
Пусть у нас есть две системы координат в $n$--мерном пространстве, $\{x_1, \ldots, x_n\}$ и $\{ t_1, \ldots, t_n \}$, и $\Omega_x$ и $\Omega_t$ --- две области (т.е. открытые связные множества) в этих координатах соответственно. Пусть также эти области не независимы, и существует такое биективное отображение $\phi: \Omega_x \to \Omega_t$, причем $\phi \in C(\Omega_x)$. Но вот проблема -- обратная функция $\phi^{-1}$ не обязательно дифференцируема, а без этого внятной замены координат в интеграле не сделаешь.

Пусть для удобства наша функция $\phi$ покоординатно складывается из других функций: $\phi(x_1, \ldots, x_n) = (\phi_1(x_1, \ldots, x_n), \ldots, \phi_n(x_1, \ldots, x_n))$. Рассмотрим тогда соответствующий ей определитель Якоби:
$$
J(\overline x) = \det \left(\frac{\partial \phi_i}{\partial x_k} \right)_{i, k = 1 \ldots n}.
$$

\begin{Statement}
Чтобы функция $\phi^{-1}$ тоже была дифференцируемой, достаточно, чтобы ее определитель Якоби не обращался в ноль на $\Omega_x$.
\end{Statement}

Пусть $\overline A_x \in \Omega_x$, $A_t = \phi(A_x)$ и $\overline A_t \in \Omega_t$. Рассмотрим функцию $f: A_t \to \R$, ограниченную на $A_t$.

\begin{Statement}
$A_x$ измеримо по Жордану тогда и только тогда, когда $A_t$ измеримо по Жордану.
\end{Statement}

Пусть далее $A_x$ измеримо.

\begin{Statement}
Функция $f(\overline t) \in \Ri(A_t)$ тогда и только тогда, когда $f(\phi(\overline x)) \cdot|J(\overline x)| \in \Ri(A_x)$. Причем в случае интегрирования эти интегралы равны: $\int\limits_{A_t} f(\overline t) \d t = \int\limits_{A_x} f(\overline x) \cdot |J(\overline{x})| \d x$.
\end{Statement}
\begin{proof}
Расскажем идею доказательства второй части.

Разбиению множества $A_x$ соответствует разбиение множества $A_t$ (просто воспользовавшись отображением $\phi$). Соответственно, мы хотим показать, что следующие суммы примерно равны (приближение будет тем лучше, чем более мелко разбиение):
$$
\sum f(\xi_j) \cdot |J(\phi^{-1}(\xi_j))| \cdot \mu \phi^{-1}(\Delta_j) \approx \sum f(\xi_j) \mu \Delta_j.
$$
Поскольку мы потребовали, чтобы $J(x) \neq 0$, то у $J(x)$ невырождена линейная составляющая. Кусочки $\Delta_j$ маленькие, так что мы хорошо можем приблизить его линейной частью:
$$
\mu \phi^{-1}(\Delta_j) \equiv \mu \Delta_j \cdot \underbrace{\left| \det \dfrac{\d x_j}{\d t_k} \right|}_{|J(\xi_j)|}.
$$

Вот примерно из этого и следует желаемое нами приближение. В целом, для экзамена достаточно понимать (и донести) идею, что при мелких разбиениях коэффициенты изменения меры равны якобиану.
\end{proof}

Рассмотрим самую популярную замену в $\R^2$ --- полярную, и попробуем перевести в нее обычный круг. Казалось бы, простейшая фигура, но уже тут мы сталкиваемся с проблемами и вообще говоря нам надо вырезать все $\eps$--окрестность $OX$, чтобы мы могли воспользоваться утверждениями выше. Хотя вообще говоря, если рассматриваемая функция непрерывна, то этот разрез можно устремить к нулю, и в пределе мы получаем что хотели изначально.

\subsection{Индикатор в кратных интегралах}

Ладно, завершающая вещь про кратные интегралы. Пусть $\Pi \in \R^n$ --- брус, $A \in \Pi$ и $I_A$ --- индикатор множества $A$.

\begin{Statement}
$I_A \in \Ri(\Pi)$ тогда и только тогда, когда $A$ --- измеримо, при этом $\mu A = \int\limits_\Pi I_A dx$.
\end{Statement}
\begin{proof}
Без доказательства. Хотя в одну сторону очевидно.
\end{proof}
Пусть $A$ --- измеримое подмножество $\R^n$, $f: A \to [0, +\infty)$ --- ограниченная функция. Тогда $A_f = \{ (x, y) \in \R^{n+1} \mid x\in  A, y \in [i, f(x)] \}$ --- многомерный аналог криволинейной трапеции.
\begin{Statement}
$A_f$ измеримо тогда и только тогда, когда $f\in \Ri(A)$, и в случае измеримости $\mu A_f = \int\limits_A f\dx$. 
\end{Statement}

\begin{Examples}
А теперь давайте посчитаем интеграл $I = \int\limits_{-\infty}^{+\infty} e^{-x^2}dx$. И сделаем это с помощью кратных интегралов.

С одной стороны:
$$
\iint_{\R^2} e^{-x^2 -y^2}\dx \d y = \int_{-\infty}^{+\infty}\dx \int_{-\infty}^{+\infty} e^{-x^2}e^{-y^2}\d y = I^2.
$$
Но с другой:
$$
\iint_{\R^2} e^{-x^2 -y^2}\dx \d y = \int_0^{2\pi}\d \phi \int_0^\infty e^{-r^2}r\d r = 2\pi\left( -\dfrac{1}{2} e^{-r^2} \Big|_{r=0}^\infty \right) = \pi.
$$
Отсюда получаем, что $I = \sqrt{\pi}$.
\end{Examples}














