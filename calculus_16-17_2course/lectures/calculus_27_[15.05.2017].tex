\pagestyle{fancy}
\section{Лекции 27-28 от 15.05.2017 \\ Критерий Дарбу}
Пусть $A$ --- непустое измеримое множество, и функция $f\colon A \to \R$ ограничена на $A$, $T = \{A_k\}_{k =1}^{K}$ --- разбиение множества $A$. Обозначим за $m_i = \inf_{A_i} f, M_i = \sup_{A_i} f$.
\begin{Def}
    Верхней и нижней суммами Дарбу называются соответственно $S^*$ и $s_*$, определённые следующим образом:
\begin{gather}
    S^*(f, T) = \sum\limits_{i =1}^{K} M_i \mu A_i,\\
    s_*(f, T) = \sum\limits_{i =1}^{K} m_i \mu A_i.
\end{gather}
\end{Def}
\begin{Def}
    Разбиение $T_1 = \{B_j\}_{j =1}^{J}$ называется измельчением разбиения $T = \{A_k\}_{k=1}^{K}$ если для всякого множества $B_j$ из разбиения $T_1$ существует такое множество $A_k$ из разбиения $T$, что $B_j \subset A_k$. 
\end{Def}
Заметим, что если $B_j \subset A_{k_1}$ и $B_j \subset A_{k_2}$ при $k_1 \neq k_2$, то $\mu B_j = 0$. Далее будем подразумевать под $B_j$ множества, составляющие измельчение разбиения $A$ на множества $A_k$. 
\begin{Statement}
    $\sum \limits_{j\colon B_j \subset A_k}\mu B_j = \mu A_k$.
\end{Statement}
\begin{proof}
    Обозначим за $\widetilde{A}_k = \bigcup\limits_{j\colon B_j \subset A_k}B_j$. Очевидно, что $\widetilde{A}_k \subset A_k$. Рассуждаем от противного: пусть $\mu \widetilde{A}_k < \mu A_k$, или, что то же самое $ \mu(A_k\setminus \widetilde{A}_k ) > 0$. Тогда найдётся такое $l$, что $\mu (B_l \cap (A_k \setminus \widetilde{A}_k)) > 0$ ( $A_k \setminus \widetilde{A}_k = \bigcup\limits_{l}((A_k \setminus \widetilde{A}_k)\cap B_l)$, и если бы $\mu((A_k \setminus \widetilde{A}_k)\cap B_l) = 0$ для всех $l$, то получили бы, что $\mu(A_k \setminus \widetilde{A}_k) = 0$). Значит, найдётся такой $k_1 \neq k$, что $B_l \subset A_k$ и $\mu(A_k \cap A_{k_1}) \geqslant \mu(A_k \cap B_l) > 0$. Противоречие.
\end{proof}
Пусть $T_1 = \{A_k\}_{k =1}^{K}$ и $T_2 = \{\widetilde{A}_j\}_{j=1}^{J}$ --- два разбиения множества $A$. Тогда существует разбиение множества, являющееся измельчением $\widetilde{T}$ для обоих разбиений сразу, определённое следующим образом:
$$
    \widetilde{T} =  \{A_k \cap \widetilde{A}_j\}_{k, j}:  A_k \cap \widetilde{A}_j \neq \emptyset.
$$
\begin{Lemma}
    Если $T_1 = \{B_k\}_{k=1}^{K}$ --- измельчение $T = \{A_j\}_{j =1}^{J}$, то 
    $$
        S^* (f, T) \geqslant S^*(f, T_1),
    $$
    $$
    s^* (f, T) \leqslant s^*(f, T_1).
    $$
\end{Lemma}
\begin{proof}
    Докажем первое неравенство (второе --- аналогично).
    
    Введём обозначения
    \begin{gather}
        M_k = \sup_{A_k}(f),\\
        \widetilde{M}_k = \sup_{B_k}(f).
    \end{gather}
    Тогда искомое неравенство есть следствие следующей цепочки неравенств:
    \begin{gather}
        S^*(f, T_1) = \sum\limits_{k = 1}^{K}\widetilde{M}_k\mu B_k =  \sum\limits_{j = 1}^{J} \sum\limits_{k\colon B_k \subset A_j} \widetilde{M}_k \mu B_k \leqslant \sum\limits_{j=1}^J  \sum\limits_{k\colon B_k \subset A_j} M_j \mu B_k = \sum\limits_{j=1}^J M_j \mu A_j = S^*(f, T)
    \end{gather}
\end{proof}
Абсолютно аналогично можно доказать и что $s_*(T_1) \geqslant s_*(T)$.
\begin{Statement}
    Для любых разбиений $T_1$ и $T_2$, $s_*(f, T_1) \leqslant S^*(f, T_2)$.
\end{Statement}
\begin{proof}
    Пусть $\widetilde{T}$ --- измельчение $T_1$ и $T_2$. Тогда,
    $$
        s_*(f, T_1) \leqslant s_*(f, \widetilde{T}) \leqslant S^*(f, \widetilde{T}) \leqslant S^*(f, T_2)
    $$
\end{proof}
\begin{Def}
    Нижним и верхним интегралами Дарбу ($I_*$ и $I^*$ соответственно) называются величины
    \begin{gather}
        I_* = I_*(f, A) = \sup_{T} s_*(f,T)\\
        I^* = I^*(f, A) = \inf_{T} S^*(f, T)
    \end{gather}
    по всем разбиениям $T$.
\end{Def}
Сразу очевидное утверждениe
\begin{Statement}
    $I_* \leqslant I^*$.
\end{Statement}
\begin{proof}
    Сразу следует из того, что для всяких $T_1, T_2$ --- разбиений $A$, $s_*(f, T_1) \leqslant S^*(f, T_2)$. Тогда $I_* \leqslant S^*(f, T_2)$ и значит $I_* \leq \inf \limits_{T_2}S^*(f; T_2) = I^*.$
\end{proof}
\begin{Theorem}[Критерий Дарбу]
Следующие утверждения эквивалентны:
\begin{enumerate}
    \item $f \in \mathcal{R}(A)$
    \item $I_* = I^*$.
\end{enumerate}
Причём если утверждение выполнено, то $I_* = I^* = \int\limits_{A}f(x) dx$.
\end{Theorem}
\begin{proof}
    $(1) \Rightarrow (2)$. Обозначим за $I = \int\limits_A fdx$. Зафиксируем произвольное $\eps > 0$. Существует такое $\delta > 0$, что для любого отмеченного разбиения $(T, \xi)$ диаметром, меньшим, чем $\delta$ выполнено $|\sigma(f, T, \xi) - I| < \eps$. Перепишем это как
    \begin{gather}
        I - \eps \leqslant \sigma(f, T, \xi) \leqslant 1 + \eps\\
        I - \eps \leqslant s_*(f, T) = \inf_\xi \sigma(f, T, \xi) \leqslant \sup_\xi \sigma(f, T, \xi) = S^*(f, T) \leqslant I +\eps\\
        I - \eps \leqslant s_*(f, T) \leqslant I_* < I^* \leqslant S^*(f, T) \leqslant I + \eps
    \end{gather}
    То есть мы получили, что с одной стороны $|I - I_*| < \eps$, с другой --- $|I - I^*| < \eps$. В силу произвольности $\eps$ это возможно лишь когда $I^* = I = I_*$.\\
    
    
    
    $(2) \Rightarrow (1)$. Обозначим за $C = \sup_A |f|$ (напомним, что рассматриваются ограниченные функции $f$). Зафиксируем произвольное $\eps > 0$ и положим $\eps_1 = \frac{\eps}{100(1 + 2С)}$.

    Найдём такие разбиения $T_1$ и $T_2$, что 
    \begin{gather}
        s_*(f, T_1) > I_* - \eps_1\\
        S^*(f, T_2) < I^* + \eps_1
    \end{gather}
    Найдём $T = \{A_k\}_{k =1}^{K}$ --- измельчение одновременно разбиений $T_1$ и $T_2$. Тогда
    $$
        I - \eps_1 < s_*(f, T) < S^*(f, T) < 1 + \eps
    $$
    Положим $D = \bigcup\limits_{k=1}^K \partial A_k$, причём в силу измеримости множеств $\mu D =0$. Найдётся простое множество $P_1$ такое, что $P_1 \supset D$ и $\mu P_1 < \frac{\eps_1}{5^n}$ (напомним, что $n$ --- размерность нашего пространства). Это множество представимо виде дизъюнктного объединения брусов, а именно
    \[
        P_1 = \bigsqcup\limits_{j=1}^J \Pi_j.
    \]
    Если среди этих брусов есть вырожденные, то заменим их на невырожденные с сохранением неравенства (в силу строгости неравенства мы можем так сделать). Пусть $\delta > 0$ --- минимальная длина ребра полученных брусов. Через $\Pi_j^*$ обозначим раздутие бруса $\Pi_j$ в 5 раз (брус, полученный гомотетией относительно центра с коэффициентом 5), положим также $P = \bigcup\limits_{j=1}^J \Pi_j^*$. Заметим, что $\mu P < \eps_1$.\\
    Далее нам потребуется одна лемма.
    \begin{Lemma}
        Пусть $B$ --- измеримое множество, причём $B \cap A \neq \varnothing$. Тогда либо $B \subset A$ либо $B \cap \partial A \neq \varnothing$.
    \end{Lemma}
    \begin{proof}
    Будем считать, что $B$ не является подмножеством $A$, так как в противном случае всё очевидно. Тогда найдём $x \in A$ и $y \in \R^n\setminus A$, лежащие в брусе. Соединим их отрезком. Будем делить отрезок пополам, на каждой итерации выбирая тот, который пересекает границу. Получим последовательность вложенных отрезков, которые имеют одну точку. Эта точка будет искомой.
    \end{proof}
    Пусть $\widetilde{T} = \{B_j\}_{j=1}^{J}$ --- разбиение $A$ с $\mathrm{diam}\; \widetilde T < \delta$. Тогда про множество $B_j$ известно, что
    \begin{enumerate}
        \item Либо $B_j \subset A_i$ для некоторого $i$
        \item Либо $B_j$ не лежит ни в одном $A_i$ для некоторого $i$.
    \end{enumerate}
    Легко понять, что во втором случае $B_j \subset P$, поскольку содержит точку границы и имеет диаметр не больше $\delta$. Тогда
    \begin{gather}
        \sigma(f, \widetilde{T}) = \sum\limits_{j =1}^{J} f(\xi_j)\mu B_j = \sum\limits_{k=1}^{K} \sum\limits_{j\colon B_j \subset A_k} f(\xi_j)\mu B_j + \sum\limits_{j \colon\not\subset A_i\forall i} f(\xi_j)\mu B_j\leqslant\\ \leqslant \sum\limits_{k=1}^K \sum\limits_{j\colon B_j\subset A_k}\sup_{B_j} f \mu B_j + \sum\limits_{j \colon\not\subset A_i\forall i} \sup_{A}|f| \mu B_j \leqslant \sum\limits_{k=1}^K \sum\limits_{j\colon B_j\subset A_k}\sup_{B_j} f \mu B_j + С\mu P \leqslant\\ \leqslant \sum\limits_{k = 1}^{K} \sup_{A_k}f\cdot\left(\mu A_k - \mu \left(A_k \setminus \bigcup_{B_j \in A_k} B_j\right)\right) + C\eps_1 \leqslant S^*(f, T) + 2C\eps_1 < I + \eps_1(1 + 2C) < I + \eps/2
    \end{gather}
    Аналогично докажем, что $\sigma(f; \tilde{T}) > I- \eps/2$. Таким образом интегрируемость $f$ доказана по определению.
\end{proof}
\begin{Statement}
    Если $f\in \mathcal{R}(A)$ и $f$ ограничена на $A$, то
    $$
        \forall \eps > 0 \exists \delta > 0\colon \; \forall T \text{ --- разбиения } A,\; \mathrm{diam}(T)< \delta\; S^*(f,T) - s_*(f, T) < \eps.
    $$
\end{Statement}
\begin{Statement}
Если $f$ ограничена на $A$ и $\forall \eps > 0\; \exists T_1, T_2 \text{ --- разбиения } A\colon S^*(f, T_1) - s_*(f, T_2) < \eps$, то $f \in \mathcal{R}(A)$. 
\end{Statement}
\begin{Consequence}
    Если $f$ ограничена и равномерно непрерывна на $A$, то $f \in \mathcal{R}(A)$. 
\end{Consequence}
\begin{Consequence}
    Если $f$ ограничена на $A$ и $B$ ($A \cap B = \varnothing$) и интегрируема нa $A$ и $B$, то $f \in \mathcal{R}(A \cup B)$.
\end{Consequence}
\begin{proof}
    Действительно, $\forall \eps>0$ найдём $T_A$ и $T_B$ --- разбиения множеств $A$ и $B$ соответственно, такие что $S^*(f; T_A) - s_*(f; T_A) < \frac{\eps}{2}$ и $S^*(f; T_B) - s_*(f; T_B) < \frac{\eps}{2}$. Тогда $T_A \cup T_B$ --- разбиение $A\cup B$ и $S^*(f; T_A \cup T_B) - s_*(f; T_A \cup T_B) < \eps$. А значит $f \in \mathcal{R}(A \cup B)$.
\end{proof}


