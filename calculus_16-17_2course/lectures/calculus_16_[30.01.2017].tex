\section{Лекция 16 от 30.01.2017 \\ Метрические и нормированные пространства (продолжение), ряды Фурье}
\subsection{Коэффициенты Фурье}
Вспомним последний результат предыдущей лекции. Пусть $H$ --- пространство со скалярным произведением, $\{e_n\}_{n=1}^{N}$ --- ортогональная система, $x \in H$. Тогда если $c_1, \ldots, c_N$ --- коэффициенты из $\R$, то 
\begin{gather*}
    \left|\left|x - \sum\limits_{n= 1}^{N}c_ne_n\right|\right|^2 = ||x||^2 +\sum\limits_{n=1}^{N}\left(c_n||e_n|| - \cfrac{(x, e_n)}{(e_n, e_n)}||e_n||\right)^2 - \sum\limits_{n =1 }^{N} \left(\cfrac{(x, e_n)}{(e_n, e_n)}\right)^2(e_n, e_n).
\end{gather*}
\begin{Def}
    Коэффициентами Фурье, соответствующими вектору $x$ и элементам ортогональной системы $\{e_n\}_{n=1}^{N}$, называются числа $\hat{x}_n = \cfrac{(x, e_n)}{(e_n, e_n)}$.
\end{Def}
Принимая это определение во внимание, исходное равенство переписывается в более красивом виде:
\[
    \left|\left|x - \sum\limits_{n= 1}^{N}c_ne_n\right|\right|^2 = \left|\left|x\right|\right|^2 + \left|\left|e_n\right|\right|^2 \sum\limits_{n=1}^{N}(c_n - \x_n)^2 - \sum\limits_{n =1}^{N}\x_n^2 
    \left|\left|e_n\right|\right|^2. \tag{*}\label{aster}
\]
\begin{Statement}[очевидное]
    Для любых коэффициентов $c_1, \ldots, c_N$
    $$
        \left|\left| x - \sum\limits_{n=1}^{N}c_n e_n\right|\right|\geqslant \left|\left|x -  \sum\limits_{n=1}^{N}\x_n e_n\right|\right|,
    $$
    причем равенство достигается тогда и только тогда, когда $c_n = \x_n$ при всех $n\in \{1\dots N\}.$
\end{Statement}
Это равенство действительно очевидно, учитывая, что в выражении~\eqref{aster}, которое, кстати, неотрицательно, у нас лишь второе (также неотрицательное) слагаемое правой части зависит от~$c_n$. Минимум достигается, если оно равно нулю.

Отметим важное свойство: \textit{коэффициент $\x_n$ не зависит от $e_i$ для $i \neq n$}.
\begin{Statement}[Тождество Бесселя]
    $$
        \left|\left|x - \sum\limits_{n=1}^{N}\x_ne_n\right|\right|^2 = \left|\left|x\right|\right|^2 - \sum\limits_{n=1}^{N} \x_n^2 \left|\left|e_n\right|\right|^2.
    $$
\end{Statement}
\begin{Consequence}[Неравенство Бесселя]
    $$\sum\limits_{n=1}^{N} \x_n^2||e_n||^2\leqslant ||x||^2$$
\end{Consequence}
\subsection{Пространство $l^2$}
Отступим в сторону.
Пусть $l^2$ --- множество всех числовых последовательностей, сумма квадратов элементов которых конечна. Заметим, что это множество будет являться линейным пространством, поскольку
\begin{align*}
    &\forall \{b_n\}_{n=1}^{\infty},\; \{a_n\}_{n=1}^{\infty} \in l^2,\; \alpha \in \R\\
    &\sum\limits_{n=1}^{\infty} \alpha^2 a_n^2 = \alpha^2 \sum\limits_{n=1}^{\infty} a_n^2 < \infty \Leftrightarrow \{\alpha a_n\}_{n=1}^{\infty} \in l^2\\
    &\sum\limits_{n=1}^{\infty} = (a_n+b_n)^2< \sum\limits_{n=1}^{\infty} (2a_n^2 + 2b_n^2) < \infty \Leftrightarrow \{a_n + b_n\}_{n=1}^{\infty} \in l^2.
\end{align*}
Обозначив $\overline{a} = \{a_n\}_{n=1}^{\infty}\in l^2$, $\overline{b} = \{b_n\}_{n=1}^{\infty} \in l^2$, мы можем ввести скалярное произведение следующим образом:
\begin{Def}
    Для пространства $l^2$, 
    \begin{align*}
        &(\overline{a}, \overline{b}) = \sum\limits_{n=1}^{\infty} a_nb_n,\\
        &||\overline{a}|| = \sqrt{\sum\limits_{n=1}^{\infty}a_n^2}.
    \end{align*}
\end{Def}
\subsection{Сходимость в номрированных пространствах}
Отойдем еще на шаг в сторону. Пусть $(L, ||\cdot||)$ --- нормированное пространство, $\{x_n\}_{n=1}^{\infty}$ --- последовательность ее элементов. В таком случае, говоря о сходимости таких последовательностей, можно ввести те же формулировки, что и в случае числовых последовательностей.
\begin{Def}
    Будем говорить, что последовательность $\{x_n\}_{n=1}^{\infty}$ сходится к $x$, если 
    $$
        ||x_n - x|| \underset{n\to \infty}{\to} 0.
    $$
\end{Def}
При помощи неравенства треугольника можно также установить и арифметические свойства пределов, например предел суммы.
\begin{Statement}
    Пусть $\{x_n\}_{n=1}^{\infty}$ сходится к $x$, $\{y_n\}_{n=1}^{\infty}$ сходится к $y$. Тогда  $\{x_n + y_n\}_{n=1}^{\infty} \underset{n\to \infty}{\to} x+y$.
\end{Statement}
\begin{proof}
    Воспользуемся неравенством треугольника:
    \begin{gather*}
        0\leqslant ||(x_n + y_n) - (x + y)|| \leqslant ||x_n-x|| + ||y_n - y|| \underset{n\to\infty}{\to} 0.
    \end{gather*}
\end{proof}
Ряды из элементов вводятся абсолютно так же, как и числовые ряды, рассмотренные нами ранее (абсолютная сходимость определяется как сходимость ряда из норм элементов).

\subsection{Ряды Фурье}
Теперь, когда мы ввели понятие сходимости в нормированных пространствах, будем двигаться дальше, к рядам Фурье. Вернемся к рассмотрению исходного пространства $H$ со скалярным произведением, в котором мы уже ввели понятие коэффициентов Фурье.
\begin{Def}
    Ряд $\sum\limits_{n}^{\infty} \hat{x}_ne_n$ называется рядом Фурье (разложением в ряд Фурье) по ортогональной системе $\{e_n\}_{n=1}^{\infty}$.
\end{Def}
Из этого определения не следует, что ряд Фурье вообще сходится.
\begin{Statement}
    Следующие утверждения эквивалентны:
    \begin{enumerate}
        \item $x = \sum\limits_{n=1}^{\infty} \hat{x}_ne_n$;
        \item $||x||^2 = \sum\limits_{n=1}^{\infty} \hat{x}^2_n ||e_n||^2$ (\textit{равенство Парсеваля}).
    \end{enumerate}
\end{Statement}
\begin{proof}
    Прямое следствие тождества Бесселя.
\end{proof}
\begin{Statement}[Единственность разложения]
    Пусть $\{e_n\}_{n=1}^{\infty}$ --- ортогональная система и $x = \sum\limits_{n=1}^{\infty}c_n e_n$ для некоторых коэффициентов $c_i$. Тогда $\x_i = c_i\;\ \forall i \in \N$.
\end{Statement}
\begin{proof}
    Зафиксируем $n_0\in \N$, пусть $N>n_0$. Тогда
    $$
        \sum\limits_{n=1}^{\infty} c_ne_n = \sum \limits_{n=1}^{N}c_n e_n + \underbrace{r_N}_{\to 0 \text{ при } N\to \infty}.
    $$
    Далее запишем, чему равно $\x_{n_0}$:
    $$
        \x_{n_0} = \cfrac{(x,e_{n_0})}{(e_{n_0}, e_{n_0})} = \cfrac{\left(\sum\limits_{n=1}^{N} c_ne_n, e_{n_0}\right) + \left(r_N, e_{n_0}\right)}{(e_{n_0}, e_{n_0})} = \cfrac{c_{n_0}(e_{n_0}, e_{n_0})}{(e_{n_0}, e_{n_0})} + \cfrac{(r_N, e_{n_0})}{(e_{n_0}, e_{n_0})} = c_{n_0} + \cfrac{(r_N, e_{n_0})}{(e_{n_0}, e_{n_0})}.
    $$
    Второе слагаемое стремится к нулю. Устремив $N \to \infty$, получим требуемое.
\end{proof}
Заметим, что если $\{e_n\}_{n=1}^{\infty}$ --- ортогональная система, то ряд $\sum\limits_{n=1}^{\infty} c_n e_n$ удовлетворяет условию Коши тогда и только тогда, когда $\sum\limits_{n=1}^{\infty}c_n^2||e_n||^2$ сходится.
$$
    \forall \eps>0\; \exists N\in \N\; \forall m>N,\; p\in \N \Rightarrow \left|\left|\sum\limits_{n=m+1}^{m+p}c_ne_n \right|\right| < \eps.
$$
Действительно,
$$
    \left(\sum\limits_{n=m+1}^{m+p}c_ne_n, \sum\limits_{n=m+1}^{m+p}c_ne_n\right) =  \sum\limits_{n=m+1}^{m+p}c_n^2 ||e_n||^2.
$$
И увидим, что искомому условию Коши удовлетворяет и желаемый ряд.
\begin{Consequence}
    Если пространство со скалярным произведением полно, то все ряды Фурье в нем сходятся.
\end{Consequence}
\begin{proof}
    Из неравенства Бесселя получаем, что
    $\sum\limits_{n=1}^{\infty} \x_n^2 ||e_n||^2 < \infty$. Следовательно, $ \sum\limits_{n=1}^{\infty} \x_ne_n 
    $ удовлетворяет условию Коши, а с учетом полноты $\sum\limits_{n=1}^{\infty}\x_n e_n$ сходится.
\end{proof}
Итак, в полных нормированных пространствах ряды Фурье сходятся, но отнюдь не всегда они сходятся куда надо.
\begin{Def}
    Ортогональная система $\{e_n\}_{n=1}^{\infty}$ замкнута, если 
    $$\forall x\in H, \forall \eps >0\; \exists e_{n_1}, e_{n_2}, \ldots, e_{n_K},\; c_1, c_2, \ldots, c_K \Rightarrow \left|\left|x - \sum\limits_{k=1}^{K} c_k e_{n_k}\right|\right|<\eps
    $$
\end{Def}
\begin{Statement}
    Если ортогональная система $\{e_n\}_{n=1}^{\infty}$ замкнута, то $\forall x\in H\; x = \sum\limits_{n=1}^{\infty}\x_n e_{n}$.
\end{Statement}
\begin{proof}
    Зафиксируем произвольное $\eps >0$, найдем такое $K_0$ и коэффициенты $c_1, \ldots, c_{K_0}$, что $\left|\left| x - \sum\limits_{n=1}^{K_0} c_ne_n \right|\right| < \eps$. Тогда
    $$
        \forall N>K_0 \; \left|\left|x - \sum\limits_{n=1}^{N}\x_ne_n\right|\right|\leqslant \left|\left|x - \sum\limits_{k=1}^{K_0}c_ke_k \right|\right|.
    $$
\end{proof}
