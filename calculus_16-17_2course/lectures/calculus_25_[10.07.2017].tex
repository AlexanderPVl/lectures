\pagestyle{fancy}
\section{Лекция 25 от 10.04.2017 \\ Критерий измеримости по Жордану и иже с ним}

Вспомним утверждение
\begin{Statement}[Критерий измеримости]
Пусть $A$ --- ограниченное подмножество $\R^n$. Тогда следующие утверждения эквивалентны:
\begin{enumerate}
    \item $A$ --- измеримо по Жордану;
    \item $\mu^* \delta A = 0$, где $\delta A$ --- граница множества.
\end{enumerate}
\end{Statement}
Для доказательства $(2) \Rightarrow (1)$ нам потребуется лемма:
\begin{Lemma}
Пусть есть брус $\Pi$ такой, что
\begin{enumerate}
    \item $\Pi \cap \mathrm{int} A \neq \varnothing$
    \item $\Pi \cap \mathrm{ext} A \neq \varnothing$
\end{enumerate}
Тогда $\Pi \cap \delta A \neq \varnothing$.
\end{Lemma}
И вначале ещё одно вспомогательное определение:
\begin{Def}
Отрезком в $\R^n$ с концами $a$ и $b$ назовём множество точек $\{x = a + t(b-a)\;|\; t\in[0;1], a,b \in \R^n\}$
\end{Def}
\begin{proof} [Первый способ доказательства леммы]
    Пусть $a \in \Pi\cap \mathrm{int}A$, $b \in \Pi \cap \mathrm{ext}A$. Отрезок с концами $a$ и $b$ полностью лежит в брусе. Будем делить его пополам и выбирать ту половину, у которой один конец в $A$, а другой --- нет. Полученная последовательность есть последовательность вложенных сегментов. У такой последовательности есть общая точка. Она и будет искомой.
\end{proof}
\begin{proof}[Второй способ доказательства леммы]
Введём функцию $f\colon \Pi \to \R$ такую, что:
$$
    f(x) = \begin{cases}
        0,\; x\notin A\\
        1,\; x \in A
    \end{cases}
$$
, то есть, иначе говоря, индикатор множества $A$. Эта функция непрерывна во всех внутренних и внешних точках $A$, а значит, в нашем предположении, и на всём брусе. Рассмотрим другую функцию $\varphi(t) = f(a + t(b-a))$. Она непрерывна на отрезке $[0;1]$, при этом принимает на нём ровно два значения. Противоречие.
\end{proof}
\begin{proof}[Доказательство критерия измеримости]
Найдём открытый брус $\Pi_0$, содержащий $A$. Зафиксируем произвольное $\eps >0$. Существует такое простое множество $P_0\supset \delta A$ и $\mu P_0 < \eps$. Будем считать $P_0 \subset \Pi_0$ (иначе возьмём вместо $P_0$ множество $P_0\cap \Pi_0$). Заметим, что $\Pi_0 \setminus P_0$). Заметим, что $\Pi_0 \setminus P_0$ --- простое множество. Представим его как пересечение попарно непересекающихся брусов.
\begin{gather}
    \Pi_0 \setminus P_0 = \pi_1 \sqcup \ldots \sqcup \pi_K\\
    \pi_k \cap \delta A = \varnothing
\end{gather}
То есть либо $\pi_k \subset \mathrm{int}A$ либо $\pi_k \subset \mathrm{ext}A$. Положим $p = \bigsqcup\limits_{k\colon \pi_k \subset \mathrm{int}A} \pi_k$, $P = p \sqcup P_0 = \Pi_0 \setminus \bigsqcup\limits_{k\colon \pi_k \subset \mathrm{ext}A}\pi_k$.
Заметим, что $p \subset A$, $P\supset A$. Тогда $\mu P \mu p = \mu(P \setminus p) = \mu(P_0) < \eps$. В силу произвольности $\eps$, получаем то, что нужно.
\end{proof}
\begin{Consequence}
Пусть множества $A$ и $B$ измеримы по Жордану. Тогда $A \cap B$, $A \cap B$, $A \setminus B$ измеримы по Жордану.
\end{Consequence}
\begin{Comment}
\begin{enumerate}
    \item Подмножество меры 0 по Жордану имеет меру 0 по Жордану.
    \item Объединение конечного числа множеств меры 0 по Жордану --- также множество меры 0 Жордану.
\end{enumerate}
\end{Comment}
\begin{Statement}
Если $A$ и $B$ измеримые подмножества $\R^n$, то $\mu(A \sqcup B) = \mu A + \mu B$.
\end{Statement}
\begin{proof}
Мы докажем лишь половину утверждения, так как вторая отличается лишь разворотом знака. Зафиксируем произвольное $\eps > 0$. Найдём такие простые множества $P_A$ и $P_B$, что
\begin{gather}
    P_A \supset A\\
    P_B \supset B\\
    \mu P_A > \mu_*A - \eps/2\\
    \mu P_A > \mu_*B - \eps/2
\end{gather}
$P_A \cup P_B$ --- простое множество, причём
$$
    P_A \sqcup P_B \subset A \cup B
$$
Тогда $\mu(P_A \sqcup P_B) = \mu P_A + \mu P_B > \mu A + \mu B - \eps$. Следовательно $\mu_*(A \sqcup B)\geqslant \mu A + \mu B$. Аналогично для верхней меры. Итого получаем
\begin{gather}
    \mu_*(A \sqcup B)\geqslant \mu A + \mu B\\
    \mu^*(A \sqcup B)\leqslant \mu A + \mu B
\end{gather}
\end{proof}
\begin{Consequence}
    Если $A$ и $B$ измеримы по Жордану, то $\mu (A \cup B) = \mu A + \mu B - \mu(A \cap B)$.
\end{Consequence}
\begin{proof}
Можно и без слов:
\begin{gather}    
    A \cup B = A \sqcup (B\setminus A)\\
    \mu (A \cup B) = \mu A + \mu(B \setminus A)\\
    B = (B \cap A) \sqcup (B \setminus A)\\
    \mu(B) = \mu(A \cap B) + \mu(B \setminus A)\\
    \mu(B \setminus A) = \mu(B) - \mu(A \cap B)\\
    \mu(A \cup B) = \mu(A) + \mu(B) - \mu(A \cap B)
\end{gather}
\end{proof}
\begin{Def}
    $A$ и $B$ будем называть неперекрывающимися, если их пересечение лежит в объединении границ: $A\cap B \subset \delta A \cup \delta B$.
\end{Def}
\begin{Comment}
    Если $A$ и $B$ неперекрывающиеся и измеримы, то $\mu(A \cap B) = 0$.
\end{Comment}
\begin{proof}
    От противного: пусть $A$ и $B$ перекрываются. Тогда $\exists x \in \mathrm{int} A \cap \mathrm{int}B$. Следовательно, существует такое множество $\delta > 0$, что $B_\delta \subset A \cap B$, отсюда следует $\mu(A \cap B) > \mu(B_\delta) > 0$. 
\end{proof}