\documentclass[../main.tex]{subfiles}

%Sample lecture


\begin{document}
	\subsection{Пространство ограниченных функций}
	Пусть $X$ -- произвольное множество.
	\begin{definition}
		\textit{Пространством ограниченных функций}, определенных на $X$ называется метрическое пространство со множеством:
		\[
		B(X)=
		\{
		f:X\to \R \ | \ 
		\sup\limits_{x\in X}|f(x)| < \infty
		\}
		\]
		и метрикой:
		\[
		\varrho(f, g)=
		\sup\limits_{x\in X}|f(x)-g(x)|
		\]
	\end{definition}

	Именно данная метрика задает равномерную сходимость:
	\[
	f_n\overset{X}{\rightrightarrows} f
	\ \Leftrightarrow \ \varrho(f_n, f)\to 0
	\]
	
	Корректность метрики очевидна, поскольку модуль разности сам по себе ведет себя как метрика, а взятие супремума мало что меняет.
	
	\begin{statement}
		$B(X)$ -- полное метрическое простанство.
	\end{statement}
	\begin{proof}
		Требуется доказать выполнимость в данном метрическом пространстве критерия Коши: последовательность ограниченных функций сходится т. и т.т., когда она является фундаментальной.
		
		По определению фундаментальной последовательности:
		\[
		\forall \varepsilon>0 \ \exists N: \ \forall n, m>N \ \sup|f_n(x)-f_m(x)|<\varepsilon
		\]
		
		Ограничение сверху на супремум можно расширить на любой возможный модуль разности:
		\[
		\forall x: |f_n(x)-f_m(x)|<\varepsilon
		\]
		
		Это означает, в свою очередь, что для всякого фиксируемого $x$ числовая последовательность $\{f_n(x)\}$ является фундаментальной. Значит, функциональная последовательность $\{f_n(x)\}$ сходится к некоторой функции $f(x)$.
		
		Теперь докажем, что если функциональная последовательность сходится к $f(x)$, то она фундаментальная. Действительно,  можно интерпретировать, как то, что при всяком $x$ последовательность $\{f_n(x)\}$ сходится к $f(x)$ как числовая последовательность в $\R$. Это значит, что она фундаментальна по тому же критерию Коши. Запишем это:
		\[
		\forall \varepsilon>0 \ \exists N: \ \forall n, m>N \ \forall x \ |f_n(x)-f_m(x)|<\varepsilon
		\]
		Устремим $m\to+\infty$ и воспользуемся сходимостью последовательности: $\forall x \ f_m(x)\to f(x)$.
		\[
		|f_n(x)-f(x)|<\varepsilon\Rightarrow
		\sup|f_n(x)-f(x)|<\varepsilon
		\]
		Таким образом, функциональная последовательность $\{f_n\}$ действительно является фундаментальной.
	\end{proof}
	
	\begin{remark}
		В доказательстве полноты какого-либо пространство всегда аппелируют к уже известным полным пространствам ($\R, \ \R^n, \ B(X)$).
	\end{remark}
	
	\begin{definition}
		Отображение $I: (X, \varrho_x)\to(Y, \varrho_y)$ называется \textit{изометрией}, если \\
		$I: X\to Y$ -- биекция и $\varrho_y\big(I(x_1), I(x_2)\big)=\varrho_x(x_1, x_2)$ (отображение сохраняет расстояние). Также говорят, что $X$ и $Y$ в этом случае \textit{неразличимы} как метрические пространства.
	\end{definition}
	
	\begin{theorem}
		Всякое метрическое пространство $(X, \varrho)$ \textit{изометрично вложено} в $B(X)$ (может быть представлено как изометрическое подмножество $B(X)$ с метрикой $\sup|x-y|$) 
	\end{theorem}
	\begin{proof}
		Будем считать, что $X$ непусто. Зафиксируем $x_0\in X$ и каждой точке $x\in X$ сопоставим функцию, которая выражается следующим образом:
		\[
		x\mapsto f_x(y)=\varrho(x, y)-\varrho(y, x_0)
		\]
		В идеале бы хотелось просто сопоставлять расстояние от $x$ до $y$, однако из неограниченности функции расстояния вычитаем расстояние до $y$ от $x_0$. Действительно, в силу неравенства треугольника $\varrho(x, y)\leqslant \varrho(x, x_0)+\varrho(y, x_0)$ следует, что $|\varrho(x, y)-\varrho(y, x_0)|=|f_x(y)|\leqslant \varrho(x, x_0)$, поэтому $f_x(y)$ ограничена. Тогда для любых фиксированных $x_1, x_2\in X$ выполняется в силу того же неравенства треугольника:
		\[
		|f_{x_1}(y)-f_{x_2}(y)|=|\varrho(x_1, y)-\varrho(x_2, y)|\leqslant \varrho(x_1, x_2)
		\]
		Видно, что равенство достигается при $y=x_2$, значит $\sup|f_{x_1}(y)-f_{x_2}(y)|=\varrho(x_1, x_2)$. 
		
		Таким образом, отображение $I: x\mapsto f_x(y)$ из $X$ в $B(X)$ действительно является изометрией из $X$ в $I(X)$.
	\end{proof}
	
	\subsection{Открытые и замкнутые множества}
	\begin{definition}
		\textit{Открытым множеством} в $(X, \varrho)$ называется множество, которое вместе с любой своей точкой содержит открытый шар с центром в этой точке.
	\end{definition}
	\begin{definition}
		\textit{Замкнутым множеством} в $(X, \varrho)$ называется дополнение к некоторому открытому множеству.
	\end{definition}
	\begin{remark}
		Существует факт из мат. анализа, что открытый шар является открытым множеством, а замкнутый, соответственно, -- замкнутым (без доказательства).
	\end{remark}
	\begin{definition}
		Точка $a$ называется \textit{предельной} точкой множества $A$, если пересечение всякого открытого шара $B(a, r)$ с $A$ бесконечно. Равносильное определение:
		пересечение всякого проколотого шара $B'(a, r)$ (точка $a$ выколота) с $A$ непусто.
	\end{definition}
	\begin{definition}
		Точка $a$ называется \textit{изолированной} в $A$, если она не является предельной, но принадлежит $A$.
	\end{definition}

	\begin{definition}
		Пересечение замкнутых множеств $\mathcal{F}$, содержащих некоторое фиксированное множество $A$, называется \textit{замыканием} данного множества и обозначается как $\bar{A}$:
		\[
		\bar{A}=\bigcap\limits_{\mathcal{F}\supset A} \mathcal{F}
		\]
	\end{definition}
	\begin{remark}
		Любое замыкание является замкнутым множеством, что следует из факта в мат. анализе (здесь без доказательства), о том, что:
		\begin{itemize}[label={--}]
			\item \textit{Любое объединение или конечное пересечение открытых множеств является открытым};
			\item \textit{Любое пересечение или конечное объединение замкнутых множеств является замкнутым};
		\end{itemize}
	\end{remark}

	\begin{statement}
		Пусть $B$  --  множество предельных точек $A$. Тогда $\bar{A}=A\cup B$.
	\end{statement}
	\begin{proof}
		Сначала поймем 2 вещи:
		\begin{itemize}[label={--}]
			\item $A\cup B$ -- замкнутое множество;
			\item $\bar{A}$ содержит все предельные точки $A$;
		\end{itemize}
		
		Второе понятно, т.к. предельная точка $A$ является предельной точкой и $\bar{A}$, ведь $A\subset \bar{A}$, а любое замкнутое множество содержит все свои предельные точки (без доказательства, из курса мат. анализа).
		
		Теперь докажем замкнутость. Возьмем $b\notin A$. Преположим, что в любой ее окрестности встречаются точки из $A\cup B$, т.е. в любой ее окрестности встречаются точки из $A$ или предельные. Присутствие точек из $A$ гарантировано их наличием в любой окрестности предельных точек данного множества. Значит $b$ сама является предельной точкой из $A$, и она лежит в $\bar{A}$, но мы ее взяли из дополнения к $A$.
		
	\end{proof}
	\begin{remark}
		Нужно заметить, что в общем случае замыкание некоторого открытого шара $B(a, r)$ не является замкнутым шаром с теми же параметрами, ведь:
		\[
		\overline{B(a, r)}=
		\overline{\{x: \ \varrho(a, x)<r\}}, \ \
		\bar{B}(a, r)=\{x: \ \varrho(a, x)\leqslant r\}
		\]
		
		Приведем пример, когда равенство выполняется: пусть в нашем метрическом пространстве используется дискретная метрика:
		\[
		\varrho(x, y)=
		\begin{cases}
		0, & \text{ если } x=y\\
		1, & \text{ иначе }\\
		\end{cases}
		\]
		Возьмем $B(a, 1)$, $a\in X$. Понятно, что это одноэлементное множество $\{a\}$, а его замыкание совпадает с ним (поскольку у одноэлементных множеств нет предельных точек). В то же время замкнутый шар $\bar{B}(a, 1)$ -- в точности все рассматриваемое метрическое пространство $X$.
	\end{remark}

	\begin{statement}
		В метрическом пространстве на множестве $\Q$ с $p$-адической метрикой и открытый, и замкнутый шар является одновременно открытым и замкнутым множеством.
	\end{statement}
	\begin{proof}
		\textit{(для открытого шара)} Пусть задан открытый шар $B(a, r)$. Возьмем точку $c\notin B(a, r)$. Рассмотрим шар $B(c, r)$. Предположим, что эти шары пересекаются, и $b$ -- их общая точка. Но в обоих шарах любая точка является центром, значит у них есть некий общий центр, притом они имеют одинаковые радиусы. Шары должны были бы совпасть, но $c\notin B(a, r)$ -- противоречие. Значит шары не пересекаются, в дополнении к шару точка каждая точка $c$ лежит в некотором целом шаре. Значит дополнение открыто, а сам шар -- замкнут, будучи открытым множеством как открытый шар в то же время.
	\end{proof}

	\begin{definition}
		Пространства, в которых есть нетривиальные одновременно открытые и замкнутые множества, называются \textit{несвязными} (они распадаются). Примеры: код Хемминга, множество Кантора, бесконечная степень двуточия ($\{0, 1\}^{\infty}$). 
	\end{definition}
\end{document}
