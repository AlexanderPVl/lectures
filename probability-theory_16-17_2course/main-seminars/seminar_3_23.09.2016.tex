\section{Семинар от 23.09.2016}
Как обычно, начнём с разбора домашнего задания.
\begin{problem}
    В ящике \(N\) различимых шаров, из которых ровно \(M\) белых. Последовательно вынимают \(n \leq N\) шаров. Пусть событие \(A_k\) означает, что \(k\)-й по счёту вынутый шар~--- белый, а событие \(B_m\)~--- что всего вынули \(m \leq M\) белых шаров. 
    
    Найдите \(\Pr(A_k \mid B_m)\), если (a) шары вынимаются без возвращения, (б) с возвращением.
\end{problem}
\begin{proof}[Решение]
    Начнём со случая, когда нельзя возвращать шары. По определению условной вероятности \(\Pr(A_k \mid B_m) = \dfrac{\Pr(A_k \cap B_m)}{\Pr(B_m)}\). Для начала посчитаем \(\Pr(B_m)\). Как это сделать? Зафиксируем набор из \(n\) шаров, в котором первые \(m\) шаров белые. Какова вероятность того, что выпадет такой набор? Она равна \[\frac{M}{N}\cdot\frac{M - 1}{N - 1}\cdot\ldots\cdot\frac{M - m + 1}{N - m + 1}\cdot\frac{N - M}{N - m}\cdot\frac{N - M - 1}{N - m - 1}\cdot\ldots\cdot\frac{N - M - (n - m) + 1}{N - n + 1}.\] Теперь заметим, что если переставить числители местами, то получится вероятность того, что выпадет какой-то другой набор из \(n\) шаров, среди которых \(m\) белых. Тогда вероятность того, что выпадет хоть какой-то набор, подходящий под это условие, равна \[\Pr(B_m) = \binom{n}{m}\frac{\frac{M!}{(M - m)!}\frac{(N - M)!}{(N - M - (n - m))!}}{\frac{N!}{(N - n)!}} = \frac{\binom{n}{m}\binom{N - n}{M - m}}{\binom{N}{M}}.\]
    Теперь перейдём к числителю. Как посчитать \(\Pr(A_k \cap B_m)\)? В принципе, точно так же, как и \(\Pr(B_m)\). Однако, в данном случае зафиксирована \(k\)-я позиция, поэтому нужно лишь выбрать \(m - 1\) позицию из \(n - 1\) для белых шаров. Тогда \[\Pr(A_k \cap B_m) = \frac{\binom{n - 1}{m - 1}\binom{N - n}{M - m}}{\binom{N}{M}}.\]
    Отсюда получаем, что \(\Pr(A_k \mid B_m) = \dfrac{\binom{n - 1}{m - 1}}{\binom{n}{m}} = \dfrac{m}{n}\).
    
    Переходим к случаю (б). Опять же, посчитаем \(\Pr[B_m]\) и \(\Pr[A_k \cap B_m]\). Рассуждения о перестановке так же имеют место, поэтому:
    \[\Pr(B_m) = \binom{n}{m}\frac{M^{m}(N - M)^{n - m}}{N^n}\]
    \[\Pr(A_k \cap B_m) = \binom{n - 1}{m - 1}\frac{M^{m}(N - M)^{n - m}}{N^n}\]
    Подставляя полученные значения в формулу условной вероятности, получаем, что \\ \(\Pr(A_k \mid B_m) = \dfrac{\binom{n - 1}{m - 1}}{\binom{n}{m}} = \dfrac{m}{n}\).
    
    \textbf{Ответ:} \(\dfrac{m}{n}\) в обоих случаях.
\end{proof}

\begin{problem}
    Ящик содержит \(a\) белых и \(b\) чёрных шаров (все шары различимы). Наудачу извлекается шар. Он возвращается обратно, и, кроме того, добавляется \(c\) шаров одного с ним цвета. Далее, подобная процедура повторяется снова. Пусть событие \(A_k\) означает, что на \(k\)-м шаге извлечён белый шар. Найдите
    \begin{enumerate}
        \item[(а)] вероятность того, что при первых \(n = n_1 + n_2\) извлечениях попалось \(n_1\) белых и \(n_2\) чёрных шаров;
        \item[(б)] вероятность события \(A_k\);
        \item[(в)] условную вероятность \(\Pr(A_m \mid A_k)\) при \(m > k\);
        \item[(г)] условную вероятность \(\Pr(A_m \mid A_k)\) при \(m > k\);
    \end{enumerate}
\end{problem}
\begin{proof}[Решение]
    Рассмотрим ситуацию, когда последовательно выпало \(n_1\) белых и \(n_2\) чёрных шаров. Какова вероятность такого события? Она равна \[\frac{a}{a + b}\cdot\frac{a + c}{a + b + c}\cdot\ldots\cdot\frac{a + (n_1 - 1)c}{a + b + (n_1 - 1)c}\cdot\frac{b}{a + b + n_1c}\cdot\frac{b + c}{a + b + (n_1 + 1)c}\cdot\ldots\cdot\frac{b + (n_2 - 1)c}{a + b + (n - 1)c}.\]
    Теперь переставим числители так, чтобы числители вида \(a + x\) и \(b + x\) были отсортированы по возрастанию. Тогда эта вероятность будет соответствовать какому-то другому набору из \(n_1\) белых и \(n_2\) чёрных. Пусть \(\Pr(a, b, n_1, n_2)\)~--- вероятность того, что из \(a\) белых и \(b\) чёрных при первых \(n = n_1 + n_2\) извлечениях попалось \(n_1\) белых и \(n_2\) чёрных шаров. Тогда
    \[\Pr(a, b, n_1, n_2) = \binom{n}{n_1}\frac{a(a + c)\ldots(a + (n_1 - 1)c)b(b + c)\ldots(b + (n_2 - 1)c)}{(a + b)(a + b + c)\ldots(a + b + (n - 1)c)}.\]
    
    Теперь посчитаем \(\Pr(A_k)\). Пусть \(C_{ki} = \{\)до \(k\)-ой процедуры вытащили ровно \(i\) белых шаров\(\}\). Очевидно, что эти события образуют разбиение вероятностного пространства. Тогда по формуле полной вероятности \(\Pr(A_k) = \sum\limits_{i = 0}^{k - 1} \Pr(A_k \cap C_{ki})\). Заметим, что \(\Pr(A_k \cap C_{ki})\) совпадает с \(\Pr(a, b, i + 1, k - i - 1)\) с тем лишь отличием, что в данном случае нужно выбрать \(i\) позиций из \(k - 1\): \[\Pr[A_k \cap C_{ki}] = \binom{k - 1}{i}\frac{a(a + c)\ldots(a + ic)b(b + c)\ldots(b + (k - i)c)}{(a + b)(a + b + c)\ldots(a + b + (n - 1)c)}.\] Тогда \[\Pr(A_k) = \frac{a}{a + b} \sum_{i = 0}^{k - 1} \binom{k - 1}{i}\frac{(a + c)\ldots(a + ic)b(b + c)\ldots(b + (k - i)c)}{(a + b + c)\ldots(a + b + (n - 1)c)}.\]
    Теперь заметим, что элемент суммы есть ни что иное, как \(\Pr(a + c, b, i, k - i - 1)\). Но \[\sum\limits_{i = 0}^{k - 1} \Pr(a + c, b, i, k - i - 1) = 1,\] так как эта сумма соответствует вероятности вытащить любой набор. Отсюда следует, что \[\Pr(A_k) = \dfrac{a}{a + b}.\]
    
    Перейдём к третьему (да и четвёртому тоже) пункту. По определению условной вероятности: \(\Pr(A_k \mid A_m) = \dfrac{\Pr(A_k \cap A_m)}{\Pr(A_m)}\). Как посчитать числитель? Точно так же, как и во втором случае. Пропустив аналогичные выкладки, выпишем ответ: \[\Pr(A_k \cap A_m) = \frac{a(a + c)}{(a + b)(a + b + c)}.\]
    Тогда получаем, что \(\Pr(A_k \mid A_m) = \dfrac{a + c}{a + b + c}.\)
\end{proof}

\begin{problem}
    Пусть \(A, B, C\)~--- попарно независимые равновероятные события, причём \(A \cap B \cap C = \emptyset\). Найти максимально возможное значение \(\Pr(A)\).
\end{problem}
\begin{proof}[Решение]
    Начнём с того, что заметим следующее: \(\Pr(A) \geq \Pr(A \cap (B \cup C)) = \Pr((A \cap B) \cup (A \cap C))\). Так как \(A \cap B \cap C = \emptyset\), то \((A \cap B) \cap (A \cap C) = \emptyset\). Следовательно, \(\Pr[A] \geq \Pr[A \cap B] + \Pr[A \cap C] = 2(\Pr[A])^2\) и \(\Pr(A) \leq 1/2\).
    
    Приведём пример, когда выполняется условие, причём \(\Pr(A) = 1/2\). Рассмотрим классическую модель \(\Omega = \{1, 2, 3, 4\}\) и события: \(A = \{1, 2\}\), \(B = \{2, 3\}\), \(C = \{3, 4\}\). Легко понять, что данные события удовлетворяют условию и \(\Pr(A) = 1/2\).
\end{proof}

\begin{problem}
    Игроки \(A\) и \(B\) играют в теннис. При розыгрыше на подаче \(A\) игрок \(A\) выигрывает с вероятностью \(p_1\), а при розыгрыше на подаче \(B\)~--- с вероятностью \(p_2\), все розыгрыши независимы. Игрок \(A\) подаёт первым, а выигрывает тот, кто первым наберёт \(n\) очков. Существует два варианта правил перехода подачи:
    \begin{itemize}
        \item[(а)] поочерёдная;
        \item[(б)] игрок подаёт до тех пор, пока не проиграет розыгрыш.
    \end{itemize}
    Покажите, что вероятность выигрыша \(A\) не зависит от правил перехода подачи, и вычислите её.
\end{problem}
\begin{proof}[Решение]
    Будем считать, что всего было проведено \(2n - 1\) розыгрышей. При таком количестве один игрок гарантированно наберёт не менее \(n\) очков, а второй~--- гарантированно меньше. Теперь опишем вероятностное пространство. Элементарные исходы будут иметь вид \(\omega = (a_1, a_2, \ldots, a_{2n - 1})\), где \(a_i \in \{0, 1\}\) (0 соответствует проигрышу, 1~--- победе). Согласно этой схеме исход будет подходящим, если в наборе будет не меньше \(n\) единиц.
    
    Начнём с пункта (а).
    
    Поймём, как посчитать вероятность какого-либо элементарного случая. Пусть \(n = 4\) и мы хотим найти вероятность элементарного исхода \(0110110\). Она равна \((1 - p_1)p_2p_1(1 - p_2)p_1p_2(1 - p_1) = p_{1}^{2}p_{2}^{2}(1 - p_{1})^{2}(1 - p_2)^{1}\). Отсюда получаем закономерность: вероятность элементарного исхода \(a_{1}a_{2}\ldots a_{2n - 1}\) равна \[p_{1}^{\sum\limits_{i = 0}^{n - 1} a_{2i + 1}}p_{2}^{\sum\limits_{i = 1}^{n - 1} a_{2i}}(1 - p_{1})^{n - \sum\limits_{i = 0}^{n - 1} a_{2i + 1}}(1 - p_{2})^{n - 1 - \sum\limits_{i = 1}^{n - 1} a_{2i}}.\]
    Пусть \(k_{1} = \sum\limits_{i = 0}^{n - 1} a_{2i + 1}\)~--- количество единиц на нечётных местах, а \(k_{2} = \sum\limits_{i = 1}^{n - 1} a_{2i}\)~--- на чётных. Тогда вероятность того, что \(A\) выиграл, будет равна
    \[\Pr = \sum\limits_{\substack{k_1, k_2 \\ k_1 + k_2 \geq n \\ k_1, k_2 \leq n}}\binom{n}{k_1}\binom{n - 1}{k_2}p_{1}^{k_1}p_{2}^{k_2}(1 - p_{1})^{n - k_1}(1 - p_{2})^{n - 1 - k_2}.\]
    
    Теперь перейдём к пункту (б).
    
    Докажем следующее: между данными методами подачи есть биекция, т.е. игре с поочерёдной подачей можно сопоставить игру с подачей до проигрыша. Рассмотрим частный случай: поочерёдно вышел исход \(0110110\). Тогда можно ``раскидать'' партии так:
    \begin{center}
        \begin{tabular}{ll}
            Подаёт первый: &0110 \\
            Подаёт второй: &101
        \end{tabular}
    \end{center}
    Биекция будет иметь вид \(0101101\). Как её построить? Разбиваем исход на подачи первого и второго игрока, тем самым получая строки длиной \(n\) и \(n - 1\). После этого строим по ним новую строку по следующему алгоритму:
    \begin{algorithm}[H]
        \caption{Построение исхода в случае подачи до проигрыша по исходу в случае поочерёдной подачи}
        \begin{algorithmic}[1]
            \State Начинаем со строки длины \(n\) (строки для первого игрока);
            \State Копируем строку посимвольно до тех пор, пока не попадём на 0;
            \State Переходим на другую строку;
            \State Повторяем два предыдущих шага до тех пор, пока не перенесём все символы.
        \end{algorithmic}
    \end{algorithm}
    Алгоритм построения исхода при поочерёдной подаче по исходу при подаче до проигрыша будет почти аналогичен. 
\end{proof}

Перейдём к задачам на тему математического ожидания и дисперсии.
\begin{problem}
    Бросили два \(N\)-гранных кубика. Пусть \(\xi\)~--- сумма выпавших очков. Найдите \(\E[\xi]\) и \(\D[\xi]\). 
\end{problem}
\begin{proof}[Решение]
    Пусть \(\E[\xi_1]\)~--- матожидание количество очков, выпавших на первом кубике. Посчитать его несложно:
    \(\E[\xi_1] = \sum\limits_{i = 1}^{N}\dfrac{i}{N} = \dfrac{N + 1}{2}\). Заметим, что \(\E[\xi] = \E[\xi_1 + \xi_2] = \E[\xi_1] + \E[\xi_2]\). Тогда \[\E[\xi] = \frac{N + 1}{2} + \frac{N + 1}{2} = N + 1.\]
    
    Дисперсию будем считать по сделующей формуле: \(\D[\xi_1] = \E[\xi_1^2] - (\E[\xi_1])^2\).
    Посчитаем первый член: \[\E[\xi_1^2] = \sum_{i = 1}^{N}\frac{i^2}{N} = \frac{N(2N + 1)(N + 1)}{6N} = \frac{(2N + 1)(N + 1)}{6}.\]
    Отсюда получаем, что
    \[\D[\xi_1] = \frac{N + 1}{2}\left(\frac{2N + 1}{3} - \frac{N + 1}{2}\right) = \frac{(N + 1)(N - 1)}{12} = \frac{N^2 - 1}{12}.\]
    Заметим, что \(\xi_1\) и \(\xi_2\) независимы (ведь кубики тоже независимы). Тогда \(\D[\xi] = \D[\xi_1] + \D[\xi_2]\) и \[\D[\xi] = \dfrac{N^2 - 1}{6}.\qedhere\]
\end{proof}

\begin{problem}
    Пусть выбрана случайная перестановка \(\sigma \in S_n\). Введём случайную величину \(\xi\), равная количеству стационарных точек (чисел \(i\) таких, что \(\sigma(i) = i\)). Найдите \(\E[\xi]\) и \(\D[\xi]\).
\end{problem}
\begin{proof}[Решение]
    Введём событие \(A_i = \{\sigma(i) = i\}\). Тогда \(\xi = \sum\limits_{i = 1}^{n} I_{A_i}\).
    По свойству линейности:
    \[\E[\xi] = \E\left[\sum_{i = 1}^{n} I_{A_i}\right] = \sum_{i = 1}^{n} \E[I_{A_i}] = \sum_{i = 1}^{n} \Pr(A_i)\]
    Так как \(\Pr(A_i) = \frac{(n - 1)!}{n!} = \frac{1}{n}\), то \(\E[\xi] = 1\).
    
    Теперь перейдём к подсчёту дисперсии. Распишем дисперсию через ковариации:
    \[\D[\xi] = \cov(\xi, \xi) = \cov\left(\sum_{i = 1}^{n} I_{A_i}, \sum_{i = 1}^{n} I_{A_i}\right) = \sum_{i = 1}^{n} \sum_{j = 1}^{n} \cov(I_{A_i}, I_{A_i}).\]
    Посчитаем \(\cov(I_{A_i}, I_{A_i})\). По свойству ковариации она равна \[\E[I_{A_i}I_{A_j}] - \E[I_{A_i}]\E[I_{A_j}] = \E[I_{A_i \cap A_j}] - \E[I_{A_i}]\E[I_{A_j}] = \Pr[A_i \cap A_j] - \Pr[A_i]\Pr[A_j].\] Возникают два случая:
    \begin{itemize}
        \item[(a)] \(i = j\). Тогда \(\cov(I_{A_i}, I_{A_j}) = \frac{1}{n} - \frac{1}{n^2}\).
        \item[(б)] \(i \neq j\). Тогда \(\Pr[A_i \cap A_j] = \frac{(n - 2)!}{n!} = \frac{1}{n(n - 1)}\) и \(\cov(I_{A_i}, I_{A_j}) = \frac{1}{n^2(n - 1)}\).
    \end{itemize}
    Отсюда получаем, что \[\D[\xi] = \frac{n(n - 1)}{2}\frac{1}{n^2(n - 1)} + n\left(\frac{1}{n} - \frac{1}{n^2}\right) = 1.\qedhere\]
\end{proof}