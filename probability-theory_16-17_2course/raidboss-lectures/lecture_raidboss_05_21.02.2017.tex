\section{Лекция от 21.02.2017}

\subsection{Производящая функция случайной величины}

Несколько лекций назад мы познакомились с производящей функцией
последовательности. А теперь давайте введём понятия для произвольной случайной
величины:

\begin{definition}
  \textit{Производящей функцией} случайной величины $\xi$ называют
  функцию $\phi_{\xi}(z) = \E{z^{\xi}}$ при $z \geqslant 0$.
\end{definition}

Начнём сразу говорить о каких-то не очень сложных и одновременно важных
свойствах.

\begin{itemize}
  \item[1.] $\phi_{\xi}(1) = 1$. Тривиально.
  \item[2.] $\phi_{\xi}'(1) = \E{\xi}$. Доказывается так же, как и в случае характеристических функций (скажем, для $|z| \leq 1$ будет достаточно).
  \item[3.] Если $\xi, \eta$ независимы, тогда
  \[
    \phi_{\xi + \eta}(z) = \phi_{\xi}(z) \cdot \phi_{\eta}(z).
  \]
  Тривиально следует из свойств матожидания.
  \item[4.] Если $\xi \in \Z_+$, тогда 
  \[
    \phi_{\xi}(z) = \sum\limits_{k = 0}^{+\infty} z^k\Pr{\xi = k}.
  \]
  Очень простое и понятное свойство. Этот ряд сходится абсолютно при $|z| \leq 1$,
  так как сумма ограничена суммой вероятностей.
  Можно заметить, что для дискретной величины свойство 2 доказывается очень легко.

  \item[5.] При $|z| < 1$ дискретную характеристическую функцию
  можно дифференцировать
  бесконечное число раз, так как по формуле Коши-Адамара радиус
  сходимости останется тот же. При $|z| = 1$ непонятно что будет происходить, так
  как ряд $k\Pr{\xi = k}$ может уже расходиться.

  \item[6.] В дискретной случайной величине $\phi_{\xi}(0) = \Pr{\xi = 0}$, если
  $k$ раз продифференцировать и подставить $z = 0$, понятно, что получим

  \[
    \Pr{\xi = k} = \frac{1}{k!}\frac{\partial^k}{\partial z^k}\phi_{\xi}(z)
  \]
\end{itemize}

Как ни странно, именно производящие функции помогают получить результаты в случайных
дискретных ветвящихся процессах.

\subsection{Вероятность вырождения}

Перед тем, как доказывать сразу окончательный результат, докажем несколько лемм.

\begin{lemma}[О зависимости $X_n$ и $X_{n - 1}$]
  Пусть $X_n$ --- ветвящийся случайный процесс с законом размножения частиц $\xi$.

  Тогда $\phi_{X_n}(z) = \phi_{X_{n - 1}}(\phi_\xi(z))$.
\end{lemma}

\begin{proof}
  \[
    \phi_{X_n}(z) = \E{z^{X_n}} = \E{z^{\sum\limits_{k = 1}^{X_{n - 1}} \xi_k^{(n)}}}
  \]

  Заметим, что $X_{n - 1}$ принимает какое-то значение дискретное значение,
  поэтому если ввести случайную величину $\eta = \sum\limits_{m = 0}^{+\infty} 
  \I\{X_{n - 1} = m\}$, то $\eta = 1$ всегда. Поэтому продолжим равенство:

  \[
    \E{z^{\sum\limits_{k = 1}^{X_{n - 1}} \xi_k^{(n)}}} =
    \E{z^{\sum\limits_{k = 1}^{X_{n - 1}} \xi_k^{(n)}} \sum\limits_{m = 0}^{+\infty} 
    \I\{X_{n - 1} = m\}} = \sum\limits_{m = 0}^{+\infty} \E{z^{\sum\limits_{k = 
    1}^{X_{n - 1}} \xi_k^{(n)}}\I\{X_{n - 1} = m\}}
  \]

  Заметим, что $X_{n - 1}$ зависит только от $\xi_k^{(\ell)}$, где $\ell \leq n - 1$,
  а $k$ --- любое, поэтому можно продолжить равенство с использованием
  независимости случайных величин (и ещё вместо $X_{n - 1}$
  подставим $m$):

  \[
    \sum\limits_{m = 0}^{+\infty} \E{z^{\sum\limits_{k = 1}^{m}
    \xi_k^{(n)}}\I\{X_{n - 1} = m\}} = \sum\limits_{m = 0}^{+\infty} \E{z^{\sum\limits_{k = 1}^{m}
    \xi_k^{(n)}}} \Pr{X_{n - 1} = m} = 
  \]

  Заметим, что $\E{z^{\sum\limits_{k = 1}^{m} \xi_k^{(n)}}} = 
  \left(\phi_{\xi}(z)\right)^m$ из-за свойства 3 производящих функций и независимости
  величин $\xi_k^{(n)}$. Откуда получаем:

  \[
    = \sum\limits_{m = 0}^{+\infty} \left(\phi_{\xi}(z)\right)^m \Pr{X_{n - 1} = m} =
    \phi_{X_{n - 1}}(\phi_{\xi}(z))
  \]
\end{proof}

\begin{consequence}
  Воспользуемся тем, что $X_0 = 1$, тогда рекурсивно получим:

  \[
    \phi_{X_n}(z) = \underbrace{\phi_{\xi}(\ldots(\phi_{\xi}(z))\ldots)}_{n \text{ раз}}
  \]

  Откуда получаем (если внимательно приглядеться к результату леммы),
  что $\phi_{X_n}(z) = \phi_{\xi}(\phi_{X_{n - 1}}(z))$.
\end{consequence}

Введем обозначение $q_n = \Pr{X_n = 0}$. Тогда сформулируем следующую лемму:

\begin{lemma}
  $q_n \leq q_{n + 1}$ и $q = \lim\limits_{n \to +\infty} q_n$, где $q$ --- вероятность
  вырождения процесса.
\end{lemma}

\begin{proof}
  Первое неравенство совершенно тривиально, так как $\{X_n = 0\} \subseteq 
  \{X_{n + 1} = 0\}$, так как если $X_n = 0$, то уж и $X_{n + 1} = 0$.

  $\{\exists n: X_n = 0\} = \bigcup\limits_{n = 1}^{+\infty} X_n$. В силу
  непрерывности вероятности получаем, что

  \[
    q = \lim\limits_{n \to +\infty} \Pr{X_n = 0} = \lim\limits_{n \to +\infty}
    q_n
  \]
\end{proof}

\begin{theorem}
  Если $q$ --- вероятность вырождения, тогда $q = \phi_{\xi}(q)$.
\end{theorem}

\begin{proof}
  $q_n = \Pr{X_n = 0} = \phi_{X_n}(0)$ (см. свойство 6). По следствию
  из леммы о зависимости $X_n$ и $X_{n - 1}$ получаем, что

  \[
    q_n = \Pr{X_n = 0} = \phi_{X_n}(0) = \phi_{\xi}(\phi_{X_{n - 1}}(0)).
  \]

  Заметим, что у $\phi_{X_n}(0)$ существует предел, так как по лемме выше у
  $q_n$ существует предел, поэтому устремим $n \to +\infty$, откуда

  \[
    q = \phi_{\xi}(q), \text{ что и требовалось доказать.}
  \]
\end{proof}

Заметим, что $q = 1$
всегда подходит, но иногда вероятность вырождения меньше единицы. Так вот, следующая
теорема утверждает, что таких $q$ не больше двух и если их два, то надо брать
наименьшее.

\begin{theorem}[О вероятности вырождения]
  Пусть $\Pr{\xi = 1} < 1$ и $\mu = \E{\xi}$ ($\mu$ необязательно конечно).

  \begin{itemize}
    \item[1)] Если $\mu \leq 1$, тогда $z = \phi_{\xi}(z)$  имеет ровно одно 
    решение --- единица;
    \item[2)] Если $\mu > 1$, тогда  $z = \phi_{\xi}(z)$ имеет один корень 
    $z_0$ на $[0, 1)$ и $q = z_0$.
  \end{itemize}
\end{theorem}

\begin{proof}
  Рассмотрим два случая в порядке их очереди.

  1) Если $\Pr{\xi = 0} = 1$, тогда утверждение очевидно. Пусть $\Pr{\xi = 0} < 1$.
  Рассмотрим производную $\phi_{\xi}'(z)$.

  \[
    \phi_{\xi}'(z) = \sum\limits_{k = 1}^{+\infty} kz^{k - 1}\Pr{\xi = k}
  \]

  Поэтому из формулы можно заключить, что $\phi_{\xi}'(z) \nearrow$ с ростом 
  $z > 0$, так как \newline $\Pr{\xi = 0} < 1$.

  Поэтому имеем по свойству 1 и формуле конечных приращений, что ($z < 1$):

  \[
    1 - \phi_{\xi}(z) = \phi_{\xi}(1) - \phi_{\xi}(z) = \phi_{\xi}'(\theta)(1 - z)
  \]

  Где $\theta \in (0, z)$. Но производная строго возрастающая функция, как мы показали
  выше, поэтому:

  \[
    \phi_{\xi}'(\theta)(1 - z) < \phi_{\xi}'(1)(1 - z) = \mu (1 - z) \leq (1 - z)
  \]

  Откуда $1 - \phi_{\xi}(z) < 1 - z$, откуда $z < \phi_{\xi}(z)$ для всех $z \in [0, 1)$.

  2) Заметим, что существует $k \geq 2$, что $\Pr{\xi = k} > 0$, иначе матожидание
  было бы не больше единицы. Поэтому $\phi_{\xi}''(z) = \sum\limits_{k = 2}^{+\infty}
  k(k - 1)z^{k - 2}\Pr{\xi = k}$ строго положительная функция.

  Введем функцию $f(z) = z - \phi_{\xi}(z)$. Тогда $f'(z) = 1 - \phi_{\xi}'(z)$.

  Заметим, что $f'(0) = 1 - \phi_{\xi}'(0) = 1 - \Pr{\xi = 1} > 0$ (по условию).
  А $f'(1) =  1 - \mu < 0$.

  Получаем, что производная у $f$ сначала возрастает, потом убывает, так как
  $\phi_{\xi}'(z)$ строго возрастающая функция. $f(1) = 0$ по свойству 1 
  производящих функций.

  $f(0) = -\phi_{\xi}(0) = -\Pr{\xi = 0}$, откуда давайте поймём, что мы начинаем из
  неположительного значения, возрастаем, а потом в единице приходим в ноль --- 
  тут будет строго один корень на $[0, 1)$ из-за непрерывности и поведения функции.

  \begin{minipage}{0.5\textwidth}
    \begin{tikzpicture}
      \draw[->] (0,0) -- (5.4,0) node[right] {$z$};
      \draw[->] (0,-3) -- (0,5.4) node[above] {$f(z)$};
      \draw (0,0) node[left] {$0$};
      \draw (5,0) node[below] {$1$};
      \draw (0.5,0) node[below right] {$z_0$};
      \draw (0,-2.5) node[left] {$-\Pr{\xi = 0}$};
      \draw[scale=1,domain=0:5,smooth,variable=\t,blue] plot ({\t},{-\t*\t + 5.5*\t - 2.5});
    \end{tikzpicture}
  \end{minipage}\hfill
  \begin{minipage}{0.5\textwidth}
    Рассмотрим 2 случая:

    1) $\Pr{\xi = 0} = 0$. Очевидно, что тогда вероятность вырождения равна нулю,
    так как каждый представитель будет иметь хотя бы одного потомка. 

    2) $\Pr{\xi = 0} > 0$, тогда обозначим этот корень за $z_0$.

    Несложно заметить, что $z - \phi_{\xi}(z) < 0$ только при $0 \leq z < z_0$.
    Остался последний вопрос про то, какое $q$ надо всё же выбрать. 

    Вообще, как мы выше показали, $q_n - q_{n + 1} \leq 0$. Но давайте в этом
    случае покажем строгое неравенство.

    Поймём, что
    \[
      q_{n + 1} = q_n + \sum\limits_{m = 1}^{+\infty} \Pr{X_n = m}(\Pr{\xi = 0})^m
    \]
  \end{minipage}
  
  То есть мы хотим, чтобы $X_n$ приняло какое-то значение и чтобы на следующем
  шаге все представители остались без потомков. Хотя бы при одном $m > 0$ мы точно
  знаем, что $\Pr{X_n = m} > 0$ (так как $\Pr{\xi > 0} > 0$, так как иначе 
  матожидание было бы меньше единицы) и $\Pr{\xi = 0} > 0$.
  Откуда действительно строгое неравенство.

  Вспомним, что $\phi_{\xi}(q_n) = q_{n + 1}$, поэтому для всех $n$
  выполняется, что $q_n - \phi_{\xi}(q_n) < 0$, значит все $q_n \in [0, z_0)$.
  Устремляем $n$ к бесконечности и получаем, что $q \in [0, z_0]$, а только в $z_0$
  функция обращается в ноль.
\end{proof}

Но в большинстве случаев это $q$ найти проблематично. Следующий пример покажет,
что даже в пуассоновском распределении будет трансцендентное уравнение.

\begin{example}
  Пусть $\xi \sim \mathrm{Pois}(c)$. Тогда

  \[
    \phi_{\xi}(z) = \sum\limits_{k = 0}^{+\infty} z^k\Pr{\xi = k} =
    \sum\limits_{k = 0}^{+\infty} \frac{(zc)^k}{k!}e^{-c} = e^{c(z - 1)}
  \]

  Откуда надо найти $q = e^{c(q - 1)}$. Введем $p = 1 - q$ --- вероятность невырождения:

  \[
    e^{-pc} + p = 1
  \]

  Если $c < 1$, тогда будет ровно один корень, а при $c \geq 1$
  будет уже два корня (оставим это утверждение для доказательства исходя только 
  из функции, как упражнение читателю).
\end{example}

Немного обозначим терминологию.

\begin{definition}
  Будем классифицировать случайные процессы в зависимости от среднего
  значения закона размножения.
  \begin{itemize}
    \item Если $\mu < 1$, тогда случайный процесс называют \textit{докритическим.}
    \item Если $\mu = 1$, тогда случайный процесс называют \textit{критическим.}
    \item Если $\mu > 1$, тогда случайный процесс называют \textit{надкритическим.}
  \end{itemize}
\end{definition}

В первых двух случаях мы доказали, что $\Pr{X_n = 0} \to 1$, то есть фактически \newline
$X_n \prto 0$. Следующую теорему мы оставим без доказательства, но она показывает,
как растет количество потомков на уровне $n$.

\begin{theorem}[О надкритическом случае]
  Пусть $\mu > 1$, $\D{\xi} = \sigma^2 < +\infty$ и $W_n = \frac{X_n}{\mu^n}$, тогда существует
  такая случайная величина $W$, что выполняются четыре утверждения:
  \begin{center}
    \begin{itemize}
      \centering
      \item[1)] $W_n \asto W$;
      \item[2)] $W_n \stackrel{L^2}{\longrightarrow} W$;
      \item[3)] $\Pr{W = 0} = q$;
      \item[4)] $\E{W} = 1, \D{W} = \dfrac{\sigma^2}{\mu^2 - \mu}$.
    \end{itemize}
  \end{center}
\end{theorem}

\subsection{Общее число частиц после $n$-го хода}

Теперь будем оценивать количество частиц/экземпляров до момента времени $n$.
Введем случайную величину $Y_n = 1 + X_1 + \ldots + X_n$. Ясно, что $Y_n$
как раз и отвечает за это самое количество.

\begin{lemma}[О зависимости $Y_n$ и $Y_{n - 1}$]
  \[
    \phi_{Y_n}(z) = z\cdot\phi_{\xi}(\phi_{Y_{n - 1}}(z))
  \]
\end{lemma}

\begin{proof}
  Аналогично распишем, как и в лемме о зависимости $X_n$ и $X_{n - 1}$.

  \[
    \phi_{Y_n}(z) = \E{z^{Y_n}} = \sum\limits_{m = 0}^{+\infty} \E{z^{Y_n}
    \I\{X_1 = m\}}
  \]

  То есть посмотрим, на сколько мы в первый раз разбились. Пусть $A_i$ --- 
  количество всего потомков у $i$-й частицы после первого
  хода. Заметим, что $A_i \eqdist Y_{n - 1}$, так как мы убрали первый уровень.

  Это разбиение нам на руку, так как $A_i$ независимы в совокупности, а также
  независимы с $X_1$.

  \[
    Y_n = 1 + \sum\limits_{k = 1}^n A_k\I\{X_1 = m\}
  \]

  Откуда мы получаем, что

  \[
    \sum\limits_{m = 0}^{+\infty} \E{z^{Y_n}
    \I\{X_1 = m\}} = z\sum\limits_{m = 0}^{+\infty} \E{z^{\sum\limits_{k = 1}^m
    A_k}} \Pr{X_1 = m}
  \]

  Легко видеть, что $\E{z^{A_k}} = \phi_{Y_{n - 1}}(z)$, так как мы пользовались
  только
  распределением. Поэтому подставим по свойству 3 производящих функций:

  \[
    z\sum\limits_{m = 0}^{+\infty} \E{z^{\sum\limits_{k = 1}^m
    A_k}} \Pr{X_1 = m} = z\sum\limits_{m = 0}^{+\infty} 
    (\phi_{Y_{n - 1}}(z))^m\Pr{X_1 = m} = z\phi_{\xi}(\phi_{Y_{n - 1}}(z))
  \]

  Последнее равенство следует из того, что $\xi \eqdist X_1$.
\end{proof}

А теперь давайте попробуем установить, будет ли какой-то предел у характеристических
функции $Y_n$. На самом деле, предыдущее равенство указывает, что при $|z| < 1$
будет <<убывание>>, давайте строго сформулируем утверждение.

\begin{lemma}[О пределе характеристических функций $Y_n$]
  Для любого $z \in [0, 1)$ существует предел $\lim\limits_{n \to +\infty} 
  \phi_{Y_n}(z) = \rho(z)$, притом $\rho(z) = z\phi_{\xi}(\rho(z))$.
\end{lemma}

\begin{proof}
  Докажем, что для любого $z \in [0, 1)$ верно, что
  $\phi_{Y_{n + 1}}(z) \leqslant \phi_{Y_n}(z)$, тогда мы докажем утверждение,
  так как снизу эти выражения ограничены нулём.

  Будем вести индукцию по $n$. Удивительным образом база доказывается дольше
  перехода. 

  База $n = 0$. Выпишем очевидные равенства из свойств производящих функций:

  \[
    \phi_{Y_1}(z) = \phi_{1 + X_1}(z) = \phi_{1}(z)\phi_{X_1}(z) = 
    \phi_{1}(z)\phi_{\xi}(z)
  \]

  Теперь у случайной величины <<один>> будет в точке $z$ значение $z$ по определению
  характеристической функции. $X_1 \eqdist \xi$, так как это количество
  первых потомков от первого. Поэтому верно последнее равенство.

  Но заметим, что $\phi_{\xi}(z) < 1$, так как это меньше суммы всех вероятностей.

  Поэтому мы получили, что $\phi_{Y_1}(z) < z = \phi_{Y_0}(z)$. Последнее равенство
  следует из того, что $Y_0 \eqdist 1$.

  Переход $n \to n + 1$. По лемме о зависимости мы получаем:

  \[
    \phi_{Y_{n + 1}}(z) = z\phi_{\xi}(\phi_{Y_n}(z))
  \]

  Воспользуемся предположением индукции и тем, что характеристическая функция
  $\xi$ не убывает с ростом $z$, откуда:

  \[
    z\phi_{\xi}(\phi_{Y_n}(z)) \leqslant z\phi_{\xi}(\phi_{Y_{n - 1}}(z)) =
    \phi_{Y_n}(z)
  \]

  Первое утверждение, о котором мы дополнительно заявляли, сразу следует из предела,
  который мы доказали.

\end{proof}

Также давайте доопределим $\rho(1) = q$.
